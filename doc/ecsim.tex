%=========================================================
% AMPS PIC/ECSIM solver analysis (LaTeX-ready writeup)
%=========================================================

\newpage
\section{AMPS PIC Electromagnetic Field Solver (ECSIM): Code-Level Analysis}

This section summarizes the structure of the AMPS implicit electromagnetic PIC field solve (ECSIM) as implemented in the uploaded sources (\texttt{pic\_field\_solver\_ecsim.cpp}, \texttt{get\_stencil.cpp}, \texttt{update\_rhs.cpp}, \texttt{LinearSystemCornerNode.h}, \texttt{pic.h}).

\subsection{What the linear system solves for: an increment \texorpdfstring{$\Delta \mathbf{E}$}{ΔE}}
The linear solver is formulated for an \emph{increment} rather than the absolute field:
\begin{equation}
\Delta \mathbf{E} \;\equiv\; \mathbf{E}^{n+\theta} - \mathbf{E}^{n}.
\end{equation}
After solving, the half-step electric field is reconstructed as
\begin{equation}
\mathbf{E}^{n+\theta} \;=\; \mathbf{E}^{n} \;+\; \frac{\mathbf{x}}{E_{\mathrm{conv}}},
\end{equation}
where $\mathbf{x}$ is the raw solver solution vector stored per corner node (3 unknowns per corner), and $E_{\mathrm{conv}}$ is the electric-field conversion factor. In this formulation, field boundary conditions are naturally imposed as
\begin{equation}
\Delta \mathbf{E}\big|_{\partial\Omega} = \mathbf{0}.
\end{equation}

\subsection{Time-step pipeline in \texttt{ECSIM::TimeStep()}}
A typical time step follows:

\begin{enumerate}
\item \textbf{Update particle moments for the implicit solve:}
\begin{itemize}
\item Compute the implicit-current predictor $\hat{\mathbf{J}}$ on corner nodes.
\item Accumulate/store the \textbf{mass matrix} coefficients $M$ on corner nodes (packed in a block-local buffer layout).
\end{itemize}

\item \textbf{Assemble/update the linear system matrix:}
\begin{itemize}
\item When needed, build the sparsity pattern (stencil connectivity).
\item Each time step, update matrix values using a decomposition into:
\begin{equation}
A_{ij} \;=\; A^{(\mathrm{param})}_{ij} \;+\; \left(4\pi\,\Delta t\,\theta\right)\,A^{(\mathrm{mm})}_{ij},
\end{equation}
where $A^{(\mathrm{mm})}_{ij}$ references mass-matrix entries via pointer-based ``support tables'' (one support entry per neighbor/component coupling).
\end{itemize}

\item \textbf{Assemble the RHS:} evaluate a sum of \emph{semantic} support contributions (corner and center samples) of the form
\begin{equation}
\mathrm{rhs} \;\leftarrow\; \mathrm{rhs} \;+\; \sum_{s\in\mathcal{S}} \alpha_s\, \mathcal{Q}_s,
\end{equation}
where each support entry $s$ encodes: (i) a quantity type (e.g., $\mathbf{E}$, $\mathbf{B}$, $\hat{\mathbf{J}}$, mass matrix, etc.), (ii) where to sample it (corner/center node + offsets), and (iii) a coefficient $\alpha_s$.

\item \textbf{Solve for $\Delta\mathbf{E}$:} 3 unknowns per corner node.

\item \textbf{Update magnetic field:} advance $\mathbf{B}$ using Faraday's law with $\mathbf{E}^{n+\theta}$.

\item \textbf{Convert $\mathbf{E}^{n+\theta}\rightarrow \mathbf{E}^{n+1}$:}
\begin{equation}
\mathbf{E}^{n+1} \;=\; \frac{\mathbf{E}^{n+\theta} - (1-\theta)\mathbf{E}^{n}}{\theta}.
\end{equation}

\item \textbf{Apply boundary conditions / exchange halos} as appropriate.
\end{enumerate}

\subsection{Stencil/operator structure constructed by \texttt{GetStencil()}}
For each component row $iVar\in\{x,y,z\}$ at a corner $(i,j,k)$, the stencil includes:

\paragraph{(i) Curl--curl operator via the identity}
The code constructs the vector operator using
\begin{equation}
\nabla\times\nabla\times \mathbf{E} \;=\; \nabla(\nabla\cdot\mathbf{E}) \;-\; \nabla^2\mathbf{E}.
\end{equation}
This appears in the implementation as the sum of a ``minus Laplacian'' block and a ``plus grad--div'' block.

\paragraph{(ii) Identity term for the increment formulation}
The diagonal (self) entry is explicitly incremented by $+1$, corresponding to the $I\,\Delta\mathbf{E}$ part of the linear system.

\paragraph{(iii) Mass-matrix coupling in the LHS}
The stencil includes couplings to up to $3\times 3\times 3$ corner-node neighbors and all vector components (up to 81 couplings per row block), whose values are computed as
\begin{equation}
A_{ij}^{(\mathrm{mm})} \;=\; (4\pi\,\Delta t\,\theta)\,M_{ij},
\end{equation}
with $M_{ij}$ stored in packed buffers and accessed through a precomputed offset table.

\subsection{RHS anatomy and a key consistency check (unit conversion)}
Because the unknown is $\Delta\mathbf{E}$, any implicit term containing $M\mathbf{E}^{n+\theta}$ expands as
\begin{equation}
M\mathbf{E}^{n+\theta} \;=\; M(\mathbf{E}^{n}+\Delta\mathbf{E}) \;=\; M\mathbf{E}^{n} \;+\; M\Delta\mathbf{E}.
\end{equation}
Thus, $M\Delta\mathbf{E}$ belongs on the LHS, while $M\mathbf{E}^n$ must appear on the RHS with the appropriate sign.

\medskip
\noindent\textbf{Important implementation consistency check:} the legacy RHS path explicitly applies field conversion factors (e.g., $E_{\mathrm{conv}}$, $B_{\mathrm{conv}}$) to samples. In the newer semantic RHS path, the conversions must be applied \emph{either} inside the sampling routine \emph{or} folded into the stencil coefficients; otherwise the semantic RHS may not be numerically equivalent to the legacy formulation when $E_{\mathrm{conv}}$ or $B_{\mathrm{conv}}$ differ from unity.

\subsection{Boundary conditions}
Boundary rows are short-circuited to enforce
\begin{equation}
\Delta \mathbf{E} = \mathbf{0}
\end{equation}
(by setting a unit diagonal and removing couplings/supports), which is consistent with the increment formulation.


%=========================================================
% Focused “physics ↔ discretization” derivation (LaTeX)
%=========================================================

\section{Focused Physics $\leftrightarrow$ Discretization Derivation (ECSIM Increment Form)}

\subsection{Continuous equations (Gaussian units, as suggested by $4\pi$ factors)}
Consider Maxwell's equations (neglecting charge conservation details here since the field solve is written in terms of $\mathbf{E}$ and $\mathbf{B}$):
\begin{align}
\frac{\partial \mathbf{B}}{\partial t} &= -c\,\nabla\times \mathbf{E}, \label{eq:faraday}\\
\frac{\partial \mathbf{E}}{\partial t} &= c\,\nabla\times \mathbf{B} - 4\pi\,\mathbf{J}. \label{eq:ampere}
\end{align}
ECSIM introduces a \emph{linearized implicit current response} at the field-solve time level,
\begin{equation}
\mathbf{J}^{n+\theta} \;\approx\; \hat{\mathbf{J}}^{\,n} \;+\; M\,\mathbf{E}^{n+\theta},
\label{eq:implicit_current_model}
\end{equation}
where
\begin{itemize}
\item $\hat{\mathbf{J}}^{\,n}$ is a predicted current (independent of $\mathbf{E}^{n+\theta}$),
\item $M$ is the (sparse) mass matrix constructed from particle data and shape functions.
\end{itemize}

\subsection{Temporal discretization with a $\theta$-scheme}
Define the intermediate (implicit) time level
\begin{equation}
\mathbf{E}^{n+\theta} \;=\; (1-\theta)\mathbf{E}^n + \theta \mathbf{E}^{n+1},\qquad
\mathbf{B}^{n+\theta} \;=\; (1-\theta)\mathbf{B}^n + \theta \mathbf{B}^{n+1}.
\end{equation}
Discretize Faraday \eqref{eq:faraday} using $\mathbf{E}^{n+\theta}$:
\begin{equation}
\mathbf{B}^{n+1} \;=\; \mathbf{B}^n \;-\; c\,\Delta t \,\nabla\times \mathbf{E}^{n+\theta}.
\label{eq:B_update_theta}
\end{equation}
Discretize Amp\`ere \eqref{eq:ampere} using $\mathbf{B}^{n+\theta}$ and $\mathbf{J}^{n+\theta}$:
\begin{equation}
\mathbf{E}^{n+1} \;=\; \mathbf{E}^n \;+\; \Delta t \left(c\,\nabla\times \mathbf{B}^{n+\theta} \;-\; 4\pi\,\mathbf{J}^{n+\theta}\right).
\label{eq:E_update_theta}
\end{equation}
Insert the implicit current model \eqref{eq:implicit_current_model}:
\begin{equation}
\mathbf{E}^{n+1} \;=\; \mathbf{E}^n \;+\; \Delta t \left(c\,\nabla\times \mathbf{B}^{n+\theta} \;-\; 4\pi\,\hat{\mathbf{J}}^{\,n} \;-\; 4\pi\,M\,\mathbf{E}^{n+\theta}\right).
\label{eq:E_update_theta_with_M}
\end{equation}

\subsection{Eliminate $\mathbf{B}^{n+\theta}$ to obtain a single equation for $\mathbf{E}^{n+\theta}$}
From \eqref{eq:B_update_theta}:
\begin{equation}
\mathbf{B}^{n+\theta} \;=\; \mathbf{B}^n \;-\; \theta\,c\,\Delta t \,\nabla\times \mathbf{E}^{n+\theta}.
\label{eq:B_n_theta_in_terms_of_E}
\end{equation}
Substitute \eqref{eq:B_n_theta_in_terms_of_E} into \eqref{eq:E_update_theta_with_M}:
\begin{align}
\mathbf{E}^{n+1}
&= \mathbf{E}^n + \Delta t \left(
c\,\nabla\times\Big[\mathbf{B}^n - \theta c\Delta t \nabla\times \mathbf{E}^{n+\theta}\Big]
-4\pi \hat{\mathbf{J}}^{\,n} - 4\pi M \mathbf{E}^{n+\theta}
\right) \nonumber \\
&= \mathbf{E}^n + c\Delta t \,\nabla\times \mathbf{B}^n
- \theta (c\Delta t)^2 \,\nabla\times\nabla\times \mathbf{E}^{n+\theta}
-4\pi\Delta t\,\hat{\mathbf{J}}^{\,n}
-4\pi\Delta t\,M\mathbf{E}^{n+\theta}.
\label{eq:E_nplus1_expanded}
\end{align}
Now use the definition of $\mathbf{E}^{n+\theta}$:
\begin{equation}
\mathbf{E}^{n+1} \;=\; \frac{1}{\theta}\mathbf{E}^{n+\theta} \;-\; \frac{1-\theta}{\theta}\mathbf{E}^n.
\label{eq:E_nplus1_from_En_theta}
\end{equation}
Substitute \eqref{eq:E_nplus1_from_En_theta} into \eqref{eq:E_nplus1_expanded} and multiply by $\theta$ to isolate $\mathbf{E}^{n+\theta}$:
\begin{align}
\mathbf{E}^{n+\theta}
&= \mathbf{E}^n + \theta c\Delta t \,\nabla\times \mathbf{B}^n
- \theta^2 (c\Delta t)^2 \,\nabla\times\nabla\times \mathbf{E}^{n+\theta}
-4\pi\theta\Delta t\,\hat{\mathbf{J}}^{\,n}
-4\pi\theta\Delta t\,M\mathbf{E}^{n+\theta}.
\end{align}
Bring all $\mathbf{E}^{n+\theta}$-terms to the LHS:
\begin{equation}
\Big[I \;+\; \theta^2(c\Delta t)^2\,\nabla\times\nabla\times \;+\; 4\pi\theta\Delta t\,M\Big]\mathbf{E}^{n+\theta}
\;=\;
\mathbf{E}^n \;+\; \theta c\Delta t\,\nabla\times\mathbf{B}^n \;-\; 4\pi\theta\Delta t\,\hat{\mathbf{J}}^{\,n}.
\label{eq:field_equation_for_En_theta}
\end{equation}
\textbf{Note:} depending on sign conventions for $\nabla\times\nabla\times$ in the discrete operator (some codes implement it as $-\nabla^2+\nabla\nabla\cdot$ with explicit minus signs in coefficients), the assembled stencil may implement the equivalent operator with opposite internal sign. The safest equivalence statement is that the code assembles the standard curl--curl operator using
\begin{equation}
\nabla\times\nabla\times \mathbf{E} = \nabla(\nabla\cdot\mathbf{E})-\nabla^2\mathbf{E},
\end{equation}
and multiplies it by $(\theta c\Delta t)^2$ (up to the code's discrete sign convention).

\subsection{Increment form used in AMPS: $\mathbf{E}^{n+\theta}=\mathbf{E}^n+\Delta\mathbf{E}$}
Define $\Delta\mathbf{E}=\mathbf{E}^{n+\theta}-\mathbf{E}^n$ so that
\begin{equation}
\mathbf{E}^{n+\theta}=\mathbf{E}^n+\Delta\mathbf{E}.
\end{equation}
Insert into \eqref{eq:field_equation_for_En_theta}:
\begin{equation}
\Big[I + \theta^2(c\Delta t)^2\,\nabla\times\nabla\times + 4\pi\theta\Delta t\,M\Big](\mathbf{E}^n+\Delta\mathbf{E})
=
\mathbf{E}^n + \theta c\Delta t\,\nabla\times\mathbf{B}^n - 4\pi\theta\Delta t\,\hat{\mathbf{J}}^{\,n}.
\end{equation}
Rearrange to obtain the linear system for $\Delta\mathbf{E}$:
\begin{equation}
\underbrace{\Big[I + \theta^2(c\Delta t)^2\,\nabla\times\nabla\times + 4\pi\theta\Delta t\,M\Big]}_{\mathcal{A}}
\Delta\mathbf{E}
=
\underbrace{\theta c\Delta t\,\nabla\times\mathbf{B}^n - 4\pi\theta\Delta t\,\hat{\mathbf{J}}^{\,n}
- \theta^2(c\Delta t)^2\,\nabla\times\nabla\times \mathbf{E}^n
- 4\pi\theta\Delta t\,M\mathbf{E}^n}_{\mathbf{b}}.
\label{eq:increment_linear_system}
\end{equation}
This is the \textbf{physics $\leftrightarrow$ discretization} mapping:
\begin{itemize}
\item $I\,\Delta\mathbf{E}$ $\leftrightarrow$ diagonal ``$+1$'' identity term in the stencil.
\item $(\theta c\Delta t)^2\,\nabla\times\nabla\times\Delta\mathbf{E}$ $\leftrightarrow$ curl--curl assembled as ``$-\nabla^2+\nabla\nabla\cdot$'' (Laplacian + grad--div blocks).
\item $4\pi\theta\Delta t\,M\Delta\mathbf{E}$ $\leftrightarrow$ mass-matrix neighbor/component couplings in the LHS matrix.
\item $\theta c\Delta t\,\nabla\times\mathbf{B}^n$ $\leftrightarrow$ center-node $\mathbf{B}$ samples contributing to RHS.
\item $-4\pi\theta\Delta t\,\hat{\mathbf{J}}^{\,n}$ $\leftrightarrow$ corner-node predicted current in RHS.
\item $-(\theta c\Delta t)^2\,\nabla\times\nabla\times\mathbf{E}^n$ $\leftrightarrow$ explicit curl--curl of the old field moved to RHS.
\item $-4\pi\theta\Delta t\,M\mathbf{E}^n$ $\leftrightarrow$ the old-field mass-matrix dot-product moved to RHS.
\end{itemize}

\subsection{Discrete operator as implemented (stencil form)}
On a corner-node grid, each row corresponds to one component (say $\Delta E_x$ at node $p$). The discrete equation is:
\begin{equation}
\sum_{q\in\mathcal{N}(p)} \sum_{\beta\in\{x,y,z\}}
\Big[
\underbrace{K^{(\mathrm{curlcurl})}_{p\alpha,q\beta}}_{\text{from }(\theta c\Delta t)^2\nabla\times\nabla\times}
\;+\;
\underbrace{4\pi\theta\Delta t\,M_{p\alpha,q\beta}}_{\text{mass matrix}}
\Big]\Delta E_{q\beta}
\;+\;
\Delta E_{p\alpha}
\;=\;
b_{p\alpha},
\end{equation}
where $\mathcal{N}(p)$ is the neighbor set covered by the stencil (typically up to $3\times3\times3$ corner nodes), and $b_{p\alpha}$ corresponds to the RHS terms in \eqref{eq:increment_linear_system} sampled on corner/center nodes.

\subsection{Boundary conditions in increment form}
Rows corresponding to boundary nodes are replaced by:
\begin{equation}
\Delta E_{p\alpha} = 0,
\end{equation}
implemented as a unit diagonal with no couplings.

\subsection{Post-solve field updates}
After solving \eqref{eq:increment_linear_system}:
\begin{align}
\mathbf{E}^{n+\theta} &= \mathbf{E}^n + \Delta\mathbf{E},\\
\mathbf{B}^{n+1} &= \mathbf{B}^n - c\Delta t\,\nabla\times \mathbf{E}^{n+\theta},\\
\mathbf{E}^{n+1} &= \frac{\mathbf{E}^{n+\theta} - (1-\theta)\mathbf{E}^n}{\theta}.
\end{align}

\subsection{Unit-conversion consistency (legacy vs semantic RHS)}
If the code stores fields in code units, but assembles the linear system in physical (or mixed) units, then the sampling operation for $\mathbf{E}$ and $\mathbf{B}$ must apply $E_{\mathrm{conv}}$ and $B_{\mathrm{conv}}$ consistently. Equivalently, the stencil coefficients must absorb these conversion factors. A mismatch causes the new semantic RHS to deviate from the legacy RHS when $E_{\mathrm{conv}}\neq 1$ and/or $B_{\mathrm{conv}}\neq 1$.

