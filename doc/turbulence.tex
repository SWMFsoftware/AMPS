

\section{Turbulence}


\subsection{Wave Transport Equation (with Kolmogorov Spectrum) for Bidirectional Alfvén Waves}

Assuming Alfvén waves propagate along the magnetic field (both {\it toward} and {\it away} from the Sun), the {\it wave energy density} $W_\pm(k, z, t)$ at wavenumber $k$ and position $z$ evolves according to the wave transport equation:

$$
\frac{\partial W_\pm(k, z, t)}{\partial t}
+ \left( V_A \mp U \right) \frac{\partial W_\pm}{\partial z}
= \Gamma_\pm(k) W_\pm(k) - \mathcal{D}_\pm(k) W_\pm(k) - \frac{\partial}{\partial k} \left[ D_{kk}^\pm \frac{\partial W_\pm}{\partial k} \right]
$$
Where: $W_+(k)$ = waves propagating away from the Sun, 
 $W_-(k)$ = waves propagating toward the Sun, 
 $V_A$ = local Alfvén speed, 
 $U$ = solar wind speed (assumed radial), 
 $\Gamma_\pm(k)$ = growth rate from streaming SEPs, 
 $\mathcal{D}_\pm(k)$ = damping rate (e.g., turbulence cascade or cyclotron damping), 
 $D_{kk}^\pm$ = spectral diffusion coefficient for nonlinear cascading (e.g., Kolmogorov-type). 


\subsection{Alfvén Wave Growth/Damping Rate ($\gamma$)}

For self-generated Alfvén waves due to streaming particles:

$$
\gamma(k) = \frac{\pi^2 e^2 v_A}{c B_0^2} \, p v \left[ \frac{\partial f(p, \mu)}{\partial \mu} \right]_{\mu = v_A/v}
$$
Where:
 $\gamma(k)$ = growth (if positive) or damping (if negative) rate for wave number $k$,
 $e$ = elementary charge,
 $v_A$ = Alfvén speed,
 $c$ = speed of light,
 $B_0$ = background magnetic field strength,
 $p$ = particle momentum,
 $v$ = particle speed,
 $f(p, \mu)$ = particle distribution function,
 $\mu$ = pitch-angle cosine.

This equation applies for {\it resonant interactions}, where:

$$
k = \frac{\Omega}{v \mu - v_A}
$$

with $\Omega$ being the particle gyrofrequency.



If you’re referring to {\it Kolmogorov-type turbulence} or cascading effects (sometimes loosely visualized as "frothy" structures in plasma), that's a different discussion involving spectral energy transfer.

If you're asking about the **overall Alfvén wave growth/damping rate** generated by the **entire particle population** (rather than for a single particle resonance), here's how it's typically formulated.



\subsection{Problem in Monte Carlo: Derivative of $f$}

In Monte Carlo models, computing $\partial f / \partial \mu$ is {\it noisy} and expensive due to statistical fluctuations. To avoid this:
 Use {\it streaming-based approximations}:
$$
\gamma(k) \propto \frac{S(k)}{W(k)}
$$



1. Identify resonant particles for each $k$ using:

$$
k = \frac{\Omega}{v \mu - v_A}
$$

2. Compute net streaming $S(k)$ of resonant particles.
3. Estimate wave growth/damping:

$$
\gamma(k) = C \cdot S(k)
$$

where $C$ includes constants like $e^2 v_A / B_0^2$.

4. Update wave energy density**:

$$
\frac{dW(k)}{dt} = 2 \gamma(k) W(k) - \text{damping terms} + \text{cascade}
$$







\subsection{Wave Growth/Damping Rates from SEPs (Modeled via Monte Carlo Particles)}

In Monte Carlo, SEPs are simulated as individual particles with positions, momenta, and pitch angles, so we must **extract macroscopic quantities** (like streaming) from these particles to compute $\Gamma_\pm(k)$:

\textbf{Growth rate (streaming-based approximation):}

$$
\Gamma_\pm(k) = \frac{\pi^2 e^2 v_A}{c B_0^2} \cdot \frac{1}{k} \cdot S_\pm(k)
$$

Where:

* $S_\pm(k)$ = **resonant streaming** of particles with wavenumber $k$, moving in direction $\pm$
* Computed from particle ensemble satisfying the resonance condition:

$$
k_{\text{res}} = \frac{\Omega}{v \mu \mp V_A}
$$

* The streaming $S(k)$ is estimated by summing over particles:

$$
S(k) = \sum_i w_i v_i \mu_i \, \delta\left(k - k_{\text{res},i}\right)
$$

where $w_i$ is the weight of particle $i$

\textbf{ Damping rate:}

Can be assumed constant or modeled as part of a Kolmogorov cascade:

$$
\mathcal{D}_\pm(k) \sim C_d \, k^{\alpha}
$$

Where $\alpha = 1/3$ for Kolmogorov scaling, or empirically determined.



\textbf{ 3. Coupling Wave Transport and Monte Carlo SEPs}


\textbf{ At each time step:}

1. Particles (Monte Carlo):

   * Advance SEPs via guiding center or Parker equation.  
   
   * At each step, compute local **pitch-angle scattering rate** $\nu(k)$ from $W(k)$, e.g.:

     $$
     D_{\mu\mu} \sim \frac{\pi^2 \Omega^2}{B_0^2} \left(1 - \mu^2\right) \frac{W(k)}{k}
     $$
     
   * Sample pitch-angle deflections accordingly.

2. Wave Growth:

   * From the **instantaneous SEP population**, compute streaming $S_\pm(k)$
   * Calculate $\Gamma_\pm(k)$, then update wave energy densities:

     $$
     \frac{dW_\pm}{dt} = \left[ \Gamma_\pm(k) - \mathcal{D}_\pm(k) \right] W_\pm + \text{cascade}
     $$

3. Wave Update:

   Integrate wave transport equation to update $W_\pm(k, z, t)$ spatially and spectrally.

4. Feedback Loop:

   New $W(k)$ updates scattering coefficients $D_{\mu\mu}$ used in the next Monte Carlo particle update.

This coupling ensures **self-consistent evolution** of both the turbulence spectrum and SEP distribution.







Because {\it growth} $\Gamma$ and {\t damping} $\mathcal D$ occur {\it locally in wavenumber space} (each scale $k$ exchanges energy with the particles and with neighbouring scales independently). If you decide to track only a {\it single, total} wave–energy quantity,

$$
\mathcal W_\pm(z,t)\;=\;\int_{k_1}^{k_2} W_\pm(k,z,t)\,dk ,
$$

you must still compute that total’s time-derivative from an integral of the $k$-dependent source terms:

$$
\frac{\partial \mathcal W_\pm}{\partial t}
\;+\;(V_A\!\mp\!U)\,\frac{\partial \mathcal W_\pm}{\partial z}
=\;2\!\int_{k_1}^{k_2}\!\Gamma_\pm(k)\,W_\pm(k)\,dk
\;-\;2\!\int_{k_1}^{k_2}\!\mathcal D_\pm(k)\,W_\pm(k)\,dk 
$$

Inside the integrals the {\it $k$-dependence remains}, because:

1. {\it Resonance is $k$-selective:}
   A particle with speed $v$ and pitch angle $\mu$ resonates only with
   $k_{\text{res}} = \Omega /(v\mu \mp V_A)$.
   The streaming that drives wave growth therefore differs from scale to scale.

2. {\it Non-linear cascade and physical dissipation are scale-selective:}
   Kolmogorov transfer, cyclotron damping, Landau damping, etc., each act most strongly at particular $k$.

3. {\it Closure still requires a spectrum:}
   To evaluate the integrals in (A) you must know (or assume) how the *total* energy $\mathcal W_\pm$ is distributed across $k$.
   A common closure is

   $$
   W_\pm(k) \;=\; \frac{\mathcal W_\pm}{N}\;k^{-5/3},
   \quad
   N=\!\!\int_{k_1}^{k_2}\!k^{-5/3}\,dk ,
   $$

   but the kernels $\Gamma(k)$ and $\mathcal D(k)$ still enter the integrals explicitly.

\subsection{Putting it into a Monte-Carlo coupling loop}

1. {\it Monte-Carlo step} – advance SEPs, tally the resonant streaming $S_\pm(k)$ in logarithmic $k$-bins.

2. {\it Growth rate per bin} 

   $$
   \Gamma_\pm(k)=\frac{\pi^2 e^2 V_A}{cB_0^2}\,\frac{S_\pm(k)}{k}.
   $$

3. {\it If you store the full spectrum} $W_\pm(k)$
   – update it with the $k$-dependent transport equation.
   {\it If you store only $\mathcal W_\pm$}
   – use your assumed spectral shape to evaluate the integrals in (A) and update the single variable $\mathcal W_\pm$.





\subsection{Coupling loop (wave and particle)}
for each global time step $\Delta t$:

     1. advance particles: 
    move particles along field, include focusing \& adiabatic deceleration.
    scatter each particle with local $D_{\mu\mu}$ built from current W(k) :
        \[
D_{\mu\mu}(\mu_i) = \frac{\pi \, \Omega_i^2}{B_0^2} (1 - \mu_i^2) \frac{W(k_{\text{res}})}{k_{\text{res}}}
\]
    update positions, $\mu_i$, energies

     2. accumulate streaming $S_{\pm}(k)$ from updated particles  

     3. compute $\Gamma_{\pm}(k)$ with eq. 

     4. integrate wave eq. (1) in (k,z) mesh for $\Delta t$
         e.g. finite-volume in z, Crank-Nicolson in k

     5. go to next step





\subsection{1.  Discretise wavenumber space}

\begin{tcolorbox}[colframe=black, colback=white]
\textbf{Log-$k$ grid:} \quad \( k_1, k_2, \ldots, k_N \) \\
Typically 30--60 bins per decade. \\
\[
\Delta k_j = k_{j+1} - k_j
\]
\end{tcolorbox}



\subsection{2.  Loop over Monte-Carlo particles and tally resonant streaming}

\begin{tcolorbox}[colframe=black, colback=white, title=Wave Accumulator Initialization and Update]
\textbf{Initialise accumulators for this spatial cell:}
\begin{align*}
S_{+}[1 \ldots N] &= 0 \quad \text{(waves moving anti-sunward)} \\
S_{-}[1 \ldots N] &= 0 \quad \text{(waves moving sunward)}
\end{align*}

\textbf{For each Monte-Carlo particle \( i \):}
\begin{align*}
v_i &= \frac{|\mathbf{p}_i|}{\gamma_i m_i} \\
\mu_i &= \cos(\text{pitch angle in local field frame}) \\
\Omega_i &= \frac{q_i B_0}{\gamma_i m_i} \quad \text{(gyro-frequency)}
\end{align*}

\textbf{Resonant \( k \) for each propagation sense:}
\begin{align*}
k_{\text{res},+} &= \frac{\Omega_i}{v_i \mu_i - V_A} \\
k_{\text{res},-} &= \frac{\Omega_i}{v_i \mu_i + V_A}
\end{align*}

\textbf{Choose the valid one (sign of denominator must match direction):}
\begin{itemize}
  \item If \( k_{\text{res},+} \) lies inside grid:
  \begin{itemize}
    \item Bin \( j = \text{index}(k_{\text{res},+}) \)
    \item \( S_{+}[j] \mathrel{+}= w_i v_i \mu_i \) \quad (where \( w_i \) is particle weight)
  \end{itemize}
  \item If \( k_{\text{res},-} \) lies inside grid:
  \begin{itemize}
    \item Bin \( j = \text{index}(k_{\text{res},-}) \)
    \item \( S_{-}[j] \mathrel{+}= w_i v_i \mu_i \)
  \end{itemize}
\end{itemize}
\end{tcolorbox}

\begin{tcolorbox}[colframe=black, colback=white, title=Units]
\( S \) is the \textbf{number flux}, with units:
\[
\text{m}^{-2} \, \text{s}^{-1}
\]
\end{tcolorbox}



\section*{3. Compute \textbf{growth rates} from streaming (avoids \( \partial f / \partial \mu \))}

For each bin \( j \) (centre \( k_j \)):

\[
\boxed{
\Gamma_{\pm}(k_j) = \frac{\pi^{2} e^{2} V_A}{c B_0^{2}} \, \frac{S_{\pm}(k_j)}{k_j}
}
\tag{1}
\]

\begin{itemize}
    \item \textbf{Positive} streaming in the same direction as the wave \textbf{amplifies} it.
    \item Opposite-sign streaming \textbf{damps} it.
\end{itemize}

If desired, smooth \( S(k) \) across neighbouring bins to suppress Monte-Carlo noise.



\begin{tabular}{@{}lll@{}}
\toprule
\textbf{Damping term} & \textbf{Expression} & \textbf{When to use} \\
\midrule
Kolmogorov cascade & 
\(
\mathcal{D}_{\pm}(k) = C_K^{-1} \epsilon^{1/3} k^{2/3}
\) (with \( C_K \approx 2 \)) &
If you assume an inertial-range cascade with cascade rate \( \epsilon \) (often set to keep background \( \delta B \) level) \\
\addlinespace
Cyclotron / Landau &
\(
\mathcal{D}_{\pm}(k) = \alpha_c k^2
\) (empirical) &
To remove energy at high \( k \) beyond the proton cyclotron scale \\
\bottomrule
\end{tabular}

\vspace{1em}

You can combine them:
\[
\mathcal{D} = \mathcal{D}_{\text{cascade}} + \mathcal{D}_{\text{phys}}.
\]

\subsection{ 5.  Update the wave spectrum}

Using an implicit or Crank-Nicolson step for stability:

$$
\frac{W_\pm^{n+1}-W_\pm^{n}}{\Delta t}
= 2\bigl[\Gamma_\pm-\mathcal{D}_\pm\bigr]\,\frac{W_\pm^{n+1}+W_\pm^{n}}{2}
-\frac{\partial}{\partial k}\!\Bigl(D_{kk}\,\partial_k W_\pm \Bigr)^{n+1/2}.
$$

$D_{kk}$ is the Kolmogorov diffusion coefficient
$D_{kk}=C_K\,\epsilon^{1/3}\,k^{7/3}$.

\subsection{Feed new $W(k)$ back to the particles}

For each particle i at its new pitch angle $\mu_i$:

$$
k_{\text{res}}  = \frac{\Omega_i}{v_i \mu_i \mp V_A},\quad
D_{\mu\mu}(\mu_i) = \frac{\pi\,\Omega_i^{2}}{B_0^{2}}\,(1-\mu_i^{2})\,
\frac{W_\pm(k_{\text{res}})}{k_{\text{res}}}.
$$

Draw a random pitch-angle kick from a Gaussian with variance $2D_{\mu\mu}\Delta t$.

---

\begin{tcolorbox}[colframe=black, colback=white, title=Time-Stepping Loop]
\begin{itemize}
    \item \textbf{while} \( t < t_{\text{end}} \):
    \begin{itemize}
        \item advance particles and accumulate \( S(k) \)
        \item \( \Gamma(k) \leftarrow \) eq.~(1)
        \item \( \mathcal{D}(k) \leftarrow \) chosen law
        \item update \( W(k) \) (implicit solve)
        \item update \( D_{\mu\mu} \) and scatter particles
        \item \( t \leftarrow t + \Delta t \)
    \end{itemize}
\end{itemize}
\end{tcolorbox}



\subsection{ 1.  Fix the spectral shape, keep only one amplitude}

Assume an inertial-range Kolmogorov spectrum between $k_{\min }$ and $k_{\max }$:

$$
\boxed{\,W_\pm(k,z,t)\;=\;A_\pm(z,t)\;k^{-5/3}},\qquad
N = \int_{k_{\min}}^{k_{\max}} k^{-5/3}\,dk
$$

Only the scalar **amplitude** $A_\pm(z,t)$ is evolved; the $k^{-5/3}$ factor is static.

Total wave energy density in the cell:

$$
\mathcal W_\pm(z,t)=A_\pm(z,t)\,N .
$$


\subsection{ 2.  Still tally resonant streaming per $k$-bin}

You must keep the {\it $k$}-resolved streaming** because resonance is scale-selective.

```pseudo

\begin{tcolorbox}[colframe=black, colback=white, title=Streaming Accumulation per Particle]
\begin{itemize}
    \item \textbf{for each particle} \( i \) \textbf{in the cell}:
    \begin{itemize}
        \item \( k_{\text{res},+} = \dfrac{\Omega_i}{v_i \mu_i - V_A} \) \quad \text{(anti-sunward waves)}
        \item \( k_{\text{res},-} = \dfrac{\Omega_i}{v_i \mu_i + V_A} \) \quad \text{(sunward waves)}
        \item \textbf{if} \( k_{\text{res},+} \in [k_{\text{min}}, k_{\text{max}}] \):
        \begin{itemize}
            \item \( j = \text{index}(k_{\text{res},+}) \)
            \item \( S_{+}[j] \mathrel{+}= w_i v_i \mu_i \)
        \end{itemize}
        \item \textbf{if} \( k_{\text{res},-} \in [k_{\text{min}}, k_{\text{max}}] \):
        \begin{itemize}
            \item \( j = \text{index}(k_{\text{res},-}) \)
            \item \( S_{-}[j] \mathrel{+}= w_i v_i \mu_i \)
        \end{itemize}
    \end{itemize}
\end{itemize}
\end{tcolorbox}



\subsection{ 3.  Growth rate per bin (unchanged)}

$$
\Gamma_\pm(k_j)=\frac{\pi^{2}e^{2}V_A}{cB_0^{2}}\,
\frac{S_\pm(k_j)}{k_j}.
$$



\subsection{ 4.  Collapse to one scalar growth term}

The {\it net} growth of the total energy is the spectrum-weighted integral:

$$
\frac{d\mathcal W_\pm}{dt}
=\;2\!\int_{k_{\min}}^{k_{\max}}\!\Gamma_\pm(k)\,W_\pm(k)\,dk
-\;2\!\int_{k_{\min}}^{k_{\max}}\!\mathcal D_\pm(k)\,W_\pm(k)\,dk .
$$

Insert $W_\pm(k)=A_\pm k^{-5/3}$:

$$
\boxed{\;
\frac{dA_\pm}{dt}=2\,A_\pm
\bigl[\langle\Gamma_\pm\rangle - \langle\mathcal D_\pm\rangle\bigr]},
\qquad
\langle X\rangle=
\frac{\displaystyle\int_{k_{\min}}^{k_{\max}}X(k)\,k^{-5/3}\,dk}
     {\displaystyle\int_{k_{\min}}^{k_{\max}}k^{-5/3}\,dk } .
$$

\begin{tcolorbox}[colframe=black, colback=white, title=Weighted Average Growth Rate]
\begin{itemize}
    \item Initialize:
    \[
    \text{num} = 0 \quad \text{(numerator for } \langle \Gamma \rangle \text{)}
    \]
    \[
    \text{den} = 0 \quad \text{(normalisation } N \text{, same each step, pre-compute)}
    \]
    \item \textbf{for each} \( k \)-bin \( j \):
    \begin{itemize}
        \item \( \text{weight} = k_j^{-5/3} \, \Delta k_j \)
        \item \( \text{num} \mathrel{+}= \Gamma_{+}[j] \times \text{weight} \quad \) (or \( \Gamma_{-}[j] \) for other sense)
        \item \( \text{den} \mathrel{+}= \text{weight} \quad \) (den \( = N \))
    \end{itemize}
    \item Compute:
    \[
    \langle \Gamma \rangle = \frac{\text{num}}{\text{den}}
    \]
\end{itemize}
\end{tcolorbox}

Do the same loop with $\mathcal D(k)$ to get $\langle\mathcal D\rangle$.

\subsection{5.  Advance the single amplitude}

\begin{tcolorbox}[colframe=black, colback=white, title=Amplitude Update]
\[
A_{\text{new}} = A_{\text{old}} \times \exp\left\{ 2 \Delta t \left( \langle \Gamma \rangle - \langle \mathcal{D} \rangle \right) \right\}
\]
\textit{(exact for constant rates in \( \Delta t \))}
\end{tcolorbox}

or, second-order Euler:


\begin{tcolorbox}[colframe=black, colback=white, title=Amplitude Update (Linearized)]
\[
A_{\text{new}} = A_{\text{old}} + \Delta t \times 2 A_{\text{old}} \left( \langle \Gamma \rangle - \langle \mathcal{D} \rangle \right)
\]
\end{tcolorbox}



\subsection{6.  Feed back to the particles}

Whenever you need the scattering coefficient for a particle with pitch-angle $\mu$:

$$
k_{\text{res}}=\frac{\Omega}{v\mu\mp V_A},\qquad
D_{\mu\mu}= \frac{\pi \Omega^{2}}{B_0^{2}}\,(1-\mu^{2})\,
\frac{A_\pm\,k_{\text{res}}^{-5/3}}{k_{\text{res}}}.
$$

Thus $D_{\mu\mu}\propto A_\pm$, so every time you update $A_\pm$ you automatically update the scattering strength everywhere in $k$.

\begin{tcolorbox}[colframe=black, colback=white, title=Why \(k\)-Resolved Streaming is Still Needed]
Although only one number \( A_{\pm} \) is stored, the averages \( \langle \Gamma \rangle \) and \( \langle \mathcal{D} \rangle \) in (2) require:
\begin{itemize}
    \item \( \Gamma(k) \) — from streaming in each bin, eq.~(1)
    \item \( \mathcal{D}(k) \) — from your cascade/dissipation law
\end{itemize}

Hence, the Monte-Carlo procedure still loops over \( k \)-bins, but the memory footprint is tiny (just two 1-D arrays for \( S_{\pm} \)), and only one scalar per direction is time-advanced.
\end{tcolorbox}

#### Bottom line

* **Kolmogorov assumption** → fixes the spectral *shape*.
* **Streaming tally per $k$-bin** → captures the correct resonance physics.
* **Equation (2)** → collapses those $k$-dependent rates to a single growth (or damping) rate for the total wave-energy amplitude that your Monte-Carlo code can evolve with one line of algebra.


Below is a compact, self-contained **physics specification** for a Monte-Carlo (MC) module that couples energetic charged particles to a background of field-aligned Alfvén turbulence whose spectrum is fixed to the inertial-range Kolmogorov form
$W(k)=A\,k^{-5/3}$ between $k_{\min }$ and $k_{\max }$.



\subsection{ 1.  Probability that the particle is scattered in $\Delta t$}

For gyro-resonant, small-angle pitch-angle diffusion the 1-D Fokker–Planck coefficient is

$$
D_{\mu\mu}(\mu_i)=\frac{\pi \Omega_i^{2}}{B_0^{2}}\bigl(1-\mu_i^{2}\bigr)
\frac{W\!\bigl(k_{\text{res}}^{\pm}\bigr)}{\,\lvert k_{\text{res}}^{\pm}\rvert\,}.
$$

Because $W(k)=A\,k^{-5/3}$,

$$
D_{\mu\mu}= \frac{\pi \Omega_i^{2}}{B_0^{2}}\bigl(1-\mu_i^{2}\bigr)
A\;|k_{\text{res}}^{\pm}|^{-8/3}.
$$

Pitch-angle diffusion is a homogeneous Wiener process, so the **survival probability** (no scatter in $\Delta t$) is

$$
P_{\rm surv}=e^{-2D_{\mu\mu}\,\Delta t},
\qquad
P_{\rm scatt}=1-P_{\rm surv}\;\simeq\;2D_{\mu\mu}\,\Delta t\;\;(\text{if }2D_{\mu\mu}\Delta t\ll1).
$$

In code: draw a uniform random number $\mathcal U\in[0,1]$; scatter if $\mathcal U<P_{\rm scatt}$.

\subsection{Scattering angle in the frame co-moving with the wave}

Work in the de-Hoffmann–Teller frame of the resonant wave, i.e. a frame translating with $V_A$ **along the field** (sign depends on wave sense).
In this frame the magnetic perturbation is time-stationary and pitch-angle diffusion is isotropic around the local field.

* Draw a Gaussian deviate $\Delta \mu$ with
  $\displaystyle \langle\Delta \mu\rangle=0,\quad
  \langle(\Delta \mu)^{2}\rangle = 2D_{\mu\mu}\,\Delta t.$

* Update the pitch angle: $\mu^{\prime}=\mu_i+\Delta \mu$.
  If $|\mu^{\prime}|>1$ reflect specularly to keep $|\mu|\le1$.

* Transform back to the plasma frame.
  Because $V_A\ll v_i$ for SEPs, the Lorentz transformation changes the magnitude of $\mu$ only by $\mathcal O(V_A/v_i)$; to first order you may keep $\mu^{\prime}$ unchanged.

**Angular deflection**

$$
\Delta\theta \;=\;\arccos μ^{\prime}-\arccos μ_i
\;\approx\;
-\frac{\Delta μ}{\sqrt{1-μ_i^{2}}}.
\tag{5}
$$

(This linearised form is what MC codes normally record for diagnostics.)

\subsection{Wave-energy increment produced by the scattering}

Each scattering event exchanges energy between particle and wave.
In QLT the **energy transferred to waves in bin $k_{\text{res}}$** is the work done by the induced electric field $E_\parallel$ over the interaction:

$$
\delta\!W(k_{\text{res}}) \;=\;
-\,\delta p_{\parallel}\,V_A/V_{\rm ph},
\qquad V_{\rm ph}\simeq V_A ,
$$

so that (sign convention: positive $\delta W$ = wave growth)

$$
\delta W(k_{\text{res}})\;=\;
-\,V_A\,\Delta p_{\parallel}.
$$

For small-angle scattering in the wave frame,

$$
\Delta p_{\parallel}=p_i\,(\mu^{\prime}-\mu_i)=p_i\,\Delta \mu .
$$

**Insert (8) into (7)** and distribute over the $k$-bin volume $\Delta k\Delta z$:

$$
\boxed{\;
\delta W(k_j)
= -\,p_i V_A \,\Delta μ \;
\bigl/(\Delta k\,\Delta z)\;}
$$

Summing (9) over all particles gives the **net spectral energy change**.
Dividing by $W(k_j)\,\Delta t$ recovers the familiar growth/damping rate

$$
\Gamma(k_j)\;=\;\frac{\pi^{2}e^{2}V_A}{cB_0^{2}}\,
\frac{S(k_j)}{k_j},
$$
where $S(k_j)=\sum_i w_i v_i \mu_i$ is the resonant streaming accumulated *before* scattering.
Consistency is automatic: equations (7)–(10) ensure that the *same* $D_{\mu\mu}$ that scatters particles also sets the wave-growth term in the turbulence transport equation.

---


\begin{tcolorbox}[colframe=black, colback=white, title=Summary of the Algorithm]
\textbf{For every particle each step:}
\begin{itemize}
    \item Evaluate \( k_{\text{res}} \) via (1);
    \item Compute \( D_{\mu\mu} \) with (3);
    \item Draw a scatter/no-scatter decision with (4);
    \item If scattered, draw \( \Delta \mu \) (Gaussian) and update \( \mu \);
    \item Add \( \delta W(k_{\text{res}}) \) from (9) to the wave-energy accumulator.
\end{itemize}

\textbf{After all particles are processed for the cell:}
\begin{itemize}
    \item Update the wave amplitude \( A \) (or full \( W(k) \) if you track it) with the accumulated \( \sum \delta W \);
    \item Proceed to the next Monte-Carlo time step.
\end{itemize}


With these three derived elements you have a physically closed Monte-Carlo model that:
\begin{itemize}
    \item reproduces Kolmogorov scattering statistics,
    \item conserves energy between particles and waves event-by-event, and
    \item yields the same macroscopic growth rate (10) as standard quasilinear theory.
\end{itemize}
\end{tcolorbox}

\subsection{Handling out-of-range pitch-angle cosine after a diffusion step}

When you update a particle’s pitch-angle cosine with

$$
\mu'=\mu+\Delta\mu ,\qquad \Delta\mu\sim\mathcal N\bigl(0,2D_{\mu\mu}\,\Delta t\bigr),
$$
the Gaussian kick can push $\mu'$ outside the **physical interval** $[-1,1]$.  In quasilinear theory the correct boundary condition is **zero probability flux** at $\mu=\pm1$;($\partial f/\partial\mu=0$).
Numerically this is enforced with **specular reflection**—exactly the rule used for a 1-D Brownian particle between two perfectly reflecting walls.

\subsubsection{ Reflection algorithm (vectorised-safe)}

\begin{tcolorbox}[colframe=black, colback=white, title=Pitch-Angle Reflection at Boundaries]
\textbf{If} \( \mu' > 1 \):
\begin{itemize}
    \item \( \mu' = 2 - \mu' \) \quad (reflect about \( \mu = +1 \))
    \item \textbf{If} \( \mu' < -1 \): \quad (overshot both walls in an extreme kick)
    \begin{itemize}
        \item \( \mu' = -2 - \mu' \) \quad (second reflection about \( \mu = -1 \))
    \end{itemize}
\end{itemize}
\textbf{Else if} \( \mu' < -1 \):
\begin{itemize}
    \item \( \mu' = -2 - \mu' \) \quad (reflect about \( \mu = -1 \))
    \item \textbf{If} \( \mu' > 1 \):
    \begin{itemize}
        \item \( \mu' = 2 - \mu' \) \quad (second reflection about \( \mu = +1 \))
    \end{itemize}
\end{itemize}
\end{tcolorbox}

\begin{tcolorbox}[colframe=black, colback=white, title=Why Reflection (and not Truncation or Redraw?)]

You may wrap the two reflections in a \textbf{while} \( |\mu'| > 1 \) loop; with realistic \( \Delta t \), two reflections are enough in practice.

The operation preserves the step's magnitude \( |\Delta \mu| \) and satisfies \( \frac{\partial f}{\partial \mu} = 0 \) at the boundaries.

\bigskip

\textbf{Why reflection (and not truncation or re-draw)?}
\begin{itemize}
    \item \textbf{Detailed balance:} Reflection conserves the diffusive probability current; truncation artificially piles up particles at \( \mu = \pm 1 \).
    \item \textbf{Energy conservation:} In the wave frame, scattering is elastic; reflecting the overshoot keeps \( |v| \) unchanged.
    \item \textbf{Computationally cheap:} Just a pair of arithmetic operations; no extra random draw.
\end{itemize}

\bigskip

\textbf{Compact formula:}

You can write the double reflection in one line:
\[
\mu' =
\begin{cases}
2 - \mu', & \text{if } \mu' > 1, \\
-2 - \mu', & \text{if } \mu' < -1.
\end{cases}
\]
Then test once more: if \( |\mu'| > 1 \), repeat (rare).

\bigskip

After this adjustment, you are guaranteed \( -1 \leq \mu' \leq 1 \), and the Monte-Carlo simulation remains faithful to quasilinear diffusion physics.

\end{tcolorbox}

\begin{tcolorbox}[colframe=black, colback=white, title=Self-Contained Monte-Carlo Recipe for Solar-Wind Ions]

The goal is to couple:
\begin{itemize}
    \item a Parker-type particle transport solved with Monte-Carlo pseudo-particles whose distribution is isotropic in the local frame, and
    \item a single, time-dependent wave–energy amplitude \( A_{\pm}(r, t) \) in the fixed spectrum:
\end{itemize}
\[
W_{\pm}(k) = A_{\pm}(r, t) \, k^{-5/3}, \quad k_{\min} \leq k \leq k_{\max}.
\]
No high-speed (\( v \gg V_A \)) approximation is made; all formulae remain valid down to \( v \sim V_A \).

\bigskip

\textbf{1. Gyro-resonant condition (valid for all \( v \))}

For each particle \( i \) with speed \( v_i \) and pitch-angle cosine \( \mu_i \):
\[
k_{\text{res},\pm} = \frac{\Omega_i}{v_i \mu_i \mp V_A}, \quad \Omega_i = \frac{q_i B_0}{m_i \gamma_i}.
\tag{1}
\]
A resonance exists only when \( |v_i \mu_i| > V_A \).

\bigskip

\textbf{2. Pitch–angle diffusion coefficient \( D_{\mu\mu} \)}
\[
D_{\mu\mu}(\mu_i, v_i) = \frac{\pi \Omega_i^2}{B_0^2} (1 - \mu_i^2) A_{\pm} |k_{\text{res},\pm}|^{-8/3}.
\tag{2}
\]

\bigskip

\textbf{3. Scattering probability in a Monte-Carlo time-step \( \Delta t \)}
\[
P_{\text{scatt}} = 1 - \exp\left( -2 D_{\mu\mu} \Delta t \right) \quad \rightarrow \quad 2 D_{\mu\mu} \Delta t \ll 1 \quad \text{(approximation)}.
\tag{3}
\]
Draw a uniform deviate \( U \in [0, 1] \); scatter if \( U < P_{\text{scatt}} \).

\bigskip

\textbf{4. Pitch-angle update with reflections}

If scattered, draw a Gaussian kick:
\[
\Delta \mu \sim \mathcal{N}(0, \, 2 D_{\mu\mu} \Delta t), \quad \mu' = \mu_i + \Delta \mu.
\]
Apply successive specular reflections until \( |\mu'| \leq 1 \):
\[
\mu' =
\begin{cases}
2 - \mu', & \text{if } \mu' > 1, \\
-2 - \mu', & \text{if } \mu' < -1.
\end{cases}
\tag{4}
\]

\bigskip

\end{tcolorbox}
\begin{tcolorbox}[colframe=black, colback=white, title=Self-Contained Monte-Carlo Recipe for Solar-Wind Ions]

\textbf{5. Energy (velocity-magnitude) diffusion — ion heating}

Quasilinear perpendicular velocity diffusion coefficient:
\[
D_{v_{\perp} v_{\perp}}(v_i, \mu_i) = \frac{\pi \Omega_i^2 V_A^2}{B_0^2} A_{\pm} |k_{\text{res},\pm}|^{-8/3}.
\tag{5}
\]
For an isotropic distribution, convert to speed diffusion:
\[
D_{vv} = \frac{1 - \mu_i^2}{2} D_{v_{\perp} v_{\perp}}.
\tag{6}
\]
Update the speed:
\[
v'_i = v_i + \sqrt{2 D_{vv} \Delta t} \, \xi, \quad \xi \sim \mathcal{N}(0, 1).
\tag{7}
\]

\bigskip

\textbf{6. Wave–energy change from each diffusive kick}

Energy conservation gives:
\[
\delta W(k_{\text{res},\pm}) = -\frac{m_i}{V_A} \left( v'_i - v_i + v_i \Delta \mu \right).
\tag{8}
\]
Accumulate \( \sum \delta W \) in the resonant \( k \)-bin.

\bigskip

\textbf{7. Single-amplitude wave update}

The net spectral energy change in the cell:
\[
\Delta W_{\pm} = \sum_{\text{particles}} \delta W, \quad W_{\pm} = A_{\pm} N, \quad N = \int_{k_{\min}}^{k_{\max}} k^{-5/3} \, \mathrm{d}k.
\]
Advance the Kolmogorov amplitude:
\[
A_{\pm}^{\text{new}} = \max \left( 0, \, A_{\pm}^{\text{old}} - \frac{\Delta W_{\pm}}{N} \right).
\tag{9}
\]
(No growth term appears because net streaming is zero for isotropic \( f \).)

\bigskip

\end{tcolorbox}
\begin{tcolorbox}[colframe=black, colback=white, title=Self-Contained Monte-Carlo Recipe for Solar-Wind Ions]

\textbf{8. Algorithm in One Glance}

\begin{itemize}
    \item \textbf{for each time-step} \( \Delta t \):
    \begin{itemize}
        \item \textbf{Loop over particles in cell}:
        \begin{itemize}
            \item \textbf{If} \( |v_i \mu_i| \leq V_A \): continue (non-resonant).
            \item Calculate \( k_{\text{res}} = \Omega_i / (v_i \mu_i \mp V_A) \).
            \item \textbf{Pitch-angle scattering}:
            \[
            D_{\mu\mu} = \text{eq.~(2)}, \quad P_{\text{scatt}} = 1 - \exp(-2 D_{\mu\mu} \Delta t).
            \]
            \item Draw random; if \( \text{random}() < P_{\text{scatt}} \):
            \[
            \Delta \mu \sim \mathcal{N}(0, \sqrt{2 D_{\mu\mu} \Delta t}), \quad \mu_{\text{new}} = \text{reflect}(\mu_i + \Delta \mu).
            \]
            Else:
            \[
            \mu_{\text{new}} = \mu_i, \quad \Delta \mu = 0.
            \]
            \item \textbf{Speed diffusion (ion heating)}:
            \[
            D_{vv} = \frac{1 - \mu_{\text{new}}^2}{2} \left( \frac{\pi \Omega_i^2 V_A^2}{B_0^2} \right) A k_{\text{res}}^{-8/3}.
            \]
            \[
            v_{\text{new}} = v_i + \text{gaussian}(0, \sqrt{2 D_{vv} \Delta t}), \quad \Delta v = v_{\text{new}} - v_i.
            \]
            \item \textbf{Energy exchange with wave}:
            \[
            \delta W = -m_i ( \Delta v + v_i \Delta \mu ) / V_A.
            \]
            Accumulate \( \delta W \) in \( \text{bin}(k_{\text{res}}) \).
            \item \textbf{Store new phase-space coordinates}:
            \[
            v_i, \mu_i = v_{\text{new}}, \mu_{\text{new}}.
            \]
        \end{itemize}
        \item \textbf{Update wave amplitude from all} \( \delta W \):
        \[
        A_{\pm} \leftarrow \max\left( 0, A_{\pm} - \frac{\sum \delta W}{N} \right).
        \]
    \end{itemize}
\end{itemize}

\bigskip

\textbf{Why it Works}
\begin{itemize}
    \item \textbf{No high-speed assumption} — all \( v \) appear explicitly; resonance switches off when \( v \mu \) falls below \( V_A \).
    \item \textbf{Parker-equation compatibility} — isotropy is preserved because pitch-angle kicks are symmetric and large \( D_{\mu\mu} \) quickly erases momentary anisotropy.
    \item \textbf{Self-consistent heating} — Eqs.~(5–9) conserve energy between particles and waves event-by-event and reduce to standard cyclotron damping expressions when averaged over a Maxwellian.
\end{itemize}

\end{tcolorbox}


\section*{Parker-equation–compatible Monte-Carlo Recipe (No \(\mu\))}

Below is a \textbf{Parker-equation–compatible Monte-Carlo recipe} that \textbf{never stores \( \mu \)}.
All pitch-angle information is handled statistically, so the particle ensemble remains \textbf{strictly isotropic} at every step.

\bigskip
\hrule
\bigskip

\section*{0. Set-up and Fixed Kolmogorov Spectrum}

\begin{itemize}
    \item \textbf{Wave spectrum} (one amplitude per propagation sense \( \pm \)):
    \[
    W_\pm(k, r, t) = A_\pm(r, t)\; k^{-5/3}, \qquad k_{\min} \leq k \leq k_{\max}.
    \]
    
    \item Particles are represented by pseudo-particles that carry only \textbf{speed} \( v \) and \textbf{position} \( r \).
    
    \item Their directions are redrawn isotropically whenever scattering occurs.
    
    \item Local background: \( B_0(r), \; V_A(r) \).
\end{itemize}

\bigskip
\hrule
\bigskip

\section*{1. Isotropic Pitch-Angle Diffusion \(\rightarrow\) One \textbf{Scattering Frequency}}

Start from the quasilinear coefficient that \emph{would} hold for a given pitch cosine \( \mu \):
\[
D_{\mu\mu}(\mu, v) = \frac{\pi \Omega^2}{B_0^2} (1 - \mu^2) A_\pm \left| k_{\text{res}}^\pm \right|^{-8/3}, \qquad
k_{\text{res}}^\pm = \frac{\Omega}{v\mu \mp V_A}.
\]

Average over an \textbf{isotropic} distribution \( P(\mu) = \frac{1}{2} \):
\[
\boxed{ \bar{D}_{\mu\mu}(v) = \frac{1}{2} \int_{-1}^{+1} D_{\mu\mu}(\mu, v) \, d\mu. }
\tag{1}
\]

Carry out the integral analytically for \( v > V_A \) (or numerically tabulate once):
\[
\bar{D}_{\mu\mu}(v) = \frac{\pi \Omega^2 A_\pm}{B_0^2} \; \mathcal{I}\left( \frac{v}{V_A} \right),
\]
where
\[
\mathcal{I}(x) = \frac{1}{2} \int_{-1}^{+1} (1 - \mu^2) \left| x\mu \mp 1 \right|^{-8/3} \, d\mu.
\]

\emph{If \( v \leq V_A \), the resonance band does not exist and \( \bar{D}_{\mu\mu} = 0 \).}

\bigskip

\textbf{Scattering frequency}
\[
\boxed{ \nu_{\text{sc}}(v) = 2 \bar{D}_{\mu\mu}(v). }
\tag{2}
\]
\emph{This single number replaces the \( \mu \)-dependent coefficient used in pitch-angle-resolved models.}

\bigskip
\hrule
\bigskip

\section*{2. Monte-Carlo Step – Isotropic Scattering}

For each particle during a time-step \( \Delta t \):
\begin{tcolorbox}[colframe=black, colback=white, title=Isotropic Scattering Step]
\begin{itemize}
    \item \( \nu = \nu_{\text{sc}}(v) \) \hfill \texttt{\# eq.~(2)}
    \item \( P_{\text{sc}} = 1 - \exp(-\nu \, \Delta t) \) \hfill \texttt{\# probability to scatter}
    \item \textbf{if} \texttt{random()} \( < P_{\text{sc}} \): \hfill \texttt{\# scatter event happens}
    \begin{itemize}
        \item \( \hat{v} = \texttt{isotropic\_unit\_vector()} \) \hfill \texttt{\# completely new random direction}
    \end{itemize}
\end{itemize}
\end{tcolorbox}

\emph{No angle bookkeeping is needed; the velocity magnitude stays \( v \) for the moment.}

\bigskip
\hrule
\bigskip

\section*{3. Energy Diffusion (Ion Heating)}

The quasilinear perpendicular-velocity coefficient, averaged over isotropy, becomes
\[
\boxed{ \bar{D}_{vv}(v) = \frac{1}{3} V_A^2 \, \nu_{\text{sc}}(v). }
\tag{3}
\]
\emph{(This follows from \( \langle 1 - \mu^2 \rangle = 2/3 \).)}

\bigskip

\textbf{Speed update}
\[
\boxed{ v' = v + \sqrt{2 \bar{D}_{vv} \, \Delta t} \; \xi, \qquad \xi \sim \mathcal{N}(0, 1). }
\tag{4}
\]
\emph{Set \( v' \geq 0 \); draw a new isotropic direction \textbf{after} the speed change so the distribution remains isotropic.}

\bigskip
\hrule
\bigskip

\section*{4. Wave-Energy Change Associated with One Particle}

The particle’s kinetic-energy increment is
\[
\boxed{ \Delta E_p = \frac{1}{2} m (v'^{2} - v^2). }
\tag{5}
\]

Energy conservation (wave frame speed \( V_A \)) gives the wave change:
\[
\boxed{ \Delta \mathcal{W}_\pm = - \Delta E_p. }
\tag{6}
\]
\emph{Because the ensemble is isotropic there is no net streaming, so waves only \textbf{damp}. Growth stays zero.}

Accumulate \( \sum \Delta \mathcal{W}_\pm \) for the cell during the step.

\bigskip
\hrule
\bigskip

\section*{5. Update the Single Kolmogorov Amplitude}

Total spectral energy in the cell:
\[
\mathcal{W}_\pm = A_\pm N, \qquad N = \int_{k_{\min}}^{k_{\max}} k^{-5/3} \, dk.
\]

After the step:
\[
\boxed{ A_\pm^{\text{new}} = \max\left( 0, \; A_\pm^{\text{old}} - \frac{\sum \Delta \mathcal{W}_\pm}{N} \right). }
\tag{7}
\]

\bigskip
\hrule
\bigskip

\section*{6. Complete Algorithm (Pseudo-code)}

\begin{lstlisting}[mathescape=true]
# pre-tabulate ν_sc(v) and Dvv(v) on a v-grid for speed.

for each global step Δt at radius r:

    E_wave_change_plus  = 0
    E_wave_change_minus = 0          # two propagation senses if both kept

    for each pseudo-particle:
        # (a) SCATTERING DECISION
        ν   = ν_sc(v)
        if random() < 1 - exp(-ν Δt):
            v_dir = isotropic_unit_vector()

        # (b) SPEED (HEATING) UPDATE
        Dvv = (1/3) * V_A^2 * ν
        v_new = v + gaussian(0, sqrt(2 Dvv Δt))
        v_new = max(v_new, 0)

        # (c) ENERGY EXCHANGE
        ΔE = 0.5 * m * (v_new**2 - v**2)
        E_wave_change_plus  -= ΔE     # choose + or – sense; here use “+”
        
        # (d) STORE NEW SPEED
        v = v_new

    # (e) WAVE AMPLITUDE UPDATE
    A_plus  = max( 0,  A_plus  - E_wave_change_plus / N )
    # repeat for A_minus if needed
\end{lstlisting}

\bigskip
\hrule
\bigskip

\section*{Key Points}
\begin{itemize}
    \item \textbf{No \( \mu \) anywhere:} only the \emph{scalar} scattering frequency \( \nu_{\text{sc}}(v) \) and speed-diffusion coefficient \( \bar{D}_{vv}(v) \) are required.
    \item \textbf{Isotropisation:} enforced by drawing a \emph{completely new} random direction at each scatter; between scatters the particle simply convects.
    \item \textbf{Heating of solar-wind ions:} obtained naturally from the speed-diffusion step (4).
    \item \textbf{Wave damping:} the negative of particle heating, ensuring energy conservation.
\end{itemize}



\begin{tcolorbox}[colframe=black, colback=white, title=Step-by-Step Coupling Scheme]

Below is a \textbf{step-by-step coupling scheme} that marries the \textbf{isotropic-Parker Monte-Carlo module} (previous reply) to a \textbf{transport equation for the turbulence–energy density}. The scheme keeps only \textbf{one Kolmogorov amplitude per propagation sense} \( A_\pm(r, t) \) yet still:
\begin{itemize}
    \item advects Alfvén-wave energy with the solar wind,
    \item accounts for nonlinear (Kolmogorov) cascade,
    \item damps/feeds the waves through \textbf{exact energy exchange with the particles}.
\end{itemize}

\section*{1. Wave-Energy Transport Equation (Single Kolmogorov Amplitude)}

Integrate the 1-D spectral wave equation over \( k \) using \( W_\pm(k) = A_\pm k^{-5/3} \). The result for the \emph{total} energy density \( \mathcal{W}_\pm = A_\pm N \) (with \( N = \int_{k_{\min}}^{k_{\max}} k^{-5/3} \, dk \)) is:
\[
\boxed{
\frac{\partial \mathcal{W}_\pm}{\partial t}
+ (V_A \mp U) \frac{\partial \mathcal{W}_\pm}{\partial r}
= - \underbrace{\frac{\mathcal{W}_\pm}{\tau_{\text{cas}}}}_{\text{Kolmogorov cascade}}
- \underbrace{Q_{\text{ion}}}_{\text{particle damping}}
+ S_{\text{inj}}(r, t)
}
\tag{1}
\]
\begin{itemize}
    \item \textbf{Advection speed:} group velocity \( V_A \mp U \).
    \item \textbf{Cascade term:} e.g., \( \tau_{\text{cas}}^{-1} = \frac{C_K}{L_\perp} \sqrt{ \delta B^2 / (4 \pi \rho) } \).
    \item \textbf{Particle term:} 
    \[
    Q_{\text{ion}} = \sum_{\text{all MC particles in cell}} \frac{-\Delta E_p}{\Delta r}
    \]
    with \( \Delta E_p \) from Eq.~(5) in the previous answer.
    \item \textbf{Source term:} \( S_{\text{inj}} \), optional Alfvén-wave source (e.g., at the coronal base).
\end{itemize}
Since \( \mathcal{W}_\pm = A_\pm N \) and \( N \) is constant, you can update either \( \mathcal{W}_\pm \) or \( A_\pm \); below we update \( A_\pm \).

\end{tcolorbox}
\begin{tcolorbox}[colframe=black, colback=white, title=Step-by-Step Coupling Scheme]

\section*{2. Discretisation in a Radial Mesh}

\begin{itemize}
    \item \textbf{Radial grid:} \( r_j, \quad j = 0 \dots J \) (log-spacing is convenient).
    \item \textbf{Time step:} \( \Delta t \) limited by CFL condition:
    \[
    \Delta t < \min_j \left\{ \frac{ \Delta r_j }{ | V_A \mp U | } \right\}.
    \]
\end{itemize}

\textbf{Finite-volume update for \( A_\pm \)}:
\[
A_\pm^{n+1}(j) = A_\pm^{n}(j) - \frac{ \Delta t }{ \Delta r_j } \left[ F_{j+1/2} - F_{j-1/2} \right]
- \frac{ \Delta t }{ N } \left[ \frac{A_\pm N}{\tau_{\text{cas}}} + Q_{\text{ion}} - S_{\text{inj}} \right]_j
\tag{2}
\]
\begin{itemize}
    \item \textbf{Upwind fluxes:}
    \[
    F_{j+1/2} = (V_A \mp U)_{j+1/2} \, A_\pm^{\text{up}}
    \]
    Use van-Leer or piecewise-linear reconstruction for second-order accuracy.
    \item Cascade, particle, and source terms are cell-centred.
\end{itemize}


\end{tcolorbox}

\begin{tcolorbox}[colframe=black, colback=white, title=Step-by-Step Coupling Scheme]

\section*{3. Monte-Carlo \(\leftrightarrow\) Turbulence Coupling at Each Global Step}

\begin{lstlisting}[mathescape=true]
============================================
(1)  Advection / Cascade sub-step for $A_\pm$
============================================
  for each radial cell j:
        # 1a. build upwind fluxes  F
        # 1b. update $A_\pm$ with eq.(2)   (no particle term yet)

============================================
(2)  Particle sub-step in each cell j
============================================
  initialise   Q_ion(j) = 0

  loop over Monte-Carlo particles:
\begin{itemize}
    \item \textbf{2a.} Scattering decision with $\nu_{\text{sc}}(v) $
    \item \textbf{2b.} Speed diffusion $v \rightarrow v'$
    \item \textbf{2c.} Energy change $
    \Delta E_p = \frac{1}{2} m (v'^2 - v^2) $
    
    \item \textbf{2d.} Update particle position by Parker drifts
\end{itemize}
============================================
(3) Particle damping $\leftrightarrow$ wave update
============================================
\text{for each radial cell } j:

$A_\pm(j) \leftarrow A_\pm(j) - \frac{\Delta t}{N} \cdot Q_{\text{ion}}(j) 
\quad \texttt{\# Eq.~(2) term}
$
$
A_\pm(j) = \max(A_\pm(j), 0) 
\quad \texttt{\# maintain positivity}
$
\end{lstlisting}

\emph{Step-ordering:} split the advection/cascade and particle feedback for clarity; operator-splitting error is \( \mathcal{O}(\Delta t) \) and can be reduced with Strang splitting.

\end{tcolorbox}
\begin{tcolorbox}[colframe=black, colback=white, title=Step-by-Step Coupling Scheme]

\section*{4. What the Coupling Achieves}


\noindent
\begin{tabularx}{\textwidth}{@{}lXl@{}}
\toprule
\textbf{Module} & \textbf{Uses} & \textbf{Provides to the other} \\
\midrule
\textbf{Monte-Carlo Parker solver} & local \( A_\pm(r, t) \) to compute \( \nu_{\text{sc}} \), \( D_{vv} \) & energy sink \( Q_{\text{ion}} \) (heating) \\
\textbf{Wave-transport solver} & \( Q_{\text{ion}} \) to damp \( \mathcal{W}_\pm \); cascade law; boundary injection & updated \( A_\pm(r, t + \Delta t) \) \\
\bottomrule
\end{tabularx}


\medskip

\emph{Energy is conserved} cell-by-cell because the exact negative of each particle’s \( \Delta E_p \) is removed from \( A_\pm N \).

\section*{5. Boundary and Initial Conditions}

\begin{itemize}
    \item At \( r = r_0 \) (coronal base) prescribe an \textbf{injected amplitude} \( A_{+,0} \) (outward) and optionally a reflection coefficient to seed \( A_- \).
    \item At the outer boundary \( r = r_{\max} \), use free-outflow for \( A_+ \) and impose a small inflowing \( A_- \) if interstellar turbulence is required.
\end{itemize}

Initial \( A_\pm(r, 0) \) can be set from empirical \( \delta B / B \propto r^{-3/2} \) or any magnetogram-based model.

\section*{6. Practical Diagnostics}

\begin{itemize}
    \item \textbf{Ion heating rate:} \( Q_{\text{ion}}(r) \).
    \item \textbf{Wave energy flux:} \( F_\pm = A_\pm N (V_A \mp U) \).
    \item \textbf{Residual energy ratio:} 
    \[
    \frac{ \delta B^2 }{ B_0^2 } = \frac{ 2 \int W_\pm(k) \, dk }{ B_0^2 } = \frac{ 2 A_\pm N }{ B_0^2 }.
    \]
\end{itemize}

\emph{Consistent trends:} outward \( A_+ \) decays by cascade and ion heating; inward \( A_- \) is generated by reflection or parametric decay—reproducing standard fast-wind observations.

\bigskip
\hrule
\bigskip

\section*{In One Sentence}

\begin{quote}
\textbf{Advance \( A_\pm \) with an upwind-cascade equation, subtract exactly the power \( Q_{\text{ion}} \) that your isotropic Parker Monte-Carlo adds to the particles, and feed the updated \( A_\pm \) back into the next step’s scattering frequency.}
\end{quote}

That closes the loop between turbulence transport and Parker-equation particle evolution \textbf{without ever tracking \( \mu \) explicitly}.

\end{tcolorbox}


\begin{tcolorbox}[colback=white, colframe=black, title={From Parker‐Equation Transport Coefficient $D_{xx}$ to a Monte-Carlo Scattering Probability}, fonttitle=\bfseries]

In an \textbf{isotropic} Parker equation the spatial diffusion coefficient parallel to the mean field is
\begin{equation}
D_{xx} = \frac{1}{3} v \lambda,
\tag{1}
\end{equation}
where
\begin{itemize}
  \item $v$ – particle speed in the plasma frame,
  \item $\lambda$ – \textit{scattering mean free path} (average distance travelled between independent pitch-angle randomisations).
\end{itemize}

\medskip

\textbf{1. Scattering frequency}

Solve Eq.~(1) for $\lambda$:
\[
\lambda = \frac{3 D_{xx}}{v}.
\]

A particle that moves at speed $v$ crosses one mean free path in the \textbf{scattering time}
\begin{equation}
\tau_{\text{sc}} = \frac{\lambda}{v} = \frac{3 D_{xx}}{v^2}.
\tag{2}
\end{equation}

Hence the \textbf{scattering frequency}
\begin{equation}
\nu_{\text{sc}} = \tau_{\text{sc}}^{-1} = \frac{v^2}{3 D_{xx}}.
\tag{3}
\end{equation}

\medskip

\textbf{2. Probability of at least one scattering in a time step $\Delta t$}

Scattering events are Poisson-distributed with rate $\nu_{\text{sc}}$. Therefore
\begin{equation}
\boxed{
P_{\text{scatt}}(\Delta t) = 1 - \exp\!\left[-\nu_{\text{sc}} \Delta t\right] = 1 - \exp\!\left[-\frac{v^2 \Delta t}{3 D_{xx}}\right]}.
\tag{4}
\end{equation}

\begin{itemize}
  \item \textbf{Small-step limit:} If $\Delta t \ll \tau_{\text{sc}}$ (typical Monte-Carlo choice),
  \[
  P_{\text{scatt}} \approx \frac{v^2}{3 D_{xx}} \Delta t.
  \]
  
  \item \textbf{Relativistic extension:} Replace $v$ with $v = \beta c$ and, if desired, scale $D_{xx}$ with the particle’s rigidity $R$; the form of Eq.~(4) is unchanged.
\end{itemize}

\medskip

\textbf{3. Implementation snippet}
\begin{lstlisting}[mathescape=true, basicstyle=\ttfamily\small]
# inputs: $v$, $D_{xx}$, $\Delta t$
$\nu_{\text{sc}} = \dfrac{v \times v}{3.0 \times D_{xx}}$
$P_{\text{sc}} = 1.0 - \exp(-\nu_{\text{sc}} \times \Delta t)$
if random() < $P_{\text{sc}}$:
    # perform isotropic re-orientation (full scattering)
\end{lstlisting}


Choose $\Delta t$ so that $P_{\text{scatt}} \lesssim 0.3$ for numerical stability; larger probabilities can be handled but require multi-scatter logic.

\medskip

\textbf{Result:}

Using only the Parker-equation coefficient $D_{xx}$, the probability that a particle undergoes an Alfvén-wave pitch-angle scattering during the next interval $\Delta t$ is given by Eq.~(4).

\end{tcolorbox}


\begin{tcolorbox}[colback=white, colframe=black, title={Setup}]
\begin{itemize}
\item Two counter-propagating Alfvén-wave populations: 
$W_{+}(k)$ – outward (anti-Sunward) waves, and $W_{-}(k)$ – inward (Sunward) waves.
\item A particle of speed $v$, pitch-angle cosine $\mu$ resonates with a single wavenumber in each population:
\[
k_{\text{res}}^{\,+} = \frac{\Omega}{v\mu - V_A}, \qquad
k_{\text{res}}^{\,-} = \frac{\Omega}{v\mu + V_A}.
\]
\item The \textbf{pitch-angle diffusion coefficient} is the sum of the two contributions:
\[
D_{\mu\mu}(\mu) = D_{\mu\mu}^{\,+}(\mu) + D_{\mu\mu}^{\,-}(\mu),
\]
where (quasi-linear theory)
\[
D_{\mu\mu}^{\,\pm}(\mu) = \frac{\pi \Omega^{2}}{B_0^{2}} (1 - \mu^{2}) 
\frac{W_{\pm}\!\left(k_{\text{res}}^{\pm}\right)}{\left|k_{\text{res}}^{\pm}\right|}.
\]
\end{itemize}

\medskip

\textbf{1. Probability that \emph{a} scattering occurs in the next interval $\Delta t$}

Each wave population supplies an independent Poisson process with rate 
$\nu_{\pm} = 2 D_{\mu\mu}^{\,\pm}(\mu)$. The total rate is
\[
\nu_{\text{tot}} = 2\left(D_{\mu\mu}^{\,+} + D_{\mu\mu}^{\,-}\right) \equiv 2 D_{\mu\mu}.
\]
Hence
\begin{equation}
\boxed{
P_{\text{scatt}}(\Delta t) = 1 - \exp\left[-\nu_{\text{tot}} \Delta t\right] = 1 - \exp\left[-2 D_{\mu\mu} \Delta t\right]
}.
\tag{1}
\end{equation}

Small-step approximation ($2 D_{\mu\mu} \Delta t \ll 1$):
\[
P_{\text{scatt}} \simeq 2 D_{\mu\mu} \Delta t.
\]

\medskip

\end{tcolorbox}
\begin{tcolorbox}[colback=white, colframe=black, title={Setup}]

\textbf{2. Probability that the scattering comes from a \emph{particular} wave band}

Divide each population into narrow $k$-bins (or ``bands'') labelled by index $j$. For band $j$ in the $+$ population let
\[
\nu_{j}^{\,+} = 2 D_{\mu\mu,j}^{\,+} 
= \frac{2\pi \Omega^{2}}{B_0^{2}} (1 - \mu^{2}) \frac{W_{+,j}}{k_{j}},
\]
and analogously $\nu_{j}^{\,-} = 2 D_{\mu\mu,j}^{\,-}$.

\medskip

\textbf{2-a Absolute probability in the next $\Delta t$}

Because the individual Poisson processes are independent,
\begin{equation}
P_{j}^{\pm}(\Delta t) = 1 - \exp\left[-\nu_{j}^{\pm} \Delta t\right]
\approx \nu_{j}^{\pm} \Delta t = 2 D_{\mu\mu,j}^{\pm} \Delta t.
\tag{2}
\end{equation}

\medskip

\textbf{2-b Conditional probability given that a scattering \emph{will} occur}

If you first decide whether any scattering happens (probability $P_{\text{scatt}}$), then choose which band caused it, the branching ratio is
\begin{equation}
\boxed{
\mathcal{P}_{j}^{\pm} = 
\frac{\nu_{j}^{\pm}}{\nu_{\text{tot}}} = 
\frac{D_{\mu\mu,j}^{\pm}}{D_{\mu\mu}^{\,+} + D_{\mu\mu}^{\,-}}
}.
\tag{3}
\end{equation}
so that $\sum_{j} \left( \mathcal{P}_{j}^{+} + \mathcal{P}_{j}^{-} \right) = 1$.

\medskip

\textbf{Monte-Carlo algorithm (per particle per step)}
\begin{lstlisting}[mathescape=true, basicstyle=\ttfamily\small]
$\Delta t$ chosen so that $\nu_{\text{tot}} \times \Delta t \lesssim 0.3$
$\nu_{j}^{+}  = 2 \times D_{\mu\mu,j}^{+}$      # for all k-bins that resonate
$\nu_{j}^{-}  = 2 \times D_{\mu\mu,j}^{-}$
$\nu_{\text{tot}} = \sum \left( \nu_{j}^{+} + \nu_{j}^{-} \right )$

# 1. decide if any scatter
if random() < 1 - exp($-\nu_{\text{tot}} \times \Delta t$):      # Eq. (1)
    # 2. pick band and direction
    r = random() $\times \nu_{\text{tot}}$                      # roulette wheel
    cumulative = 0
    for each j:
        cumulative += $\nu_{j}^{+}$
        if r < cumulative: choose band j, direction +
        break
        cumulative += $\nu_{j}^{-}$
        if r < cumulative: choose band j, direction -
        break
    # 3. draw Gaussian kick $\Delta \mu$ with variance $2 D_{\mu\mu,j}^{\pm} \times \Delta t$
\end{lstlisting}

This procedure exactly realizes the probabilities (1)–(3).

\end{tcolorbox}

\begin{tcolorbox}[colback=white, colframe=black, title={Pitch–Angle Cosine After a Single Monte-Carlo Scattering Step}]

\textbf{1. Frames of reference}
\begin{itemize}
\item \textbf{Plasma frame (lab)} – the Parker equation and your Monte-Carlo particles are expressed here; pitch-angle cosine is
\[
\mu \equiv \cos\theta = \frac{\mathbf{v} \cdot \mathbf{B}_0}{v B_0}.
\]
\item \textbf{Wave frame} – the frame translating along the mean field at the Alfvén speed $V_A$ in the direction of the resonant wave packet ($+$ for anti-Sunward, $-$ for Sunward waves).
In this frame the electric field of the Alfvén wave vanishes and the gyro-resonant diffusion coefficient $D_{\mu\mu}$ is derived.
\end{itemize}

You \emph{apply} the stochastic pitch-angle kick in the \textbf{wave frame}, then convert the new cosine back to the plasma frame.

\medskip

\textbf{2. Diffusive kick in the wave frame}

For a time step $\Delta t$, the quasilinear theory gives
\[
\langle(\Delta \mu_{\text{wf}})^2\rangle = 2 D_{\mu\mu}^{\text{(res)}} \Delta t.
\]
Draw one Gaussian deviate $\xi \sim \mathcal{N}(0,1)$ and set
\begin{equation}
\boxed{
\mu_{\text{wf}}' = \mu_{\text{wf}} + \sqrt{2 D_{\mu\mu}^{\text{(res)}} \Delta t} \, \xi
}
\tag{1}
\end{equation}
where $\mu_{\text{wf}}$ is the particle’s current pitch-angle cosine measured in the wave frame:
\[
\mu_{\text{wf}} = \frac{\mu - \sigma \beta_A}{1 - \sigma \beta_A \mu},
\qquad
\beta_A \equiv \frac{V_A}{c}, \quad
\sigma = \begin{cases}
+1 & (\text{outward wave})\\[3pt]
-1 & (\text{inward wave}).
\end{cases}
\]

\medskip

\textbf{3. Boundary reflection}

If $|\mu_{\text{wf}}'| > 1$ after Eq.~(1), use specular reflection to keep it inside $[-1,1]$:
\[
\mu_{\text{wf}}' \gets
\begin{cases}
2 - \mu_{\text{wf}}', & \mu_{\text{wf}}' > 1, \\[2pt]
-2 - \mu_{\text{wf}}', & \mu_{\text{wf}}' < -1.
\end{cases}
\]
(Iterate once more if an extreme kick overshoots both walls.)

\medskip

\textbf{4. Transform back to the plasma frame}

A Lorentz boost of velocity components parallel to $B_0$ gives
\begin{equation}
\boxed{
\mu' = \frac{\mu_{\text{wf}}' + \sigma \beta_A}{1 + \sigma \beta_A \mu_{\text{wf}}'}
}
\tag{2}
\end{equation}
with the same sign $\sigma$ used in step 2.

\medskip

\textit{Non-relativistic or SEP limit $v \gg V_A$:}
\[
\beta_A \ll 1 \quad \Rightarrow \quad \mu' \simeq \mu_{\text{wf}}'
\]
to better than $V_A / v$.

\medskip

\end{tcolorbox}
\begin{tcolorbox}[colback=white, colframe=black, title={Pitch–Angle Cosine After a Single Monte-Carlo Scattering Step}]

\textbf{5. Summary algorithm (per particle)}
\begin{lstlisting}[mathescape=true, basicstyle=\ttfamily\small]
# inputs: $\mu$ (plasma frame), sign $\sigma$ (wave direction), $D_{\mu\mu}^{\text{(res)}}$, $\Delta t$
$\beta_A$   = $V_A / c$
$\mu_{\text{wf}}$ = ( $\mu - \sigma \beta_A$ ) / ( $1 - \sigma \beta_A \mu$ )

# Gaussian kick in wave frame
$\Delta \mu$ = sqrt( $2 D_{\mu\mu}^{\text{(res)}} \Delta t$ ) * normal_rand()
$\mu_{\text{wf}}^{\text{new}}$ = $\mu_{\text{wf}} + \Delta \mu$
$\mu_{\text{wf}}^{\text{new}}$ = reflect( $\mu_{\text{wf}}^{\text{new}}$ )      # keep in [-1,1]

# transform back to plasma frame
$\mu^{\text{new}}$ = ( $\mu_{\text{wf}}^{\text{new}} + \sigma \beta_A$ ) / ( $1 + \sigma \beta_A \mu_{\text{wf}}^{\text{new}}$ )
\end{lstlisting}


\medskip

\textbf{Result:}

Equation (1) gives the stochastic update of pitch angle \emph{in the wave frame}; Equation (2) returns the new cosine $\mu'$ that your Monte-Carlo solver stores in the plasma (Parker) frame.

\end{tcolorbox}


\begin{tcolorbox}[colback=white, colframe=black, title={Why Does the Frame‐Change Formula Use $V_A/c$ Instead of Just $V_A$?}]

\textbf{1. Two frames and one boost}

\medskip

\begin{tabular}{@{}p{0.35\linewidth} p{0.25\linewidth} p{0.35\linewidth}@{}}
\toprule
\textbf{Frame} & \textbf{Symbol} & \textbf{Velocity of the boost} \\
\midrule
Plasma (lab) frame & unprimed quantities $(\mu, v_\parallel, v)$ & -- \\
Wave frame (co-moving with an Alfvén wave packet) & ``wf'' quantities $(\mu_{\rm wf})$ & $+V_A$ along $\mathbf{B}_0$ for outward waves ($\sigma = +1$); $-V_A$ for inward waves ($\sigma = -1$) \\
\bottomrule
\end{tabular}

\medskip

To move from the plasma frame to the wave frame you perform a \textbf{Lorentz boost} of speed $\sigma V_A$ along the magnetic-field direction.

\medskip

\textbf{2. Lorentz transformation of a velocity component}

\medskip

For any velocity component parallel to the boost, special relativity gives
\[
v_\parallel^{\rm(wf)} = \frac{v_\parallel - \sigma V_A}{1 - \sigma \dfrac{v_\parallel V_A}{c^2}},
\]
where $c$ is the \textbf{speed of light in vacuum}.

\medskip

\textbf{3. Expressing the result as a pitch-angle cosine}

\medskip

Divide by the particle speed $v$ to get pitch-angle cosine:
\[
\mu_{\rm wf}
= \frac{v_\parallel^{\rm(wf)}}{v}
= \frac{\mu - \sigma \beta_A}{1 - \sigma \beta_A \mu}, 
\qquad 
\beta_A \equiv \frac{V_A}{c}.
\]

Hence the appearance of the $\boldsymbol{V_A/c}$ ratio.

\medskip

\textbf{4. Why use the full relativistic formula even for non-relativistic SEPs?}

\begin{itemize}
    \item \textbf{Correct for all energies} – Some Monte-Carlo particles may reach relativistic speeds; the exact boost keeps the algorithm valid without extra branches.
    \item \textbf{Consistency across modules} – Wave growth and damping formulas already rely on relativistic gyro-frequency $\Omega = qB/(\gamma m)$; using the same kinematics avoids subtle bookkeeping errors.
    \item \textbf{Negligible overhead} – Computing with $\beta_A = V_A/c \lesssim 10^{-3}$ adds one multiply and divide per particle — insignificant compared to random-number generation.
\end{itemize}

\medskip

In the common SEP limit where $v \gg V_A$ (e.g., $v \sim 30{,}000\,\text{km/s}$ vs $V_A \sim 50\,\text{km/s}$),
$\beta_A \ll 1$ and the formula reduces to the intuitive Galilean shift:
\[
\mu_{\rm wf} \simeq \mu - \sigma \frac{V_A}{v},
\]
but writing it with $V_A/c$ ensures exactness whenever it matters.

\medskip

\end{tcolorbox}
\begin{tcolorbox}[colback=white, colframe=black, title={Why Does the Frame‐Change Formula Use $V_A/c$ Instead of Just $V_A$?}]

\textbf{5. What is $c$ here?}

\medskip

$c$ is the fundamental constant 
\[
c = 2.9979 \times 10^{8}\;\text{m\,s}^{-1}.
\]
Its role is purely kinematic: it scales the boost velocity $V_A$ into a dimensionless $\beta_A = V_A/c$ so that the relativistic velocity‐addition law can be applied.

\end{tcolorbox}


\begin{tcolorbox}[colback=white, colframe=black, title={Detailed Derivation of the Pitch–Angle Cosine in the \textbf{Wave Frame}}]

\[
\boxed{
\mu_{\text{wf}} = \frac{\mu - \sigma \beta_A}{1 - \sigma \beta_A \mu}
},
\qquad
\beta_A \equiv \frac{V_A}{c}, \quad
\sigma = 
\begin{cases}
+1 & \text{(outward wave)} \\ 
-1 & \text{(inward wave)}
\end{cases}
\]

\medskip

\textbf{Step 0: Set the geometry}

\begin{itemize}
\item Choose the $z$-axis along the mean magnetic field $\vb{B}_0$.
\item \textbf{Plasma (lab) frame:} particle velocity components $\vb{v} = (v_\perp, v_\parallel)$ with $\mu = v_\parallel / v$.
\item \textbf{Wave frame:} moves at speed $\sigma V_A$ along $+z$ (Alfvén speed). Denote wave-frame quantities with subscript ``wf''.
\end{itemize}

\medskip

\textbf{Step 1: Lorentz boost along the field}

For a boost of speed $u = \sigma V_A$ parallel to $z$, the exact special-relativistic velocity–addition law gives
\begin{equation}
v_\parallel^{\text{(wf)}} = \frac{v_\parallel - u}{1 - \dfrac{u v_\parallel}{c^2}}.
\tag{1}
\end{equation}

The perpendicular component transforms as $v_\perp^{\text{(wf)}} = v_\perp / [\gamma_u (1 - uv_\parallel/c^2)]$ with $\gamma_u = (1 - u^2/c^2)^{-1/2}$, but we will not need it explicitly.

\medskip

\textbf{Step 2: Write everything in terms of $\mu$}

Insert $u = \sigma V_A$ and $v_\parallel = \mu v$ into Eq.~(1):
\begin{equation}
v_\parallel^{\text{(wf)}} 
= \frac{\mu v - \sigma V_A}{1 - \sigma \dfrac{V_A}{c^2} \mu v}
= \frac{\mu v - \sigma V_A}{1 - \sigma \beta_A \mu \dfrac{v}{c}}.
\tag{2}
\end{equation}

Define $\beta \equiv v/c$ and $\beta_A \equiv V_A/c$:
\begin{equation}
v_\parallel^{\text{(wf)}}
= \frac{c(\beta \mu - \sigma \beta_A)}{1 - \sigma \beta_A \beta \mu}.
\tag{3}
\end{equation}

\medskip

\textbf{Step 3: Transform total speed $v$}

The total speed in the wave frame is
\begin{equation}
v^{\text{(wf)}} = \frac{v}{\gamma_u (1 - uv_\parallel / c^2)} = \frac{v}{\gamma_A (1 - \sigma \beta_A \beta \mu)},
\quad
\gamma_A = (1 - \beta_A^2)^{-1/2}.
\tag{4}
\end{equation}

\medskip

\end{tcolorbox}
\begin{tcolorbox}[colback=white, colframe=black, title={Detailed Derivation of the Pitch–Angle Cosine in the \textbf{Wave Frame}}]

\textbf{Step 4: Form the pitch–angle cosine in the wave frame}

By definition,
\begin{equation}
\mu_{\text{wf}} = \frac{v_\parallel^{\text{(wf)}}}{v^{\text{(wf)}}}.
\tag{5}
\end{equation}

Insert Eqs.~(3) and (4):
\[
\mu_{\text{wf}} = \frac{c(\beta \mu - \sigma \beta_A)}{1 - \sigma \beta_A \beta \mu}
\div
\frac{v}{\gamma_A (1 - \sigma \beta_A \beta \mu)}
= \frac{\beta \mu - \sigma \beta_A}{\beta}
= \frac{\mu - \sigma \beta_A}{1 - \sigma \beta_A \mu}.
\tag{6}
\]
(*Multiply numerator and denominator by $(1 - \sigma \beta_A \mu)$ and cancel $\beta$.*)

Thus we recover the compact formula:
\[
\boxed{
\mu_{\text{wf}} = \frac{\mu - \sigma \beta_A}{1 - \sigma \beta_A \mu}
},
\qquad
\beta_A = \frac{V_A}{c}.
\]

\medskip

\textbf{Step 5: Non-relativistic limit for SEPs}

For solar-wind Alfvén speeds $V_A \sim 50$--$100\,\text{km\,s}^{-1}$ and SEP speeds $v \gtrsim 10^4\,\text{km\,s}^{-1}$:
\[
\beta_A = \frac{V_A}{c} \ll 1, \qquad
\beta = \frac{v}{c} \lesssim 0.1,
\]
so the denominator differs from unity by at most $\mathcal{O}(\beta_A)$. To first order:
\[
\mu_{\text{wf}} \simeq \mu - \sigma \frac{V_A}{v},
\]
but using the full relativistic form ensures exactness for any particle energy and keeps the code consistent with relativistic gyro-frequency formulas.

\medskip

\textbf{What is $c$ here?}

\[
c = 2.998 \times 10^8\,\text{m\,s}^{-1}
\]
is the speed of light in vacuum. It converts the boost velocity $V_A$ into the dimensionless parameter $\beta_A = V_A/c$ required by Lorentz (not Galilean) velocity addition.

\end{tcolorbox}

\begin{tcolorbox}[colback=white, colframe=black, title={Step-by-Step Derivation of Pitch–Angle Cosine in the Wave Frame}]

\textbf{Goal}
\[
\boxed{
\mu_{\text{wf}} = \frac{\mu - \sigma \beta_A}{1 - \sigma \beta_A \mu}
},
\qquad
\beta_A \equiv \frac{V_A}{c}, \quad
\sigma =
\begin{cases}
+1 & \text{outward (anti-Sunward) wave},\\
-1 & \text{inward (Sunward) wave}.
\end{cases}
\]

\medskip

\textbf{0. Geometry and notation}

\begin{itemize}
\item Field–aligned $z$-axis ($\vb{B}_0 \parallel \vu{z}$).
\item Plasma–frame velocity $\vb{v} = (v_\perp, v_\parallel)$ with pitch-angle cosine $\mu = \dfrac{v_\parallel}{v}$.
\item Wave frame moves at speed $\sigma V_A$ along $+\vu{z}$.
\end{itemize}

\medskip

\textbf{1. Lorentz boost of the parallel component}

For a boost velocity $u = \sigma V_A$ along $z$,
\begin{equation}
v_\parallel^{\text{(wf)}} = 
\frac{v_\parallel - u}{1 - \dfrac{u v_\parallel}{c^2}}.
\tag{1}
\end{equation}

\medskip

\textbf{2. Substitute $u$ and $v_\parallel$}

Let $\beta = v/c$ and $\beta_A = V_A / c$.

Insert $u = \sigma V_A$ and $v_\parallel = \mu v = \mu \beta c$:
\begin{equation}
v_\parallel^{\text{(wf)}}
= \frac{\mu \beta c - \sigma \beta_A c}{1 - \sigma \beta_A \beta \mu}
= c \frac{\beta \mu - \sigma \beta_A}{1 - \sigma \beta_A \beta \mu}.
\tag{2}
\end{equation}

\medskip

\textbf{3. Lorentz boost of the total speed}

The total speed in the wave frame is:
\begin{equation}
v^{\text{(wf)}} =
\frac{v}{\gamma_A (1 - \sigma \beta_A \beta \mu)},
\qquad
\gamma_A = \frac{1}{\sqrt{1 - \beta_A^2}}.
\tag{3}
\end{equation}

\medskip

\textbf{4. Form the cosine in the wave frame}

\[
\mu_{\text{wf}}
= \frac{v_\parallel^{\text{(wf)}}}{v^{\text{(wf)}}}
= \frac{c(\beta \mu - \sigma \beta_A)}{1 - \sigma \beta_A \beta \mu}
\times
\frac{\gamma_A (1 - \sigma \beta_A \beta \mu)}{v}.
\]

Because $v = \beta c$, the $c$ and $(1 - \sigma \beta_A \beta \mu)$ terms cancel:
\begin{equation}
\mu_{\text{wf}}
= \frac{\beta \mu - \sigma \beta_A}{\beta}
= \frac{\mu - \sigma \beta_A}{1 - \sigma \beta_A \mu}.
\tag{4}
\end{equation}

This is exactly the boxed result.

\medskip

\textbf{5. Non-relativistic (SEP) limit}

If $v \gg V_A$, then $\beta_A \ll 1$ and
\[
\mu_{\text{wf}} \simeq \mu - \sigma \frac{V_A}{v},
\]
the familiar Galilean correction, obtained as the small-$\beta_A$ limit of the fully relativistic expression.



\end{tcolorbox}
