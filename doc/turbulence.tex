

\section{Turbulence}


\subsection{Wave Transport Equation (with Kolmogorov Spectrum) for Bidirectional Alfvén Waves}

Assuming Alfvén waves propagate along the magnetic field (both {\it toward} and {\it away} from the Sun), the {\it wave energy density} $W_\pm(k, z, t)$ at wavenumber $k$ and position $z$ evolves according to the wave transport equation:

$$
\frac{\partial W_\pm(k, z, t)}{\partial t}
+ \left( V_A \mp U \right) \frac{\partial W_\pm}{\partial z}
= \Gamma_\pm(k) W_\pm(k) - \mathcal{D}_\pm(k) W_\pm(k) - \frac{\partial}{\partial k} \left[ D_{kk}^\pm \frac{\partial W_\pm}{\partial k} \right]
$$
Where: $W_+(k)$ = waves propagating away from the Sun, 
 $W_-(k)$ = waves propagating toward the Sun, 
 $V_A$ = local Alfvén speed, 
 $U$ = solar wind speed (assumed radial), 
 $\Gamma_\pm(k)$ = growth rate from streaming SEPs, 
 $\mathcal{D}_\pm(k)$ = damping rate (e.g., turbulence cascade or cyclotron damping), 
 $D_{kk}^\pm$ = spectral diffusion coefficient for nonlinear cascading (e.g., Kolmogorov-type). 


\subsection{Alfvén Wave Growth/Damping Rate ($\gamma$)}

For self-generated Alfvén waves due to streaming particles:

$$
\gamma(k) = \frac{\pi^2 e^2 v_A}{c B_0^2} \, p v \left[ \frac{\partial f(p, \mu)}{\partial \mu} \right]_{\mu = v_A/v}
$$
Where:
 $\gamma(k)$ = growth (if positive) or damping (if negative) rate for wave number $k$,
 $e$ = elementary charge,
 $v_A$ = Alfvén speed,
 $c$ = speed of light,
 $B_0$ = background magnetic field strength,
 $p$ = particle momentum,
 $v$ = particle speed,
 $f(p, \mu)$ = particle distribution function,
 $\mu$ = pitch-angle cosine.

This equation applies for {\it resonant interactions}, where:

$$
k = \frac{\Omega}{v \mu - v_A}
$$

with $\Omega$ being the particle gyrofrequency.



If you’re referring to {\it Kolmogorov-type turbulence} or cascading effects (sometimes loosely visualized as "frothy" structures in plasma), that's a different discussion involving spectral energy transfer.

If you're asking about the **overall Alfvén wave growth/damping rate** generated by the **entire particle population** (rather than for a single particle resonance), here's how it's typically formulated.



\subsection{Problem in Monte Carlo: Derivative of $f$}

In Monte Carlo models, computing $\partial f / \partial \mu$ is {\it noisy} and expensive due to statistical fluctuations. To avoid this:
 Use {\it streaming-based approximations}:
$$
\gamma(k) \propto \frac{S(k)}{W(k)}
$$



1. Identify resonant particles for each $k$ using:

$$
k = \frac{\Omega}{v \mu - v_A}
$$

2. Compute net streaming $S(k)$ of resonant particles.
3. Estimate wave growth/damping:

$$
\gamma(k) = C \cdot S(k)
$$

where $C$ includes constants like $e^2 v_A / B_0^2$.

4. Update wave energy density**:

$$
\frac{dW(k)}{dt} = 2 \gamma(k) W(k) - \text{damping terms} + \text{cascade}
$$







\subsection{Wave Growth/Damping Rates from SEPs (Modeled via Monte Carlo Particles)}

In Monte Carlo, SEPs are simulated as individual particles with positions, momenta, and pitch angles, so we must **extract macroscopic quantities** (like streaming) from these particles to compute $\Gamma_\pm(k)$:

\textbf{Growth rate (streaming-based approximation):}

$$
\Gamma_\pm(k) = \frac{\pi^2 e^2 v_A}{c B_0^2} \cdot \frac{1}{k} \cdot S_\pm(k)
$$

Where:

* $S_\pm(k)$ = **resonant streaming** of particles with wavenumber $k$, moving in direction $\pm$
* Computed from particle ensemble satisfying the resonance condition:

$$
k_{\text{res}} = \frac{\Omega}{v \mu \mp V_A}
$$

* The streaming $S(k)$ is estimated by summing over particles:

$$
S(k) = \sum_i w_i v_i \mu_i \, \delta\left(k - k_{\text{res},i}\right)
$$

where $w_i$ is the weight of particle $i$

\textbf{ Damping rate:}

Can be assumed constant or modeled as part of a Kolmogorov cascade:

$$
\mathcal{D}_\pm(k) \sim C_d \, k^{\alpha}
$$

Where $\alpha = 1/3$ for Kolmogorov scaling, or empirically determined.



\textbf{ 3. Coupling Wave Transport and Monte Carlo SEPs}


\textbf{ At each time step:}

1. Particles (Monte Carlo):

   * Advance SEPs via guiding center or Parker equation.  
   
   * At each step, compute local **pitch-angle scattering rate** $\nu(k)$ from $W(k)$, e.g.:

     $$
     D_{\mu\mu} \sim \frac{\pi^2 \Omega^2}{B_0^2} \left(1 - \mu^2\right) \frac{W(k)}{k}
     $$
     
   * Sample pitch-angle deflections accordingly.

2. Wave Growth:

   * From the **instantaneous SEP population**, compute streaming $S_\pm(k)$
   * Calculate $\Gamma_\pm(k)$, then update wave energy densities:

     $$
     \frac{dW_\pm}{dt} = \left[ \Gamma_\pm(k) - \mathcal{D}_\pm(k) \right] W_\pm + \text{cascade}
     $$

3. Wave Update:

   Integrate wave transport equation to update $W_\pm(k, z, t)$ spatially and spectrally.

4. Feedback Loop:

   New $W(k)$ updates scattering coefficients $D_{\mu\mu}$ used in the next Monte Carlo particle update.

This coupling ensures **self-consistent evolution** of both the turbulence spectrum and SEP distribution.







Because {\it growth} $\Gamma$ and {\t damping} $\mathcal D$ occur {\it locally in wavenumber space} (each scale $k$ exchanges energy with the particles and with neighbouring scales independently). If you decide to track only a {\it single, total} wave–energy quantity,

$$
\mathcal W_\pm(z,t)\;=\;\int_{k_1}^{k_2} W_\pm(k,z,t)\,dk ,
$$

you must still compute that total’s time-derivative from an integral of the $k$-dependent source terms:

$$
\frac{\partial \mathcal W_\pm}{\partial t}
\;+\;(V_A\!\mp\!U)\,\frac{\partial \mathcal W_\pm}{\partial z}
=\;2\!\int_{k_1}^{k_2}\!\Gamma_\pm(k)\,W_\pm(k)\,dk
\;-\;2\!\int_{k_1}^{k_2}\!\mathcal D_\pm(k)\,W_\pm(k)\,dk 
$$

Inside the integrals the {\it $k$-dependence remains}, because:

1. {\it Resonance is $k$-selective:}
   A particle with speed $v$ and pitch angle $\mu$ resonates only with
   $k_{\text{res}} = \Omega /(v\mu \mp V_A)$.
   The streaming that drives wave growth therefore differs from scale to scale.

2. {\it Non-linear cascade and physical dissipation are scale-selective:}
   Kolmogorov transfer, cyclotron damping, Landau damping, etc., each act most strongly at particular $k$.

3. {\it Closure still requires a spectrum:}
   To evaluate the integrals in (A) you must know (or assume) how the *total* energy $\mathcal W_\pm$ is distributed across $k$.
   A common closure is

   $$
   W_\pm(k) \;=\; \frac{\mathcal W_\pm}{N}\;k^{-5/3},
   \quad
   N=\!\!\int_{k_1}^{k_2}\!k^{-5/3}\,dk ,
   $$

   but the kernels $\Gamma(k)$ and $\mathcal D(k)$ still enter the integrals explicitly.

\subsection{Putting it into a Monte-Carlo coupling loop}

1. {\it Monte-Carlo step} – advance SEPs, tally the resonant streaming $S_\pm(k)$ in logarithmic $k$-bins.

2. {\it Growth rate per bin} 

   $$
   \Gamma_\pm(k)=\frac{\pi^2 e^2 V_A}{cB_0^2}\,\frac{S_\pm(k)}{k}.
   $$

3. {\it If you store the full spectrum} $W_\pm(k)$
   – update it with the $k$-dependent transport equation.
   {\it If you store only $\mathcal W_\pm$}
   – use your assumed spectral shape to evaluate the integrals in (A) and update the single variable $\mathcal W_\pm$.





\subsection{Coupling loop (wave and particle)}
for each global time step $\Delta t$:

     1. advance particles: 
    move particles along field, include focusing \& adiabatic deceleration.
    scatter each particle with local $D_{\mu\mu}$ built from current W(k) :
        \[
D_{\mu\mu}(\mu_i) = \frac{\pi \, \Omega_i^2}{B_0^2} (1 - \mu_i^2) \frac{W(k_{\text{res}})}{k_{\text{res}}}
\]
    update positions, $\mu_i$, energies

     2. accumulate streaming $S_{\pm}(k)$ from updated particles  

     3. compute $\Gamma_{\pm}(k)$ with eq. 

     4. integrate wave eq. (1) in (k,z) mesh for $\Delta t$
         e.g. finite-volume in z, Crank-Nicolson in k

     5. go to next step





\subsection{1.  Discretise wavenumber space}

\begin{tcolorbox}[colframe=black, colback=white]
\textbf{Log-$k$ grid:} \quad \( k_1, k_2, \ldots, k_N \) \\
Typically 30--60 bins per decade. \\
\[
\Delta k_j = k_{j+1} - k_j
\]
\end{tcolorbox}



\subsection{2.  Loop over Monte-Carlo particles and tally resonant streaming}

\begin{tcolorbox}[colframe=black, colback=white, title=Wave Accumulator Initialization and Update]
\textbf{Initialise accumulators for this spatial cell:}
\begin{align*}
S_{+}[1 \ldots N] &= 0 \quad \text{(waves moving anti-sunward)} \\
S_{-}[1 \ldots N] &= 0 \quad \text{(waves moving sunward)}
\end{align*}

\textbf{For each Monte-Carlo particle \( i \):}
\begin{align*}
v_i &= \frac{|\mathbf{p}_i|}{\gamma_i m_i} \\
\mu_i &= \cos(\text{pitch angle in local field frame}) \\
\Omega_i &= \frac{q_i B_0}{\gamma_i m_i} \quad \text{(gyro-frequency)}
\end{align*}

\textbf{Resonant \( k \) for each propagation sense:}
\begin{align*}
k_{\text{res},+} &= \frac{\Omega_i}{v_i \mu_i - V_A} \\
k_{\text{res},-} &= \frac{\Omega_i}{v_i \mu_i + V_A}
\end{align*}

\textbf{Choose the valid one (sign of denominator must match direction):}
\begin{itemize}
  \item If \( k_{\text{res},+} \) lies inside grid:
  \begin{itemize}
    \item Bin \( j = \text{index}(k_{\text{res},+}) \)
    \item \( S_{+}[j] \mathrel{+}= w_i v_i \mu_i \) \quad (where \( w_i \) is particle weight)
  \end{itemize}
  \item If \( k_{\text{res},-} \) lies inside grid:
  \begin{itemize}
    \item Bin \( j = \text{index}(k_{\text{res},-}) \)
    \item \( S_{-}[j] \mathrel{+}= w_i v_i \mu_i \)
  \end{itemize}
\end{itemize}
\end{tcolorbox}

\begin{tcolorbox}[colframe=black, colback=white, title=Units]
\( S \) is the \textbf{number flux}, with units:
\[
\text{m}^{-2} \, \text{s}^{-1}
\]
\end{tcolorbox}



\section*{3. Compute \textbf{growth rates} from streaming (avoids \( \partial f / \partial \mu \))}

For each bin \( j \) (centre \( k_j \)):

\[
\boxed{
\Gamma_{\pm}(k_j) = \frac{\pi^{2} e^{2} V_A}{c B_0^{2}} \, \frac{S_{\pm}(k_j)}{k_j}
}
\tag{1}
\]

\begin{itemize}
    \item \textbf{Positive} streaming in the same direction as the wave \textbf{amplifies} it.
    \item Opposite-sign streaming \textbf{damps} it.
\end{itemize}

If desired, smooth \( S(k) \) across neighbouring bins to suppress Monte-Carlo noise.



\begin{tabular}{@{}lll@{}}
\toprule
\textbf{Damping term} & \textbf{Expression} & \textbf{When to use} \\
\midrule
Kolmogorov cascade & 
\(
\mathcal{D}_{\pm}(k) = C_K^{-1} \epsilon^{1/3} k^{2/3}
\) (with \( C_K \approx 2 \)) &
If you assume an inertial-range cascade with cascade rate \( \epsilon \) (often set to keep background \( \delta B \) level) \\
\addlinespace
Cyclotron / Landau &
\(
\mathcal{D}_{\pm}(k) = \alpha_c k^2
\) (empirical) &
To remove energy at high \( k \) beyond the proton cyclotron scale \\
\bottomrule
\end{tabular}

\vspace{1em}

You can combine them:
\[
\mathcal{D} = \mathcal{D}_{\text{cascade}} + \mathcal{D}_{\text{phys}}.
\]

\subsection{ 5.  Update the wave spectrum}

Using an implicit or Crank-Nicolson step for stability:

$$
\frac{W_\pm^{n+1}-W_\pm^{n}}{\Delta t}
= 2\bigl[\Gamma_\pm-\mathcal{D}_\pm\bigr]\,\frac{W_\pm^{n+1}+W_\pm^{n}}{2}
-\frac{\partial}{\partial k}\!\Bigl(D_{kk}\,\partial_k W_\pm \Bigr)^{n+1/2}.
$$

$D_{kk}$ is the Kolmogorov diffusion coefficient
$D_{kk}=C_K\,\epsilon^{1/3}\,k^{7/3}$.

\subsection{Feed new $W(k)$ back to the particles}

For each particle i at its new pitch angle $\mu_i$:

$$
k_{\text{res}}  = \frac{\Omega_i}{v_i \mu_i \mp V_A},\quad
D_{\mu\mu}(\mu_i) = \frac{\pi\,\Omega_i^{2}}{B_0^{2}}\,(1-\mu_i^{2})\,
\frac{W_\pm(k_{\text{res}})}{k_{\text{res}}}.
$$

Draw a random pitch-angle kick from a Gaussian with variance $2D_{\mu\mu}\Delta t$.

---

\begin{tcolorbox}[colframe=black, colback=white, title=Time-Stepping Loop]
\begin{itemize}
    \item \textbf{while} \( t < t_{\text{end}} \):
    \begin{itemize}
        \item advance particles and accumulate \( S(k) \)
        \item \( \Gamma(k) \leftarrow \) eq.~(1)
        \item \( \mathcal{D}(k) \leftarrow \) chosen law
        \item update \( W(k) \) (implicit solve)
        \item update \( D_{\mu\mu} \) and scatter particles
        \item \( t \leftarrow t + \Delta t \)
    \end{itemize}
\end{itemize}
\end{tcolorbox}



\subsection{ 1.  Fix the spectral shape, keep only one amplitude}

Assume an inertial-range Kolmogorov spectrum between $k_{\min }$ and $k_{\max }$:

$$
\boxed{\,W_\pm(k,z,t)\;=\;A_\pm(z,t)\;k^{-5/3}},\qquad
N = \int_{k_{\min}}^{k_{\max}} k^{-5/3}\,dk
$$

Only the scalar **amplitude** $A_\pm(z,t)$ is evolved; the $k^{-5/3}$ factor is static.

Total wave energy density in the cell:

$$
\mathcal W_\pm(z,t)=A_\pm(z,t)\,N .
$$


\subsection{ 2.  Still tally resonant streaming per $k$-bin}

You must keep the {\it $k$}-resolved streaming** because resonance is scale-selective.

```pseudo

\begin{tcolorbox}[colframe=black, colback=white, title=Streaming Accumulation per Particle]
\begin{itemize}
    \item \textbf{for each particle} \( i \) \textbf{in the cell}:
    \begin{itemize}
        \item \( k_{\text{res},+} = \dfrac{\Omega_i}{v_i \mu_i - V_A} \) \quad \text{(anti-sunward waves)}
        \item \( k_{\text{res},-} = \dfrac{\Omega_i}{v_i \mu_i + V_A} \) \quad \text{(sunward waves)}
        \item \textbf{if} \( k_{\text{res},+} \in [k_{\text{min}}, k_{\text{max}}] \):
        \begin{itemize}
            \item \( j = \text{index}(k_{\text{res},+}) \)
            \item \( S_{+}[j] \mathrel{+}= w_i v_i \mu_i \)
        \end{itemize}
        \item \textbf{if} \( k_{\text{res},-} \in [k_{\text{min}}, k_{\text{max}}] \):
        \begin{itemize}
            \item \( j = \text{index}(k_{\text{res},-}) \)
            \item \( S_{-}[j] \mathrel{+}= w_i v_i \mu_i \)
        \end{itemize}
    \end{itemize}
\end{itemize}
\end{tcolorbox}



\subsection{ 3.  Growth rate per bin (unchanged)}

$$
\Gamma_\pm(k_j)=\frac{\pi^{2}e^{2}V_A}{cB_0^{2}}\,
\frac{S_\pm(k_j)}{k_j}.
$$



\subsection{ 4.  Collapse to one scalar growth term}

The {\it net} growth of the total energy is the spectrum-weighted integral:

$$
\frac{d\mathcal W_\pm}{dt}
=\;2\!\int_{k_{\min}}^{k_{\max}}\!\Gamma_\pm(k)\,W_\pm(k)\,dk
-\;2\!\int_{k_{\min}}^{k_{\max}}\!\mathcal D_\pm(k)\,W_\pm(k)\,dk .
$$

Insert $W_\pm(k)=A_\pm k^{-5/3}$:

$$
\boxed{\;
\frac{dA_\pm}{dt}=2\,A_\pm
\bigl[\langle\Gamma_\pm\rangle - \langle\mathcal D_\pm\rangle\bigr]},
\qquad
\langle X\rangle=
\frac{\displaystyle\int_{k_{\min}}^{k_{\max}}X(k)\,k^{-5/3}\,dk}
     {\displaystyle\int_{k_{\min}}^{k_{\max}}k^{-5/3}\,dk } .
$$

\begin{tcolorbox}[colframe=black, colback=white, title=Weighted Average Growth Rate]
\begin{itemize}
    \item Initialize:
    \[
    \text{num} = 0 \quad \text{(numerator for } \langle \Gamma \rangle \text{)}
    \]
    \[
    \text{den} = 0 \quad \text{(normalisation } N \text{, same each step, pre-compute)}
    \]
    \item \textbf{for each} \( k \)-bin \( j \):
    \begin{itemize}
        \item \( \text{weight} = k_j^{-5/3} \, \Delta k_j \)
        \item \( \text{num} \mathrel{+}= \Gamma_{+}[j] \times \text{weight} \quad \) (or \( \Gamma_{-}[j] \) for other sense)
        \item \( \text{den} \mathrel{+}= \text{weight} \quad \) (den \( = N \))
    \end{itemize}
    \item Compute:
    \[
    \langle \Gamma \rangle = \frac{\text{num}}{\text{den}}
    \]
\end{itemize}
\end{tcolorbox}

Do the same loop with $\mathcal D(k)$ to get $\langle\mathcal D\rangle$.

\subsection{5.  Advance the single amplitude}

\begin{tcolorbox}[colframe=black, colback=white, title=Amplitude Update]
\[
A_{\text{new}} = A_{\text{old}} \times \exp\left\{ 2 \Delta t \left( \langle \Gamma \rangle - \langle \mathcal{D} \rangle \right) \right\}
\]
\textit{(exact for constant rates in \( \Delta t \))}
\end{tcolorbox}

or, second-order Euler:


\begin{tcolorbox}[colframe=black, colback=white, title=Amplitude Update (Linearized)]
\[
A_{\text{new}} = A_{\text{old}} + \Delta t \times 2 A_{\text{old}} \left( \langle \Gamma \rangle - \langle \mathcal{D} \rangle \right)
\]
\end{tcolorbox}



\subsection{6.  Feed back to the particles}

Whenever you need the scattering coefficient for a particle with pitch-angle $\mu$:

$$
k_{\text{res}}=\frac{\Omega}{v\mu\mp V_A},\qquad
D_{\mu\mu}= \frac{\pi \Omega^{2}}{B_0^{2}}\,(1-\mu^{2})\,
\frac{A_\pm\,k_{\text{res}}^{-5/3}}{k_{\text{res}}}.
$$

Thus $D_{\mu\mu}\propto A_\pm$, so every time you update $A_\pm$ you automatically update the scattering strength everywhere in $k$.

\begin{tcolorbox}[colframe=black, colback=white, title=Why \(k\)-Resolved Streaming is Still Needed]
Although only one number \( A_{\pm} \) is stored, the averages \( \langle \Gamma \rangle \) and \( \langle \mathcal{D} \rangle \) in (2) require:
\begin{itemize}
    \item \( \Gamma(k) \) — from streaming in each bin, eq.~(1)
    \item \( \mathcal{D}(k) \) — from your cascade/dissipation law
\end{itemize}

Hence, the Monte-Carlo procedure still loops over \( k \)-bins, but the memory footprint is tiny (just two 1-D arrays for \( S_{\pm} \)), and only one scalar per direction is time-advanced.
\end{tcolorbox}

#### Bottom line

* **Kolmogorov assumption** → fixes the spectral *shape*.
* **Streaming tally per $k$-bin** → captures the correct resonance physics.
* **Equation (2)** → collapses those $k$-dependent rates to a single growth (or damping) rate for the total wave-energy amplitude that your Monte-Carlo code can evolve with one line of algebra.


Below is a compact, self-contained **physics specification** for a Monte-Carlo (MC) module that couples energetic charged particles to a background of field-aligned Alfvén turbulence whose spectrum is fixed to the inertial-range Kolmogorov form
$W(k)=A\,k^{-5/3}$ between $k_{\min }$ and $k_{\max }$.



\subsection{ 1.  Probability that the particle is scattered in $\Delta t$}

For gyro-resonant, small-angle pitch-angle diffusion the 1-D Fokker–Planck coefficient is

$$
D_{\mu\mu}(\mu_i)=\frac{\pi \Omega_i^{2}}{B_0^{2}}\bigl(1-\mu_i^{2}\bigr)
\frac{W\!\bigl(k_{\text{res}}^{\pm}\bigr)}{\,\lvert k_{\text{res}}^{\pm}\rvert\,}.
$$

Because $W(k)=A\,k^{-5/3}$,

$$
D_{\mu\mu}= \frac{\pi \Omega_i^{2}}{B_0^{2}}\bigl(1-\mu_i^{2}\bigr)
A\;|k_{\text{res}}^{\pm}|^{-8/3}.
$$

Pitch-angle diffusion is a homogeneous Wiener process, so the **survival probability** (no scatter in $\Delta t$) is

$$
P_{\rm surv}=e^{-2D_{\mu\mu}\,\Delta t},
\qquad
P_{\rm scatt}=1-P_{\rm surv}\;\simeq\;2D_{\mu\mu}\,\Delta t\;\;(\text{if }2D_{\mu\mu}\Delta t\ll1).
$$

In code: draw a uniform random number $\mathcal U\in[0,1]$; scatter if $\mathcal U<P_{\rm scatt}$.

\subsection{Scattering angle in the frame co-moving with the wave}

Work in the de-Hoffmann–Teller frame of the resonant wave, i.e. a frame translating with $V_A$ **along the field** (sign depends on wave sense).
In this frame the magnetic perturbation is time-stationary and pitch-angle diffusion is isotropic around the local field.

* Draw a Gaussian deviate $\Delta \mu$ with
  $\displaystyle \langle\Delta \mu\rangle=0,\quad
  \langle(\Delta \mu)^{2}\rangle = 2D_{\mu\mu}\,\Delta t.$

* Update the pitch angle: $\mu^{\prime}=\mu_i+\Delta \mu$.
  If $|\mu^{\prime}|>1$ reflect specularly to keep $|\mu|\le1$.

* Transform back to the plasma frame.
  Because $V_A\ll v_i$ for SEPs, the Lorentz transformation changes the magnitude of $\mu$ only by $\mathcal O(V_A/v_i)$; to first order you may keep $\mu^{\prime}$ unchanged.

**Angular deflection**

$$
\Delta\theta \;=\;\arccos μ^{\prime}-\arccos μ_i
\;\approx\;
-\frac{\Delta μ}{\sqrt{1-μ_i^{2}}}.
\tag{5}
$$

(This linearised form is what MC codes normally record for diagnostics.)

\subsection{Wave-energy increment produced by the scattering}

Each scattering event exchanges energy between particle and wave.
In QLT the **energy transferred to waves in bin $k_{\text{res}}$** is the work done by the induced electric field $E_\parallel$ over the interaction:

$$
\delta\!W(k_{\text{res}}) \;=\;
-\,\delta p_{\parallel}\,V_A/V_{\rm ph},
\qquad V_{\rm ph}\simeq V_A ,
$$

so that (sign convention: positive $\delta W$ = wave growth)

$$
\delta W(k_{\text{res}})\;=\;
-\,V_A\,\Delta p_{\parallel}.
$$

For small-angle scattering in the wave frame,

$$
\Delta p_{\parallel}=p_i\,(\mu^{\prime}-\mu_i)=p_i\,\Delta \mu .
$$

**Insert (8) into (7)** and distribute over the $k$-bin volume $\Delta k\Delta z$:

$$
\boxed{\;
\delta W(k_j)
= -\,p_i V_A \,\Delta μ \;
\bigl/(\Delta k\,\Delta z)\;}
$$

Summing (9) over all particles gives the **net spectral energy change**.
Dividing by $W(k_j)\,\Delta t$ recovers the familiar growth/damping rate

$$
\Gamma(k_j)\;=\;\frac{\pi^{2}e^{2}V_A}{cB_0^{2}}\,
\frac{S(k_j)}{k_j},
$$
where $S(k_j)=\sum_i w_i v_i \mu_i$ is the resonant streaming accumulated *before* scattering.
Consistency is automatic: equations (7)–(10) ensure that the *same* $D_{\mu\mu}$ that scatters particles also sets the wave-growth term in the turbulence transport equation.

---


\begin{tcolorbox}[colframe=black, colback=white, title=Summary of the Algorithm]
\textbf{For every particle each step:}
\begin{itemize}
    \item Evaluate \( k_{\text{res}} \) via (1);
    \item Compute \( D_{\mu\mu} \) with (3);
    \item Draw a scatter/no-scatter decision with (4);
    \item If scattered, draw \( \Delta \mu \) (Gaussian) and update \( \mu \);
    \item Add \( \delta W(k_{\text{res}}) \) from (9) to the wave-energy accumulator.
\end{itemize}

\textbf{After all particles are processed for the cell:}
\begin{itemize}
    \item Update the wave amplitude \( A \) (or full \( W(k) \) if you track it) with the accumulated \( \sum \delta W \);
    \item Proceed to the next Monte-Carlo time step.
\end{itemize}


With these three derived elements you have a physically closed Monte-Carlo model that:
\begin{itemize}
    \item reproduces Kolmogorov scattering statistics,
    \item conserves energy between particles and waves event-by-event, and
    \item yields the same macroscopic growth rate (10) as standard quasilinear theory.
\end{itemize}
\end{tcolorbox}

\subsection{Handling out-of-range pitch-angle cosine after a diffusion step}

When you update a particle’s pitch-angle cosine with

$$
\mu'=\mu+\Delta\mu ,\qquad \Delta\mu\sim\mathcal N\bigl(0,2D_{\mu\mu}\,\Delta t\bigr),
$$
the Gaussian kick can push $\mu'$ outside the **physical interval** $[-1,1]$.  In quasilinear theory the correct boundary condition is **zero probability flux** at $\mu=\pm1$;($\partial f/\partial\mu=0$).
Numerically this is enforced with **specular reflection**—exactly the rule used for a 1-D Brownian particle between two perfectly reflecting walls.

\subsubsection{ Reflection algorithm (vectorised-safe)}

\begin{tcolorbox}[colframe=black, colback=white, title=Pitch-Angle Reflection at Boundaries]
\textbf{If} \( \mu' > 1 \):
\begin{itemize}
    \item \( \mu' = 2 - \mu' \) \quad (reflect about \( \mu = +1 \))
    \item \textbf{If} \( \mu' < -1 \): \quad (overshot both walls in an extreme kick)
    \begin{itemize}
        \item \( \mu' = -2 - \mu' \) \quad (second reflection about \( \mu = -1 \))
    \end{itemize}
\end{itemize}
\textbf{Else if} \( \mu' < -1 \):
\begin{itemize}
    \item \( \mu' = -2 - \mu' \) \quad (reflect about \( \mu = -1 \))
    \item \textbf{If} \( \mu' > 1 \):
    \begin{itemize}
        \item \( \mu' = 2 - \mu' \) \quad (second reflection about \( \mu = +1 \))
    \end{itemize}
\end{itemize}
\end{tcolorbox}

\begin{tcolorbox}[colframe=black, colback=white, title=Why Reflection (and not Truncation or Redraw?)]

You may wrap the two reflections in a \textbf{while} \( |\mu'| > 1 \) loop; with realistic \( \Delta t \), two reflections are enough in practice.

The operation preserves the step's magnitude \( |\Delta \mu| \) and satisfies \( \frac{\partial f}{\partial \mu} = 0 \) at the boundaries.

\bigskip

\textbf{Why reflection (and not truncation or re-draw)?}
\begin{itemize}
    \item \textbf{Detailed balance:} Reflection conserves the diffusive probability current; truncation artificially piles up particles at \( \mu = \pm 1 \).
    \item \textbf{Energy conservation:} In the wave frame, scattering is elastic; reflecting the overshoot keeps \( |v| \) unchanged.
    \item \textbf{Computationally cheap:} Just a pair of arithmetic operations; no extra random draw.
\end{itemize}

\bigskip

\textbf{Compact formula:}

You can write the double reflection in one line:
\[
\mu' =
\begin{cases}
2 - \mu', & \text{if } \mu' > 1, \\
-2 - \mu', & \text{if } \mu' < -1.
\end{cases}
\]
Then test once more: if \( |\mu'| > 1 \), repeat (rare).

\bigskip

After this adjustment, you are guaranteed \( -1 \leq \mu' \leq 1 \), and the Monte-Carlo simulation remains faithful to quasilinear diffusion physics.

\end{tcolorbox}

\begin{tcolorbox}[colframe=black, colback=white, title=Self-Contained Monte-Carlo Recipe for Solar-Wind Ions]

The goal is to couple:
\begin{itemize}
    \item a Parker-type particle transport solved with Monte-Carlo pseudo-particles whose distribution is isotropic in the local frame, and
    \item a single, time-dependent wave–energy amplitude \( A_{\pm}(r, t) \) in the fixed spectrum:
\end{itemize}
\[
W_{\pm}(k) = A_{\pm}(r, t) \, k^{-5/3}, \quad k_{\min} \leq k \leq k_{\max}.
\]
No high-speed (\( v \gg V_A \)) approximation is made; all formulae remain valid down to \( v \sim V_A \).

\bigskip

\textbf{1. Gyro-resonant condition (valid for all \( v \))}

For each particle \( i \) with speed \( v_i \) and pitch-angle cosine \( \mu_i \):
\[
k_{\text{res},\pm} = \frac{\Omega_i}{v_i \mu_i \mp V_A}, \quad \Omega_i = \frac{q_i B_0}{m_i \gamma_i}.
\tag{1}
\]
A resonance exists only when \( |v_i \mu_i| > V_A \).

\bigskip

\textbf{2. Pitch–angle diffusion coefficient \( D_{\mu\mu} \)}
\[
D_{\mu\mu}(\mu_i, v_i) = \frac{\pi \Omega_i^2}{B_0^2} (1 - \mu_i^2) A_{\pm} |k_{\text{res},\pm}|^{-8/3}.
\tag{2}
\]

\bigskip

\textbf{3. Scattering probability in a Monte-Carlo time-step \( \Delta t \)}
\[
P_{\text{scatt}} = 1 - \exp\left( -2 D_{\mu\mu} \Delta t \right) \quad \rightarrow \quad 2 D_{\mu\mu} \Delta t \ll 1 \quad \text{(approximation)}.
\tag{3}
\]
Draw a uniform deviate \( U \in [0, 1] \); scatter if \( U < P_{\text{scatt}} \).

\bigskip

\textbf{4. Pitch-angle update with reflections}

If scattered, draw a Gaussian kick:
\[
\Delta \mu \sim \mathcal{N}(0, \, 2 D_{\mu\mu} \Delta t), \quad \mu' = \mu_i + \Delta \mu.
\]
Apply successive specular reflections until \( |\mu'| \leq 1 \):
\[
\mu' =
\begin{cases}
2 - \mu', & \text{if } \mu' > 1, \\
-2 - \mu', & \text{if } \mu' < -1.
\end{cases}
\tag{4}
\]

\bigskip

\end{tcolorbox}
\begin{tcolorbox}[colframe=black, colback=white, title=Self-Contained Monte-Carlo Recipe for Solar-Wind Ions]

\textbf{5. Energy (velocity-magnitude) diffusion — ion heating}

Quasilinear perpendicular velocity diffusion coefficient:
\[
D_{v_{\perp} v_{\perp}}(v_i, \mu_i) = \frac{\pi \Omega_i^2 V_A^2}{B_0^2} A_{\pm} |k_{\text{res},\pm}|^{-8/3}.
\tag{5}
\]
For an isotropic distribution, convert to speed diffusion:
\[
D_{vv} = \frac{1 - \mu_i^2}{2} D_{v_{\perp} v_{\perp}}.
\tag{6}
\]
Update the speed:
\[
v'_i = v_i + \sqrt{2 D_{vv} \Delta t} \, \xi, \quad \xi \sim \mathcal{N}(0, 1).
\tag{7}
\]

\bigskip

\textbf{6. Wave–energy change from each diffusive kick}

Energy conservation gives:
\[
\delta W(k_{\text{res},\pm}) = -\frac{m_i}{V_A} \left( v'_i - v_i + v_i \Delta \mu \right).
\tag{8}
\]
Accumulate \( \sum \delta W \) in the resonant \( k \)-bin.

\bigskip

\textbf{7. Single-amplitude wave update}

The net spectral energy change in the cell:
\[
\Delta W_{\pm} = \sum_{\text{particles}} \delta W, \quad W_{\pm} = A_{\pm} N, \quad N = \int_{k_{\min}}^{k_{\max}} k^{-5/3} \, \mathrm{d}k.
\]
Advance the Kolmogorov amplitude:
\[
A_{\pm}^{\text{new}} = \max \left( 0, \, A_{\pm}^{\text{old}} - \frac{\Delta W_{\pm}}{N} \right).
\tag{9}
\]
(No growth term appears because net streaming is zero for isotropic \( f \).)

\bigskip

\end{tcolorbox}
\begin{tcolorbox}[colframe=black, colback=white, title=Self-Contained Monte-Carlo Recipe for Solar-Wind Ions]

\textbf{8. Algorithm in One Glance}

\begin{itemize}
    \item \textbf{for each time-step} \( \Delta t \):
    \begin{itemize}
        \item \textbf{Loop over particles in cell}:
        \begin{itemize}
            \item \textbf{If} \( |v_i \mu_i| \leq V_A \): continue (non-resonant).
            \item Calculate \( k_{\text{res}} = \Omega_i / (v_i \mu_i \mp V_A) \).
            \item \textbf{Pitch-angle scattering}:
            \[
            D_{\mu\mu} = \text{eq.~(2)}, \quad P_{\text{scatt}} = 1 - \exp(-2 D_{\mu\mu} \Delta t).
            \]
            \item Draw random; if \( \text{random}() < P_{\text{scatt}} \):
            \[
            \Delta \mu \sim \mathcal{N}(0, \sqrt{2 D_{\mu\mu} \Delta t}), \quad \mu_{\text{new}} = \text{reflect}(\mu_i + \Delta \mu).
            \]
            Else:
            \[
            \mu_{\text{new}} = \mu_i, \quad \Delta \mu = 0.
            \]
            \item \textbf{Speed diffusion (ion heating)}:
            \[
            D_{vv} = \frac{1 - \mu_{\text{new}}^2}{2} \left( \frac{\pi \Omega_i^2 V_A^2}{B_0^2} \right) A k_{\text{res}}^{-8/3}.
            \]
            \[
            v_{\text{new}} = v_i + \text{gaussian}(0, \sqrt{2 D_{vv} \Delta t}), \quad \Delta v = v_{\text{new}} - v_i.
            \]
            \item \textbf{Energy exchange with wave}:
            \[
            \delta W = -m_i ( \Delta v + v_i \Delta \mu ) / V_A.
            \]
            Accumulate \( \delta W \) in \( \text{bin}(k_{\text{res}}) \).
            \item \textbf{Store new phase-space coordinates}:
            \[
            v_i, \mu_i = v_{\text{new}}, \mu_{\text{new}}.
            \]
        \end{itemize}
        \item \textbf{Update wave amplitude from all} \( \delta W \):
        \[
        A_{\pm} \leftarrow \max\left( 0, A_{\pm} - \frac{\sum \delta W}{N} \right).
        \]
    \end{itemize}
\end{itemize}

\bigskip

\textbf{Why it Works}
\begin{itemize}
    \item \textbf{No high-speed assumption} — all \( v \) appear explicitly; resonance switches off when \( v \mu \) falls below \( V_A \).
    \item \textbf{Parker-equation compatibility} — isotropy is preserved because pitch-angle kicks are symmetric and large \( D_{\mu\mu} \) quickly erases momentary anisotropy.
    \item \textbf{Self-consistent heating} — Eqs.~(5–9) conserve energy between particles and waves event-by-event and reduce to standard cyclotron damping expressions when averaged over a Maxwellian.
\end{itemize}

\end{tcolorbox}


\section*{Parker-equation–compatible Monte-Carlo Recipe (No \(\mu\))}

Below is a \textbf{Parker-equation–compatible Monte-Carlo recipe} that \textbf{never stores \( \mu \)}.
All pitch-angle information is handled statistically, so the particle ensemble remains \textbf{strictly isotropic} at every step.

\bigskip
\hrule
\bigskip

\section*{0. Set-up and Fixed Kolmogorov Spectrum}

\begin{itemize}
    \item \textbf{Wave spectrum} (one amplitude per propagation sense \( \pm \)):
    \[
    W_\pm(k, r, t) = A_\pm(r, t)\; k^{-5/3}, \qquad k_{\min} \leq k \leq k_{\max}.
    \]
    
    \item Particles are represented by pseudo-particles that carry only \textbf{speed} \( v \) and \textbf{position} \( r \).
    
    \item Their directions are redrawn isotropically whenever scattering occurs.
    
    \item Local background: \( B_0(r), \; V_A(r) \).
\end{itemize}

\bigskip
\hrule
\bigskip

\section*{1. Isotropic Pitch-Angle Diffusion \(\rightarrow\) One \textbf{Scattering Frequency}}

Start from the quasilinear coefficient that \emph{would} hold for a given pitch cosine \( \mu \):
\[
D_{\mu\mu}(\mu, v) = \frac{\pi \Omega^2}{B_0^2} (1 - \mu^2) A_\pm \left| k_{\text{res}}^\pm \right|^{-8/3}, \qquad
k_{\text{res}}^\pm = \frac{\Omega}{v\mu \mp V_A}.
\]

Average over an \textbf{isotropic} distribution \( P(\mu) = \frac{1}{2} \):
\[
\boxed{ \bar{D}_{\mu\mu}(v) = \frac{1}{2} \int_{-1}^{+1} D_{\mu\mu}(\mu, v) \, d\mu. }
\tag{1}
\]

Carry out the integral analytically for \( v > V_A \) (or numerically tabulate once):
\[
\bar{D}_{\mu\mu}(v) = \frac{\pi \Omega^2 A_\pm}{B_0^2} \; \mathcal{I}\left( \frac{v}{V_A} \right),
\]
where
\[
\mathcal{I}(x) = \frac{1}{2} \int_{-1}^{+1} (1 - \mu^2) \left| x\mu \mp 1 \right|^{-8/3} \, d\mu.
\]

\emph{If \( v \leq V_A \), the resonance band does not exist and \( \bar{D}_{\mu\mu} = 0 \).}

\bigskip

\textbf{Scattering frequency}
\[
\boxed{ \nu_{\text{sc}}(v) = 2 \bar{D}_{\mu\mu}(v). }
\tag{2}
\]
\emph{This single number replaces the \( \mu \)-dependent coefficient used in pitch-angle-resolved models.}

\bigskip
\hrule
\bigskip

\section*{2. Monte-Carlo Step – Isotropic Scattering}

For each particle during a time-step \( \Delta t \):
\begin{tcolorbox}[colframe=black, colback=white, title=Isotropic Scattering Step]
\begin{itemize}
    \item \( \nu = \nu_{\text{sc}}(v) \) \hfill \texttt{\# eq.~(2)}
    \item \( P_{\text{sc}} = 1 - \exp(-\nu \, \Delta t) \) \hfill \texttt{\# probability to scatter}
    \item \textbf{if} \texttt{random()} \( < P_{\text{sc}} \): \hfill \texttt{\# scatter event happens}
    \begin{itemize}
        \item \( \hat{v} = \texttt{isotropic\_unit\_vector()} \) \hfill \texttt{\# completely new random direction}
    \end{itemize}
\end{itemize}
\end{tcolorbox}

\emph{No angle bookkeeping is needed; the velocity magnitude stays \( v \) for the moment.}

\bigskip
\hrule
\bigskip

\section*{3. Energy Diffusion (Ion Heating)}

The quasilinear perpendicular-velocity coefficient, averaged over isotropy, becomes
\[
\boxed{ \bar{D}_{vv}(v) = \frac{1}{3} V_A^2 \, \nu_{\text{sc}}(v). }
\tag{3}
\]
\emph{(This follows from \( \langle 1 - \mu^2 \rangle = 2/3 \).)}

\bigskip

\textbf{Speed update}
\[
\boxed{ v' = v + \sqrt{2 \bar{D}_{vv} \, \Delta t} \; \xi, \qquad \xi \sim \mathcal{N}(0, 1). }
\tag{4}
\]
\emph{Set \( v' \geq 0 \); draw a new isotropic direction \textbf{after} the speed change so the distribution remains isotropic.}

\bigskip
\hrule
\bigskip

\section*{4. Wave-Energy Change Associated with One Particle}

The particle’s kinetic-energy increment is
\[
\boxed{ \Delta E_p = \frac{1}{2} m (v'^{2} - v^2). }
\tag{5}
\]

Energy conservation (wave frame speed \( V_A \)) gives the wave change:
\[
\boxed{ \Delta \mathcal{W}_\pm = - \Delta E_p. }
\tag{6}
\]
\emph{Because the ensemble is isotropic there is no net streaming, so waves only \textbf{damp}. Growth stays zero.}

Accumulate \( \sum \Delta \mathcal{W}_\pm \) for the cell during the step.

\bigskip
\hrule
\bigskip

\section*{5. Update the Single Kolmogorov Amplitude}

Total spectral energy in the cell:
\[
\mathcal{W}_\pm = A_\pm N, \qquad N = \int_{k_{\min}}^{k_{\max}} k^{-5/3} \, dk.
\]

After the step:
\[
\boxed{ A_\pm^{\text{new}} = \max\left( 0, \; A_\pm^{\text{old}} - \frac{\sum \Delta \mathcal{W}_\pm}{N} \right). }
\tag{7}
\]

\bigskip
\hrule
\bigskip

\section*{6. Complete Algorithm (Pseudo-code)}

\begin{lstlisting}[mathescape=true]
# pre-tabulate ν_sc(v) and Dvv(v) on a v-grid for speed.

for each global step Δt at radius r:

    E_wave_change_plus  = 0
    E_wave_change_minus = 0          # two propagation senses if both kept

    for each pseudo-particle:
        # (a) SCATTERING DECISION
        ν   = ν_sc(v)
        if random() < 1 - exp(-ν Δt):
            v_dir = isotropic_unit_vector()

        # (b) SPEED (HEATING) UPDATE
        Dvv = (1/3) * V_A^2 * ν
        v_new = v + gaussian(0, sqrt(2 Dvv Δt))
        v_new = max(v_new, 0)

        # (c) ENERGY EXCHANGE
        ΔE = 0.5 * m * (v_new**2 - v**2)
        E_wave_change_plus  -= ΔE     # choose + or – sense; here use “+”
        
        # (d) STORE NEW SPEED
        v = v_new

    # (e) WAVE AMPLITUDE UPDATE
    A_plus  = max( 0,  A_plus  - E_wave_change_plus / N )
    # repeat for A_minus if needed
\end{lstlisting}

\bigskip
\hrule
\bigskip

\section*{Key Points}
\begin{itemize}
    \item \textbf{No \( \mu \) anywhere:} only the \emph{scalar} scattering frequency \( \nu_{\text{sc}}(v) \) and speed-diffusion coefficient \( \bar{D}_{vv}(v) \) are required.
    \item \textbf{Isotropisation:} enforced by drawing a \emph{completely new} random direction at each scatter; between scatters the particle simply convects.
    \item \textbf{Heating of solar-wind ions:} obtained naturally from the speed-diffusion step (4).
    \item \textbf{Wave damping:} the negative of particle heating, ensuring energy conservation.
\end{itemize}



\begin{tcolorbox}[colframe=black, colback=white, title=Step-by-Step Coupling Scheme]

Below is a \textbf{step-by-step coupling scheme} that marries the \textbf{isotropic-Parker Monte-Carlo module} (previous reply) to a \textbf{transport equation for the turbulence–energy density}. The scheme keeps only \textbf{one Kolmogorov amplitude per propagation sense} \( A_\pm(r, t) \) yet still:
\begin{itemize}
    \item advects Alfvén-wave energy with the solar wind,
    \item accounts for nonlinear (Kolmogorov) cascade,
    \item damps/feeds the waves through \textbf{exact energy exchange with the particles}.
\end{itemize}

\section*{1. Wave-Energy Transport Equation (Single Kolmogorov Amplitude)}

Integrate the 1-D spectral wave equation over \( k \) using \( W_\pm(k) = A_\pm k^{-5/3} \). The result for the \emph{total} energy density \( \mathcal{W}_\pm = A_\pm N \) (with \( N = \int_{k_{\min}}^{k_{\max}} k^{-5/3} \, dk \)) is:
\[
\boxed{
\frac{\partial \mathcal{W}_\pm}{\partial t}
+ (V_A \mp U) \frac{\partial \mathcal{W}_\pm}{\partial r}
= - \underbrace{\frac{\mathcal{W}_\pm}{\tau_{\text{cas}}}}_{\text{Kolmogorov cascade}}
- \underbrace{Q_{\text{ion}}}_{\text{particle damping}}
+ S_{\text{inj}}(r, t)
}
\tag{1}
\]
\begin{itemize}
    \item \textbf{Advection speed:} group velocity \( V_A \mp U \).
    \item \textbf{Cascade term:} e.g., \( \tau_{\text{cas}}^{-1} = \frac{C_K}{L_\perp} \sqrt{ \delta B^2 / (4 \pi \rho) } \).
    \item \textbf{Particle term:} 
    \[
    Q_{\text{ion}} = \sum_{\text{all MC particles in cell}} \frac{-\Delta E_p}{\Delta r}
    \]
    with \( \Delta E_p \) from Eq.~(5) in the previous answer.
    \item \textbf{Source term:} \( S_{\text{inj}} \), optional Alfvén-wave source (e.g., at the coronal base).
\end{itemize}
Since \( \mathcal{W}_\pm = A_\pm N \) and \( N \) is constant, you can update either \( \mathcal{W}_\pm \) or \( A_\pm \); below we update \( A_\pm \).

\end{tcolorbox}
\begin{tcolorbox}[colframe=black, colback=white, title=Step-by-Step Coupling Scheme]

\section*{2. Discretisation in a Radial Mesh}

\begin{itemize}
    \item \textbf{Radial grid:} \( r_j, \quad j = 0 \dots J \) (log-spacing is convenient).
    \item \textbf{Time step:} \( \Delta t \) limited by CFL condition:
    \[
    \Delta t < \min_j \left\{ \frac{ \Delta r_j }{ | V_A \mp U | } \right\}.
    \]
\end{itemize}

\textbf{Finite-volume update for \( A_\pm \)}:
\[
A_\pm^{n+1}(j) = A_\pm^{n}(j) - \frac{ \Delta t }{ \Delta r_j } \left[ F_{j+1/2} - F_{j-1/2} \right]
- \frac{ \Delta t }{ N } \left[ \frac{A_\pm N}{\tau_{\text{cas}}} + Q_{\text{ion}} - S_{\text{inj}} \right]_j
\tag{2}
\]
\begin{itemize}
    \item \textbf{Upwind fluxes:}
    \[
    F_{j+1/2} = (V_A \mp U)_{j+1/2} \, A_\pm^{\text{up}}
    \]
    Use van-Leer or piecewise-linear reconstruction for second-order accuracy.
    \item Cascade, particle, and source terms are cell-centred.
\end{itemize}


\end{tcolorbox}

\begin{tcolorbox}[colframe=black, colback=white, title=Step-by-Step Coupling Scheme]

\section*{3. Monte-Carlo \(\leftrightarrow\) Turbulence Coupling at Each Global Step}

\begin{lstlisting}[mathescape=true]
============================================
(1)  Advection / Cascade sub-step for $A_\pm$
============================================
  for each radial cell j:
        # 1a. build upwind fluxes  F
        # 1b. update $A_\pm$ with eq.(2)   (no particle term yet)

============================================
(2)  Particle sub-step in each cell j
============================================
  initialise   Q_ion(j) = 0

  loop over Monte-Carlo particles:
\begin{itemize}
    \item \textbf{2a.} Scattering decision with $\nu_{\text{sc}}(v) $
    \item \textbf{2b.} Speed diffusion $v \rightarrow v'$
    \item \textbf{2c.} Energy change $
    \Delta E_p = \frac{1}{2} m (v'^2 - v^2) $
    
    \item \textbf{2d.} Update particle position by Parker drifts
\end{itemize}
============================================
(3) Particle damping $\leftrightarrow$ wave update
============================================
\text{for each radial cell } j:

$A_\pm(j) \leftarrow A_\pm(j) - \frac{\Delta t}{N} \cdot Q_{\text{ion}}(j) 
\quad \texttt{\# Eq.~(2) term}
$
$
A_\pm(j) = \max(A_\pm(j), 0) 
\quad \texttt{\# maintain positivity}
$
\end{lstlisting}

\emph{Step-ordering:} split the advection/cascade and particle feedback for clarity; operator-splitting error is \( \mathcal{O}(\Delta t) \) and can be reduced with Strang splitting.

\end{tcolorbox}
\begin{tcolorbox}[colframe=black, colback=white, title=Step-by-Step Coupling Scheme]

\section*{4. What the Coupling Achieves}


\noindent
\begin{tabularx}{\textwidth}{@{}lXl@{}}
\toprule
\textbf{Module} & \textbf{Uses} & \textbf{Provides to the other} \\
\midrule
\textbf{Monte-Carlo Parker solver} & local \( A_\pm(r, t) \) to compute \( \nu_{\text{sc}} \), \( D_{vv} \) & energy sink \( Q_{\text{ion}} \) (heating) \\
\textbf{Wave-transport solver} & \( Q_{\text{ion}} \) to damp \( \mathcal{W}_\pm \); cascade law; boundary injection & updated \( A_\pm(r, t + \Delta t) \) \\
\bottomrule
\end{tabularx}


\medskip

\emph{Energy is conserved} cell-by-cell because the exact negative of each particle’s \( \Delta E_p \) is removed from \( A_\pm N \).

\section*{5. Boundary and Initial Conditions}

\begin{itemize}
    \item At \( r = r_0 \) (coronal base) prescribe an \textbf{injected amplitude} \( A_{+,0} \) (outward) and optionally a reflection coefficient to seed \( A_- \).
    \item At the outer boundary \( r = r_{\max} \), use free-outflow for \( A_+ \) and impose a small inflowing \( A_- \) if interstellar turbulence is required.
\end{itemize}

Initial \( A_\pm(r, 0) \) can be set from empirical \( \delta B / B \propto r^{-3/2} \) or any magnetogram-based model.

\section*{6. Practical Diagnostics}

\begin{itemize}
    \item \textbf{Ion heating rate:} \( Q_{\text{ion}}(r) \).
    \item \textbf{Wave energy flux:} \( F_\pm = A_\pm N (V_A \mp U) \).
    \item \textbf{Residual energy ratio:} 
    \[
    \frac{ \delta B^2 }{ B_0^2 } = \frac{ 2 \int W_\pm(k) \, dk }{ B_0^2 } = \frac{ 2 A_\pm N }{ B_0^2 }.
    \]
\end{itemize}

\emph{Consistent trends:} outward \( A_+ \) decays by cascade and ion heating; inward \( A_- \) is generated by reflection or parametric decay—reproducing standard fast-wind observations.

\bigskip
\hrule
\bigskip

\section*{In One Sentence}

\begin{quote}
\textbf{Advance \( A_\pm \) with an upwind-cascade equation, subtract exactly the power \( Q_{\text{ion}} \) that your isotropic Parker Monte-Carlo adds to the particles, and feed the updated \( A_\pm \) back into the next step’s scattering frequency.}
\end{quote}

That closes the loop between turbulence transport and Parker-equation particle evolution \textbf{without ever tracking \( \mu \) explicitly}.

\end{tcolorbox}


With this formulation you can couple a Parker (isotropic) transport Monte-Carlo to a Kolmogorov Alfvénic turbulence model \textbf{without ever introducing \( \mu \) explicitly}, yet still retain the essential physics of pitch-angle scattering and ion heating by resonant waves.





With this formulation you can couple a Parker (isotropic) transport Monte-Carlo to a Kolmogorov Alfvénic turbulence model **without ever introducing $\mu$ explicitly**, yet still retain the essential physics of pitch-angle scattering and ion heating by resonant waves.


Below is a **step-by-step coupling scheme** that marries the **isotropic-Parker Monte-Carlo module** (previous reply) to a **transport equation for the turbulence–energy density**.  The scheme keeps only **one Kolmogorov amplitude per propagation sense** $A_\pm(r,t)$ yet still:

* advects Alfvén-wave energy with the solar wind,
* accounts for nonlinear (Kolmogorov) cascade,
* damps/feeds the waves through **exact energy exchange with the particles**.

---

## 1 Wave-energy transport equation (single Kolmogorov amplitude)

Integrate the 1-D spectral wave equation over $k$ using
$W_\pm(k)=A_\pm k^{-5/3}$.  The result for the *total* energy density $\mathcal W_\pm=A_\pm N$ (with $N=\!\int_{k_{\min}}^{k_{\max}}k^{-5/3}dk$) is

$$
\boxed{
\frac{\partial\mathcal W_\pm}{\partial t}
+\bigl(V_A\!\mp\!U\bigr)\frac{\partial\mathcal W_\pm}{\partial r}
= -\,\underbrace{\frac{\mathcal W_\pm}{\tau_{\rm cas}}}_{\text{Kolmogorov cascade}}
\;-\;\underbrace{Q_{\rm ion}}_{\text{particle damping}}
\;+\;S_{\rm inj}(r,t)} \tag{1}
$$

* **Advection speed:** group velocity $V_A\mp U$.
* **Cascade term:** choose e.g.
  $\displaystyle \tau_{\rm cas}^{-1}=\frac{C_K}{L_\perp}\sqrt{\frac{\delta B^2}{4\pi\rho}}\!)
  or adopt a constant fraction of the nonlinear time; remove the energy from \(\mathcal W_\pm$ and deposit it at $k\!>\!k_{\max}$ (not followed).
* **Particle term:**
  $Q_{\rm ion}=\sum_{\text{all MC particles in cell}}\!\!(-\Delta E_p)/\Delta r$
  with $\Delta E_p$ from Eq.(5) in the previous answer.
* **$S_{\rm inj}$:** optional Alfvén-wave source (e.g. at the coronal base).

Since $\mathcal W_\pm=A_\pm N$ and $N$ is a constant, you can update **either** $\mathcal W_\pm$ **or** $A_\pm$; below we keep $A_\pm$.

---

## 2 Discretisation in a radial mesh

*Radial grid:* $r_j,\;j=0\dots J$ (log-spacing is convenient).
*Time step:* $\Delta t$ limited by CFL:
$\Delta t < \min_j\{\Delta r_j / |V_A\!\mp\!U|\}$.

### Finite-volume update for $A_\pm$

$$
A_\pm^{\,n+1}(j)=A_\pm^{\,n}(j)
-\frac{\Delta t}{\Delta r_j} \Bigl[F_{j+1/2}-F_{j-1/2}\Bigr]
-\frac{\Delta t}{N}\Bigl[\frac{A_\pm N}{\tau_{\rm cas}}+Q_{\rm ion}-S_{\rm inj}\Bigr]_j
\tag{2}
$$

* **Upwind fluxes**
  $F_{j+1/2}=(V_A\!\mp\!U)_{j+1/2}\,A_\pm^{\text{up}}$.
  Use van-Leer or piecewise-linear reconstruction for second-order accuracy.

* **Cascade / particle / source** terms are cell-centred.

---

## 3 Monte-Carlo ↔ turbulence coupling at each global step

```pseudo
============================================
(1)  Advection / Cascade sub-step for A±
============================================
  for each radial cell j:
        # 1a. build upwind fluxes  F
        # 1b. update A_± with eq.(2)   (no particle term yet)

============================================
(2)  Particle sub-step in each cell j
============================================
  initialise   Q_ion(j) = 0

  loop over Monte-Carlo particles:
         # 2a. scattering decision with overall ν_sc(v)   (equ. (2) previous reply)
         # 2b. speed diffusion  v  →  v'                 (equ. (4))
         # 2c. energy change    ΔE_p = ½m(v'^2 - v^2)
         # 2d. update particle position by Parker drifts
         Q_ion(home_cell)  -= ΔE_p / Δt      # power density

============================================
(3)  Particle damping ↔ wave update
============================================
  for each radial cell j:
        A_±(j)  ←  A_±(j)  -  (Δt / N) · Q_ion(j)        # Eq.(2) term

        A_±(j)  =  max(A_±(j), 0)    # maintain positivity
```

*Step-ordering:* split the advection/cascade and particle feedback for clarity; operator-splitting error is $\mathcal O(\Delta t)$ and can be reduced with Strang splitting.

---

## 4 What the coupling achieves

| Module                        | Uses                                                                    | Provides to the other               |
| ----------------------------- | ----------------------------------------------------------------------- | ----------------------------------- |
| **Monte-Carlo Parker solver** | local $A_\pm(r,t)$ to compute $ν_{\rm sc}$ and $D_{vv}$                 | energy sink $Q_{\rm ion}$ (heating) |
| **Wave-transport solver**     | $Q_{\rm ion}$ to damp $\mathcal W_\pm$; cascade law; boundary injection | updated $A_\pm(r,t+\Delta t)$       |

*Energy is conserved* cell-by-cell because the exact negative of each particle’s $\Delta E_p$ is removed from $A_\pm N$.

---

## 5 Boundary and initial conditions

* At $r=r_0$ (coronal base) prescribe an **injected amplitude** $A_{+,0}$ (outward) and optionally a reflection coefficient to seed $A_-$.
* At the outer boundary $r=r_{\max}$ use free-outflow for $A_+$ and impose a small inflowing $A_-$ if interstellar turbulence is required.

Initial $A_\pm(r,0)$ can be set from empirical $\delta B/B\propto r^{-3/2}$ or any magnetogram-based model.

---

## 6 Practical diagnostics

* **Ion heating rate:** $Q_{\rm ion}(r)$.
* **Wave energy flux:** $F_\pm=A_\pm N (V_A\!\mp\!U)$.
* **Residual energy ratio:** $\delta B^2/B_0^2 = 2\! \int W_\pm(k)\,dk /B_0^2 = 2A_\pm N/B_0^2$.

Consistent trends—outward $A_+$ decays by cascade and ion heating; inward $A_-$ is generated by reflection or parametric decay—reproduce standard fast-wind observations.

---

### In one sentence

> **Advance $A_\pm$ with an upwind-cascade equation, subtract exactly the power $Q_{\rm ion}$ that your isotropic Parker Monte-Carlo adds to the particles, and feed the updated $A_\pm$ back into the next step’s scattering frequency.**

That closes the loop between turbulence transport and Parker-equation particle evolution without ever tracking $\mu$ explicitly.


**Turbulence cascade** is the chain‐reaction process by which energy (or some other conserved quantity such as cross-helicity or magnetic flux) that is injected into a fluid or plasma at **large spatial scales** is passed down, step-by-step, to **progressively smaller scales** until it reaches scales where microscopic dissipation processes can convert it irreversibly into heat.

---

### 1.  The big picture

| Stage                           | Characteristic scale                       | What happens                                                                                                                                                                                                                                                                                    |
| ------------------------------- | ------------------------------------------ | ----------------------------------------------------------------------------------------------------------------------------------------------------------------------------------------------------------------------------------------------------------------------------------------------- |
| **Injection / driving**         | Largest eddies or wave packets, size $L$   | Energy is supplied by stirring, shear, shocks, reconnection, etc.                                                                                                                                                                                                                               |
| **Inertial (Kolmogorov) range** | $L \gg \ell \gg \ell_d$                    | Non-linear interactions redistribute the energy from one scale to the next without significant loss: the **cascade**. A classical result is Kolmogorov’s $E(k)\propto k^{-5/3}$ spectrum in hydrodynamics (and often observed in MHD/plasma turbulence too). ([Wikipedia][1], [Altair Help][2]) |
| **Dissipation range**           | Smallest eddies or kinetic scales $\ell_d$ | Viscosity (fluids) or kinetic effects such as Landau/cyclotron damping (plasmas) convert the cascading energy into particle heat. ([Frontiers][3])                                                                                                                                              |

*In plasmas* (e.g., the solar wind) the same picture holds, but the “eddies” can be **Alfvénic wave packets** and the final dissipation often occurs at the ion or electron gyroradius scales instead of the viscous scale of a neutral fluid. ([Frontiers][3])

---

### 2.  How the cascade works mathematically

In Fourier (wavenumber) space the energy density per unit $k$ is $E(k)$.  Non-linear terms in the Navier–Stokes or MHD equations couple neighbouring modes, producing a conservative flux $\varepsilon$ that is **constant with $k$** throughout the inertial range:

$$
\frac{\partial E(k,t)}{\partial t}
   +\frac{\partial}{\partial k}\Bigl[\varepsilon(k)\Bigr]=0,
\qquad
\varepsilon(k)\simeq\text{const}= \varepsilon.
$$

Setting $\varepsilon$ constant leads directly to Kolmogorov’s famous scaling $E(k)\propto \varepsilon^{2/3}k^{-5/3}$ for an isotropic, steady cascade. ([Cushman][4])

In numerical models this conservative transfer is often represented by a **spectral diffusion term**

$$
-\frac{\partial}{\partial k}\!\left[D_{kk}(k)\,\frac{\partial E}{\partial k}\right],
\quad\text{with }D_{kk}\propto k^{7/3},
$$

which mimics the net effect of many local, non-linear triad interactions.

---

### 3.  Forward vs. inverse cascades

* **Forward (direct) cascade** – energy moves from **large to small scales**; this is the default in 3-D hydrodynamics and most space-plasma turbulence.
* **Inverse cascade** – certain invariants (e.g., 2-D kinetic energy, magnetic helicity) can flow **from small to large scales**, building system-sized structures; but even then, a simultaneous forward cascade of another quantity (enstrophy, cross-helicity) often co-exists. ([Wikipedia][1])

---

### 4.  Why the cascade matters in heliophysics

* **Heating:** The eventually dissipated cascade power explains why the solar wind remains hotter than adiabatic expansion alone predicts. ([Frontiers][3])
* **Particle transport:** The scale-dependent wave energy $W(k)$ that the cascade maintains sets the scattering rate of energetic particles (SEPs, cosmic rays) through gyro-resonance.
* **Chronometer of energy flow:** Measuring $E(k)$ and its evolution gives a direct diagnostic of how much mechanical or magnetic energy is being converted into heat per unit time in astrophysical plasmas.

---

### 5.  Connection to your Monte-Carlo + Parker model

In the coupled algorithm we discussed earlier:

* The **cascade term** $-\mathcal W_\pm/\tau_{\text{cas}}$ in the turbulence-energy transport equation is the large-scale representation of the forward cascade: it drains energy from the Kolmogorov range at a rate $\varepsilon = \mathcal W_\pm/\tau_{\text{cas}}$ and ultimately hands it to dissipation (or to smaller-scale waves not modelled explicitly).
* You do **not** need to resolve every intermediate scale in detail; prescribing a cascade time $\tau_{\text{cas}} \sim (L_\perp / \delta v)$ or using the $D_{kk}$-diffusion form reproduces the net flux that sustains the $k^{-5/3}$ shape while letting you track only the total amplitude $A_\pm(r,t)$.

---

**In one sentence:**
*A turbulence cascade is the scale-by-scale hand-off of energy (or any conserved invariant) from the big structures that are stirred by drivers all the way down to the tiny structures where microscopic physics can finally convert that energy into heat.*

[1]: https://en.wikipedia.org/wiki/Energy_cascade?utm_source=chatgpt.com "Energy cascade - Wikipedia"
[2]: https://help.altair.com/hwcfdsolvers/acusolve/topics/acusolve/training_manual/turb_scales_energy_cascasde_r.htm?utm_source=chatgpt.com "Turbulence Scales and Energy Cascade"
[3]: https://www.frontiersin.org/journals/astronomy-and-space-sciences/articles/10.3389/fspas.2024.1371058/full?utm_source=chatgpt.com "The unsolved problem of solar-wind turbulence - Frontiers"
[4]: https://cushman.host.dartmouth.edu/courses/engs250/Kolmogorov.pdf?utm_source=chatgpt.com "[PDF] Kolmogorov's Theory of Inertial Turbulence - Benoit Cushman-Roisin"




=================================


