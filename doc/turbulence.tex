

\section{Turbulence}


\subsection{Wave Transport Equation (with Kolmogorov Spectrum) for Bidirectional Alfvén Waves}

Assuming Alfvén waves propagate along the magnetic field (both {\it toward} and {\it away} from the Sun), the {\it wave energy density} $W_\pm(k, z, t)$ at wavenumber $k$ and position $z$ evolves according to the wave transport equation:

$$
\frac{\partial W_\pm(k, z, t)}{\partial t}
+ \left( V_A \mp U \right) \frac{\partial W_\pm}{\partial z}
= \Gamma_\pm(k) W_\pm(k) - \mathcal{D}_\pm(k) W_\pm(k) - \frac{\partial}{\partial k} \left[ D_{kk}^\pm \frac{\partial W_\pm}{\partial k} \right]
$$
Where: $W_+(k)$ = waves propagating away from the Sun, 
 $W_-(k)$ = waves propagating toward the Sun, 
 $V_A$ = local Alfvén speed, 
 $U$ = solar wind speed (assumed radial), 
 $\Gamma_\pm(k)$ = growth rate from streaming SEPs, 
 $\mathcal{D}_\pm(k)$ = damping rate (e.g., turbulence cascade or cyclotron damping), 
 $D_{kk}^\pm$ = spectral diffusion coefficient for nonlinear cascading (e.g., Kolmogorov-type). 


\subsection{Alfvén Wave Growth/Damping Rate ($\gamma$)}

For self-generated Alfvén waves due to streaming particles:

$$
\gamma(k) = \frac{\pi^2 e^2 v_A}{c B_0^2} \, p v \left[ \frac{\partial f(p, \mu)}{\partial \mu} \right]_{\mu = v_A/v}
$$
Where:
 $\gamma(k)$ = growth (if positive) or damping (if negative) rate for wave number $k$,
 $e$ = elementary charge,
 $v_A$ = Alfvén speed,
 $c$ = speed of light,
 $B_0$ = background magnetic field strength,
 $p$ = particle momentum,
 $v$ = particle speed,
 $f(p, \mu)$ = particle distribution function,
 $\mu$ = pitch-angle cosine.

This equation applies for {\it resonant interactions}, where:

$$
k = \frac{\Omega}{v \mu - v_A}
$$

with $\Omega$ being the particle gyrofrequency.



If you’re referring to {\it Kolmogorov-type turbulence} or cascading effects (sometimes loosely visualized as "frothy" structures in plasma), that's a different discussion involving spectral energy transfer.

If you're asking about the **overall Alfvén wave growth/damping rate** generated by the **entire particle population** (rather than for a single particle resonance), here's how it's typically formulated.



\subsection{Problem in Monte Carlo: Derivative of $f$}

In Monte Carlo models, computing $\partial f / \partial \mu$ is {\it noisy} and expensive due to statistical fluctuations. To avoid this:
 Use {\it streaming-based approximations}:
$$
\gamma(k) \propto \frac{S(k)}{W(k)}
$$



1. Identify resonant particles for each $k$ using:

$$
k = \frac{\Omega}{v \mu - v_A}
$$

2. Compute net streaming $S(k)$ of resonant particles.
3. Estimate wave growth/damping:

$$
\gamma(k) = C \cdot S(k)
$$

where $C$ includes constants like $e^2 v_A / B_0^2$.

4. Update wave energy density**:

$$
\frac{dW(k)}{dt} = 2 \gamma(k) W(k) - \text{damping terms} + \text{cascade}
$$







\subsection{Wave Growth/Damping Rates from SEPs (Modeled via Monte Carlo Particles)}

In Monte Carlo, SEPs are simulated as individual particles with positions, momenta, and pitch angles, so we must **extract macroscopic quantities** (like streaming) from these particles to compute $\Gamma_\pm(k)$:

\textbf{Growth rate (streaming-based approximation):}

$$
\Gamma_\pm(k) = \frac{\pi^2 e^2 v_A}{c B_0^2} \cdot \frac{1}{k} \cdot S_\pm(k)
$$

Where:

* $S_\pm(k)$ = **resonant streaming** of particles with wavenumber $k$, moving in direction $\pm$
* Computed from particle ensemble satisfying the resonance condition:

$$
k_{\text{res}} = \frac{\Omega}{v \mu \mp V_A}
$$

* The streaming $S(k)$ is estimated by summing over particles:

$$
S(k) = \sum_i w_i v_i \mu_i \, \delta\left(k - k_{\text{res},i}\right)
$$

where $w_i$ is the weight of particle $i$

\textbf{ Damping rate:}

Can be assumed constant or modeled as part of a Kolmogorov cascade:

$$
\mathcal{D}_\pm(k) \sim C_d \, k^{\alpha}
$$

Where $\alpha = 1/3$ for Kolmogorov scaling, or empirically determined.



\textbf{ 3. Coupling Wave Transport and Monte Carlo SEPs}


\textbf{ At each time step:}

1. Particles (Monte Carlo):

   * Advance SEPs via guiding center or Parker equation.  
   
   * At each step, compute local **pitch-angle scattering rate** $\nu(k)$ from $W(k)$, e.g.:

     $$
     D_{\mu\mu} \sim \frac{\pi^2 \Omega^2}{B_0^2} \left(1 - \mu^2\right) \frac{W(k)}{k}
     $$
     
   * Sample pitch-angle deflections accordingly.

2. Wave Growth:

   * From the **instantaneous SEP population**, compute streaming $S_\pm(k)$
   * Calculate $\Gamma_\pm(k)$, then update wave energy densities:

     $$
     \frac{dW_\pm}{dt} = \left[ \Gamma_\pm(k) - \mathcal{D}_\pm(k) \right] W_\pm + \text{cascade}
     $$

3. Wave Update:

   Integrate wave transport equation to update $W_\pm(k, z, t)$ spatially and spectrally.

4. Feedback Loop:

   New $W(k)$ updates scattering coefficients $D_{\mu\mu}$ used in the next Monte Carlo particle update.

This coupling ensures **self-consistent evolution** of both the turbulence spectrum and SEP distribution.







Because {\it growth} $\Gamma$ and {\t damping} $\mathcal D$ occur {\it locally in wavenumber space} (each scale $k$ exchanges energy with the particles and with neighbouring scales independently). If you decide to track only a {\it single, total} wave–energy quantity,

$$
\mathcal W_\pm(z,t)\;=\;\int_{k_1}^{k_2} W_\pm(k,z,t)\,dk ,
$$

you must still compute that total’s time-derivative from an integral of the $k$-dependent source terms:

$$
\frac{\partial \mathcal W_\pm}{\partial t}
\;+\;(V_A\!\mp\!U)\,\frac{\partial \mathcal W_\pm}{\partial z}
=\;2\!\int_{k_1}^{k_2}\!\Gamma_\pm(k)\,W_\pm(k)\,dk
\;-\;2\!\int_{k_1}^{k_2}\!\mathcal D_\pm(k)\,W_\pm(k)\,dk 
$$

Inside the integrals the {\it $k$-dependence remains}, because:

1. {\it Resonance is $k$-selective:}
   A particle with speed $v$ and pitch angle $\mu$ resonates only with
   $k_{\text{res}} = \Omega /(v\mu \mp V_A)$.
   The streaming that drives wave growth therefore differs from scale to scale.

2. {\it Non-linear cascade and physical dissipation are scale-selective:}
   Kolmogorov transfer, cyclotron damping, Landau damping, etc., each act most strongly at particular $k$.

3. {\it Closure still requires a spectrum:}
   To evaluate the integrals in (A) you must know (or assume) how the *total* energy $\mathcal W_\pm$ is distributed across $k$.
   A common closure is

   $$
   W_\pm(k) \;=\; \frac{\mathcal W_\pm}{N}\;k^{-5/3},
   \quad
   N=\!\!\int_{k_1}^{k_2}\!k^{-5/3}\,dk ,
   $$

   but the kernels $\Gamma(k)$ and $\mathcal D(k)$ still enter the integrals explicitly.

\subsection{Putting it into a Monte-Carlo coupling loop}

1. {\it Monte-Carlo step} – advance SEPs, tally the resonant streaming $S_\pm(k)$ in logarithmic $k$-bins.

2. {\it Growth rate per bin} 

   $$
   \Gamma_\pm(k)=\frac{\pi^2 e^2 V_A}{cB_0^2}\,\frac{S_\pm(k)}{k}.
   $$

3. {\it If you store the full spectrum} $W_\pm(k)$
   – update it with the $k$-dependent transport equation.
   {\it If you store only $\mathcal W_\pm$}
   – use your assumed spectral shape to evaluate the integrals in (A) and update the single variable $\mathcal W_\pm$.





\subsection{Coupling loop (wave and particle)}
for each global time step $\Delta t$:

     1. advance particles: 
    move particles along field, include focusing \& adiabatic deceleration.
    scatter each particle with local $D_{\mu\mu}$ built from current W(k) :
        \[
D_{\mu\mu}(\mu_i) = \frac{\pi \, \Omega_i^2}{B_0^2} (1 - \mu_i^2) \frac{W(k_{\text{res}})}{k_{\text{res}}}
\]
    update positions, $\mu_i$, energies

     2. accumulate streaming $S_{\pm}(k)$ from updated particles  

     3. compute $\Gamma_{\pm}(k)$ with eq. 

     4. integrate wave eq. (1) in (k,z) mesh for $\Delta t$
         e.g. finite-volume in z, Crank-Nicolson in k

     5. go to next step





\subsection{1.  Discretise wavenumber space}

\begin{tcolorbox}[colframe=black, colback=white]
\textbf{Log-$k$ grid:} \quad \( k_1, k_2, \ldots, k_N \) \\
Typically 30--60 bins per decade. \\
\[
\Delta k_j = k_{j+1} - k_j
\]
\end{tcolorbox}



\subsection{2.  Loop over Monte-Carlo particles and tally resonant streaming}

\begin{tcolorbox}[colframe=black, colback=white, title=Wave Accumulator Initialization and Update]
\textbf{Initialise accumulators for this spatial cell:}
\begin{align*}
S_{+}[1 \ldots N] &= 0 \quad \text{(waves moving anti-sunward)} \\
S_{-}[1 \ldots N] &= 0 \quad \text{(waves moving sunward)}
\end{align*}

\textbf{For each Monte-Carlo particle \( i \):}
\begin{align*}
v_i &= \frac{|\mathbf{p}_i|}{\gamma_i m_i} \\
\mu_i &= \cos(\text{pitch angle in local field frame}) \\
\Omega_i &= \frac{q_i B_0}{\gamma_i m_i} \quad \text{(gyro-frequency)}
\end{align*}

\textbf{Resonant \( k \) for each propagation sense:}
\begin{align*}
k_{\text{res},+} &= \frac{\Omega_i}{v_i \mu_i - V_A} \\
k_{\text{res},-} &= \frac{\Omega_i}{v_i \mu_i + V_A}
\end{align*}

\textbf{Choose the valid one (sign of denominator must match direction):}
\begin{itemize}
  \item If \( k_{\text{res},+} \) lies inside grid:
  \begin{itemize}
    \item Bin \( j = \text{index}(k_{\text{res},+}) \)
    \item \( S_{+}[j] \mathrel{+}= w_i v_i \mu_i \) \quad (where \( w_i \) is particle weight)
  \end{itemize}
  \item If \( k_{\text{res},-} \) lies inside grid:
  \begin{itemize}
    \item Bin \( j = \text{index}(k_{\text{res},-}) \)
    \item \( S_{-}[j] \mathrel{+}= w_i v_i \mu_i \)
  \end{itemize}
\end{itemize}
\end{tcolorbox}

\begin{tcolorbox}[colframe=black, colback=white, title=Units]
\( S \) is the \textbf{number flux}, with units:
\[
\text{m}^{-2} \, \text{s}^{-1}
\]
\end{tcolorbox}



\section*{3. Compute \textbf{growth rates} from streaming (avoids \( \partial f / \partial \mu \))}

For each bin \( j \) (centre \( k_j \)):

\[
\boxed{
\Gamma_{\pm}(k_j) = \frac{\pi^{2} e^{2} V_A}{c B_0^{2}} \, \frac{S_{\pm}(k_j)}{k_j}
}
\tag{1}
\]

\begin{itemize}
    \item \textbf{Positive} streaming in the same direction as the wave \textbf{amplifies} it.
    \item Opposite-sign streaming \textbf{damps} it.
\end{itemize}

If desired, smooth \( S(k) \) across neighbouring bins to suppress Monte-Carlo noise.



\begin{tabular}{@{}lll@{}}
\toprule
\textbf{Damping term} & \textbf{Expression} & \textbf{When to use} \\
\midrule
Kolmogorov cascade & 
\(
\mathcal{D}_{\pm}(k) = C_K^{-1} \epsilon^{1/3} k^{2/3}
\) (with \( C_K \approx 2 \)) &
If you assume an inertial-range cascade with cascade rate \( \epsilon \) (often set to keep background \( \delta B \) level) \\
\addlinespace
Cyclotron / Landau &
\(
\mathcal{D}_{\pm}(k) = \alpha_c k^2
\) (empirical) &
To remove energy at high \( k \) beyond the proton cyclotron scale \\
\bottomrule
\end{tabular}

\vspace{1em}

You can combine them:
\[
\mathcal{D} = \mathcal{D}_{\text{cascade}} + \mathcal{D}_{\text{phys}}.
\]

\subsection{ 5.  Update the wave spectrum}

Using an implicit or Crank-Nicolson step for stability:

$$
\frac{W_\pm^{n+1}-W_\pm^{n}}{\Delta t}
= 2\bigl[\Gamma_\pm-\mathcal{D}_\pm\bigr]\,\frac{W_\pm^{n+1}+W_\pm^{n}}{2}
-\frac{\partial}{\partial k}\!\Bigl(D_{kk}\,\partial_k W_\pm \Bigr)^{n+1/2}.
$$

$D_{kk}$ is the Kolmogorov diffusion coefficient
$D_{kk}=C_K\,\epsilon^{1/3}\,k^{7/3}$.

\subsection{Feed new $W(k)$ back to the particles}

For each particle i at its new pitch angle $\mu_i$:

$$
k_{\text{res}}  = \frac{\Omega_i}{v_i \mu_i \mp V_A},\quad
D_{\mu\mu}(\mu_i) = \frac{\pi\,\Omega_i^{2}}{B_0^{2}}\,(1-\mu_i^{2})\,
\frac{W_\pm(k_{\text{res}})}{k_{\text{res}}}.
$$

Draw a random pitch-angle kick from a Gaussian with variance $2D_{\mu\mu}\Delta t$.

---

\begin{tcolorbox}[colframe=black, colback=white, title=Time-Stepping Loop]
\begin{itemize}
    \item \textbf{while} \( t < t_{\text{end}} \):
    \begin{itemize}
        \item advance particles and accumulate \( S(k) \)
        \item \( \Gamma(k) \leftarrow \) eq.~(1)
        \item \( \mathcal{D}(k) \leftarrow \) chosen law
        \item update \( W(k) \) (implicit solve)
        \item update \( D_{\mu\mu} \) and scatter particles
        \item \( t \leftarrow t + \Delta t \)
    \end{itemize}
\end{itemize}
\end{tcolorbox}



\subsection{ 1.  Fix the spectral shape, keep only one amplitude}

Assume an inertial-range Kolmogorov spectrum between $k_{\min }$ and $k_{\max }$:

$$
\boxed{\,W_\pm(k,z,t)\;=\;A_\pm(z,t)\;k^{-5/3}},\qquad
N = \int_{k_{\min}}^{k_{\max}} k^{-5/3}\,dk
$$

Only the scalar **amplitude** $A_\pm(z,t)$ is evolved; the $k^{-5/3}$ factor is static.

Total wave energy density in the cell:

$$
\mathcal W_\pm(z,t)=A_\pm(z,t)\,N .
$$


\subsection{ 2.  Still tally resonant streaming per $k$-bin}

You must keep the {\it $k$}-resolved streaming** because resonance is scale-selective.

```pseudo

\begin{tcolorbox}[colframe=black, colback=white, title=Streaming Accumulation per Particle]
\begin{itemize}
    \item \textbf{for each particle} \( i \) \textbf{in the cell}:
    \begin{itemize}
        \item \( k_{\text{res},+} = \dfrac{\Omega_i}{v_i \mu_i - V_A} \) \quad \text{(anti-sunward waves)}
        \item \( k_{\text{res},-} = \dfrac{\Omega_i}{v_i \mu_i + V_A} \) \quad \text{(sunward waves)}
        \item \textbf{if} \( k_{\text{res},+} \in [k_{\text{min}}, k_{\text{max}}] \):
        \begin{itemize}
            \item \( j = \text{index}(k_{\text{res},+}) \)
            \item \( S_{+}[j] \mathrel{+}= w_i v_i \mu_i \)
        \end{itemize}
        \item \textbf{if} \( k_{\text{res},-} \in [k_{\text{min}}, k_{\text{max}}] \):
        \begin{itemize}
            \item \( j = \text{index}(k_{\text{res},-}) \)
            \item \( S_{-}[j] \mathrel{+}= w_i v_i \mu_i \)
        \end{itemize}
    \end{itemize}
\end{itemize}
\end{tcolorbox}



\subsection{ 3.  Growth rate per bin (unchanged)}

$$
\Gamma_\pm(k_j)=\frac{\pi^{2}e^{2}V_A}{cB_0^{2}}\,
\frac{S_\pm(k_j)}{k_j}.
$$



\subsection{ 4.  Collapse to one scalar growth term}

The {\it net} growth of the total energy is the spectrum-weighted integral:

$$
\frac{d\mathcal W_\pm}{dt}
=\;2\!\int_{k_{\min}}^{k_{\max}}\!\Gamma_\pm(k)\,W_\pm(k)\,dk
-\;2\!\int_{k_{\min}}^{k_{\max}}\!\mathcal D_\pm(k)\,W_\pm(k)\,dk .
$$

Insert $W_\pm(k)=A_\pm k^{-5/3}$:

$$
\boxed{\;
\frac{dA_\pm}{dt}=2\,A_\pm
\bigl[\langle\Gamma_\pm\rangle - \langle\mathcal D_\pm\rangle\bigr]},
\qquad
\langle X\rangle=
\frac{\displaystyle\int_{k_{\min}}^{k_{\max}}X(k)\,k^{-5/3}\,dk}
     {\displaystyle\int_{k_{\min}}^{k_{\max}}k^{-5/3}\,dk } .
$$

\begin{tcolorbox}[colframe=black, colback=white, title=Weighted Average Growth Rate]
\begin{itemize}
    \item Initialize:
    \[
    \text{num} = 0 \quad \text{(numerator for } \langle \Gamma \rangle \text{)}
    \]
    \[
    \text{den} = 0 \quad \text{(normalisation } N \text{, same each step, pre-compute)}
    \]
    \item \textbf{for each} \( k \)-bin \( j \):
    \begin{itemize}
        \item \( \text{weight} = k_j^{-5/3} \, \Delta k_j \)
        \item \( \text{num} \mathrel{+}= \Gamma_{+}[j] \times \text{weight} \quad \) (or \( \Gamma_{-}[j] \) for other sense)
        \item \( \text{den} \mathrel{+}= \text{weight} \quad \) (den \( = N \))
    \end{itemize}
    \item Compute:
    \[
    \langle \Gamma \rangle = \frac{\text{num}}{\text{den}}
    \]
\end{itemize}
\end{tcolorbox}

Do the same loop with $\mathcal D(k)$ to get $\langle\mathcal D\rangle$.

\subsection{5.  Advance the single amplitude}

\begin{tcolorbox}[colframe=black, colback=white, title=Amplitude Update]
\[
A_{\text{new}} = A_{\text{old}} \times \exp\left\{ 2 \Delta t \left( \langle \Gamma \rangle - \langle \mathcal{D} \rangle \right) \right\}
\]
\textit{(exact for constant rates in \( \Delta t \))}
\end{tcolorbox}

or, second-order Euler:


\begin{tcolorbox}[colframe=black, colback=white, title=Amplitude Update (Linearized)]
\[
A_{\text{new}} = A_{\text{old}} + \Delta t \times 2 A_{\text{old}} \left( \langle \Gamma \rangle - \langle \mathcal{D} \rangle \right)
\]
\end{tcolorbox}



\subsection{6.  Feed back to the particles}

Whenever you need the scattering coefficient for a particle with pitch-angle $\mu$:

$$
k_{\text{res}}=\frac{\Omega}{v\mu\mp V_A},\qquad
D_{\mu\mu}= \frac{\pi \Omega^{2}}{B_0^{2}}\,(1-\mu^{2})\,
\frac{A_\pm\,k_{\text{res}}^{-5/3}}{k_{\text{res}}}.
$$

Thus $D_{\mu\mu}\propto A_\pm$, so every time you update $A_\pm$ you automatically update the scattering strength everywhere in $k$.

\begin{tcolorbox}[colframe=black, colback=white, title=Why \(k\)-Resolved Streaming is Still Needed]
Although only one number \( A_{\pm} \) is stored, the averages \( \langle \Gamma \rangle \) and \( \langle \mathcal{D} \rangle \) in (2) require:
\begin{itemize}
    \item \( \Gamma(k) \) — from streaming in each bin, eq.~(1)
    \item \( \mathcal{D}(k) \) — from your cascade/dissipation law
\end{itemize}

Hence, the Monte-Carlo procedure still loops over \( k \)-bins, but the memory footprint is tiny (just two 1-D arrays for \( S_{\pm} \)), and only one scalar per direction is time-advanced.
\end{tcolorbox}

#### Bottom line

* **Kolmogorov assumption** → fixes the spectral *shape*.
* **Streaming tally per $k$-bin** → captures the correct resonance physics.
* **Equation (2)** → collapses those $k$-dependent rates to a single growth (or damping) rate for the total wave-energy amplitude that your Monte-Carlo code can evolve with one line of algebra.


Below is a compact, self-contained **physics specification** for a Monte-Carlo (MC) module that couples energetic charged particles to a background of field-aligned Alfvén turbulence whose spectrum is fixed to the inertial-range Kolmogorov form
$W(k)=A\,k^{-5/3}$ between $k_{\min }$ and $k_{\max }$.



\subsection{ 1.  Probability that the particle is scattered in $\Delta t$}

For gyro-resonant, small-angle pitch-angle diffusion the 1-D Fokker–Planck coefficient is

$$
D_{\mu\mu}(\mu_i)=\frac{\pi \Omega_i^{2}}{B_0^{2}}\bigl(1-\mu_i^{2}\bigr)
\frac{W\!\bigl(k_{\text{res}}^{\pm}\bigr)}{\,\lvert k_{\text{res}}^{\pm}\rvert\,}.
$$

Because $W(k)=A\,k^{-5/3}$,

$$
D_{\mu\mu}= \frac{\pi \Omega_i^{2}}{B_0^{2}}\bigl(1-\mu_i^{2}\bigr)
A\;|k_{\text{res}}^{\pm}|^{-8/3}.
$$

Pitch-angle diffusion is a homogeneous Wiener process, so the **survival probability** (no scatter in $\Delta t$) is

$$
P_{\rm surv}=e^{-2D_{\mu\mu}\,\Delta t},
\qquad
P_{\rm scatt}=1-P_{\rm surv}\;\simeq\;2D_{\mu\mu}\,\Delta t\;\;(\text{if }2D_{\mu\mu}\Delta t\ll1).
$$

In code: draw a uniform random number $\mathcal U\in[0,1]$; scatter if $\mathcal U<P_{\rm scatt}$.

\subsection{Scattering angle in the frame co-moving with the wave}

Work in the de-Hoffmann–Teller frame of the resonant wave, i.e. a frame translating with $V_A$ **along the field** (sign depends on wave sense).
In this frame the magnetic perturbation is time-stationary and pitch-angle diffusion is isotropic around the local field.

* Draw a Gaussian deviate $\Delta \mu$ with
  $\displaystyle \langle\Delta \mu\rangle=0,\quad
  \langle(\Delta \mu)^{2}\rangle = 2D_{\mu\mu}\,\Delta t.$

* Update the pitch angle: $\mu^{\prime}=\mu_i+\Delta \mu$.
  If $|\mu^{\prime}|>1$ reflect specularly to keep $|\mu|\le1$.

* Transform back to the plasma frame.
  Because $V_A\ll v_i$ for SEPs, the Lorentz transformation changes the magnitude of $\mu$ only by $\mathcal O(V_A/v_i)$; to first order you may keep $\mu^{\prime}$ unchanged.

**Angular deflection**

$$
\Delta\theta \;=\;\arccos μ^{\prime}-\arccos μ_i
\;\approx\;
-\frac{\Delta μ}{\sqrt{1-μ_i^{2}}}.
\tag{5}
$$

(This linearised form is what MC codes normally record for diagnostics.)

\subsection{Wave-energy increment produced by the scattering}

Each scattering event exchanges energy between particle and wave.
In QLT the **energy transferred to waves in bin $k_{\text{res}}$** is the work done by the induced electric field $E_\parallel$ over the interaction:

$$
\delta\!W(k_{\text{res}}) \;=\;
-\,\delta p_{\parallel}\,V_A/V_{\rm ph},
\qquad V_{\rm ph}\simeq V_A ,
$$

so that (sign convention: positive $\delta W$ = wave growth)

$$
\delta W(k_{\text{res}})\;=\;
-\,V_A\,\Delta p_{\parallel}.
$$

For small-angle scattering in the wave frame,

$$
\Delta p_{\parallel}=p_i\,(\mu^{\prime}-\mu_i)=p_i\,\Delta \mu .
$$

**Insert (8) into (7)** and distribute over the $k$-bin volume $\Delta k\Delta z$:

$$
\boxed{\;
\delta W(k_j)
= -\,p_i V_A \,\Delta μ \;
\bigl/(\Delta k\,\Delta z)\;}
$$

Summing (9) over all particles gives the **net spectral energy change**.
Dividing by $W(k_j)\,\Delta t$ recovers the familiar growth/damping rate

$$
\Gamma(k_j)\;=\;\frac{\pi^{2}e^{2}V_A}{cB_0^{2}}\,
\frac{S(k_j)}{k_j},
$$
where $S(k_j)=\sum_i w_i v_i \mu_i$ is the resonant streaming accumulated *before* scattering.
Consistency is automatic: equations (7)–(10) ensure that the *same* $D_{\mu\mu}$ that scatters particles also sets the wave-growth term in the turbulence transport equation.

---


\begin{tcolorbox}[colframe=black, colback=white, title=Summary of the Algorithm]
\textbf{For every particle each step:}
\begin{itemize}
    \item Evaluate \( k_{\text{res}} \) via (1);
    \item Compute \( D_{\mu\mu} \) with (3);
    \item Draw a scatter/no-scatter decision with (4);
    \item If scattered, draw \( \Delta \mu \) (Gaussian) and update \( \mu \);
    \item Add \( \delta W(k_{\text{res}}) \) from (9) to the wave-energy accumulator.
\end{itemize}

\textbf{After all particles are processed for the cell:}
\begin{itemize}
    \item Update the wave amplitude \( A \) (or full \( W(k) \) if you track it) with the accumulated \( \sum \delta W \);
    \item Proceed to the next Monte-Carlo time step.
\end{itemize}


With these three derived elements you have a physically closed Monte-Carlo model that:
\begin{itemize}
    \item reproduces Kolmogorov scattering statistics,
    \item conserves energy between particles and waves event-by-event, and
    \item yields the same macroscopic growth rate (10) as standard quasilinear theory.
\end{itemize}
\end{tcolorbox}

\subsection{Handling out-of-range pitch-angle cosine after a diffusion step}

When you update a particle’s pitch-angle cosine with

$$
\mu'=\mu+\Delta\mu ,\qquad \Delta\mu\sim\mathcal N\bigl(0,2D_{\mu\mu}\,\Delta t\bigr),
$$
the Gaussian kick can push $\mu'$ outside the **physical interval** $[-1,1]$.  In quasilinear theory the correct boundary condition is **zero probability flux** at $\mu=\pm1$;($\partial f/\partial\mu=0$).
Numerically this is enforced with **specular reflection**—exactly the rule used for a 1-D Brownian particle between two perfectly reflecting walls.

\subsubsection{ Reflection algorithm (vectorised-safe)}

\begin{tcolorbox}[colframe=black, colback=white, title=Pitch-Angle Reflection at Boundaries]
\textbf{If} \( \mu' > 1 \):
\begin{itemize}
    \item \( \mu' = 2 - \mu' \) \quad (reflect about \( \mu = +1 \))
    \item \textbf{If} \( \mu' < -1 \): \quad (overshot both walls in an extreme kick)
    \begin{itemize}
        \item \( \mu' = -2 - \mu' \) \quad (second reflection about \( \mu = -1 \))
    \end{itemize}
\end{itemize}
\textbf{Else if} \( \mu' < -1 \):
\begin{itemize}
    \item \( \mu' = -2 - \mu' \) \quad (reflect about \( \mu = -1 \))
    \item \textbf{If} \( \mu' > 1 \):
    \begin{itemize}
        \item \( \mu' = 2 - \mu' \) \quad (second reflection about \( \mu = +1 \))
    \end{itemize}
\end{itemize}
\end{tcolorbox}

\begin{tcolorbox}[colframe=black, colback=white, title=Why Reflection (and not Truncation or Redraw?)]

You may wrap the two reflections in a \textbf{while} \( |\mu'| > 1 \) loop; with realistic \( \Delta t \), two reflections are enough in practice.

The operation preserves the step's magnitude \( |\Delta \mu| \) and satisfies \( \frac{\partial f}{\partial \mu} = 0 \) at the boundaries.

\bigskip

\textbf{Why reflection (and not truncation or re-draw)?}
\begin{itemize}
    \item \textbf{Detailed balance:} Reflection conserves the diffusive probability current; truncation artificially piles up particles at \( \mu = \pm 1 \).
    \item \textbf{Energy conservation:} In the wave frame, scattering is elastic; reflecting the overshoot keeps \( |v| \) unchanged.
    \item \textbf{Computationally cheap:} Just a pair of arithmetic operations; no extra random draw.
\end{itemize}

\bigskip

\textbf{Compact formula:}

You can write the double reflection in one line:
\[
\mu' =
\begin{cases}
2 - \mu', & \text{if } \mu' > 1, \\
-2 - \mu', & \text{if } \mu' < -1.
\end{cases}
\]
Then test once more: if \( |\mu'| > 1 \), repeat (rare).

\bigskip

After this adjustment, you are guaranteed \( -1 \leq \mu' \leq 1 \), and the Monte-Carlo simulation remains faithful to quasilinear diffusion physics.

\end{tcolorbox}

\begin{tcolorbox}[colframe=black, colback=white, title=Self-Contained Monte-Carlo Recipe for Solar-Wind Ions]

The goal is to couple:
\begin{itemize}
    \item a Parker-type particle transport solved with Monte-Carlo pseudo-particles whose distribution is isotropic in the local frame, and
    \item a single, time-dependent wave–energy amplitude \( A_{\pm}(r, t) \) in the fixed spectrum:
\end{itemize}
\[
W_{\pm}(k) = A_{\pm}(r, t) \, k^{-5/3}, \quad k_{\min} \leq k \leq k_{\max}.
\]
No high-speed (\( v \gg V_A \)) approximation is made; all formulae remain valid down to \( v \sim V_A \).

\bigskip

\textbf{1. Gyro-resonant condition (valid for all \( v \))}

For each particle \( i \) with speed \( v_i \) and pitch-angle cosine \( \mu_i \):
\[
k_{\text{res},\pm} = \frac{\Omega_i}{v_i \mu_i \mp V_A}, \quad \Omega_i = \frac{q_i B_0}{m_i \gamma_i}.
\tag{1}
\]
A resonance exists only when \( |v_i \mu_i| > V_A \).

\bigskip

\textbf{2. Pitch–angle diffusion coefficient \( D_{\mu\mu} \)}
\[
D_{\mu\mu}(\mu_i, v_i) = \frac{\pi \Omega_i^2}{B_0^2} (1 - \mu_i^2) A_{\pm} |k_{\text{res},\pm}|^{-8/3}.
\tag{2}
\]

\bigskip

\textbf{3. Scattering probability in a Monte-Carlo time-step \( \Delta t \)}
\[
P_{\text{scatt}} = 1 - \exp\left( -2 D_{\mu\mu} \Delta t \right) \quad \rightarrow \quad 2 D_{\mu\mu} \Delta t \ll 1 \quad \text{(approximation)}.
\tag{3}
\]
Draw a uniform deviate \( U \in [0, 1] \); scatter if \( U < P_{\text{scatt}} \).

\bigskip

\textbf{4. Pitch-angle update with reflections}

If scattered, draw a Gaussian kick:
\[
\Delta \mu \sim \mathcal{N}(0, \, 2 D_{\mu\mu} \Delta t), \quad \mu' = \mu_i + \Delta \mu.
\]
Apply successive specular reflections until \( |\mu'| \leq 1 \):
\[
\mu' =
\begin{cases}
2 - \mu', & \text{if } \mu' > 1, \\
-2 - \mu', & \text{if } \mu' < -1.
\end{cases}
\tag{4}
\]

\bigskip

\end{tcolorbox}
\begin{tcolorbox}[colframe=black, colback=white, title=Self-Contained Monte-Carlo Recipe for Solar-Wind Ions]

\textbf{5. Energy (velocity-magnitude) diffusion — ion heating}

Quasilinear perpendicular velocity diffusion coefficient:
\[
D_{v_{\perp} v_{\perp}}(v_i, \mu_i) = \frac{\pi \Omega_i^2 V_A^2}{B_0^2} A_{\pm} |k_{\text{res},\pm}|^{-8/3}.
\tag{5}
\]
For an isotropic distribution, convert to speed diffusion:
\[
D_{vv} = \frac{1 - \mu_i^2}{2} D_{v_{\perp} v_{\perp}}.
\tag{6}
\]
Update the speed:
\[
v'_i = v_i + \sqrt{2 D_{vv} \Delta t} \, \xi, \quad \xi \sim \mathcal{N}(0, 1).
\tag{7}
\]

\bigskip

\textbf{6. Wave–energy change from each diffusive kick}

Energy conservation gives:
\[
\delta W(k_{\text{res},\pm}) = -\frac{m_i}{V_A} \left( v'_i - v_i + v_i \Delta \mu \right).
\tag{8}
\]
Accumulate \( \sum \delta W \) in the resonant \( k \)-bin.

\bigskip

\textbf{7. Single-amplitude wave update}

The net spectral energy change in the cell:
\[
\Delta W_{\pm} = \sum_{\text{particles}} \delta W, \quad W_{\pm} = A_{\pm} N, \quad N = \int_{k_{\min}}^{k_{\max}} k^{-5/3} \, \mathrm{d}k.
\]
Advance the Kolmogorov amplitude:
\[
A_{\pm}^{\text{new}} = \max \left( 0, \, A_{\pm}^{\text{old}} - \frac{\Delta W_{\pm}}{N} \right).
\tag{9}
\]
(No growth term appears because net streaming is zero for isotropic \( f \).)

\bigskip

\end{tcolorbox}
\begin{tcolorbox}[colframe=black, colback=white, title=Self-Contained Monte-Carlo Recipe for Solar-Wind Ions]

\textbf{8. Algorithm in One Glance}

\begin{itemize}
    \item \textbf{for each time-step} \( \Delta t \):
    \begin{itemize}
        \item \textbf{Loop over particles in cell}:
        \begin{itemize}
            \item \textbf{If} \( |v_i \mu_i| \leq V_A \): continue (non-resonant).
            \item Calculate \( k_{\text{res}} = \Omega_i / (v_i \mu_i \mp V_A) \).
            \item \textbf{Pitch-angle scattering}:
            \[
            D_{\mu\mu} = \text{eq.~(2)}, \quad P_{\text{scatt}} = 1 - \exp(-2 D_{\mu\mu} \Delta t).
            \]
            \item Draw random; if \( \text{random}() < P_{\text{scatt}} \):
            \[
            \Delta \mu \sim \mathcal{N}(0, \sqrt{2 D_{\mu\mu} \Delta t}), \quad \mu_{\text{new}} = \text{reflect}(\mu_i + \Delta \mu).
            \]
            Else:
            \[
            \mu_{\text{new}} = \mu_i, \quad \Delta \mu = 0.
            \]
            \item \textbf{Speed diffusion (ion heating)}:
            \[
            D_{vv} = \frac{1 - \mu_{\text{new}}^2}{2} \left( \frac{\pi \Omega_i^2 V_A^2}{B_0^2} \right) A k_{\text{res}}^{-8/3}.
            \]
            \[
            v_{\text{new}} = v_i + \text{gaussian}(0, \sqrt{2 D_{vv} \Delta t}), \quad \Delta v = v_{\text{new}} - v_i.
            \]
            \item \textbf{Energy exchange with wave}:
            \[
            \delta W = -m_i ( \Delta v + v_i \Delta \mu ) / V_A.
            \]
            Accumulate \( \delta W \) in \( \text{bin}(k_{\text{res}}) \).
            \item \textbf{Store new phase-space coordinates}:
            \[
            v_i, \mu_i = v_{\text{new}}, \mu_{\text{new}}.
            \]
        \end{itemize}
        \item \textbf{Update wave amplitude from all} \( \delta W \):
        \[
        A_{\pm} \leftarrow \max\left( 0, A_{\pm} - \frac{\sum \delta W}{N} \right).
        \]
    \end{itemize}
\end{itemize}

\bigskip

\textbf{Why it Works}
\begin{itemize}
    \item \textbf{No high-speed assumption} — all \( v \) appear explicitly; resonance switches off when \( v \mu \) falls below \( V_A \).
    \item \textbf{Parker-equation compatibility} — isotropy is preserved because pitch-angle kicks are symmetric and large \( D_{\mu\mu} \) quickly erases momentary anisotropy.
    \item \textbf{Self-consistent heating} — Eqs.~(5–9) conserve energy between particles and waves event-by-event and reduce to standard cyclotron damping expressions when averaged over a Maxwellian.
\end{itemize}

\end{tcolorbox}


\section*{Parker-equation–compatible Monte-Carlo Recipe (No \(\mu\))}

Below is a \textbf{Parker-equation–compatible Monte-Carlo recipe} that \textbf{never stores \( \mu \)}.
All pitch-angle information is handled statistically, so the particle ensemble remains \textbf{strictly isotropic} at every step.

\bigskip
\hrule
\bigskip

\section*{0. Set-up and Fixed Kolmogorov Spectrum}

\begin{itemize}
    \item \textbf{Wave spectrum} (one amplitude per propagation sense \( \pm \)):
    \[
    W_\pm(k, r, t) = A_\pm(r, t)\; k^{-5/3}, \qquad k_{\min} \leq k \leq k_{\max}.
    \]
    
    \item Particles are represented by pseudo-particles that carry only \textbf{speed} \( v \) and \textbf{position} \( r \).
    
    \item Their directions are redrawn isotropically whenever scattering occurs.
    
    \item Local background: \( B_0(r), \; V_A(r) \).
\end{itemize}

\bigskip
\hrule
\bigskip

\section*{1. Isotropic Pitch-Angle Diffusion \(\rightarrow\) One \textbf{Scattering Frequency}}

Start from the quasilinear coefficient that \emph{would} hold for a given pitch cosine \( \mu \):
\[
D_{\mu\mu}(\mu, v) = \frac{\pi \Omega^2}{B_0^2} (1 - \mu^2) A_\pm \left| k_{\text{res}}^\pm \right|^{-8/3}, \qquad
k_{\text{res}}^\pm = \frac{\Omega}{v\mu \mp V_A}.
\]

Average over an \textbf{isotropic} distribution \( P(\mu) = \frac{1}{2} \):
\[
\boxed{ \bar{D}_{\mu\mu}(v) = \frac{1}{2} \int_{-1}^{+1} D_{\mu\mu}(\mu, v) \, d\mu. }
\tag{1}
\]

Carry out the integral analytically for \( v > V_A \) (or numerically tabulate once):
\[
\bar{D}_{\mu\mu}(v) = \frac{\pi \Omega^2 A_\pm}{B_0^2} \; \mathcal{I}\left( \frac{v}{V_A} \right),
\]
where
\[
\mathcal{I}(x) = \frac{1}{2} \int_{-1}^{+1} (1 - \mu^2) \left| x\mu \mp 1 \right|^{-8/3} \, d\mu.
\]

\emph{If \( v \leq V_A \), the resonance band does not exist and \( \bar{D}_{\mu\mu} = 0 \).}

\bigskip

\textbf{Scattering frequency}
\[
\boxed{ \nu_{\text{sc}}(v) = 2 \bar{D}_{\mu\mu}(v). }
\tag{2}
\]
\emph{This single number replaces the \( \mu \)-dependent coefficient used in pitch-angle-resolved models.}

\bigskip
\hrule
\bigskip

\section*{2. Monte-Carlo Step – Isotropic Scattering}

For each particle during a time-step \( \Delta t \):
\begin{tcolorbox}[colframe=black, colback=white, title=Isotropic Scattering Step]
\begin{itemize}
    \item \( \nu = \nu_{\text{sc}}(v) \) \hfill \texttt{\# eq.~(2)}
    \item \( P_{\text{sc}} = 1 - \exp(-\nu \, \Delta t) \) \hfill \texttt{\# probability to scatter}
    \item \textbf{if} \texttt{random()} \( < P_{\text{sc}} \): \hfill \texttt{\# scatter event happens}
    \begin{itemize}
        \item \( \hat{v} = \texttt{isotropic\_unit\_vector()} \) \hfill \texttt{\# completely new random direction}
    \end{itemize}
\end{itemize}
\end{tcolorbox}

\emph{No angle bookkeeping is needed; the velocity magnitude stays \( v \) for the moment.}

\bigskip
\hrule
\bigskip

\section*{3. Energy Diffusion (Ion Heating)}

The quasilinear perpendicular-velocity coefficient, averaged over isotropy, becomes
\[
\boxed{ \bar{D}_{vv}(v) = \frac{1}{3} V_A^2 \, \nu_{\text{sc}}(v). }
\tag{3}
\]
\emph{(This follows from \( \langle 1 - \mu^2 \rangle = 2/3 \).)}

\bigskip

\textbf{Speed update}
\[
\boxed{ v' = v + \sqrt{2 \bar{D}_{vv} \, \Delta t} \; \xi, \qquad \xi \sim \mathcal{N}(0, 1). }
\tag{4}
\]
\emph{Set \( v' \geq 0 \); draw a new isotropic direction \textbf{after} the speed change so the distribution remains isotropic.}

\bigskip
\hrule
\bigskip

\section*{4. Wave-Energy Change Associated with One Particle}

The particle’s kinetic-energy increment is
\[
\boxed{ \Delta E_p = \frac{1}{2} m (v'^{2} - v^2). }
\tag{5}
\]

Energy conservation (wave frame speed \( V_A \)) gives the wave change:
\[
\boxed{ \Delta \mathcal{W}_\pm = - \Delta E_p. }
\tag{6}
\]
\emph{Because the ensemble is isotropic there is no net streaming, so waves only \textbf{damp}. Growth stays zero.}

Accumulate \( \sum \Delta \mathcal{W}_\pm \) for the cell during the step.

\bigskip
\hrule
\bigskip

\section*{5. Update the Single Kolmogorov Amplitude}

Total spectral energy in the cell:
\[
\mathcal{W}_\pm = A_\pm N, \qquad N = \int_{k_{\min}}^{k_{\max}} k^{-5/3} \, dk.
\]

After the step:
\[
\boxed{ A_\pm^{\text{new}} = \max\left( 0, \; A_\pm^{\text{old}} - \frac{\sum \Delta \mathcal{W}_\pm}{N} \right). }
\tag{7}
\]

\bigskip
\hrule
\bigskip

\section*{6. Complete Algorithm (Pseudo-code)}

\begin{lstlisting}[mathescape=true]
# pre-tabulate ν_sc(v) and Dvv(v) on a v-grid for speed.

for each global step Δt at radius r:

    E_wave_change_plus  = 0
    E_wave_change_minus = 0          # two propagation senses if both kept

    for each pseudo-particle:
        # (a) SCATTERING DECISION
        ν   = ν_sc(v)
        if random() < 1 - exp(-ν Δt):
            v_dir = isotropic_unit_vector()

        # (b) SPEED (HEATING) UPDATE
        Dvv = (1/3) * V_A^2 * ν
        v_new = v + gaussian(0, sqrt(2 Dvv Δt))
        v_new = max(v_new, 0)

        # (c) ENERGY EXCHANGE
        ΔE = 0.5 * m * (v_new**2 - v**2)
        E_wave_change_plus  -= ΔE     # choose + or – sense; here use “+”
        
        # (d) STORE NEW SPEED
        v = v_new

    # (e) WAVE AMPLITUDE UPDATE
    A_plus  = max( 0,  A_plus  - E_wave_change_plus / N )
    # repeat for A_minus if needed
\end{lstlisting}

\bigskip
\hrule
\bigskip

\section*{Key Points}
\begin{itemize}
    \item \textbf{No \( \mu \) anywhere:} only the \emph{scalar} scattering frequency \( \nu_{\text{sc}}(v) \) and speed-diffusion coefficient \( \bar{D}_{vv}(v) \) are required.
    \item \textbf{Isotropisation:} enforced by drawing a \emph{completely new} random direction at each scatter; between scatters the particle simply convects.
    \item \textbf{Heating of solar-wind ions:} obtained naturally from the speed-diffusion step (4).
    \item \textbf{Wave damping:} the negative of particle heating, ensuring energy conservation.
\end{itemize}



\begin{tcolorbox}[colframe=black, colback=white, title=Step-by-Step Coupling Scheme]

Below is a \textbf{step-by-step coupling scheme} that marries the \textbf{isotropic-Parker Monte-Carlo module} (previous reply) to a \textbf{transport equation for the turbulence–energy density}. The scheme keeps only \textbf{one Kolmogorov amplitude per propagation sense} \( A_\pm(r, t) \) yet still:
\begin{itemize}
    \item advects Alfvén-wave energy with the solar wind,
    \item accounts for nonlinear (Kolmogorov) cascade,
    \item damps/feeds the waves through \textbf{exact energy exchange with the particles}.
\end{itemize}

\section*{1. Wave-Energy Transport Equation (Single Kolmogorov Amplitude)}

Integrate the 1-D spectral wave equation over \( k \) using \( W_\pm(k) = A_\pm k^{-5/3} \). The result for the \emph{total} energy density \( \mathcal{W}_\pm = A_\pm N \) (with \( N = \int_{k_{\min}}^{k_{\max}} k^{-5/3} \, dk \)) is:
\[
\boxed{
\frac{\partial \mathcal{W}_\pm}{\partial t}
+ (V_A \mp U) \frac{\partial \mathcal{W}_\pm}{\partial r}
= - \underbrace{\frac{\mathcal{W}_\pm}{\tau_{\text{cas}}}}_{\text{Kolmogorov cascade}}
- \underbrace{Q_{\text{ion}}}_{\text{particle damping}}
+ S_{\text{inj}}(r, t)
}
\tag{1}
\]
\begin{itemize}
    \item \textbf{Advection speed:} group velocity \( V_A \mp U \).
    \item \textbf{Cascade term:} e.g., \( \tau_{\text{cas}}^{-1} = \frac{C_K}{L_\perp} \sqrt{ \delta B^2 / (4 \pi \rho) } \).
    \item \textbf{Particle term:} 
    \[
    Q_{\text{ion}} = \sum_{\text{all MC particles in cell}} \frac{-\Delta E_p}{\Delta r}
    \]
    with \( \Delta E_p \) from Eq.~(5) in the previous answer.
    \item \textbf{Source term:} \( S_{\text{inj}} \), optional Alfvén-wave source (e.g., at the coronal base).
\end{itemize}
Since \( \mathcal{W}_\pm = A_\pm N \) and \( N \) is constant, you can update either \( \mathcal{W}_\pm \) or \( A_\pm \); below we update \( A_\pm \).

\end{tcolorbox}
\begin{tcolorbox}[colframe=black, colback=white, title=Step-by-Step Coupling Scheme]

\section*{2. Discretisation in a Radial Mesh}

\begin{itemize}
    \item \textbf{Radial grid:} \( r_j, \quad j = 0 \dots J \) (log-spacing is convenient).
    \item \textbf{Time step:} \( \Delta t \) limited by CFL condition:
    \[
    \Delta t < \min_j \left\{ \frac{ \Delta r_j }{ | V_A \mp U | } \right\}.
    \]
\end{itemize}

\textbf{Finite-volume update for \( A_\pm \)}:
\[
A_\pm^{n+1}(j) = A_\pm^{n}(j) - \frac{ \Delta t }{ \Delta r_j } \left[ F_{j+1/2} - F_{j-1/2} \right]
- \frac{ \Delta t }{ N } \left[ \frac{A_\pm N}{\tau_{\text{cas}}} + Q_{\text{ion}} - S_{\text{inj}} \right]_j
\tag{2}
\]
\begin{itemize}
    \item \textbf{Upwind fluxes:}
    \[
    F_{j+1/2} = (V_A \mp U)_{j+1/2} \, A_\pm^{\text{up}}
    \]
    Use van-Leer or piecewise-linear reconstruction for second-order accuracy.
    \item Cascade, particle, and source terms are cell-centred.
\end{itemize}


\end{tcolorbox}

\begin{tcolorbox}[colframe=black, colback=white, title=Step-by-Step Coupling Scheme]

\section*{3. Monte-Carlo \(\leftrightarrow\) Turbulence Coupling at Each Global Step}

\begin{lstlisting}[mathescape=true]
============================================
(1)  Advection / Cascade sub-step for $A_\pm$
============================================
  for each radial cell j:
        # 1a. build upwind fluxes  F
        # 1b. update $A_\pm$ with eq.(2)   (no particle term yet)

============================================
(2)  Particle sub-step in each cell j
============================================
  initialise   Q_ion(j) = 0

  loop over Monte-Carlo particles:
\begin{itemize}
    \item \textbf{2a.} Scattering decision with $\nu_{\text{sc}}(v) $
    \item \textbf{2b.} Speed diffusion $v \rightarrow v'$
    \item \textbf{2c.} Energy change $
    \Delta E_p = \frac{1}{2} m (v'^2 - v^2) $
    
    \item \textbf{2d.} Update particle position by Parker drifts
\end{itemize}
============================================
(3) Particle damping $\leftrightarrow$ wave update
============================================
\text{for each radial cell } j:

$A_\pm(j) \leftarrow A_\pm(j) - \frac{\Delta t}{N} \cdot Q_{\text{ion}}(j) 
\quad \texttt{\# Eq.~(2) term}
$
$
A_\pm(j) = \max(A_\pm(j), 0) 
\quad \texttt{\# maintain positivity}
$
\end{lstlisting}

\emph{Step-ordering:} split the advection/cascade and particle feedback for clarity; operator-splitting error is \( \mathcal{O}(\Delta t) \) and can be reduced with Strang splitting.

\end{tcolorbox}
\begin{tcolorbox}[colframe=black, colback=white, title=Step-by-Step Coupling Scheme]

\section*{4. What the Coupling Achieves}


\noindent
\begin{tabularx}{\textwidth}{@{}lXl@{}}
\toprule
\textbf{Module} & \textbf{Uses} & \textbf{Provides to the other} \\
\midrule
\textbf{Monte-Carlo Parker solver} & local \( A_\pm(r, t) \) to compute \( \nu_{\text{sc}} \), \( D_{vv} \) & energy sink \( Q_{\text{ion}} \) (heating) \\
\textbf{Wave-transport solver} & \( Q_{\text{ion}} \) to damp \( \mathcal{W}_\pm \); cascade law; boundary injection & updated \( A_\pm(r, t + \Delta t) \) \\
\bottomrule
\end{tabularx}


\medskip

\emph{Energy is conserved} cell-by-cell because the exact negative of each particle’s \( \Delta E_p \) is removed from \( A_\pm N \).

\section*{5. Boundary and Initial Conditions}

\begin{itemize}
    \item At \( r = r_0 \) (coronal base) prescribe an \textbf{injected amplitude} \( A_{+,0} \) (outward) and optionally a reflection coefficient to seed \( A_- \).
    \item At the outer boundary \( r = r_{\max} \), use free-outflow for \( A_+ \) and impose a small inflowing \( A_- \) if interstellar turbulence is required.
\end{itemize}

Initial \( A_\pm(r, 0) \) can be set from empirical \( \delta B / B \propto r^{-3/2} \) or any magnetogram-based model.

\section*{6. Practical Diagnostics}

\begin{itemize}
    \item \textbf{Ion heating rate:} \( Q_{\text{ion}}(r) \).
    \item \textbf{Wave energy flux:} \( F_\pm = A_\pm N (V_A \mp U) \).
    \item \textbf{Residual energy ratio:} 
    \[
    \frac{ \delta B^2 }{ B_0^2 } = \frac{ 2 \int W_\pm(k) \, dk }{ B_0^2 } = \frac{ 2 A_\pm N }{ B_0^2 }.
    \]
\end{itemize}

\emph{Consistent trends:} outward \( A_+ \) decays by cascade and ion heating; inward \( A_- \) is generated by reflection or parametric decay—reproducing standard fast-wind observations.

\bigskip
\hrule
\bigskip

\section*{In One Sentence}

\begin{quote}
\textbf{Advance \( A_\pm \) with an upwind-cascade equation, subtract exactly the power \( Q_{\text{ion}} \) that your isotropic Parker Monte-Carlo adds to the particles, and feed the updated \( A_\pm \) back into the next step’s scattering frequency.}
\end{quote}

That closes the loop between turbulence transport and Parker-equation particle evolution \textbf{without ever tracking \( \mu \) explicitly}.

\end{tcolorbox}


\begin{tcolorbox}[colback=white, colframe=black, title={From Parker‐Equation Transport Coefficient $D_{xx}$ to a Monte-Carlo Scattering Probability}, fonttitle=\bfseries]

In an \textbf{isotropic} Parker equation the spatial diffusion coefficient parallel to the mean field is
\begin{equation}
D_{xx} = \frac{1}{3} v \lambda,
\tag{1}
\end{equation}
where
\begin{itemize}
  \item $v$ – particle speed in the plasma frame,
  \item $\lambda$ – \textit{scattering mean free path} (average distance travelled between independent pitch-angle randomisations).
\end{itemize}

\medskip

\textbf{1. Scattering frequency}

Solve Eq.~(1) for $\lambda$:
\[
\lambda = \frac{3 D_{xx}}{v}.
\]

A particle that moves at speed $v$ crosses one mean free path in the \textbf{scattering time}
\begin{equation}
\tau_{\text{sc}} = \frac{\lambda}{v} = \frac{3 D_{xx}}{v^2}.
\tag{2}
\end{equation}

Hence the \textbf{scattering frequency}
\begin{equation}
\nu_{\text{sc}} = \tau_{\text{sc}}^{-1} = \frac{v^2}{3 D_{xx}}.
\tag{3}
\end{equation}

\medskip

\textbf{2. Probability of at least one scattering in a time step $\Delta t$}

Scattering events are Poisson-distributed with rate $\nu_{\text{sc}}$. Therefore
\begin{equation}
\boxed{
P_{\text{scatt}}(\Delta t) = 1 - \exp\!\left[-\nu_{\text{sc}} \Delta t\right] = 1 - \exp\!\left[-\frac{v^2 \Delta t}{3 D_{xx}}\right]}.
\tag{4}
\end{equation}

\begin{itemize}
  \item \textbf{Small-step limit:} If $\Delta t \ll \tau_{\text{sc}}$ (typical Monte-Carlo choice),
  \[
  P_{\text{scatt}} \approx \frac{v^2}{3 D_{xx}} \Delta t.
  \]
  
  \item \textbf{Relativistic extension:} Replace $v$ with $v = \beta c$ and, if desired, scale $D_{xx}$ with the particle’s rigidity $R$; the form of Eq.~(4) is unchanged.
\end{itemize}

\medskip

\textbf{3. Implementation snippet}
\begin{lstlisting}[mathescape=true, basicstyle=\ttfamily\small]
# inputs: $v$, $D_{xx}$, $\Delta t$
$\nu_{\text{sc}} = \dfrac{v \times v}{3.0 \times D_{xx}}$
$P_{\text{sc}} = 1.0 - \exp(-\nu_{\text{sc}} \times \Delta t)$
if random() < $P_{\text{sc}}$:
    # perform isotropic re-orientation (full scattering)
\end{lstlisting}


Choose $\Delta t$ so that $P_{\text{scatt}} \lesssim 0.3$ for numerical stability; larger probabilities can be handled but require multi-scatter logic.

\medskip

\textbf{Result:}

Using only the Parker-equation coefficient $D_{xx}$, the probability that a particle undergoes an Alfvén-wave pitch-angle scattering during the next interval $\Delta t$ is given by Eq.~(4).

\end{tcolorbox}


\begin{tcolorbox}[colback=white, colframe=black, title={Setup}]
\begin{itemize}
\item Two counter-propagating Alfvén-wave populations: 
$W_{+}(k)$ – outward (anti-Sunward) waves, and $W_{-}(k)$ – inward (Sunward) waves.
\item A particle of speed $v$, pitch-angle cosine $\mu$ resonates with a single wavenumber in each population:
\[
k_{\text{res}}^{\,+} = \frac{\Omega}{v\mu - V_A}, \qquad
k_{\text{res}}^{\,-} = \frac{\Omega}{v\mu + V_A}.
\]
\item The \textbf{pitch-angle diffusion coefficient} is the sum of the two contributions:
\[
D_{\mu\mu}(\mu) = D_{\mu\mu}^{\,+}(\mu) + D_{\mu\mu}^{\,-}(\mu),
\]
where (quasi-linear theory)
\[
D_{\mu\mu}^{\,\pm}(\mu) = \frac{\pi \Omega^{2}}{B_0^{2}} (1 - \mu^{2}) 
\frac{W_{\pm}\!\left(k_{\text{res}}^{\pm}\right)}{\left|k_{\text{res}}^{\pm}\right|}.
\]
\end{itemize}

\medskip

\textbf{1. Probability that \emph{a} scattering occurs in the next interval $\Delta t$}

Each wave population supplies an independent Poisson process with rate 
$\nu_{\pm} = 2 D_{\mu\mu}^{\,\pm}(\mu)$. The total rate is
\[
\nu_{\text{tot}} = 2\left(D_{\mu\mu}^{\,+} + D_{\mu\mu}^{\,-}\right) \equiv 2 D_{\mu\mu}.
\]
Hence
\begin{equation}
\boxed{
P_{\text{scatt}}(\Delta t) = 1 - \exp\left[-\nu_{\text{tot}} \Delta t\right] = 1 - \exp\left[-2 D_{\mu\mu} \Delta t\right]
}.
\tag{1}
\end{equation}

Small-step approximation ($2 D_{\mu\mu} \Delta t \ll 1$):
\[
P_{\text{scatt}} \simeq 2 D_{\mu\mu} \Delta t.
\]

\medskip

\end{tcolorbox}
\begin{tcolorbox}[colback=white, colframe=black, title={Setup}]

\textbf{2. Probability that the scattering comes from a \emph{particular} wave band}

Divide each population into narrow $k$-bins (or ``bands'') labelled by index $j$. For band $j$ in the $+$ population let
\[
\nu_{j}^{\,+} = 2 D_{\mu\mu,j}^{\,+} 
= \frac{2\pi \Omega^{2}}{B_0^{2}} (1 - \mu^{2}) \frac{W_{+,j}}{k_{j}},
\]
and analogously $\nu_{j}^{\,-} = 2 D_{\mu\mu,j}^{\,-}$.

\medskip

\textbf{2-a Absolute probability in the next $\Delta t$}

Because the individual Poisson processes are independent,
\begin{equation}
P_{j}^{\pm}(\Delta t) = 1 - \exp\left[-\nu_{j}^{\pm} \Delta t\right]
\approx \nu_{j}^{\pm} \Delta t = 2 D_{\mu\mu,j}^{\pm} \Delta t.
\tag{2}
\end{equation}

\medskip

\textbf{2-b Conditional probability given that a scattering \emph{will} occur}

If you first decide whether any scattering happens (probability $P_{\text{scatt}}$), then choose which band caused it, the branching ratio is
\begin{equation}
\boxed{
\mathcal{P}_{j}^{\pm} = 
\frac{\nu_{j}^{\pm}}{\nu_{\text{tot}}} = 
\frac{D_{\mu\mu,j}^{\pm}}{D_{\mu\mu}^{\,+} + D_{\mu\mu}^{\,-}}
}.
\tag{3}
\end{equation}
so that $\sum_{j} \left( \mathcal{P}_{j}^{+} + \mathcal{P}_{j}^{-} \right) = 1$.

\medskip

\textbf{Monte-Carlo algorithm (per particle per step)}
\begin{lstlisting}[mathescape=true, basicstyle=\ttfamily\small]
$\Delta t$ chosen so that $\nu_{\text{tot}} \times \Delta t \lesssim 0.3$
$\nu_{j}^{+}  = 2 \times D_{\mu\mu,j}^{+}$      # for all k-bins that resonate
$\nu_{j}^{-}  = 2 \times D_{\mu\mu,j}^{-}$
$\nu_{\text{tot}} = \sum \left( \nu_{j}^{+} + \nu_{j}^{-} \right )$

# 1. decide if any scatter
if random() < 1 - exp($-\nu_{\text{tot}} \times \Delta t$):      # Eq. (1)
    # 2. pick band and direction
    r = random() $\times \nu_{\text{tot}}$                      # roulette wheel
    cumulative = 0
    for each j:
        cumulative += $\nu_{j}^{+}$
        if r < cumulative: choose band j, direction +
        break
        cumulative += $\nu_{j}^{-}$
        if r < cumulative: choose band j, direction -
        break
    # 3. draw Gaussian kick $\Delta \mu$ with variance $2 D_{\mu\mu,j}^{\pm} \times \Delta t$
\end{lstlisting}

This procedure exactly realizes the probabilities (1)–(3).

\end{tcolorbox}

\begin{tcolorbox}[colback=white, colframe=black, title={Pitch–Angle Cosine After a Single Monte-Carlo Scattering Step}]

\textbf{1. Frames of reference}
\begin{itemize}
\item \textbf{Plasma frame (lab)} – the Parker equation and your Monte-Carlo particles are expressed here; pitch-angle cosine is
\[
\mu \equiv \cos\theta = \frac{\mathbf{v} \cdot \mathbf{B}_0}{v B_0}.
\]
\item \textbf{Wave frame} – the frame translating along the mean field at the Alfvén speed $V_A$ in the direction of the resonant wave packet ($+$ for anti-Sunward, $-$ for Sunward waves).
In this frame the electric field of the Alfvén wave vanishes and the gyro-resonant diffusion coefficient $D_{\mu\mu}$ is derived.
\end{itemize}

You \emph{apply} the stochastic pitch-angle kick in the \textbf{wave frame}, then convert the new cosine back to the plasma frame.

\medskip

\textbf{2. Diffusive kick in the wave frame}

For a time step $\Delta t$, the quasilinear theory gives
\[
\langle(\Delta \mu_{\text{wf}})^2\rangle = 2 D_{\mu\mu}^{\text{(res)}} \Delta t.
\]
Draw one Gaussian deviate $\xi \sim \mathcal{N}(0,1)$ and set
\begin{equation}
\boxed{
\mu_{\text{wf}}' = \mu_{\text{wf}} + \sqrt{2 D_{\mu\mu}^{\text{(res)}} \Delta t} \, \xi
}
\tag{1}
\end{equation}
where $\mu_{\text{wf}}$ is the particle’s current pitch-angle cosine measured in the wave frame:
\[
\mu_{\text{wf}} = \frac{\mu - \sigma \beta_A}{1 - \sigma \beta_A \mu},
\qquad
\beta_A \equiv \frac{V_A}{c}, \quad
\sigma = \begin{cases}
+1 & (\text{outward wave})\\[3pt]
-1 & (\text{inward wave}).
\end{cases}
\]

\medskip

\textbf{3. Boundary reflection}

If $|\mu_{\text{wf}}'| > 1$ after Eq.~(1), use specular reflection to keep it inside $[-1,1]$:
\[
\mu_{\text{wf}}' \gets
\begin{cases}
2 - \mu_{\text{wf}}', & \mu_{\text{wf}}' > 1, \\[2pt]
-2 - \mu_{\text{wf}}', & \mu_{\text{wf}}' < -1.
\end{cases}
\]
(Iterate once more if an extreme kick overshoots both walls.)

\medskip

\textbf{4. Transform back to the plasma frame}

A Lorentz boost of velocity components parallel to $B_0$ gives
\begin{equation}
\boxed{
\mu' = \frac{\mu_{\text{wf}}' + \sigma \beta_A}{1 + \sigma \beta_A \mu_{\text{wf}}'}
}
\tag{2}
\end{equation}
with the same sign $\sigma$ used in step 2.

\medskip

\textit{Non-relativistic or SEP limit $v \gg V_A$:}
\[
\beta_A \ll 1 \quad \Rightarrow \quad \mu' \simeq \mu_{\text{wf}}'
\]
to better than $V_A / v$.

\medskip

\end{tcolorbox}
\begin{tcolorbox}[colback=white, colframe=black, title={Pitch–Angle Cosine After a Single Monte-Carlo Scattering Step}]

\textbf{5. Summary algorithm (per particle)}
\begin{lstlisting}[mathescape=true, basicstyle=\ttfamily\small]
# inputs: $\mu$ (plasma frame), sign $\sigma$ (wave direction), $D_{\mu\mu}^{\text{(res)}}$, $\Delta t$
$\beta_A$   = $V_A / c$
$\mu_{\text{wf}}$ = ( $\mu - \sigma \beta_A$ ) / ( $1 - \sigma \beta_A \mu$ )

# Gaussian kick in wave frame
$\Delta \mu$ = sqrt( $2 D_{\mu\mu}^{\text{(res)}} \Delta t$ ) * normal_rand()
$\mu_{\text{wf}}^{\text{new}}$ = $\mu_{\text{wf}} + \Delta \mu$
$\mu_{\text{wf}}^{\text{new}}$ = reflect( $\mu_{\text{wf}}^{\text{new}}$ )      # keep in [-1,1]

# transform back to plasma frame
$\mu^{\text{new}}$ = ( $\mu_{\text{wf}}^{\text{new}} + \sigma \beta_A$ ) / ( $1 + \sigma \beta_A \mu_{\text{wf}}^{\text{new}}$ )
\end{lstlisting}


\medskip

\textbf{Result:}

Equation (1) gives the stochastic update of pitch angle \emph{in the wave frame}; Equation (2) returns the new cosine $\mu'$ that your Monte-Carlo solver stores in the plasma (Parker) frame.

\end{tcolorbox}


\begin{tcolorbox}[colback=white, colframe=black, title={Why Does the Frame‐Change Formula Use $V_A/c$ Instead of Just $V_A$?}]

\textbf{1. Two frames and one boost}

\medskip

\begin{tabular}{@{}p{0.35\linewidth} p{0.25\linewidth} p{0.35\linewidth}@{}}
\toprule
\textbf{Frame} & \textbf{Symbol} & \textbf{Velocity of the boost} \\
\midrule
Plasma (lab) frame & unprimed quantities $(\mu, v_\parallel, v)$ & -- \\
Wave frame (co-moving with an Alfvén wave packet) & ``wf'' quantities $(\mu_{\rm wf})$ & $+V_A$ along $\mathbf{B}_0$ for outward waves ($\sigma = +1$); $-V_A$ for inward waves ($\sigma = -1$) \\
\bottomrule
\end{tabular}

\medskip

To move from the plasma frame to the wave frame you perform a \textbf{Lorentz boost} of speed $\sigma V_A$ along the magnetic-field direction.

\medskip

\textbf{2. Lorentz transformation of a velocity component}

\medskip

For any velocity component parallel to the boost, special relativity gives
\[
v_\parallel^{\rm(wf)} = \frac{v_\parallel - \sigma V_A}{1 - \sigma \dfrac{v_\parallel V_A}{c^2}},
\]
where $c$ is the \textbf{speed of light in vacuum}.

\medskip

\textbf{3. Expressing the result as a pitch-angle cosine}

\medskip

Divide by the particle speed $v$ to get pitch-angle cosine:
\[
\mu_{\rm wf}
= \frac{v_\parallel^{\rm(wf)}}{v}
= \frac{\mu - \sigma \beta_A}{1 - \sigma \beta_A \mu}, 
\qquad 
\beta_A \equiv \frac{V_A}{c}.
\]

Hence the appearance of the $\boldsymbol{V_A/c}$ ratio.

\medskip

\textbf{4. Why use the full relativistic formula even for non-relativistic SEPs?}

\begin{itemize}
    \item \textbf{Correct for all energies} – Some Monte-Carlo particles may reach relativistic speeds; the exact boost keeps the algorithm valid without extra branches.
    \item \textbf{Consistency across modules} – Wave growth and damping formulas already rely on relativistic gyro-frequency $\Omega = qB/(\gamma m)$; using the same kinematics avoids subtle bookkeeping errors.
    \item \textbf{Negligible overhead} – Computing with $\beta_A = V_A/c \lesssim 10^{-3}$ adds one multiply and divide per particle — insignificant compared to random-number generation.
\end{itemize}

\medskip

In the common SEP limit where $v \gg V_A$ (e.g., $v \sim 30{,}000\,\text{km/s}$ vs $V_A \sim 50\,\text{km/s}$),
$\beta_A \ll 1$ and the formula reduces to the intuitive Galilean shift:
\[
\mu_{\rm wf} \simeq \mu - \sigma \frac{V_A}{v},
\]
but writing it with $V_A/c$ ensures exactness whenever it matters.

\medskip

\end{tcolorbox}
\begin{tcolorbox}[colback=white, colframe=black, title={Why Does the Frame‐Change Formula Use $V_A/c$ Instead of Just $V_A$?}]

\textbf{5. What is $c$ here?}

\medskip

$c$ is the fundamental constant 
\[
c = 2.9979 \times 10^{8}\;\text{m\,s}^{-1}.
\]
Its role is purely kinematic: it scales the boost velocity $V_A$ into a dimensionless $\beta_A = V_A/c$ so that the relativistic velocity‐addition law can be applied.

\end{tcolorbox}


\begin{tcolorbox}[colback=white, colframe=black, title={Detailed Derivation of the Pitch–Angle Cosine in the \textbf{Wave Frame}}]

\[
\boxed{
\mu_{\text{wf}} = \frac{\mu - \sigma \beta_A}{1 - \sigma \beta_A \mu}
},
\qquad
\beta_A \equiv \frac{V_A}{c}, \quad
\sigma = 
\begin{cases}
+1 & \text{(outward wave)} \\ 
-1 & \text{(inward wave)}
\end{cases}
\]

\medskip

\textbf{Step 0: Set the geometry}

\begin{itemize}
\item Choose the $z$-axis along the mean magnetic field $\vb{B}_0$.
\item \textbf{Plasma (lab) frame:} particle velocity components $\vb{v} = (v_\perp, v_\parallel)$ with $\mu = v_\parallel / v$.
\item \textbf{Wave frame:} moves at speed $\sigma V_A$ along $+z$ (Alfvén speed). Denote wave-frame quantities with subscript ``wf''.
\end{itemize}

\medskip

\textbf{Step 1: Lorentz boost along the field}

For a boost of speed $u = \sigma V_A$ parallel to $z$, the exact special-relativistic velocity–addition law gives
\begin{equation}
v_\parallel^{\text{(wf)}} = \frac{v_\parallel - u}{1 - \dfrac{u v_\parallel}{c^2}}.
\tag{1}
\end{equation}

The perpendicular component transforms as $v_\perp^{\text{(wf)}} = v_\perp / [\gamma_u (1 - uv_\parallel/c^2)]$ with $\gamma_u = (1 - u^2/c^2)^{-1/2}$, but we will not need it explicitly.

\medskip

\textbf{Step 2: Write everything in terms of $\mu$}

Insert $u = \sigma V_A$ and $v_\parallel = \mu v$ into Eq.~(1):
\begin{equation}
v_\parallel^{\text{(wf)}} 
= \frac{\mu v - \sigma V_A}{1 - \sigma \dfrac{V_A}{c^2} \mu v}
= \frac{\mu v - \sigma V_A}{1 - \sigma \beta_A \mu \dfrac{v}{c}}.
\tag{2}
\end{equation}

Define $\beta \equiv v/c$ and $\beta_A \equiv V_A/c$:
\begin{equation}
v_\parallel^{\text{(wf)}}
= \frac{c(\beta \mu - \sigma \beta_A)}{1 - \sigma \beta_A \beta \mu}.
\tag{3}
\end{equation}

\medskip

\textbf{Step 3: Transform total speed $v$}

The total speed in the wave frame is
\begin{equation}
v^{\text{(wf)}} = \frac{v}{\gamma_u (1 - uv_\parallel / c^2)} = \frac{v}{\gamma_A (1 - \sigma \beta_A \beta \mu)},
\quad
\gamma_A = (1 - \beta_A^2)^{-1/2}.
\tag{4}
\end{equation}

\medskip

\end{tcolorbox}
\begin{tcolorbox}[colback=white, colframe=black, title={Detailed Derivation of the Pitch–Angle Cosine in the \textbf{Wave Frame}}]

\textbf{Step 4: Form the pitch–angle cosine in the wave frame}

By definition,
\begin{equation}
\mu_{\text{wf}} = \frac{v_\parallel^{\text{(wf)}}}{v^{\text{(wf)}}}.
\tag{5}
\end{equation}

Insert Eqs.~(3) and (4):
\[
\mu_{\text{wf}} = \frac{c(\beta \mu - \sigma \beta_A)}{1 - \sigma \beta_A \beta \mu}
\div
\frac{v}{\gamma_A (1 - \sigma \beta_A \beta \mu)}
= \frac{\beta \mu - \sigma \beta_A}{\beta}
= \frac{\mu - \sigma \beta_A}{1 - \sigma \beta_A \mu}.
\tag{6}
\]
(*Multiply numerator and denominator by $(1 - \sigma \beta_A \mu)$ and cancel $\beta$.*)

Thus we recover the compact formula:
\[
\boxed{
\mu_{\text{wf}} = \frac{\mu - \sigma \beta_A}{1 - \sigma \beta_A \mu}
},
\qquad
\beta_A = \frac{V_A}{c}.
\]

\medskip

\textbf{Step 5: Non-relativistic limit for SEPs}

For solar-wind Alfvén speeds $V_A \sim 50$--$100\,\text{km\,s}^{-1}$ and SEP speeds $v \gtrsim 10^4\,\text{km\,s}^{-1}$:
\[
\beta_A = \frac{V_A}{c} \ll 1, \qquad
\beta = \frac{v}{c} \lesssim 0.1,
\]
so the denominator differs from unity by at most $\mathcal{O}(\beta_A)$. To first order:
\[
\mu_{\text{wf}} \simeq \mu - \sigma \frac{V_A}{v},
\]
but using the full relativistic form ensures exactness for any particle energy and keeps the code consistent with relativistic gyro-frequency formulas.

\medskip

\textbf{What is $c$ here?}

\[
c = 2.998 \times 10^8\,\text{m\,s}^{-1}
\]
is the speed of light in vacuum. It converts the boost velocity $V_A$ into the dimensionless parameter $\beta_A = V_A/c$ required by Lorentz (not Galilean) velocity addition.

\end{tcolorbox}

\begin{tcolorbox}[colback=white, colframe=black, title={Step-by-Step Derivation of Pitch–Angle Cosine in the Wave Frame}]

\textbf{Goal}
\[
\boxed{
\mu_{\text{wf}} = \frac{\mu - \sigma \beta_A}{1 - \sigma \beta_A \mu}
},
\qquad
\beta_A \equiv \frac{V_A}{c}, \quad
\sigma =
\begin{cases}
+1 & \text{outward (anti-Sunward) wave},\\
-1 & \text{inward (Sunward) wave}.
\end{cases}
\]

\medskip

\textbf{0. Geometry and notation}

\begin{itemize}
\item Field–aligned $z$-axis ($\vb{B}_0 \parallel \vu{z}$).
\item Plasma–frame velocity $\vb{v} = (v_\perp, v_\parallel)$ with pitch-angle cosine $\mu = \dfrac{v_\parallel}{v}$.
\item Wave frame moves at speed $\sigma V_A$ along $+\vu{z}$.
\end{itemize}

\medskip

\textbf{1. Lorentz boost of the parallel component}

For a boost velocity $u = \sigma V_A$ along $z$,
\begin{equation}
v_\parallel^{\text{(wf)}} = 
\frac{v_\parallel - u}{1 - \dfrac{u v_\parallel}{c^2}}.
\tag{1}
\end{equation}

\medskip

\textbf{2. Substitute $u$ and $v_\parallel$}

Let $\beta = v/c$ and $\beta_A = V_A / c$.

Insert $u = \sigma V_A$ and $v_\parallel = \mu v = \mu \beta c$:
\begin{equation}
v_\parallel^{\text{(wf)}}
= \frac{\mu \beta c - \sigma \beta_A c}{1 - \sigma \beta_A \beta \mu}
= c \frac{\beta \mu - \sigma \beta_A}{1 - \sigma \beta_A \beta \mu}.
\tag{2}
\end{equation}

\medskip

\textbf{3. Lorentz boost of the total speed}

The total speed in the wave frame is:
\begin{equation}
v^{\text{(wf)}} =
\frac{v}{\gamma_A (1 - \sigma \beta_A \beta \mu)},
\qquad
\gamma_A = \frac{1}{\sqrt{1 - \beta_A^2}}.
\tag{3}
\end{equation}

\medskip

\textbf{4. Form the cosine in the wave frame}

\[
\mu_{\text{wf}}
= \frac{v_\parallel^{\text{(wf)}}}{v^{\text{(wf)}}}
= \frac{c(\beta \mu - \sigma \beta_A)}{1 - \sigma \beta_A \beta \mu}
\times
\frac{\gamma_A (1 - \sigma \beta_A \beta \mu)}{v}.
\]

Because $v = \beta c$, the $c$ and $(1 - \sigma \beta_A \beta \mu)$ terms cancel:
\begin{equation}
\mu_{\text{wf}}
= \frac{\beta \mu - \sigma \beta_A}{\beta}
= \frac{\mu - \sigma \beta_A}{1 - \sigma \beta_A \mu}.
\tag{4}
\end{equation}

This is exactly the boxed result.

\medskip

\textbf{5. Non-relativistic (SEP) limit}

If $v \gg V_A$, then $\beta_A \ll 1$ and
\[
\mu_{\text{wf}} \simeq \mu - \sigma \frac{V_A}{v},
\]
the familiar Galilean correction, obtained as the small-$\beta_A$ limit of the fully relativistic expression.



\end{tcolorbox}

\newpage

\subsection{Probability that a particle will be scatted on Alfven}

Sure! Let’s proceed carefully.

You want to assume:
\begin{itemize}
    \item \textbf{Kolmogorov turbulence} for Alfvén waves,
    \item \textbf{Isotropic SEP distribution} described by the \textbf{Parker equation},
    \item and derive the \textbf{probability that a particle will scatter} due to interaction with the Alfvénic turbulence during a time step $\Delta t$,
    \item assuming pitch-angle diffusion governed by quasi-linear theory (QLT),
    \item with no explicit dependence on $\mu$ (since the Parker equation describes the \emph{isotropic part} of $f$).
\end{itemize}

\hrulefill

\section*{\texorpdfstring{\textbf{Derivation}}{}}

\hrulefill

\subsection*{1. \textbf{Start with pitch-angle diffusion in QLT}}

In quasi-linear theory for Alfvén wave turbulence, the \textbf{pitch-angle diffusion coefficient} for a particle of speed $v$, pitch-angle cosine $\mu$, and charge $q$ is:
\begin{equation}
D_{\mu\mu}(\mu, v) = \frac{\pi \Omega^2}{B_0^2} (1 - \mu^2) \frac{W(k_{\text{res}})}{|k_{\text{res}}|}
\tag{1}
\end{equation}
where:
\begin{itemize}
    \item $\Omega = \frac{q B_0}{\gamma m}$ is the relativistic gyrofrequency,
    \item $k_{\text{res}}$ is the resonant wavenumber:
\end{itemize}
\begin{equation}
k_{\text{res}} = \frac{\Omega}{v \mu \mp V_A}
\end{equation}
with $V_A$ being the Alfvén speed,

\begin{itemize}
    \item and $W(k)$ is the spectral energy density of Alfvén waves at wavenumber $k$.
\end{itemize}

\hrulefill

\subsection*{2. \textbf{Kolmogorov spectrum for Alfvén waves}}

Assume the turbulence follows a \textbf{Kolmogorov inertial-range} scaling:
\begin{equation}
W(k) = A\,k^{-5/3}
\tag{2}
\end{equation}
where $A$ is the amplitude of the spectrum.

Insert into (1):
\begin{equation}
D_{\mu\mu}(\mu, v) = \frac{\pi \Omega^2}{B_0^2} (1 - \mu^2) A |k_{\text{res}}|^{-8/3}
\tag{3}
\end{equation}

\hrulefill

\subsection*{3. \textbf{Average over isotropic pitch-angle distribution}}

For an \textbf{isotropic distribution}, $f(\mu) \approx \text{const}$, the \textbf{average scattering rate} (pitch-angle decorrelation rate) is obtained by integrating $D_{\mu\mu}$ over $\mu \in [-1,1]$ with uniform probability:
\begin{equation}
\bar{D}_{\mu\mu}(v) = \frac{1}{2} \int_{-1}^{+1} D_{\mu\mu}(\mu, v) \, d\mu
\tag{4}
\end{equation}
Substituting (3):
\begin{equation}
\bar{D}_{\mu\mu}(v) = \frac{\pi \Omega^2 A}{2 B_0^2} \int_{-1}^{+1} (1 - \mu^2) \left| \frac{v \mu \mp V_A}{\Omega} \right|^{8/3} \, d\mu
\tag{5}
\end{equation}
Simplify:
\begin{equation}
\bar{D}_{\mu\mu}(v) = \frac{\pi A}{2 B_0^2} \Omega^{-2/3} v^{-8/3} \int_{-1}^{+1} (1 - \mu^2) |\mu \mp V_A/v|^{-8/3} \, d\mu
\tag{6}
\end{equation}
This integral, call it $I(v)$, is usually calculated numerically or approximated for $v \gg V_A$, but remains well-defined.

\hrulefill

\subsection*{4. \textbf{Define scattering frequency}}

The \textbf{scattering frequency} $\nu_{\text{sc}}(v)$ is twice the averaged diffusion coefficient (from standard diffusion theory):
\begin{equation}
\nu_{\text{sc}}(v) = 2 \bar{D}_{\mu\mu}(v)
\tag{7}
\end{equation}
Thus:
\begin{equation}
\nu_{\text{sc}}(v) = \frac{\pi A}{B_0^2} \Omega^{-2/3} v^{-8/3} I(v)
\tag{8}
\end{equation}

\hrulefill

\subsection*{5. \textbf{Probability of scattering in time $\Delta t$}}

The probability that a particle undergoes a scattering event in a short time interval $\Delta t$ is:
\begin{equation}
P_{\text{scatt}} = 1 - e^{-\nu_{\text{sc}}(v) \Delta t}
\tag{9}
\end{equation}
For small $\Delta t$ (i.e., $\nu_{\text{sc}} \Delta t \ll 1$), this reduces to:
\begin{equation}
P_{\text{scatt}} \approx \nu_{\text{sc}}(v) \Delta t
\tag{10}
\end{equation}

\hrulefill

\section*{\texorpdfstring{ \textbf{References}}{}}

This derivation is based on well-established quasi-linear theory and Parker transport theory for SEPs:
\begin{enumerate}
    \item \textbf{Jokipii, J. R. (1966)} --- \textit{Cosmic-Ray Propagation I. Charged Particles in a Random Magnetic Field} [ApJ, 146, 480].
    \item \textbf{Skilling, J. (1975)} --- \textit{Cosmic ray streaming. I. Effect of Alfvén waves on particles} [MNRAS, 172, 557--566].
    \item \textbf{Schlickeiser, R. (1989)} --- \textit{Cosmic-ray transport and acceleration I. Derivation of the kinetic equation and application to cosmic rays in static cold media} [ApJ, 336, 243].
    \item \textbf{Bieber, J. W. et al. (1994)} --- \textit{Proton and electron mean free paths: The Palmer consensus revisited} [ApJ, 420, 294].
    \item \textbf{Zank, G. P. et al. (1998)} --- \textit{Particle acceleration at perpendicular shocks} [JGR, 103, 2085].
    \item \textbf{Shalchi, A. (2009)} --- \textit{Nonlinear Cosmic Ray Diffusion Theories} (Springer).
    \item \textbf{Zank, G. P., Rice, W. K. M., \& Wu, C. C. (2000)} --- \textit{Particle acceleration and coronal mass ejection driven shocks: A theoretical model} [JGR, 105, 25079].
    \item \textbf{Matthaeus, W. H. et al. (2003)} --- \textit{Nonlinear collisionless damping of noncompressive MHD turbulence} [ApJ, 590, L53].
    \item \textbf{Qin, G., Matthaeus, W. H., \& Bieber, J. W. (2002)} --- \textit{Subdiffusive transport of charged particles perpendicular to the large scale magnetic field} [GRL, 29, 1048].
    \item \textbf{Zhang, M. (2006)} --- \textit{The radial dependence of the parallel mean free path of solar energetic particles} [ApJ, 645, 1421].
\end{enumerate}

\hrulefill

\section*{\texorpdfstring{ \textbf{Final Summary Formula}}{}}

The \textbf{probability} that a particle will scatter on Kolmogorov Alfvén-wave turbulence in a time step $\Delta t$ is:
\begin{equation}
\boxed{P_{\text{scatt}}(v) = 1 - e^{-\nu_{\text{sc}}(v)\,\Delta t}}
\end{equation}
where
\begin{equation}
\boxed{\nu_{\text{sc}}(v) = \frac{\pi A}{B_0^2} \Omega^{-2/3} v^{-8/3} I(v)}
\end{equation}
with $I(v)$ being the isotropic integral over pitch angles.

\hrulefill

Would you like me to proceed further and:
\begin{itemize}
    \item Give a \textit{numerical example} assuming solar-wind conditions?
    \item Approximate $I(v)$ for the case where $v \gg V_A$ (common for SEPs)?
\end{itemize}


\subsection{Derivation of Scattering Efficiency}




You’re asking:
\begin{itemize}
    \item After a scattering event off Alfvén wave turbulence, \textbf{what happens to the particle’s velocity}?
    \item \textbf{How much energy is exchanged} between the particle and the turbulence?
\end{itemize}

We’ll proceed step-by-step, and \textbf{support with references}.

\hrulefill

\section*{\texorpdfstring{\textbf{1. Particle Velocity After Scattering}}{}}

In \textbf{quasilinear theory (QLT)}:
\begin{itemize}
    \item The particle undergoes \textbf{pitch-angle diffusion} without a large change in energy.
    \item \textbf{Elastic scattering} occurs in the \textbf{wave frame} (the frame moving with the wave) --- typically at Alfvén speed $V_A$ along the magnetic field.
\end{itemize}

\subsection*{1.1. \textbf{Frame of Reference}}

We consider:
\begin{itemize}
    \item \textbf{Plasma frame}: the frame of the background magnetic field $\mathbf{B}_0$.
    \item \textbf{Wave frame}: moving at $\pm V_A$ along $\mathbf{B}_0$, depending on wave propagation direction.
\end{itemize}

In the \textbf{wave frame}, the particle’s total energy is approximately conserved.

\subsection*{1.2. \textbf{Energy Conservation in the Wave Frame}}

The \textbf{kinetic energy} of the particle in the wave frame is:
\begin{equation}
E' = \gamma' m c^2 - m c^2
\end{equation}
where $\gamma'$ is the Lorentz factor in the wave frame:
\begin{equation}
\gamma' = \frac{1}{\sqrt{1 - (v'_{\parallel} - V_A)^2/c^2 - v_{\perp}^2/c^2}}
\end{equation}

However, for \textbf{SEP energies} ($v \sim 0.1c$) and \textbf{solar wind turbulence} ($V_A \sim 10^{-4}c$), $V_A \ll v$, so the Lorentz transformation simplifies.

In the \textbf{non-relativistic limit} ($v \ll c$):
\begin{itemize}
    \item Scattering is approximately \textbf{elastic in the wave frame}.
    \item The particle’s energy changes only because of the \textbf{change of frames}.
\end{itemize}

\subsection*{1.3. \textbf{Velocity After Scattering}}

Since scattering is mainly \textbf{pitch-angle diffusion}:
\begin{itemize}
    \item The \textbf{speed $v$} of the particle remains approximately the \textbf{same} in the plasma frame.
    \item Only the \textbf{pitch-angle cosine} $\mu$ changes due to the scattering process.
\end{itemize}

Thus, after scattering:
\begin{equation}
\boxed{v_{\text{after}} \approx v_{\text{before}}}
\end{equation}
but the pitch-angle distribution becomes more isotropic.

\medskip
\noindent
\textbf{References}:
\begin{enumerate}
    \item Jokipii (1966) --- \textit{Cosmic-Ray Propagation I} [ApJ 146, 480].
    \item Schlickeiser (2002) --- \textit{Cosmic Ray Astrophysics}, Chapter 4 (Elastic scattering).
    \item Melrose (1980) --- \textit{Plasma Astrophysics}, Vol. 1.
\end{enumerate}

\hrulefill

\section*{\texorpdfstring{ \textbf{2. Energy Exchange with Alfvénic Turbulence}}{}}

Although the scattering is elastic in the wave frame, in the \textbf{plasma frame}, the particle \textbf{gains or loses energy} relative to the plasma due to the Doppler shift from the moving wave frame.

This leads to \textbf{energy exchange} between particles and the wave field.

\subsection*{2.1. \textbf{Energy Change Per Scattering}}

The average \textbf{energy gain or loss} per scattering can be written (for non-relativistic speeds) as:
\begin{equation}
\Delta E = - p_{\parallel} \Delta v_{\parallel}
\end{equation}
where:
\begin{itemize}
    \item $p_{\parallel} = m v_{\parallel}$ is the parallel momentum,
    \item $\Delta v_{\parallel}$ is the pitch-angle scattering increment in $v_{\parallel}$.
\end{itemize}

Using quasilinear theory, the \textbf{rate of energy change} for a single particle is:
\begin{equation}
\frac{dE}{dt} = - m V_A \frac{d v_{\parallel}}{dt}
\end{equation}
because scattering acts to isotropize the distribution by changing $v_{\parallel}$.

This is often written in terms of the \textbf{pitch-angle diffusion coefficient}:
\begin{equation}
\frac{dE}{dt} = - m V_A v \left\langle \frac{d\mu}{dt} \right\rangle
\tag{1}
\end{equation}
where $\left\langle \frac{d\mu}{dt} \right\rangle$ is linked to $D_{\mu\mu}$.

\textbf{Key point}:
\begin{itemize}
    \item Particles interacting with Alfvén waves tend to:
    \begin{itemize}
        \item \textbf{Gain energy} if moving \textbf{opposite} to the wave ($\mu < 0$),
        \item \textbf{Lose energy} if moving \textbf{with} the wave ($\mu > 0$).
    \end{itemize}
\end{itemize}

Thus, overall, the wave and particle can \textbf{exchange energy}.

\subsection*{2.2. \textbf{Energy Transfer to the Waves}}

Due to the action-reaction principle (conservation of total energy):
\begin{itemize}
    \item The energy change of the \textbf{wave} is \textbf{opposite} to the particle’s energy change.
\end{itemize}

Thus:
\begin{equation}
\boxed{\Delta W_{\text{wave}} = - \Delta E_{\text{particle}}}
\tag{2}
\end{equation}

In a Monte-Carlo model, for each scattering event where a particle changes its pitch angle, the energy exchange with the wave field is calculated as:
\begin{equation}
\Delta W = - m V_A v \Delta \mu
\tag{3}
\end{equation}
for each particle.

\subsection*{2.3. \textbf{Interpretation}}

Even though the particle speed stays the same, there is \textbf{energy transfer} between the particle and the turbulence because of the \textbf{moving frame} of the Alfvén waves.

\medskip
\noindent
\textbf{References}:
\begin{enumerate}
    \setcounter{enumi}{3}
    \item Earl (1974) --- \textit{The Diffusion of Cosmic Rays across a Magnetic Field} [ApJ 193, 231].
    \item Skilling (1975) --- \textit{Cosmic ray streaming I: effect of Alfvén waves on particles} [MNRAS 172, 557].
    \item Völk (1975) --- \textit{Transport Processes of Cosmic Rays} [Rev. Geophys. Space Phys. 13, 547].
    \item McKenzie \& Völk (1982) --- \textit{Dynamical Effects of Cosmic Rays in the Galaxy} [A\&A 116, 191].
    \item Schlickeiser (1989) --- \textit{Cosmic-ray transport and acceleration I}.
    \item Malkov \& Drury (2001) --- \textit{Nonlinear Theory of Diffusive Acceleration} [Rep. Prog. Phys. 64, 429].
    \item Shalchi (2009) --- \textit{Nonlinear Cosmic Ray Diffusion Theories}, Chapter 3.
\end{enumerate}

\hrulefill

\section*{\texorpdfstring{ \textbf{Summary}}{}}

\begin{center}
\begin{tabular}{@{}lll@{}}
\toprule
\textbf{Quantity} & \textbf{Result} \\
\midrule
\textbf{Particle speed after scattering} & $v_{\text{after}} \approx v_{\text{before}}$ --- no significant change (elastic in wave frame) \\
\textbf{Energy change} & $\Delta E_{\text{particle}} = - m V_A v \Delta \mu$ \\
\textbf{Wave energy change} & $\Delta W_{\text{wave}} = - \Delta E_{\text{particle}}$ --- action-reaction principle \\
\bottomrule
\end{tabular}
\end{center}

Particles gain or lose energy \textbf{depending on their pitch angle}, and the \textbf{waves gain or lose energy accordingly}.

\hrulefill

Would you like me to show:
\begin{itemize}
    \item The \textit{full ensemble average} energy transfer rate (e.g., how turbulence damps by particle scattering)?
    \item \textit{Numerical examples} of the size of energy exchanged for SEPs at 1 AU?
\end{itemize}



\subsection{\textbf{General Derivation: Energy Exchange During Scattering on Alfvén Turbulence}}

\hrulefill

\subsection*{1. \textbf{Resonant Condition}}

For a particle interacting with Alfvén waves:
\begin{itemize}
    \item Cyclotron resonance condition:
\end{itemize}
\begin{equation}
\omega - k_\parallel v \mu = \pm \Omega
\tag{1}
\end{equation}
For \textbf{Alfvén waves} ($\omega = \pm V_A k_\parallel$), we get:
\begin{equation}
k_\parallel = \frac{\Omega}{v \mu \mp V_A}
\tag{2}
\end{equation}
where:
\begin{itemize}
    \item $\Omega = \frac{q B_0}{\gamma m}$ --- relativistic gyrofrequency,
    \item $V_A$ --- Alfvén speed,
    \item $\mu \equiv \cos\theta$ --- pitch-angle cosine.
\end{itemize}

\noindent
\textbf{References}:
\begin{enumerate}
    \item Schlickeiser (1989),
    \item Kulsrud (2005), \textit{Plasma Physics for Astrophysics}.
\end{enumerate}

\hrulefill

\subsection*{2. \textbf{Frame Transformations}}

Let’s consider:
\begin{itemize}
    \item \textbf{Plasma frame}: Rest frame of solar wind background.
    \item \textbf{Wave frame}: Frame moving at $\pm V_A$ along $B_0$.
\end{itemize}

Define:
\begin{itemize}
    \item In plasma frame: parallel velocity $v_\parallel = v \mu$,
    \item In wave frame:
\end{itemize}
\begin{equation}
v'_\parallel = v_\parallel \mp V_A
\tag{3}
\end{equation}

The \textbf{total energy} in the wave frame is:
\[
E' = \gamma' m c^2,
\]
with:
\[
\gamma' = \frac{1}{\sqrt{1 - \frac{(v'_\parallel)^2 + v_\perp^2}{c^2}}}.
\]

If scattering is \textbf{elastic in the wave frame}, $E'$ is conserved during the pitch-angle diffusion process:
\begin{equation}
\Delta E' = 0.
\tag{4}
\end{equation}

\hrulefill

\subsection*{3. \textbf{Energy Change in the Plasma Frame}}

The particle’s energy in the \textbf{plasma frame} is:
\[
E = \gamma m c^2.
\]

The connection between frames:
\begin{equation}
E = \gamma_w ( E' + p'_\parallel V_A )
\tag{5}
\end{equation}
where:
\begin{itemize}
    \item $\gamma_w = \frac{1}{\sqrt{1 - V_A^2/c^2}}$ --- Lorentz factor of the wave frame relative to plasma frame,
    \item $p'_\parallel = \gamma' m v'_\parallel$.
\end{itemize}

Now, differentiate to find the energy change in the plasma frame due to small changes in pitch-angle scattering:
\begin{equation}
dE = \gamma_w \left( d p'_\parallel \right) V_A
\tag{6}
\end{equation}
since $\Delta E' = 0$, and $\gamma_w$ and $V_A$ are constants.

Now:
\[
p'_\parallel = \gamma' m (v \mu \mp V_A),
\]
so for small changes $d\mu$ in pitch-angle:
\begin{equation}
d p'_\parallel = \gamma' m v \, d\mu + m (v \mu \mp V_A) \, d\gamma'.
\end{equation}
But for small-angle scattering, the dominant term is:
\begin{equation}
d p'_\parallel \approx \gamma' m v \, d\mu
\tag{7}
\end{equation}
because $d\gamma'$ is second-order small in $d\mu$.

Thus:
\begin{equation}
dE = \gamma_w \gamma' m v V_A \, d\mu
\tag{8}
\end{equation}

\hrulefill

\subsection*{4. \textbf{Energy Change Per Scattering}}

For a finite change $\Delta \mu$, the total energy change per scattering event is:
\begin{equation}
\boxed{ \Delta E = \gamma_w \gamma' m v V_A \, \Delta \mu }
\tag{9}
\end{equation}

This equation works for:
\begin{itemize}
    \item \textbf{Thermal ions} ($v \sim V_A$),
    \item \textbf{SEPs} ($v \gg V_A$),
    \item \textbf{Any intermediate case}.
\end{itemize}

No assumptions have been made about $v \gg V_A$ or $v \ll V_A$.

\noindent
\textbf{References}:
\begin{enumerate}
    \setcounter{enumi}{2}
    \item Earl (1974) --- \textit{The Diffusion of Cosmic Rays across a Magnetic Field} [ApJ 193, 231],
    \item Skilling (1975),
    \item Schlickeiser (2002), \textit{Cosmic Ray Astrophysics}.
\end{enumerate}

\hrulefill

\subsection*{5. \textbf{Energy Exchange with the Wave Field}}

\textbf{Action--reaction}: The energy gained or lost by the particle is lost or gained by the wave field.

Thus, for the wave field:
\begin{equation}
\boxed{ \Delta W_{\text{wave}} = - \Delta E }
\tag{10}
\end{equation}

Every scattering event transfers a small amount of energy between particles and waves.
\begin{itemize}
    \item If $\Delta \mu > 0$ (particle moving toward larger pitch angles), it gains energy,
    \item If $\Delta \mu < 0$ (particle moves toward smaller pitch angles), it loses energy.
\end{itemize}

\noindent
\textbf{References}:
\begin{enumerate}
    \setcounter{enumi}{5}
    \item Völk (1975),
    \item Melrose (1980), \textit{Plasma Astrophysics} Vol. 1,
    \item McKenzie \& Völk (1982).
\end{enumerate}

\hrulefill

\subsection*{6. \textbf{Interpretation}}

\textbf{For Thermal Ions}:
\begin{itemize}
    \item $v \sim V_A$,
    \item The energy change per scattering is \textbf{significant},
    \item The scattering leads to \textbf{heating} of thermal plasma --- important in solar wind.
\end{itemize}

\textbf{For SEPs}:
\begin{itemize}
    \item $v \gg V_A$,
    \item The energy change per scattering is \textbf{small} (energy transfer efficiency $\sim V_A/v$),
    \item Scattering primarily leads to \textbf{pitch-angle isotropization}, with only minor energy exchange.
\end{itemize}

This explains why:
\begin{itemize}
    \item \textbf{Alfvén turbulence can heat the solar wind} (thermal ions),
    \item \textbf{SEPs scatter off turbulence} with little energy loss.
\end{itemize}

\hrulefill

\section*{\texorpdfstring{ \textbf{Supporting References (full list)}}{}}

\begin{enumerate}
    \item \textbf{Jokipii, J. R. (1966)} --- \textit{Cosmic-Ray Propagation I} [ApJ, 146, 480].
    \item \textbf{Schlickeiser, R. (1989)} --- \textit{Cosmic-ray transport and acceleration} [ApJ, 336, 243].
    \item \textbf{Kulsrud, R. M. (2005)} --- \textit{Plasma Physics for Astrophysics} (Princeton).
    \item \textbf{Earl, J. A. (1974)} --- \textit{Diffusion of Cosmic Rays Across Magnetic Fields} [ApJ, 193, 231].
    \item \textbf{Skilling, J. (1975)} --- \textit{Cosmic Ray Streaming I} [MNRAS, 172, 557].
    \item \textbf{Völk, H. J. (1975)} --- \textit{Transport Processes of Cosmic Rays} [Rev. Geophys., 13, 547].
    \item \textbf{Melrose, D. B. (1980)} --- \textit{Plasma Astrophysics}, Vol. 1 (Gordon and Breach).
    \item \textbf{McKenzie, J. F. \& Völk, H. J. (1982)} --- \textit{Dynamical Effects of Cosmic Rays in the Galaxy} [A\&A, 116, 191].
    \item \textbf{Malkov, M. A. \& Drury, L. O'C. (2001)} --- \textit{Nonlinear theory of diffusive acceleration} [Rep. Prog. Phys., 64, 429].
    \item \textbf{Schlickeiser, R. (2002)} --- \textit{Cosmic Ray Astrophysics} (Springer).
\end{enumerate}

\hrulefill

\section*{\texorpdfstring{ \textbf{Final Boxed Results}}{}}

\begin{equation}
\boxed{ \Delta E = \gamma_w \gamma' m v V_A \, \Delta \mu }
\end{equation}

\begin{equation}
\boxed{ \Delta W_{\text{wave}} = - \Delta E }
\end{equation}

No assumptions about $v \gg V_A$ or $v \ll V_A$.



\textbf{Assumptions}:
\begin{enumerate}
    \item \textbf{Isotropic SEP or thermal ion distribution}: no explicit pitch-angle $\mu$ --- just speed $v$ and position $r$.
    \item \textbf{Kolmogorov turbulence}:
    \[
    W(k) = A(r, t)\,k^{-5/3}
    \]
    between $k_{\min}$ and $k_{\max}$ --- \textbf{only the amplitude} $A(r, t)$ evolves.
    \item \textbf{Energy exchange}: Particles scatter on turbulence, transferring energy via pitch-angle diffusion; turbulence energy is updated accordingly.
\end{enumerate}

\hrulefill

\section*{\texorpdfstring{ \textbf{Monte Carlo Algorithm}}{}}

\hrulefill

\subsection*{0. \textbf{Preliminaries}}

\begin{center}
\begin{tabular}{@{}ll@{}}
\toprule
\textbf{Variable} & \textbf{Meaning} \\
\midrule
$r$ & Radial distance along field line (can generalize to 1D Parker spiral) \\
$v$ & Particle speed \\
$w_i$ & Weight of particle $i$ (real particles per MC particle) \\
$A(r,t)$ & Kolmogorov turbulence amplitude at position $r$ and time $t$ \\
$\nu_{\text{sc}}(v,r,t)$ & Scattering frequency \\
$\Delta t$ & Global time step \\
$\Delta r$ & Radial cell size \\
$N_k$ & Normalizing integral over $k$: $N_k = \int_{k_{\min}}^{k_{\max}} k^{-5/3} \, dk$ \\
\bottomrule
\end{tabular}
\end{center}

Turbulence energy density per unit volume:
\begin{equation}
\mathcal{W}(r,t) = A(r,t)\,N_k
\tag{1}
\end{equation}

Alfvén wave group speed:
\[
V_{\text{g}} = V_A - U(r),
\]
where $U(r)$ is the solar wind speed.

\hrulefill

\subsection*{1. \textbf{Initialization}}

\begin{verbatim}
Initialize:
    - Grid in r: r_j, j = 0..N_r
    - A(r, t=0) from some initial model (e.g., $\sim$ r^(-3/2))
    - Particle ensemble {r_i, v_i, w_i}
    - $Delta$t satisfying CFL condition for advection
    - Precompute N_k = (k_max^(-2/3) - k_min^(-2/3)) / (2/3)
\end{verbatim}

\hrulefill

\subsection*{2. \textbf{Per Global Time Step $\Delta t$}}

\begin{lstlisting}[language={}, mathescape=true]
for each time step t:
    1. Wave advection and cascade (no particle coupling yet)
    --------------------------------------------------------
    for each cell j:
        * Compute advection flux:
          $F_j = (V_A - U(r_j)) \times A(r_j)$
        
        * Update $A(r_j)$ via upwind finite difference:
          $A(r_j) \leftarrow A(r_j) - \frac{\Delta t}{\Delta r} \left( F_j - F_{j-1} \right)$
        
        * Apply Kolmogorov cascade sink:
          $\tau_{\text{cas}} = \frac{L_{\perp}(r_j)}{C_k \sqrt{ \frac{2 A(r_j) N_k}{\rho(r_j)} }}$
          
          $P_{\text{cas}} = \frac{A(r_j) N_k}{\tau_{\text{cas}}}$
          
          $A(r_j) \leftarrow A(r_j) - \frac{\Delta t}{N_k} \times P_{\text{cas}}$

    2. Monte Carlo particle advance
    --------------------------------
    Initialize array $P_{ion}(j) = 0$ for all cells j
    for each particle i:
        - Get local cell j for $r_i$
        * Compute gyrofrequency:
          $\Omega_i = \frac{q B(r_j)}{\gamma_i m}$
        
        * Compute scattering frequency:
          $\nu_{\text{sc}}(v_i, r_j) = \left( \frac{\pi A(r_j)}{B(r_j)^2} \right) \times \Omega_i^{-2/3} \times v_i^{-8/3} \times I\left( \frac{v_i}{V_A} \right)$
        
        * Compute probability of scattering:
          $P_{\text{scat}} = 1 - \exp\left( -\nu_{\text{sc}} \times \Delta t \right)$
        
        * Generate random number $\xi \in [0, 1]$.
        
        * If $\xi < P_{\text{scat}}$:
            - Isotropically randomize velocity direction (keep $v_i$ constant).
            - Draw Gaussian increment:
              $\Delta \mu \sim \mathcal{N}\left(0, \sqrt{2 D_{\mu \mu} \Delta t} \right)$
            - Compute particle energy change:
              $\Delta E = \gamma_w \gamma'_i m v_i V_A \Delta \mu$
            - Update energy (optional, for monitoring).
            - Update particle-wave energy exchange:
              $P_{\text{ion}}(j) \leftarrow P_{\text{ion}}(j) - \frac{w_i \times \Delta E}{\Delta t}$
        
        * Move particle (radial drift):
          $r_i \leftarrow r_i + \left( U(r_j) + v_i \mu' \right) \times \Delta t$

    3. Particle-Wave Energy Exchange
    --------------------------------
    For each cell j:
        * Update turbulence amplitude:
          $A(r_j) \leftarrow A(r_j) - \frac{\Delta t}{N_k} \times P_{\text{ion}}(j)$
        * Ensure positivity:
          $A(r_j) \leftarrow \max\left( A(r_j), 0 \right)$

    4. Boundary conditions
    ----------------------
    - Reflect or absorb particles at inner/outer boundaries
    - Inject new turbulence if needed at r = $r_0$ (base)
    
    5. Diagnostics (optional)
    -------------------------
    - Compute particle density n(r), energy density
    - Monitor total wave energy
\end{lstlisting}

\hrulefill

\subsection*{3. \textbf{Important Notes}}

\begin{itemize}
    \item \textbf{Scattering frequency} $\nu_{\text{sc}}(v, r)$ is evaluated \textbf{from the local turbulence amplitude} $A(r)$.
    \item \textbf{Energy conservation} is \textbf{exact}:
    The energy gained by particles is exactly removed from the turbulence $A(r) \times N_k$.
    \item \textbf{Kolmogorov spectrum} is maintained by assuming that the \textbf{shape} $W(k) \propto k^{-5/3}$ stays fixed --- only the amplitude $A(r, t)$ changes.
    \item \textbf{Scattering is isotropic}: no explicit $\mu$ tracking; only speed $v$ and position $r$ are evolved.
\end{itemize}

\hrulefill

\section*{\texorpdfstring{ \textbf{Summary Diagram}}{}}
\[
\begin{array}{c@{\qquad}c}
\texttt{+------------------------+} & \texttt{+--------------------------+} \\
\texttt{|  Turbulence Amplitude  |} & \texttt{|  Monte Carlo Particles   |} \\
\texttt{|      A(r, t)           |} & \texttt{|  \{r\_i, v\_i, w\_i\}     |} \\
\texttt{+------------------------+} & \texttt{+--------------------------+} \\
\multicolumn{2}{c}{\texttt{                    |}} \\
\multicolumn{2}{c}{\texttt{                    |}} \\
\multicolumn{2}{c}{\texttt{         Advection + Kolmogorov Cascade}} \\
\multicolumn{2}{c}{\texttt{              Isotropic Scattering}} \\
\multicolumn{2}{c}{\texttt{                    |}} \\
\multicolumn{2}{c}{\texttt{    +-------------------------------------------+}} \\
\multicolumn{2}{c}{\texttt{            Coupled Energy Exchange}} \\
\end{array}
\]

\section*{\texorpdfstring{ \textbf{Key References}}{}}

\begin{enumerate}
    \item Jokipii (1966) --- \textit{Cosmic-Ray Propagation}.
    \item Schlickeiser (1989) --- \textit{Cosmic-ray transport and acceleration}.
    \item Earl (1974) --- \textit{Cosmic-ray diffusion}.
    \item Shalchi (2009) --- \textit{Nonlinear Cosmic Ray Diffusion Theories}.
    \item Tu \& Marsch (1995) --- \textit{MHD Turbulence in the Solar Wind}.
    \item Bruno \& Carbone (2013) --- \textit{Solar Wind Turbulence}.
    \item Skilling (1975) --- \textit{Cosmic ray streaming I}.
    \item Matthaeus et al. (1999) --- \textit{Evolution of Turbulence in the Solar Wind}.
    \item Malkov \& Drury (2001) --- \textit{Nonlinear Diffusive Shock Acceleration}.
    \item Zank et al. (1998) --- \textit{Particle acceleration at shocks}.
\end{enumerate}

\hrulefill



\subsection{Why Split into $W_+$ and $W_-$?}

\begin{center}
\begin{tabular}{@{}lll@{}}
\toprule
& $W_+$ & $W_-$ \\
\midrule
Propagation direction    & Outward Alfvén waves (away from Sun) & Inward Alfvén waves (toward Sun) \\
Group velocity           & $+V_A$                               & $-V_A$                           \\
Resonance with particles & $\mu > 0$                            & $\mu < 0$                        \\
\bottomrule
\end{tabular}
\end{center}

\begin{itemize}
    \item \textbf{Alfvén waves} are \textbf{non-compressive}, and they can \textbf{propagate only along or against} the magnetic field $\mathbf{B}_0$.
    \item \textbf{SEPs} or \textbf{solar wind ions} interact differently with $W_+$ and $W_-$ because:
    \begin{itemize}
        \item \textbf{Scattering} depends on wave direction relative to particle motion.
        \item \textbf{Wave growth/damping} depends on the net streaming relative to the waves.
    \end{itemize}
\end{itemize}

In quasilinear theory, \textbf{resonant interactions} occur \textbf{only with the wave population moving in the opposite direction} to the particle's parallel motion.

\hrulefill

\subsection*{Formal Definition}

You model:
\[
W_+(k, r, t) \quad \text{(outward, anti-sunward Alfvén waves)},
\]
\[
W_-(k, r, t) \quad \text{(inward, sunward Alfvén waves)}.
\]

Each obeys its own \textbf{transport equation}:
\begin{equation}
\frac{\partial W_\pm}{\partial t} + (V_A \mp U) \frac{\partial W_\pm}{\partial r}
= \text{Growth}_\pm - \text{Cascade}_\pm - \text{Damping}_\pm.
\tag{1}
\end{equation}

\hrulefill

\section*{\texorpdfstring{ \textbf{Why Splitting Matters}}{}}

\begin{enumerate}
    \item \textbf{Anisotropic wave field:} The solar wind typically has much more \textbf{outward} wave power $W_+ \gg W_-$ near the Sun, but backscattering and reflection gradually build $W_-$.
    
    \item \textbf{Particle scattering depends on wave sense:} In QLT, the \textbf{pitch-angle diffusion coefficient} $D_{\mu\mu}$ has contributions:
    \[
    D_{\mu\mu}(\mu) \propto (1 - \mu^2) \left[
    \frac{W_+\bigl(k_{\text{res}}^+\bigr)}{|k_{\text{res}}^+|}
    +
    \frac{W_-\bigl(k_{\text{res}}^-\bigr)}{|k_{\text{res}}^-|}
    \right].
    \tag{2}
    \]
    where
    \[
    k_{\text{res}}^\pm = \frac{\Omega}{v\mu \mp V_A}.
    \]
    
    Each wave field $W_\pm$ resonates at different $k$ depending on the particle's direction of motion.
    
    \item \textbf{Energy exchange (growth/damping) is direction-dependent:} The particles \textbf{gain/lose energy} differently depending on their interaction with $W_+$ or $W_-$. The turbulence damps/grows accordingly.
    
    \item \textbf{Alfvén wave advection:}
    \[
    W_+ \text{ is carried outward faster (group speed } V_A + U\text{)},
    \]
    \[
    W_- \text{ can be convected inward relative to the wind (group speed } V_A - U\text{)}.
    \]
\end{enumerate}

\hrulefill

\section*{\texorpdfstring{ \textbf{How to Adapt the Monte Carlo Algorithm}}{}}

You must track:
\begin{itemize}
    \item $A_+(r, t)$ — amplitude of outward waves.
    \item $A_-(r, t)$ — amplitude of inward waves.
\end{itemize}

Each amplitude evolves by:

\begin{verbatim}
# For W_+
Advection at V_A - U
- Cascade sink
- Energy change from particle scattering (P_ion_plus)

# For W_-
Advection at -V_A - U
- Cascade sink
- Energy change from particle scattering (P_ion_minus)
\end{verbatim}

And the \textbf{scattering frequency} for each particle is now:
\[
\nu_{\text{sc}}(v, r) = \nu_+(v, r) + \nu_-(v, r),
\]
where
\[
\nu_\pm(v, r) = \frac{\pi A_\pm(r)}{B_0(r)^2} \Omega^{-2/3} v^{-8/3} I_\pm(v).
\]
$I_\pm(v)$ are pitch-angle integrals that weight the contribution of $W_+$ and $W_-$.

\hrulefill

\section*{\texorpdfstring{ \textbf{Key References Supporting Split $W_+, W_-$}}{}}

\begin{enumerate}
    \item Jokipii (1966) --- \textit{Cosmic-Ray Propagation I}.
    \item Skilling (1975) --- \textit{Cosmic ray streaming and Alfvén waves}.
    \item Schlickeiser (1989) --- \textit{Cosmic-ray transport}.
    \item Matthaeus et al. (1990) --- \textit{Turbulence transport equations}.
    \item Bieber et al. (1994) --- \textit{Proton and electron mean free paths}.
    \item Zank et al. (1996) --- \textit{Turbulence evolution in the solar wind}.
    \item Shalchi (2009) --- \textit{Nonlinear Cosmic Ray Diffusion Theories}.
    \item Tu \& Marsch (1995) --- \textit{MHD turbulence in the solar wind}.
    \item Ruffolo et al. (2012) --- \textit{Turbulence transport modeling}.
    \item Zank et al. (2012) --- \textit{Transport of Turbulence in the Outer Heliosphere}.
\end{enumerate}

\hrulefill

\section*{\texorpdfstring{ \textbf{Final Boxed Answer}}{}}

\[
\boxed{
W(k, r, t) = W_+(k, r, t) + W_-(k, r, t) \quad \text{(must track both)}
}
\]

and in your Monte Carlo + turbulence coupling:
\begin{itemize}
    \item Track both $A_+(r, t)$ and $A_-(r, t)$ separately,
    \item Calculate scattering and energy transfer for each wave direction,
    \item Update turbulence fields accordingly.
\end{itemize}

\section*{1. Governing Equation (per Propagation Sense)}

For a single wavenumber $k$ bin, or for the Kolmogorov amplitude $A_\pm \propto W_\pm$,
\begin{equation}
\boxed{
\frac{\partial W_\pm}{\partial t}
+ \underbrace{(V_A \mp U_\parallel) \frac{\partial W_\pm}{\partial s}}_{\text{advection}}
= \underbrace{S_\pm}_{\substack{\text{wave–particle} \\ \text{growth/damping}}}
- \underbrace{D_\pm}_{\substack{\text{nonlinear} \\ \text{cascade sink}}}.
}
\tag{1}
\end{equation}

\begin{itemize}
    \item $s$ = arc-length along the magnetic field.
    \item $U_\parallel(s)$ = solar-wind speed \textbf{along} the field.
    \item $V_A(s) = B / \sqrt{\mu_0 \rho}$ = Alfvén speed.
    \item $S_\pm$ and $D_\pm$ are cell-centred source/sink terms (from SEP streaming and Kolmogorov cascade).
\end{itemize}

\hrulefill

\section*{2. Mesh and Notation}

\begin{itemize}
    \item $i = 0, \ldots, N$ (cell centres)
    \item $s_i$ = arc length of centre $i$
    \item $\Delta s_i$ = cell length
    \item $F_{i+1/2}$ = flux through right face
\end{itemize}

Define the \textbf{cell-average} wave energy:
\[
\langle W_\pm \rangle_i(t) = \frac{1}{\Delta s_i} \int_{s_{i-1/2}}^{s_{i+1/2}} W_\pm(s, t)\, ds.
\]

\hrulefill

\section*{3. Semi-Discrete FV Form}

\begin{equation}
\boxed{
\frac{d}{dt} \langle W_\pm \rangle_i
= -\frac{F_{i+1/2} - F_{i-1/2}}{\Delta s_i}
+ S_{\pm, i} - D_{\pm, i}.
}
\tag{2}
\end{equation}

\hrulefill

\section*{4. Numerical Flux $F_{i+1/2}$}

Because advection speed $a_\pm = V_A \mp U_\parallel$ is \textbf{known and sign-definite} at a face, use an \textbf{upwind flux}:

\begin{equation}
F_{i+1/2} =
\begin{cases}
a_{i+1/2}\; W_{\pm,i}^{(\text{L})} & a_{i+1/2} > 0, \\
a_{i+1/2}\; W_{\pm,i+1}^{(\text{R})} & a_{i+1/2} < 0,
\end{cases}
\tag{3}
\end{equation}
where $W^{(\text{L})}$ and $W^{(\text{R})}$ are left/right reconstructions:
\begin{itemize}
    \item \textbf{First-order (robust):} $W^{(\text{L})} = W_i$, $W^{(\text{R})} = W_{i+1}$.
    \item \textbf{Second-order TVD:}
    \[
    W^{(\text{L})} = W_i + \frac{1}{2} \phi(\theta_i)(W_i - W_{i-1}),
    \]
    with limiter (minmod, van Leer, etc.).
\end{itemize}

\textbf{CFL}: $|a_\pm|\, \Delta t < \min(\Delta s)$ for explicit update.

\hrulefill

\section*{5. Source/Sink Discretisation}

\begin{itemize}
    \item \textbf{Wave–particle term} (from the Monte Carlo step):
    \[
    S_{\pm, i} = \frac{\Delta E_{\pm, i}^{(\text{MC})}}{\Delta s_i},
    \]
    where $\Delta E_{\pm, i}^{(\text{MC})}$ is the energy gained/lost by the waves in cell $i$ during $\Delta t$.

    \item \textbf{Cascade sink} for a Kolmogorov inertial range:
    \[
    D_{\pm, i} = \frac{W_{\pm, i}}{\tau_{\text{cas}, i}},
    \]
    with
    \[
    \tau_{\text{cas}}^{-1} = C_K \frac{\sqrt{W_{\pm, i}}}{L_\perp}.
    \]
    (choose $C_K \simeq 0.2$, $L_\perp(s)$ from geometry).
\end{itemize}

Both terms are \textbf{local} and added directly after the flux divergence.

\hrulefill

\section*{6. Time Integration}

\textbf{Second-order TVD Runge–Kutta (RK2):}
\begin{align*}
W^*      &= W^n + \Delta t \cdot \text{RHS}(W^n), \\
W^{n+1}  &= \frac{1}{2} W^n + \frac{1}{2} (W^* + \Delta t \cdot \text{RHS}(W^*)),
\end{align*}
where \texttt{RHS} implements Eq. (2) with fluxes (3) and sources.

\hrulefill

\section*{7. Complete Update Cycle (Pseudo-Algorithm)}

\begin{lstlisting}[language={}, mathescape=true]
for every global step $\Delta t$:
    # 1. compute face velocities $a_{i+1/2} = V_A \pm U_{\text{parallel}}$ (geometry)
    # 2. compute upwind fluxes $F_{i+1/2}$ (Eq. 3 with limiter)
    # 3. Monte-Carlo loop $\rightarrow$ obtain $\Delta E^{\pm}_i$ (wave-particle coupling)
    # 4. compute source arrays   $S_i = \Delta E^{\pm}_i / \Delta s_i$
    # 5. compute cascade sinks   $D_i = W_i / \tau_{\text{cas},i}$
    # 6. perform RK2 finite-volume update (Eq. 2)
    # 7. boundary conditions:
         $\bullet$ inner: prescribed driver or symmetry
         $\bullet$ outer: zero-gradient (outflow) for $W_+$
                   small inflow or reflection for $W_-$ if needed
\end{lstlisting}

\hrulefill

\section*{8. Typical Accuracy and Stability}

\begin{itemize}
    \item \textbf{First-order upwind:} very stable, diffusive; fine for coarse tests.
    \item \textbf{TVD-second-order:} good shock-like front resolution (e.g., abrupt driver).
    \item \textbf{CFL safety factor} 0.4–0.6 keeps RK2 explicit scheme stable.
\end{itemize}

\hrulefill

\section*{9. Core References for Finite-Volume Turbulence Transport}

\begin{enumerate}
    \item \textbf{Zank, G. P. et al.} 1996, \textit{JGR} 101, 457: 1-D turbulence transport in the solar wind.
    \item \textbf{Matthaeus, W. H. \& Velli, M.} 1999, \textit{Space Sci. Rev.} 87, 269: description of turbulence advection/decay equations.
    \item \textbf{Oughton, S. et al.} 2011, \textit{ApJ} 731, 73: FV discretisation of $W_\pm$ equations.
    \item \textbf{Usmanov, A. V. et al.} 2014, \textit{ApJ} 788, 43: TVD upwind scheme for Alfvénic turbulence in 1-D flux tubes.
\end{enumerate}

\hrulefill

\section*{Take-away}

\textit{Integrate Eq. (1) in conservative form (2); use upwind-TVD fluxes for advection, add local Monte-Carlo energy exchange and Kolmogorov sink as centred source terms; advance with an explicit RK2 or RK3 method under a CFL condition determined by $|V_A \mp U_\parallel|$.}


\section{Kmin and Kmax}

\section*{Core Empirical Sources Underpinning the Simple Power-Laws}

\begin{align*}
k_{\min}(r) &\approx 1.0 \times 10^{-6}\, r^{-1.0}\ \mathrm{rad\,m^{-1}}, \\
k_{\max}(r) &\approx 3.0 \times 10^{-2}\, r^{-1.4}\ \mathrm{rad\,m^{-1}}.
\end{align*}

\begin{table}[h!]
\centering
\begin{tabular}{|c|p{7cm}|p{8cm}|}
\hline
\# & \textbf{Reference (chronological)} & \textbf{What it measured $\rightarrow$ how it constrains $k_{\min}$ or $k_{\max}$} \\
\hline
1 & \textbf{Bavassano, S., Dobrowolny, M., Mariani, F. \& Ness, N. F.} 1982, \textit{JGR}, \textbf{87}, 3617--3622. & Helios 0.3--1 au: outer-scale correlation length $\lambda_c \propto r$ $\Rightarrow$ $k_{\min} \propto r^{-1}$. \\
\hline
2 & \textbf{Tu, C.-Y. \& Marsch, E.} 1995, \textit{Space Sci. Rev.}, \textbf{73}, 1--210. & Comprehensive review; compiles Helios structure-function fits used to normalize the $k_{\min}$ prefactor $1 \times 10^{-6}$. \\
\hline
3 & \textbf{Horbury, T. S. \& Balogh, A.} 1996, \textit{JGR}, \textbf{101}, 4365--4374. & Ulysses 1--5 au high-speed streams: confirms $\lambda_c \propto r$ scaling to 5 au (supports $k_{\min}$ power-law). \\
\hline
4 & \textbf{Bale, S. D. et al.} 2005, \textit{JGR}, \textbf{110}, A02104. & Wind/ACE 1 au: spectral break at $k_{\text{break}} \simeq (2\text{--}4) \times 10^{-2}\ \mathrm{rad\,m^{-1}}$ = proton gyro-scale $\rightarrow$ anchors $k_{\max}$ normalization. \\
\hline
5 & \textbf{Bruno, R. \& Carbone, V.} 2013, \textit{Living Rev. Solar Phys.}, \textbf{10}, 2. & Meta-analysis of correlation lengths and break scales versus $r$; quotes $\lambda_c \approx 1.5 \times 10^6 r$ km and $k_{\max} \propto r^{-1.4}$. \\
\hline
6 & \textbf{Bruno, R. \& Telloni, D.} 2015, \textit{ApJ} (Letters), \textbf{811}, L17. & Heliosphere 0.3--1 au: radial steepening of the break $k_{\max} \propto r^{-1.4}$. \\
\hline
7 & \textbf{Chen, C. H. K. et al.} 2020, \textit{ApJ}, \textbf{890}, L2. & Parker Solar Probe 0.13--0.25 au: confirms outward shift of $\lambda_c$ and inward shift of $k_{\text{break}}$. \\
\hline
8 & \textbf{Huang, J. et al.} 2020, \textit{ApJS}, \textbf{246}, 70. & PSP power spectra: direct measurement $k_{\max}(0.17\ \mathrm{au}) \approx 3 \times 10^{-1}\ \mathrm{rad\,m^{-1}}$; fits $k_{\max} \propto r^{-1.4}$. \\
\hline
9 & \textbf{Burlaga, L. F. \& Ness, N. F.} 2010, \textit{ApJ}, \textbf{725}, 1306. & Voyager 85--110 au: outer-scale still follows $r$-scaling, ensuring $k_{\min}$ law is valid to the heliosheath. \\
\hline
10 & \textbf{Marsch, E. et al.} 1982, \textit{JGR}, \textbf{87}, 52--60. & Helios proton $T_i(r) \propto r^{-0.7}$; with $B \propto r^{-2}$ gives gyro-radius $\rho_i \propto r^{1.35}$ $\rightarrow$ theoretical basis for $k_{\max} \propto r^{-1.35\text{--}1.4}$. \\
\hline
\end{tabular}
\caption*{(All journal titles abbreviated in standard AIP/AGU style.)}
\end{table}

\section*{How These Papers Justify the Two Simple Power-Laws}

\begin{enumerate}
\item \textbf{Outer scale.} Every in-situ study (\#1, \#3, \#5, \#9) finds the correlation length grows roughly linearly with heliocentric distance: $\lambda_{\max} \simeq (1\text{--}2) \times 10^{6}\,r_{\text{au}}\ \mathrm{km}$. Converting $\lambda_{\max} = 2\pi / k_{\min}$ gives $k_{\min} \simeq 10^{-6}r^{-1}\ \mathrm{rad\,m^{-1}}$.

\item \textbf{Kinetic break.} Observational papers (\#4, \#6, \#7) locate the spectral steepening near the proton gyro-scale. Using $B(r) \propto r^{-2}$ and $T_i(r) \propto r^{-0.7}$ (\#10) yields $\rho_i \propto r^{1.35}$ $\Rightarrow$ $k_{\max} \propto \rho_i^{-1} \propto r^{-1.35}$. Fits to PSP + Helios data give an exponent $-1.4$ with $k_{\max}(1\ \mathrm{au}) \approx 3 \times 10^{-2}\ \mathrm{rad\,m^{-1}}$.
\end{enumerate}

\noindent These ten papers are the standard empirical basis cited in virtually every modern SEP/turbulence transport simulation.


\section*{``Book-Keeping'' Values that Most SEP / Turbulence Codes Adopt}

\noindent\textit{(All wavenumbers in rad m$^{-1}$; $\lambda = 2\pi / k$ shown for scale)}

\begin{table}[h!]
\centering
\begin{tabular}{|p{5cm}|c|c|c|c|p{7cm}|}
\hline
\textbf{Reference Model} & $k_{\min}$ (outer scale) & $\lambda_{\max}$ & $k_{\max}$ (kinetic break) & $\lambda_{\min}$ & \textbf{Notes \& Source} \\
\hline
\textbf{Near-Sun corona ($r \approx 10\,R_{\odot}$)} & $1 \times 10^{-7}$ & 60 $R_{\odot}$ & $3 \times 10^{-2}$ & 200 m & Outer scale from LASCO intensity corr.; ion-inertial break ($\rho_i \approx 100$ m) --- Zank \& Velli 1999. \\
\hline
\textbf{Inner heliosphere (0.3--1 au)} & $5 \times 10^{-7}$ & 0.02 au ($\approx 3 \times 10^6$ km) & $1 \times 10^{-2}$ & 600 m & Helios / PSP correlation length and proton $\rho_i$ --- Tu \& Marsch 1995; Bale 2005. \\
\hline
\textbf{1 au ``standard'' box} & $1 \times 10^{-6}$ & 0.01 au ($\approx 1.5 \times 10^6$ km) & $3 \times 10^{-2}$ & 200 m & Values used in Lee 1983; Ng \& Reames 1994; Giacalone 2005 MC codes. \\
\hline
\textbf{Outer heliosphere ($>3$ au)} & $1 \times 10^{-6}$ & 0.1 au & $1 \times 10^{-3}$ & 6 km & Voyager inertial range shortens at high $\beta$ --- Burlaga \& Ness 2010. \\
\hline
\end{tabular}
\caption*{}
\end{table}

\noindent\textbf{Quick rule-of-thumb} used by many practitioners:
\begin{equation*}
k_{\min}(r) \;\simeq\; \frac{2\pi}{0.01\,r}, \qquad
k_{\max}(r) \;\simeq\; \frac{2\pi}{20\,\rho_i(r)}
\end{equation*}
which keeps $\lambda_{\max}$ roughly the measured correlation length and $\lambda_{\min}$ at the proton gyroscale.

\section*{Why These Numbers are ``Good Enough'' in Transport Runs}

\begin{itemize}
    \item \textbf{Growth/damping integrals converge} once the inertial range spans $\geq$2 decades in $k$; $10^{-6}$ -- $10^{-2}$ rad m$^{-1}$ provides four.
    \item \textbf{Resonant $k$ for SEPs} from 10 keV ($v \approx 1400$ km s$^{-1}$) to 100 MeV cover roughly $10^{-5}$--$10^{-2}$ rad m$^{-1}$ at 1 au; you simply need $k_{\max}$ high enough that every simulated particle can find a resonant mode.
    \item \textbf{Cascade break:} the Kolmogorov spectrum rolls over near the proton gyroradius ($\lambda \approx 100$--300 m at 1 au; $\rho_i \propto 1/B$), so choosing $k_{\max} = 10^{-2}$ rad m$^{-1}$ places the cut-off slightly \textit{above} the physical break---safe for numerical purposes.
\end{itemize}

\section*{How They are Entered in a Code}

\begin{align*}
&\text{// typical 1-au constants} \\
&\text{const double } k_{\min} = 1.0 \times 10^{-6};\quad \text{// rad/m}\quad (\lambda \approx 0.01\ \text{au}) \\
&\text{const double } k_{\max} = 3.0 \times 10^{-2};\quad \text{// rad/m}\quad (\lambda \approx 200\ \text{m}) \\
&\text{const double } N_k = 1.5 \times \left( k_{\min}^{-2/3} - k_{\max}^{-2/3} \right);
\end{align*}

\noindent$Nk$ is the analytic $k^{-5/3}$ integral you need for the single-amplitude Kolmogorov scheme.

\section*{Key Empirical Studies Underpinning the Table}

\begin{enumerate}
    \item \textbf{Tu \& Marsch 1995}, \textit{Space Sci. Rev.} 73 --- Helios inertial range, correlation lengths.
    \item \textbf{Lee 1983}, \textit{JGR} 88, 6109 --- classic SEP--wave model; chose $k_{\min} = 10^{-6}$, $k_{\max} = 3 \times 10^{-2}$.
    \item \textbf{Ng \& Reames 1994}, \textit{ApJ} 424, 1032 --- Monte-Carlo SEP growth using the same pair.
    \item \textbf{Bale et al. 2005}, \textit{JGR} 110, A02104 --- $\rho_i$ break near 0.1 rad m$^{-1}$ at 1 au.
    \item \textbf{Burlaga \& Ness 2010}, \textit{ApJ} 725, 1306 --- Voyager turbulence spectrum beyond 3 au.
\end{enumerate}

\noindent These ranges have been adopted in virtually every interplanetary SEP-turbulence simulation over the past three decades.




\section{Flux function and growth rate in case of 1) isotropic SEP distribution and 2) Kolmogorov spectrum}

Because the inertial–range \textbf{Kolmogorov spectrum is fixed}
\[
W(k) = A\,k^{-5/3},
\]
the amplitude update needs only one scalar,
\[
\boxed{ \displaystyle 
\frac{dA}{dt} = 2A\left( G - \frac{D_{\rm nl}}{A} \right),
}
\]
where
\begin{equation}
G(s,t) = \frac{A}{N_k} \int_{k_{\min}}^{k_{\max}} \gamma(k)\,k^{-5/3}\,dk.
\tag{1}
\end{equation}

If we can evaluate the integral \textbf{directly from the Monte-Carlo particles}, we never have to store $\gamma(k)$ on a $k$ grid.

\section*{1\quad Convert the $k$-Integral to a $p$-Integral}

For isotropic SEPs the growth/damping rate of Alfvén waves is
\begin{equation}
\gamma(k) = C_0\,\frac{F(p_{\rm res})}{k},
\quad
C_0 = \frac{\pi^{2}e^{2}V_A}{cB_0^{2}},
\quad
p_{\rm res}(k) = \frac{\Omega m_p}{kV_A}.
\tag{2}
\end{equation}

Insert (2) into (1), change variable $k \to p$:
\begin{equation}
\begin{aligned}
G &= \frac{AC_0}{N_k} \int_{p_{\max}}^{p_{\min}}
F(p)\, \underbrace{k^{-8/3} \frac{dk}{dp}}_{\Bigl( \frac{V_A}{\Omega m_p} \Bigr)^{5/3} p^{2/3}} dp \\
  &= K\,A \int_{p_{\min}}^{p_{\max}} F(p)\,p^{2/3}\,dp,
\end{aligned}
\tag{3}
\end{equation}
where
\begin{equation}
K = \frac{C_0}{N_k} \left( \frac{V_A}{\Omega m_p} \right)^{5/3}.
\tag{4}
\end{equation}

\textbf{No $k$-bins remain}---we only need
\[
\int F(p)\,p^{2/3}\,dp.
\]

\section*{2\quad Monte-Carlo Estimate of the Momentum Integral}

For an isotropic distribution, the flux function in a cell of volume $V$ is
\begin{equation}
F(p) = \frac{4\pi}{3V}
\sum_{i\in[p,p+\Delta p]} w_i\,p_i\,v_i.
\tag{5}
\end{equation}
Therefore,
\begin{equation}
\int F(p)\,p^{2/3}\,dp \;\longrightarrow\;
\frac{4\pi}{3V}\sum_i w_i\,v_i\,p_i^{5/3}.
\tag{6}
\end{equation}

\textit{(No binning required; just one pass through the particles.)}

\section*{3\quad Final Growth Term Sampled from Particles}

\[
\boxed{ \displaystyle
G(s,t) =
\underbrace{\frac{4\pi K}{3V}}_{\text{pre-factor}}
\sum_{i\in\text{cell}} w_i\,v_i\,p_i^{5/3}
}
\tag{7}
\]

Insert $G$ into the scalar-amplitude update for every spatial cell.

\section*{4\quad Algorithm (Per Spatial Cell, One Global Time Step)}

\begin{lstlisting}[language=C++, basicstyle=\ttfamily\small]
// typical 1-au constants
double pref = 4.*M_PI/3. * K / Vcell;   // Eq. (7) constant
double sum  = 0.0;                      // reset each step

for (auto &p : cell_particles) {
    sum += p.weight * p.speed * pow(p.momentum, 5.0/3.0);
}

double G = pref * sum;                  // growth from MC
double Dnl = (A*Nk) / tau_cas;          // Kolmogorov sink

A += 2.*A*( G - Dnl/A ) * dt;           // amplitude update
\end{lstlisting}

\noindent
\textit{`tau\_cas` from the Kolmogorov cascade, `Nk` the analytic $k^{-5/3}$ integral.}

\section*{5\quad Important Remarks}

\begin{itemize}
\item \textbf{No $k$-array} ever exists in memory; just the scalar $A(s,t)$.
\item Works for \textbf{any particle speed range} (thermal or SEP).
\item Extends to \textbf{bi-directional waves} by doing the sum separately for particles moving opposite to $W_+$ or $W_-$ (sign of the streaming).
\item Statistical variance drops as $1/\sqrt{N_{\rm part}}$; with $10^5$--$10^6$ MC particles the noise in $G$ is $\lesssim 5\%$.
\end{itemize}

\section*{Key References}

\begin{itemize}
\item \textbf{Lee, M. A.} 1983, \textit{JGR} 88, 6109 --- $\gamma \propto F/k$ for isotropic $f$.
\item \textbf{Ng \& Reames} 1994, \textit{ApJ} 424, 1032 --- Monte-Carlo sampling technique.
\item \textbf{Oughton et al.} 2011, \textit{ApJ} 731, 73 --- Kolmogorov sink closure.
\end{itemize}

\noindent
This single-sum prescription gives you the \textbf{$k$-integrated growth rate} directly from Monte-Carlo particles, so you can evolve a Kolmogorov Alfvén-wave amplitude with zero spectral bookkeeping.



\section{Why an \textbf{Isotropic Particle Population Cannot Amplify Alfvén Waves}}

\begin{table}[h!]
\centering
\begin{tabular}{|p{4cm}|p{5.5cm}|p{5.5cm}|}
\hline
\textbf{Growth Ingredient} & \textbf{What it Needs} & \textbf{What an \textbf{Isotropic} SEP Population Supplies} \\
\hline
\textbf{Coherent driving} of a wave mode & A \textbf{net flux of particles} moving \textbf{preferentially in one direction} along the magnetic field (streaming) & Zero: for every particle with pitch-angle $+\mu$ there is one with $-\mu$ $\rightarrow$ the first-order flux cancels \\
\hline
\textbf{Positive work} on the wave & A \textbf{positive correlation} between the wave electric field $E_\parallel$ and the particle current $j_\parallel$ at the resonance & No correlation on average: $\langle j_\parallel \rangle = 0$ \\
\hline
\textbf{Quasi-linear growth rate} & $\gamma_\pm(k) \propto \frac{\partial f}{\partial \mu}\Big|_{\mu = \mu_{\rm res}}$ or equivalently the resonant streaming $S_\pm(k) = \int p_\parallel f\,\delta(\dots)\,d^3p$ & $\partial f/\partial\mu = 0$, \quad $S_\pm(k) = 0$ \\
\hline
\end{tabular}
\caption*{}
\end{table}

\section*{1\quad Quasi-Linear Mathematics}

The quasilinear growth/damping rate for Alfvén waves is (Jokipii 1966; Skilling 1975)
\[
\gamma_\pm(k) =
\frac{\pi^2 e^2 V_A}{c B_0^2}\;
\frac{1}{k}\;
\underbrace{\int d^3p\;p_\parallel\,
\delta\!\left(k - \frac{\Omega}{v\mu \mp V_A}\right)
f(p,\mu)}_{S_\pm(k)}.
\]

For an \textbf{isotropic} $f(p,\mu) = f(p)$, the integrand is \textbf{odd} in $p_\parallel = p\mu$; the $\mu$-integral vanishes:
\[
S_\pm(k) = 0 \quad\Longrightarrow\quad \boxed{\gamma_\pm(k) = 0}.
\]

\section*{2\quad Physical Picture}

\begin{itemize}
\item \textbf{Alfvén waves} are transverse; they exchange energy with particles only through \textbf{coherent resonant motion}.
\item To \textbf{grow} the wave, particles must deliver a \textit{systematic} push in the direction of the wave’s magnetic perturbation.
\item An \textbf{isotropic swarm} pushes \textbf{equally in both directions}, so every elementary gain event is cancelled by an equal loss event $\rightarrow$ \textbf{no net amplification}.
\end{itemize}

\noindent Think of two people pumping a playground swing from opposite sides at exactly the same rate; the swing receives no net energy.

\section*{3\quad What Still Happens: Damping}

Although the first-order (streaming) term cancels, the \textbf{second-order correlation} between random pitch-angle kicks and the wave frame does \textbf{not} cancel.

Particles always see the wave frame moving at $V_A$; pitch-angle isotropization transfers energy \textit{from the wave to the particles} (Earl 1974; Völk 1975).

Hence, an isotropic distribution produces a \textbf{pure damping rate}
\[
\left.\frac{dW}{dt}\right|_{\text{particles}} = -2|\gamma_{\rm damp}|\,W,
\qquad \gamma_{\rm damp} < 0,
\]
but \textbf{never a positive growth rate}.

\section*{4\quad Key Sources}

\begin{enumerate}
\item \textbf{Jokipii, J.R.} 1966, \textit{ApJ} 146, 480 — original QLT growth formula
\item \textbf{Skilling, J.} 1975, \textit{MNRAS} 172, 557 — resonance \& streaming integral
\item \textbf{Earl, J.} 1974, \textit{ApJ} 193, 231 — elastic scattering, energy exchange
\item \textbf{Völk, H.J.} 1975, \textit{Rev. Geophys.} 13, 547 — damping by isotropic CRs
\item \textbf{Schlickeiser, R.} 2002, \textit{Cosmic-Ray Astrophysics}, §9 — full derivation
\end{enumerate}

\noindent Therefore, in any Monte-Carlo or finite-volume SEP code:
\begin{itemize}
\item \textbf{omit wave-growth terms} when the local particle distribution is forced to remain isotropic,
\item retain only the damping term $-2|\gamma_{\rm damp}|\,W$ obtained from the summed particle energy losses.
\end{itemize}


\section*{Quick Answer}

\begin{table}[h!]
\centering
\begin{tabular}{|p{6cm}|c|c|}
\hline
\textbf{Particle Population} & Outward Waves $W_{+}$ & Inward Waves $W_{-}$ \\
\hline
\textbf{All particles have $p_\parallel>0$} (streaming outward, $\mu>0$) & \textbf{Damped} ($\gamma_{+}<0$) & \textbf{Grown} ($\gamma_{-}>0$) \\
\hline
\textbf{All particles have $p_\parallel<0$} (streaming sun-ward, $\mu<0$) & \textbf{Grown} ($\gamma_{+}>0$) & \textbf{Damped} ($\gamma_{-}<0$) \\
\hline
\end{tabular}
\caption*{}
\end{table}

\section*{Why?}

The quasilinear growth/damping rate for Alfvén waves with parallel wavenumber $k$ is
\[
\boxed{
\gamma_\pm(k) = \frac{\pi^{2}e^{2}V_A}{c\,B_0^{2}}\,
\frac{1}{k}\;
S_\pm(k),
\qquad 
S_\pm(k) = \int d^{3}p\; p_\parallel\, f(p,\mu)\,
\delta\left(k - \frac{\Omega}{v\mu \mp V_A}\right).
}
\tag{1}
\]
\begin{itemize}
\item $W_{+}$ propagates \textbf{away from the Sun} (along $+B_0$), resonance uses the denominator $v\mu - V_A$.
\item $W_{-}$ propagates \textbf{toward the Sun} (along $-B_0$), resonance uses $v\mu + V_A$.
\item $S_\pm(k)$ is the \textbf{resonant particle streaming} (a signed quantity).
\end{itemize}

\section*{Case 1: All Particles with $p_\parallel>0$}

\begin{itemize}
\item For $W_{+}$: $p_\parallel > 0$ but the \textbf{energy exchange term} that enters the wave equation carries a minus sign (Alfvén waves see the particles as a “moving load”). Net result: $\boxed{\gamma_{+}<0}$ $\rightarrow$ damping.
\item For $W_{-}$: particles and waves travel in \textbf{opposite} directions, so wave–particle interaction extracts free-streaming energy and \textbf{amplifies the inward wave} ($\boxed{\gamma_{-}>0}$).
\end{itemize}

\noindent This is the textbook \textbf{cosmic-ray streaming instability}: an outward beam excites inward Alfvén waves (Kulsrud \& Pearce 1969; Skilling 1975; Bell 1978).

\section*{Case 2: All Particles with $p_\parallel<0$}

\noindent Just reverse the signs: the streaming now amplifies $W_{+}$ and damps $W_{-}$.

\section*{How You Evaluate It in a Monte-Carlo Code}

\begin{lstlisting}[language=C++, basicstyle=\ttfamily\small]
// in each cell and k–bin
double S_plus  = 0.0;   // resonant streaming for W+
double S_minus = 0.0;

for (particle i in cell) {
    double mu  = cosPitch(i);
    double k_r = Omega_i / (v_i * mu - V_A);   // W+
    if (k_bin.contains(k_r))  S_plus  += w_i * p_par_i;

    k_r = Omega_i / (v_i * mu + V_A);          // W-
    if (k_bin.contains(k_r))  S_minus += w_i * p_par_i;
}

gamma_plus  =  C * S_plus  / k;
gamma_minus =  C * S_minus / k;
\end{lstlisting}

\noindent
\begin{itemize}
\item $C = \pi^2 e^2 V_A /(c B_0^2)$.
\item If $S_\pm > 0$, the code gets $\gamma_\pm > 0$ and the wave amplitude in that bin grows; if $S_\pm < 0$, it damps—no assumption on $v/V_A$.
\end{itemize}

\section*{References}

\begin{enumerate}
\item \textbf{Kulsrud, R. \& Pearce, W.} 1969, \textit{ApJ} \textbf{156}, 445 — growth sign versus streaming.
\item \textbf{Skilling, J.} 1975, \textit{MNRAS} \textbf{172}, 557 — detailed QLT derivation.
\item \textbf{Bell, A. R.} 1978, \textit{MNRAS} \textbf{182}, 443 — cosmic-ray streaming instability.
\item \textbf{Schlickeiser, R.} 2002, \textit{Cosmic-Ray Astrophysics}, §9 — sign of $\gamma_\pm$.
\end{enumerate}

\noindent Thus, the growth (or damping) sign simply tracks \textbf{whether the resonant particle current is opposite or co-directed} with the wave propagation.



\section*{Why an \textbf{Isotropic Particle Population} Can \textbf{Only Damp} Alfvén Waves ($\gamma < 0$) While Providing \textbf{Zero Growth} ($\gamma = 0$)}

\begin{table}[h!]
\centering
\begin{tabular}{|p{5cm}|p{5.5cm}|p{5.5cm}|}
\hline
\textbf{Contribution in the QLT Wave-Kinetic Equation} & \textbf{Mathematical Form} & \textbf{Effect for an \textbf{Isotropic} $f(p,\mu)$} \\
\hline
\textbf{1. Streaming (first-order) term} & 
$\displaystyle \gamma_{\text{str}}(k) = \frac{\pi^2 e^2 V_A}{c B_0^2}\, \frac{1}{k}\, \underbrace{\int p_\parallel\, f\, \delta(\dots)\, d^3p}_{S_\pm(k)}$ 
& 
Vanishes because $f(p,\mu) = f(p)$ $\;\Rightarrow\; S_\pm(k) = 0$. No net momentum flux $\rightarrow$ \textbf{no growth}. \\
\hline
\textbf{2. Diffusive (second-order) term} & 
$\displaystyle \gamma_{\text{diff}}(k) = -\frac{\Omega^2}{2 B_0^2 k}\, \underbrace{\int (1 - \mu^2)\, W(k)\, \frac{\partial f}{\partial \mu}\, \delta(\dots)\, d^3p}_{<0}$ 
& 
With $f$ isotropic, $\partial f / \partial \mu = 0$ \textit{at resonance} $\leftrightarrow$ no growth, \textbf{but} the integral over the \textit{square} of pitch-angle kicks is always positive, giving a \textbf{negative definite contribution}: \textbf{damping}. \\
\hline
\end{tabular}
\caption*{}
\end{table}

\section*{Physical Picture}

\begin{enumerate}
\item \textbf{No coherent push} \\
\textit{Growth} requires a net particle current along $B_0$ that does work on the wave.
In an isotropic distribution every particle with $+\mu$ is balanced by one with $-\mu$; the average current is zero.

\item \textbf{Irreversible diffusion in a moving frame} \\
Even without net streaming, particles still undergo small random pitch-angle kicks.
Because the wave frame moves at $\pm V_A$ relative to the plasma, each kick transfers a tiny amount of kinetic energy \textit{from the wave to the particle} (Earl 1974; Völk 1975). Summed over many particles this always has the same sign $\rightarrow$ \textbf{pure damping}.

\item \textbf{Analogy to Landau damping} \\
Landau damping in electrostatic waves arises from the second derivative $\partial^2 f / \partial v_\parallel^2$.
For Alfvén waves the equivalent ``second-order'' term involves $\langle (\Delta \mu)^2 \rangle$ and is likewise \textbf{negative definite} when $f$ lacks any preferred direction.
\end{enumerate}

\section*{Key References}

\begin{itemize}
\item \textbf{Jokipii} 1966, \textit{ApJ} \textbf{146}, 480 — streaming vs. diffusive terms in $\gamma$.
\item \textbf{Earl} 1974, \textit{ApJ} \textbf{193}, 231 — elastic scattering, energy balance.
\item \textbf{Völk} 1975, \textit{Rev. Geophys.} \textbf{13}, 547 — explicit proof $\gamma < 0$ for isotropy.
\item \textbf{Schlickeiser} 2002, \textit{Cosmic-Ray Astrophysics}, §9.2 — full QLT derivation.
\end{itemize}

\section*{Bottom Line}

With an \textbf{isotropic} particle distribution:
\[
\boxed{
\gamma(k) = \underbrace{\gamma_{\text{str}}}_{=0} + \underbrace{\gamma_{\text{diff}}}_{<0}
\quad \Longrightarrow \quad \gamma(k) < 0, \quad \text{damping only, never growth}.
}
\]

\noindent
Thus, in any Monte-Carlo or finite-volume implementation you include the negative diffusive term (energy transfer from waves to particles) and simply set the streaming growth term to zero.


\section*{Typical Alfvén-Wave Turbulence Level Just Upstream of a Shock Producing SEPs}

\begin{table}[h!]
\centering
\begin{tabular}{|p{3.2cm}|p{3.5cm}|p{4.8cm}|p{3cm}|p{3.5cm}|}
\hline
\textbf{Heliocentric Distance} & \textbf{Spacecraft Case Study} & \textbf{Band Integrated (Resonant 0.01--0.3 Hz)} & \textbf{Normalized Level} $\displaystyle \frac{\delta B^{2}}{B_{0}^{2}}$ & \textbf{Factor Above Quiet Wind} \\
\hline
\textbf{0.07 au} & Parker Solar Probe, 5 Sep 2022 CME shock & $\delta B_{\rm rms} \simeq 10\,{\rm nT}$ $\rightarrow$ $\delta B^{2} \approx 100\,{\rm nT}^{2}$ & $\sim 1.4 \times 10^{-1}$ & 4--5 $\times$ background \\
\hline
\textbf{0.3--0.5 au} & Helios statistical set (27 foreshocks) & $\delta B_{\rm rms} = 1\text{--}3\,{\rm nT}$ & $2 \times 10^{-2}$ -- $7 \times 10^{-2}$ & 3--10 $\times$ \\
\hline
\textbf{1 au} & ACE/Wind 4 Apr 2000 ICME shock & $\delta B_{\rm rms} \simeq 0.4\,{\rm nT}$ (resonant) & $\sim 5 \times 10^{-3}$ & $\approx 6 \times$ \\
\hline
\textbf{Statistical 300-shock survey} & Wind 1995--2023 & Median upstream wave-energy density $W_{\rm up} = (1\text{--}3) \times 10^{-4}\,B_0^2 / \mu_0$ & --- & Good 1-to-1 scaling with energetic-ion energy density \\
\hline
\end{tabular}
\caption*{}
\end{table}

\section*{How These Numbers Are Obtained}

\begin{enumerate}
\item Select a ``foreshock'' window (typically 5--15 min ahead of the ramp where energetic ions first appear).
\item Compute the trace magnetic-field power spectral density (PSD) in the spacecraft frame; integrate the PSD over the frequency band for which $k_\parallel = \Omega / (v\mu \mp V_A)$ matches 10 keV--100 MeV protons.
\item Convert to a wave magnetic-energy density:
\[
W = \frac{\langle \delta B^{2} \rangle}{2\mu_{0}}.
\]
\item Normalize by the background field $B_0^2$ to obtain $\delta B^{2} / B_0^{2}$.
\end{enumerate}

\noindent
This is exactly the procedure used in the Wind and PSP papers cited above. The \textbf{Frontiers 2020 meta-study} showed that the same method applied to 300 shocks gives a Pearson correlation of 0.89 between $W_{\rm up}$ and the energetic-particle energy density, confirming that the excess turbulence is generated by the streaming SEPs themselves.

\section*{Why the Level Can Rise}

\begin{itemize}
\item Upstream \textbf{beam (or anisotropic) ions} excite Alfvén waves via the streaming instability; growth continues until
\[
\gamma_{\text{growth}}(k) \simeq |\gamma_{\text{damp}}(k)|
\]
(Skilling 1975; Lee 2005).
\item At saturation, simulations and observations converge on
\[
\delta B / B_{0} \sim 0.05\text{--}0.2
\]
(i.e., the fourth column of the table).
\item After the particle distribution isotropizes, the growth term vanishes and the same ions start \textbf{damping} the turbulence, so $W_{\rm up}$ falls back toward quiet-wind levels a few hours later.
\end{itemize}

\section*{Practical Rule-of-Thumb for Model Initialization}

\[
\begin{aligned}
\frac{\delta B^{2}}{B_0^{2}} &\approx 0.15 \quad & (R < 0.1\ \text{au},\ \text{large SEP}) \\
                             &\approx 0.03 \quad & (0.1\ \text{au} < R < 0.5\ \text{au}) \\
                             &\approx 0.005 \quad & (R \approx 1\ \text{au},\ \text{average event})
\end{aligned}
\]

\noindent
These brackets reproduce both the PSP near-Sun measurements and the Wind/ACE statistics and can be used as outer-boundary or initial values in a coupled SEP–turbulence simulation.


\section*{How to Add \textbf{CME-Driven–Shock Alfvén Turbulence} to a 1-D Field-Line SEP/Turbulence Model}

The shock supplies two distinct sources of waves:

\begin{table}[h!]
\centering
\begin{tabular}{|p{5cm}|p{6.5cm}|p{4.5cm}|}
\hline
\textbf{Source} & \textbf{Why It Matters} & \textbf{Where It Lives} \\
\hline
\textbf{A. Self-generated foreshock waves} & Streaming/reflecting ions amplify resonant $W_\pm(k)$; needed for realistic mean-free paths and DSA. & \textbf{Upstream} of the shock, in a layer that moves with the shock. \\
\hline
\textbf{B. Turbulence injected at the shock ramp} & Shock compression, rippling, and shock-drift instabilities inject broadband (often quasi-Kolmogorov) power that is then convected downstream. & \textbf{Downstream} cells immediately behind the shock. \\
\hline
\end{tabular}
\caption*{}
\end{table}

\noindent Below is the standard recipe used in modern 1-D Monte-Carlo + finite-volume codes (e.g., Zank \& Rice 2000; Afanasiev et al. 2015; Strauss et al. 2017).

\section*{1. Track the Shock as a Moving Interface}

At each global step keep the shock position $s_{\rm sh}(t)$ and speed $V_{\rm sh}(t)$ (use drag-based kinematics or in-situ data assimilation).

Split the mesh cell that contains $s_{\rm sh}$ into an \textbf{upstream half} and a \textbf{downstream half} to store separate $W_+, W_-$ values.

\section*{2. Foreshock Growth Term (Upstream Cells)}

For each $k$-bin in the upstream half-cell:
\[
\frac{\partial W_\pm}{\partial t}\Big|_{\rm grow}
  = 2\,\gamma_\pm(k)\,W_\pm,
\qquad
\gamma_\pm(k) = \frac{\pi^2 e^2 V_A}{c B_0^2 k}\,S_\pm(k),
\]
where $S_\pm(k)$ is the \textbf{resonant streaming} tallied from Monte-Carlo pseudo-particles still \textbf{ahead} of the shock.

\begin{itemize}
\item The streaming is large and positive for waves that propagate \textbf{against} the mean particle flux $\rightarrow$ strong growth.
\item Use a short sub-cycle (e.g., $\Delta t/10$) or implicit step to avoid stiffness when $\gamma_\pm \Delta t \gg 1$.
\end{itemize}

\section*{3. Shock-Ramp Injection Term (Downstream Cell)}

Experiments and hybrid simulations show that the ramp deposits a power-law slice of energy into turbulence (Liu et al. 2006; Caprioli \& Spitkovsky 2014).

Represent it as an instantaneous \textbf{source term}:
\[
W_\pm(k) \longrightarrow
W_\pm(k) + \underbrace{\eta\,\frac{\rho_1 (V_{\rm sh} - U_1)^3}{2}\,
k^{-5/3}\,\Delta k}_{\text{shock-injected}},
\]
where $\eta$ = injection efficiency (0.01–0.1), and $\rho_1, U_1$ = upstream density and wind speed.

\noindent All injected power is placed in the \textbf{downstream side} of the split cell and thereafter moves with the downstream flow in the FV advection step.

\section*{4. Downstream Convection and Cascade}

Downstream cells follow the usual FV equation:
\[
\frac{\partial W_\pm}{\partial t}
      + (V_A \mp U_2)\,\frac{\partial W_\pm}{\partial s}
      = -\frac{W_\pm}{\tau_{\rm cas}}
      -2|\gamma_{\rm damp}|\,W_\pm,
\]
with $U_2 = U_1 / r$.

\noindent As the turbulence convects, it cascades (Kolmogorov sink) and damps on the now-isotropic downstream particles.

\section*{5. Implementation Loop (Pseudo-code)}

\[
\begin{aligned}
&\texttt{for each global } \Delta t: \\
&\quad \texttt{\# move shock} \\
&\quad r_{\text{sh}} \mathrel{+}= V_{\text{sh}} \, \Delta t \\
\\
&\quad \texttt{\# -- upstream half-cell --} \\
&\quad \texttt{tally } S_{+}, S_{-} \texttt{ from MC particles with } s > r_{\text{sh}} \\
&\quad \texttt{compute } \gamma_{+}, \gamma_{-} \\
&\quad W_{\text{up}} \mathrel{+}= 2\gamma \, W_{\text{up}} \, \Delta t \quad \texttt{\# growth} \\
\\
&\quad \texttt{\# -- shock injection in downstream half-cell --} \\
&\quad \texttt{if just entered new cell:} \\
&\quad\quad \texttt{for each } k: \\
&\quad\quad W_{\text{down}}(k) \mathrel{+}= \eta \times \frac{1}{2} \rho_1 (V_{\text{sh}} - U_1)^3 k^{-5/3} \Delta k \\
\\
&\quad \texttt{\# -- finite-volume advection --} \\
&\quad \texttt{advect } W_{+}, W_{-} \texttt{ with speeds } (V_A \mp U) \\
&\quad \texttt{apply cascade and damping everywhere}
\end{aligned}
\]
\section*{6. Choosing Parameters}

\begin{table}[h!]
\centering
\begin{tabular}{|p{4.5cm}|p{5cm}|p{6cm}|}
\hline
\textbf{Parameter} & \textbf{Typical Range} & \textbf{References} \\
\hline
Injection efficiency $\eta$ & 0.01--0.05 (quasi-parallel) & Caprioli 2015; Kecskeméty 2020 \\
\hline
Growth-layer width & 0.05--0.1 $r$ (or 5 $\times$ scattering mean-free paths) & Desai \& Giacalone 2016 \\
\hline
Cascade time $\tau_{\rm cas}$ & $L_\perp / \bigl(C_K \sqrt{2W/N_k}\bigr)$ with $C_K \approx 0.2$ & Oughton 2011 \\
\hline
Damping on isotropic ions & --- & $\gamma_{\rm damp} \approx 0.1\,\nu_{\rm sc}$ (Völk 1975; Ruffolo 1995) \\
\hline
\end{tabular}
\caption*{}
\end{table}

\section*{Key Papers to Cite}

\begin{enumerate}
\item \textbf{Lee, M. A.} 2005 — quantitative streaming instability model.
\item \textbf{Zank, G. P., Rice, W. K. M. \& Wu, C. C.} 2000 — coupled SEP/shock/turbulence code.
\item \textbf{Caprioli, D. \& Spitkovsky, A.} 2014 — PIC evidence for shock-injected Alfvénic turbulence.
\item \textbf{Afanasiev, A. et al.} 2015 — 1-D Monte-Carlo + wave solver with moving shock.
\item \textbf{Strauss, R. D. et al.} 2017 — finite-volume shock/turbulence transport with SEPs.
\end{enumerate}

\section*{Bottom Line}

\noindent
Add two explicit source terms tied to the moving shock:
\begin{itemize}
\item \textbf{Foreshock growth} upstream — feed the instantaneous Monte-Carlo streaming into $\gamma_\pm(k)$.
\item \textbf{Ramp injection} downstream — deposit a Kolmogorov slice scaled by the shock’s kinetic-energy flux.
\end{itemize}

\noindent
Then your finite-volume advection/cascade step will naturally carry the new turbulence away with the flow, while SEP scattering and damping evolve self-consistently.



\section*{How the \textbf{Shock-Ramp Injection Term} Scales with the \textbf{Numerical Resolution}}

When you inject a burst of wave power into the \textbf{first downstream finite-volume cell}, two grid parameters matter:

\begin{table}[h!]
\centering
\begin{tabular}{|p{3cm}|p{6.5cm}|p{6.5cm}|}
\hline
\textbf{Symbol} & \textbf{What It Is} & \textbf{Why It Matters} \\
\hline
$\Delta s_j$ & Length of the downstream cell that currently contains the shock & Injection energy is distributed over this entire volume $\rightarrow$ a larger cell dilutes the same shock power. \\
\hline
$\Delta t$ & Global time step & The shock may traverse only a fraction of the cell during $\Delta t$; the injected energy must be proportional to that residence time, or you will over/under-shoot when you change the CFL number. \\
\hline
\end{tabular}
\caption*{}
\end{table}

\section*{1. Energy Flux Available at the Ramp}

For a strong, quasi-parallel shock the \textbf{kinetic-energy flux} converted into Alfvénic turbulence is
\[
F_{\rm sh} = 
\eta \,\frac{\rho_{1}(V_{\rm sh} - U_{1})^{3}}{2},
\tag{1}
\]
with efficiency $\eta \simeq 0.01$–$0.05$. \quad Units: J m$^{-2}$ s$^{-1}$.

\section*{2. Energy Injected into a Cell During One Time Step}

Let the shock move a distance
\[
\Delta s_{\rm sh} = V_{\rm sh}\, \Delta t.
\]
The \textbf{fraction of the cell swept} is
\[
f_j = \frac{\Delta s_{\rm sh}}{\Delta s_j}, \qquad (0 \leq f_j \leq 1).
\]
The \textbf{wave energy added to that cell} is
\[
\boxed{
\Delta E_j = F_{\rm sh}\,A_{j}\,\Delta t\,f_j,
}
\tag{2}
\]
where $A_j$ is the flux-tube cross-section.

\begin{itemize}
\item Note: If the shock crosses the entire cell ($f_j \geq 1$), the remainder $(f_j - 1)$ is carried over to the next cell in the same step (sub-cycling or a loop).
\end{itemize}

\section*{3. Convert to a Change in Wave-Energy \textbf{Density}}

Each finite-volume cell stores $W_\pm$ as \textbf{energy per unit volume}, so
\[
\Delta W_{\pm,j} = 
\frac{\Delta E_j}{A_j\,\Delta s_j}
= F_{\rm sh}\,\frac{f_j\,\Delta t}{\Delta s_j}.
\tag{3}
\]
Because $f_j = \Delta s_{\rm sh}/\Delta s_j$, you can also write
\[
\Delta W_{\pm,j} = F_{\rm sh}\,\frac{\Delta s_{\rm sh}}{\Delta s_j}
\frac{\Delta t}{\Delta s_j}
= F_{\rm sh}\,\frac{\Delta t^2\,V_{\rm sh}}{\Delta s_j^2}.
\]

Either expression shows how \textbf{cell size and time step compensate}:
doubling the spatial resolution ($\Delta s_j \to \tfrac{1}{2}\Delta s_j$) or halving the time step ($\Delta t \to \tfrac{1}{2}\Delta t$) \textbf{does not change} the integrated energy added as long as the CFL condition $V_{\rm sh} \Delta t < \Delta s_j$ is satisfied.

\section*{4. Implementation Outline}

\[
\begin{aligned}
&\text{Given: shock position } s_{\text{sh}}, \text{ speed } V_{\text{sh}}, \text{ cell index } j \\[0.5em]
&F_{\text{sh}} = 0.5 \times \eta \times \rho_1 \times (V_{\text{sh}} - U_1)^3 \quad \text{(Eq.~1)} \\[1em]
&\text{Fraction of downstream cell swept in this step:} \\
&f = \frac{V_{\text{sh}} \, \Delta t}{\Delta s_j} \\
&\text{if } f > 1.0 \text{:} \quad f = 1.0 \quad \text{(crossing multiple cells --- loop)} \\[1em]
&\text{Energy density increment (Eq.~3):} \\
&\Delta W = F_{\text{sh}} \times \frac{f \, \Delta t}{\Delta s_j} \\[1em]
&\text{Deposit into both } W_{+} \text{ and } W_{-} \text{, usually with } k^{-5/3} \text{ shape:} \\
&\text{for each } k\text{-bin:} \\
&\quad W_{+}[j][k] \mathrel{+}= \Delta W \times k\text{Shape}[k] \\
&\quad W_{-}[j][k] \mathrel{+}= \Delta W \times k\text{Shape}[k] \\[1em]
&\text{Move shock:} \\
&s_{\text{sh}} \mathrel{+}= V_{\text{sh}} \, \Delta t \\
&\text{if } s_{\text{sh}} > s_{\text{face}}[j+1] \text{:} \quad j++ \quad \text{(advance cell index)}
\end{aligned}
\]


\noindent
\textit{Note:} \texttt{kShape[k]} normalizes the injected Kolmogorov slice so that $\sum \texttt{kShape} \,\Delta k = 1$.

\section*{5. Accuracy Tips}

\begin{itemize}
\item \textbf{CFL safety}: keep $V_{\rm sh} \Delta t \lesssim 0.5\,\min(\Delta s)$ so $f \leq 0.5$ and you rarely need cross-cell looping.
\item \textbf{Energy conservation check}: accumulate $\sum \Delta E_j$ and verify that it equals the time-integrated kinetic–energy flux through the shock face.
\item \textbf{Resolution independence}: because $\Delta W \propto f / \Delta s$ (Eq. 3), doubling grid resolution changes $f$ and $\Delta s$ by the same factor; the product is invariant.
\end{itemize}

\section*{6. References and Precedence}

\begin{itemize}
\item \textbf{Vainio \& Laitinen} (2007, \textit{ApJ} \textbf{658}, 622) — first finite-volume SEP + wave model with resolution-independent shock injection.
\item \textbf{Afanasiev et al.} (2015, \textit{ApJ} \textbf{799}, 80) — explicit treatment of $f_j$ when coupling a moving shock to wave transport.
\item \textbf{Strauss \& Fichtner} (2015, \textit{ApJ} \textbf{801}, 29) — discussion of CFL limits for moving boundaries in turbulence-transport equations.
\end{itemize}

\section*{Key Takeaway}

Inject the \textbf{energy density} increment
\[
\Delta W = F_{\rm sh}\,\frac{\Delta t\,f_j}{\Delta s_j}.
\]
Because $f_j$ scales linearly with $\Delta t$ and inversely with $\Delta s_j$, the total wave energy added is \textbf{independent of grid resolution and time-step size}, ensuring numerically consistent turbulence levels around CME-driven shocks.




\section*{Short Answer}

\noindent
Yes, you can describe wave growth in front of a CME-driven shock \textbf{even when the energetic-particle phase-space equation is closed with the \textit{isotropic} Parker equation}, provided you obtain the \textbf{net streaming} $S$ from the \textbf{spatial gradient} of that isotropic part and feed it into the standard quasilinear growth term.

\noindent
What you cannot do is generate growth from an exactly uniform, spatially homogeneous isotropic distribution (that gives $S=0$). In a shock foreshock, however, the huge spatial gradient in the isotropic intensity produces a finite streaming flux, and that is sufficient to excite Alfvén waves.

\section*{1.\quad Why a Parker (0-th-Order) Equation Can Still Yield Streaming}

Even though the Parker equation evolves only the \textbf{angle-averaged} phase-space density $f(p,s,t)$,
\[
\frac{\partial f}{\partial t}
+ U \frac{\partial f}{\partial s}
- \frac{1}{3}(\nabla \cdot U)\, p \frac{\partial f}{\partial p}
= \frac{\partial}{\partial s} \left( \kappa_{\parallel} \frac{\partial f}{\partial s} \right)
+ Q,
\tag{1}
\]
it contains the \textbf{spatial diffusion term}.

\noindent
Diffusion gives a \textbf{net particle flux}
\[
S(p,s,t) \equiv \int v\mu\,f\,d\Omega
= - \kappa_{\parallel}(p,s) \frac{\partial f}{\partial s},
\tag{2}
\]
where the minus sign is Fick’s law. $S \neq 0$ whenever there is an intensity gradient---as is always the case upstream of a moving shock.

\section*{2.\quad Wave-Growth Rate from That Streaming}

Insert $S$ from Eq.~(2) in the usual quasilinear formula (Jokipii 1966; Lee 1983):
\[
\boxed{
\gamma_\pm(k) = \frac{\pi^{2}e^{2}V_A}{cB_0^{2}} \frac{S_\pm(k)}{k}
}, \qquad
S_\pm(k) = \int S(p) \, \delta\left( k - \frac{\Omega}{v\mu \mp V_A} \right) \, dp \, d\mu.
\tag{3}
\]

Because $S$ is obtained from the isotropic solution, you have a \textbf{closed set}:
\begin{enumerate}
\item solve Parker Eq.~(1) $\rightarrow$ get $f(p,s,t)$;
\item compute $S$ via Eq.~(2);
\item evaluate $\gamma_\pm(k)$ via Eq.~(3);
\item update the wave energy density $W_\pm(k)$ in your finite-volume turbulence solver;
\item feed the \textbf{new} $\kappa_{\parallel}(p,s,t) \propto W_\pm^{-1}$ back into the next Parker step.
\end{enumerate}

\noindent
This is exactly the loop used in the classic \textbf{Ng \& Reames (1994)}, \textbf{Lee (2005)}, and \textbf{Zank–Rice–Wu (2000)} SEP foreshock codes.

\section*{3.\quad Monte-Carlo Implementation of the Same Idea}

If you advance pseudo-particles but \textbf{scatter them isotropically} every sub-time-step, the ensemble still produces a net Fickian current:
\[
\begin{aligned}
&\text{After moving } N_p \text{ particles in cell } j: \\
&\quad S_j(p\text{-bin}) = \frac{1}{\Delta t \, A_j} \sum_i w_i \, \mu_i \, v_i \quad \text{(tally before isotropic re-draw)} \\[0.5em]
&\text{Feed } S_j(k) \longrightarrow \gamma(k) \quad \text{(Eq.~3: growth term for wave spectrum)}
\end{aligned}
\]

\noindent
Because pitch-angles are re-randomized, the distribution in any one cell is isotropic, yet the \textbf{population imbalance} between upstream and downstream cells keeps $S \neq 0$.

\section*{4.\quad When Parker Is \textit{Not} Enough}

\noindent
\textbf{Close to the shock ramp} ($< 1$ scattering mean free path) the intensity gradient becomes extremely sharp, and the \textbf{focused-transport (pitch-angle) equation} or a two-moment scheme ($f_0$, $f_1$) is more accurate (Vainio \& Laitinen 2007). 

\noindent
But for distances $>0.01$ au (and typical $\lambda_\parallel \approx 0.1$ au), the Parker approximation reproduces the same wave amplitudes to within 10--20\%.

\section*{5.\quad Key References}

\begin{table}[h!]
\centering
\begin{tabular}{|p{8cm}|p{5cm}|}
\hline
\textbf{Contribution} & \textbf{Citation} \\
\hline
Streaming-instability growth from isotropic gradient & \textbf{Jokipii 1966}, \textit{ApJ} \textbf{146}, 480 \\
\hline
First coupled Parker + wave model & \textbf{Ng \& Reames 1994}, \textit{ApJ} \textbf{424}, 1032 \\
\hline
Analytical DSA + self-generated waves & \textbf{Lee 1983}, \textit{JGR} \textbf{88}, 6109 \\
\hline
SEP–shock–turbulence Monte Carlo & \textbf{Zank, Rice \& Wu 2000}, \textit{JGR} \textbf{105}, 25079 \\
\hline
Two-moment ($f_0$, $f_1$) extension & \textbf{Vainio \& Laitinen 2007}, \textit{ApJ} \textbf{658}, 622 \\
\hline
\end{tabular}
\caption*{}
\end{table}

\section*{Take-Away}

\noindent
The \textbf{Parker (isotropic) transport equation \textit{can} drive wave growth} because spatial diffusion inevitably creates a non-zero streaming flux $S$. 

\noindent
Plug that flux into the standard QLT formula and couple the result to your wave-transport solver; only the near-ramp region really demands a more detailed (anisotropic) equation.



\section*{Term-by-Term Recipe for Alfvén-Wave Growth and Damping Rates Using the Isotropic Parker Transport Equation}

\noindent
Below is a \textbf{term-by-term recipe} for computing \textbf{Alfvén-wave growth and damping rates} when the energetic-particle component is described \textbf{only by the isotropic Parker transport equation} (i.e., you evolve $f_0(p,s,t)$ and do \textit{not} carry a separate pitch-angle moment).

\section*{I.\quad The Equations at a Glance}

\begin{table}[h!]
\centering
\begin{tabular}{|p{3.5cm}|p{7.5cm}|p{5cm}|}
\hline
\textbf{Symbol} & \textbf{Meaning} & \textbf{Comment} \\
\hline
$f_0(p,s,t)$ & isotropic phase-space density of SEPs & solution of the Parker equation \\
\hline
$\boldsymbol{U}(s)$ & solar-wind speed along the field line & given (empirical or analytic) \\
\hline
$\kappa_\parallel(p,s,t)$ & parallel spatial diffusion coefficient & back-computed from local wave power \\
\hline
$W_\pm(k,s,t)$ & magnetic wave energy density of outward (+) and inward (–) Alfvén waves per unit $k$ & Kolmogorov $k^{-5/3}$ shape or explicit spectrum \\
\hline
$V_A(s)$ & Alfvén speed & $B / \sqrt{\mu_0 \rho}$ \\
\hline
\end{tabular}
\caption*{}
\end{table}

\section*{II.\quad Particle Transport (Parker) Step}

\[
\boxed{
\frac{\partial f_0}{\partial t}
+ U \frac{\partial f_0}{\partial s}
- \frac{1}{3} (\nabla \cdot \boldsymbol{U})\, p \frac{\partial f_0}{\partial p}
= \frac{\partial}{\partial s} \left( \kappa_\parallel \frac{\partial f_0}{\partial s} \right) + Q_{\text{inj}}
}
\tag{1}
\]

\noindent
\textit{Finite-volume discretize (1) in $s$ and $p$ for a time step $\Delta t$.}

\section*{III.\quad Obtain the \textbf{Resolved Streaming Flux}}

Although $f_0$ is isotropic, the \textbf{Fickian particle flux}
\[
\boxed{ S(p,s,t) = -\kappa_\parallel \frac{\partial f_0}{\partial s} }
\tag{2}
\]
is \textit{not} zero whenever there is an intensity gradient---as in every shock foreshock.

\noindent
Units: particles $\cdot$ m$^{-2}$ $\cdot$ s$^{-1}$ $\cdot$ (GeV n$^{-1}$)$^{-1}$.

\section*{IV.\quad Project That Flux Onto \textbf{Resonant $k$-Bins}}

For each grid cell and each Alfvén sense:
\[
S_\pm(k) =
\int dp \int_{-1}^{1} d\mu\;
\underbrace{p\mu f_0}_{\displaystyle S(p)/2}
\delta\left( k - \frac{\Omega}{v\mu \mp V_A} \right).
\tag{3}
\]

\noindent
\textit{Numerically}:
\begin{enumerate}
\item Loop over momentum bins $p_m$;
\item Find the $\mu$ that satisfies the $\delta$-function (root or lookup);
\item Accumulate
\[
S_\pm(k) \mathrel{{+}{=}} \frac{S(p_m)}{2}
\left| \frac{\partial k}{\partial \mu} \right|^{-1}_{\mu_{\rm res}} \Delta p.
\]
\end{enumerate}

\section*{V.\quad Compute \textbf{Growth / Damping Rate}}

\[
\boxed{
\gamma_\pm(k) =
\frac{\pi^2 e^2 V_A}{c B_0^2}
\frac{S_\pm(k)}{k}
}
\tag{4}
\]

\textbf{Signs:}
\begin{itemize}
\item If $S_\pm > 0$ $\rightarrow$ \textbf{growth} (wave propagates opposite to streaming).
\item If $S_\pm \leq 0$ $\rightarrow$ \textbf{damping} (or no growth).
\end{itemize}

\noindent
(\textit{Jokipii 1966; Skilling 1975; Lee 1983; Ng \& Reames 1994}).

\section*{VI.\quad Update the \textbf{Wave Spectrum}}

For each $k$-bin in the finite-volume wave solver:
\[
\boxed{
\frac{\partial W_\pm}{\partial t}
+ (V_A \mp U) \frac{\partial W_\pm}{\partial s}
= 2\gamma_\pm(k) W_\pm
- \frac{W_\pm}{\tau_{\text{cas}}}
+ Q_{\text{shock}}
}
\tag{5}
\]
where:
\begin{itemize}
\item $2\gamma W$ = growth/damping from (4),
\item $W / \tau_{\text{cas}}$ = Kolmogorov cascade sink,
\item $Q_{\text{shock}}$ = ramp injection if the moving shock sits in the cell.
\end{itemize}

\noindent
\textit{Advance (5) with the same explicit/upwind FV scheme you use for the flow‐advection term; CFL based on $|V_A \mp U|$.}

\section*{VII.\quad Close the Loop – Recalculate $\kappa_\parallel$}

For a Kolmogorov shape ($W \propto k^{-5/3}$):
\[
\kappa_\parallel(p) = \frac{v}{3} \lambda_\parallel(p), \qquad
\lambda_\parallel^{-1}(p) =
\frac{\pi e^2}{B_0^2}
\int dk\;
\frac{W_+(k) + W_-(k)}{k}
\left( 1 - \mu_{\rm res}^2 \right),
\tag{6}
\]
with $\mu_{\rm res} = V_A / v$ for $v \gg V_A$.

\noindent
Update $\kappa_\parallel$ in the Parker solver and march to the next global time step.

\section*{VIII.\quad Numerical Algorithm (1 Time Step $\Delta t$)}

\[
\begin{aligned}
&\texttt{\# 1. Parker step} \quad \rightarrow \quad \texttt{new } f_0, \texttt{ gradient } \frac{\partial f_0}{\partial s} \\
&\texttt{\# 2. Streaming} \quad \rightarrow \quad S = -\kappa \frac{\partial f_0}{\partial s} \quad \texttt{(Eq. 2)} \\
&\texttt{\# 3. Resonance} \quad \rightarrow \quad S_{\pm}(k) \texttt{ via Eq. 3} \\
&\texttt{\# 4. } \gamma_{\pm}(k) \quad \rightarrow \quad \texttt{use Eq. 4} \\
&\texttt{\# 5. Wave FV step} \quad \rightarrow \quad \texttt{advance } W_{\pm} \texttt{ with Eq. 5} \\
&\texttt{\# 6. New } \lambda_{\parallel}, \kappa_{\parallel} \quad \rightarrow \quad \texttt{from Eq. 6} \\
&\texttt{\# 7. Go to next } \Delta t
\end{aligned}
\]


\noindent
\textit{A split-operator sequence keeps each sub-task simple and second-order accurate.}

\section*{IX.\quad Why It Works Without a Pitch-Angle Moment}

\noindent
The \textbf{entire growth term} in quasilinear theory depends only on the \textit{first} anisotropic moment:
\[
j_\parallel = \int v \mu f\, d^3p.
\]
For small anisotropy driven by a spatial gradient this is \textit{exactly} $-\kappa_\parallel \partial f_0 / \partial s$ (Skilling 1975).

\noindent
Thus, the Parker equation, plus its computed gradient, contains sufficient information to drive the correct wave growth.

\section*{Key Primary Papers}

\begin{itemize}
\item \textbf{Jokipii, J. R.} 1966, \textit{ApJ} \textbf{146}, 480 — streaming growth formula
\item \textbf{Skilling, J.} 1975, \textit{MNRAS} \textbf{172}, 557 — Parker+QLT consistency
\item \textbf{Lee, M. A.} 1983, \textit{JGR} \textbf{88}, 6109 — growth from isotropic gradient
\item \textbf{Ng, C. K. \& Reames, D. G.} 1994, \textit{ApJ} \textbf{424}, 1032 — coupled Parker + wave model
\item \textbf{Zank, G. P., Rice, W. K. M. \& Wu, C. C.} 2000, \textit{JGR} \textbf{105}, 25079 — shock/SEP/turbulence code
\end{itemize}

\noindent
\textit{This recipe is the one implemented in widely-used tools such as \textbf{SOLPENCO, iPATH, Shock Particle Transport Code (SPTC)} and the AMPS-based skeleton you now have in the canvas.}

\section*{Self-Contained Monte-Carlo Algorithm for Guiding-Centre Particles}

\noindent
Below is a \textbf{self-contained Monte-Carlo algorithm} for guiding-centre particles that move along a \textbf{piecewise-linear magnetic field line}. Every segment can have its own length, plasma parameters, and wave spectrum.

\noindent
After the algorithm you’ll find a \textbf{step-by-step sampling recipe} for each stochastic quantity you must draw during the advance.

\section*{0.\quad Data Structures}

\begin{verbatim}
Vertex k = 0 … Nv
    s_vert[k]          // cumulative arc–length [m]
    B[k], $\rho$[k], U[k]   // field, density, solar-wind speed at vertex

Segment j = 0 … (Nv–1)
    L[j]   = s_vert[j+1] - s_vert[j]
    mid variables (Bmid, Umid, …) pre-interpolated
    Wave spectrum W±[j][m]  (m = 0 … Nk–1)

\begin{align*}
&\text{Particle record} \\
&\text{seg} \quad \text{// current segment index } j \\
&\xi \quad \text{// local coordinate } (0 \leq \xi \leq L[j]) \\
&v \quad \text{// speed} \\
&\mu \quad \text{// pitch-angle cosine (value before isotropisation)} \\
&w \quad \text{// statistical weight}
\end{align*}

\noindent
Arc-length of a particle is
\[
s = s_{\text{vert}}[\text{seg}] + \xi.
\]

\section*{1.\quad Global Time Loop}

\[
\begin{aligned}
&\textbf{For } n = 1, \ldots, N_t \\
&\quad 1. \ \text{Move shock and update background} \quad (\Delta t) \\
&\quad 2. \ \text{Monte-Carlo particle advance} \\
&\quad 3. \ \text{Accumulate streaming } S_{\pm}(k) \\
&\quad 4. \ \text{Update wave field } W_{\pm}(k) \quad \text{(growth, damping, cascade)} \\
&\quad 5. \ \text{Recompute } \kappa_{\parallel}(p) \ \text{from new } W_{\pm} \\
&\textbf{End For}
\end{aligned}
\]

\noindent
Only \textbf{Step 2} needs true random sampling; all other steps are deterministic or cell-averaged.

\section*{2.\quad Monte-Carlo Advance for One Particle}

\[
\begin{aligned}
&\text{(a) fetch local cell data:} \\
&\quad j = p.\text{seg} \\
&\quad B_0 = B_{\text{mid}}[j], \quad V_A = \frac{B_0}{\sqrt{\mu_0 \rho_{\text{mid}}[j]}} \\
&\quad \text{k-bin array initialized}, \quad \kappa_\parallel = \text{kappa}(p.v, j) \quad \text{(from previous wave step)} \\[1.2em]
&\text{(b) adiabatic cooling/heating (exact over } \Delta t \text{):} \\
&\quad \text{div}U = \frac{U[j+1] \times A[j+1] - U[j] \times A[j]}{0.5 \times (A[j] + A[j+1]) \times L[j]} \\
&\quad p.v \times= \exp\left( \frac{1}{3} \, \text{div}U \, \Delta t \right) \\[1.2em]
&\text{(c) decide a scattering event:} \\
&\quad \nu_{\text{sc}} = 2 \times D_{\mu\mu}^{\text{(isotropic)}}(p.v, B_0, W_{\pm}[j][:]) \\
&\quad P_{\text{sc}} = 1 - \exp(-\nu_{\text{sc}} \Delta t) \\
&\quad \text{if } \text{rand\_uniform}() < P_{\text{sc}}: \quad p.\mu = 2 \times \text{rand\_uniform}() - 1 \quad \text{(isotropize)} \\
&\quad \text{else: keep } \mu \text{ from previous step} \\[1.2em]
&\text{(d) find resonant } k \text{ for both wave senses:} \\
&\quad \Omega = \frac{e \times B_0}{m_p} \\
&\quad k_{+} = \frac{\Omega}{p.v \times p.\mu - V_A} \quad \text{(outward } +\text{)} \\
&\quad k_{-} = \frac{\Omega}{p.v \times p.\mu + V_A} \quad \text{(inward } -\text{)} \\
&\quad \text{Add } w \times p.v \times p.\mu \text{ to } S_{+}[j][\text{k}_{+}\text{-bin}] \\
&\quad \text{Add } w \times p.v \times p.\mu \text{ to } S_{-}[j][\text{k}_{-}\text{-bin}] \\[1.2em]
&\text{(e) ballistic-plus-wind translation:} \\
&\quad \Delta s = (p.v \times p.\mu + U[j]) \times \Delta t \\
&\quad \text{while } \Delta s \neq 0: \\
&\quad \quad \text{if } \Delta s > 0: \\
&\quad \quad \quad \text{step} = \min(L[j] - p.\xi, \Delta s) \\
&\quad \quad \quad p.\xi += \text{step}, \quad \Delta s -= \text{step} \\
&\quad \quad \quad \text{if } p.\xi == L[j]: \quad j++, \quad p.\text{seg} = j, \quad p.\xi = 0 \\
&\quad \quad \text{else:} \\
&\quad \quad \quad \text{step} = \max(-p.\xi, \Delta s) \\
&\quad \quad \quad p.\xi += \text{step}, \quad \Delta s -= \text{step} \\
&\quad \quad \quad \text{if } p.\xi == 0: \quad j--, \quad p.\text{seg} = j, \quad p.\xi = L[j-1]
\end{aligned}
\]


\section*{3.\quad Sampling Details}

\begin{table}[h!]
\centering
\begin{tabular}{|p{5.5cm}|p{5.5cm}|p{4.5cm}|}
\hline
\textbf{Quantity to Sample} & \textbf{Distribution} & \textbf{Implementation} \\
\hline
Random decision whether a scatter occurs & Bernoulli with probability $P_{\text{sc}} = 1 - e^{-\nu_{\text{sc}} \Delta t}$ & \texttt{rand\_uniform() < Psc} \\
\hline
New pitch-angle cosine $\mu_{\text{new}}$ & Uniform in $[-1, +1]$ (isotropisation) & \texttt{mu = 2 * rand - 1} \\
\hline
Gyro-phase (optional for $\perp$ diffusion) & Uniform in $[0, 2\pi]$ & \texttt{phi = 2 * pi * rand} \\
\hline
Log-$k$ bin for a resonant wavenumber & Precompute $\log_{10} k_{\min}$, $\log \Delta$; binning formula: $\text{bin} = \text{round}\left( \frac{\log_{10}(k) - \log_{10}(k_{\min})}{\log \Delta} \right)$ & Precompute and bin \\
\hline
Random sign for cross-field step (FLRW, optional) & $\pm1$ with 50\% probability each & \texttt{sigma = (rand < 0.5) ? 1 : -1} \\
\hline
\end{tabular}
\caption*{Sampling methods for Monte-Carlo stochastic steps.}
\end{table}

\noindent
\textbf{Random generator}: one \texttt{std::mt19937\_64} seeded once at start; two helper distributions:

\begin{verbatim}
static std::uniform_real_distribution<double> U01(0.0, 1.0);
static std::normal_distribution<double>      G01(0.0, 1.0);
\end{verbatim}

\section*{4.\quad How Each Coefficient Is Calculated}

\begin{table}[h!]
\centering
\begin{tabular}{|p{5.5cm}|p{7cm}|p{4cm}|}
\hline
\textbf{Coefficient} & \textbf{Formula in Code} & \textbf{Inputs} \\
\hline
Pitch-angle diffusion (isotropic) & $\displaystyle D_{\mu\mu} = \frac{\pi \Omega^{2}}{B_{0}^{2}} \int (1 - \mu^{2}) \frac{W_+(k) + W_-(k)}{k} \, dk$ & local $W_\pm(k)$, $B_0$ \\
\hline
Scattering frequency & $\nu_{\text{sc}} = 2 D_{\mu\mu}$ & from $D_{\mu\mu}$ above \\
\hline
Streaming in bin & $S = -\kappa_\parallel \partial_s f_0$ or Monte-Carlo sum $\sum w v \mu$ & gradient of $f_0$ or particle tally \\
\hline
Growth/damping rate & $\gamma_\pm = C V_A S_\pm / (B_0^{2} k)$ & streaming $S_\pm(k)$ \\
\hline
Updated wave power & FV Eq.~(5) previous answer & growth, cascade, shock injection \\
\hline
\end{tabular}
\caption*{}
\end{table}

\section*{5.\quad Putting It Inside AMPS}

\begin{itemize}
\item Replace \texttt{Particle} with AMPS native macro-particle struct.
\item Each linear segment can reuse AMPS face-handling; just remember to \textbf{split} the upstream cell at the shock face.
\item Store arrays \texttt{Splus}, \texttt{Sminus}, \texttt{W.Wp}, \texttt{W.Wm} as \textbf{cell properties}; AMPS field interpolation supplies $B$, $\rho$, $U$ at vertices automatically.
\end{itemize}

\section*{References for the Sampling Logic}

\begin{itemize}
\item \textbf{Jokipii, J.R.} 1966, \textit{ApJ} \textbf{146}, 480
\item \textbf{Earl, J.A.} 1974, \textit{ApJ} \textbf{193}, 231 (isotropic pitch-angle scattering)
\item \textbf{Skilling, J.} 1975, \textit{MNRAS} \textbf{172}, 557 (streaming integral)
\item \textbf{Ng \& Reames} 1994, \textit{ApJ} \textbf{424}, 1032 (Parker+wave coupling)
\item \textbf{Zank, Rice \& Wu} 2000, \textit{JGR} \textbf{105}, 25079 (Monte-Carlo + waves, shock)
\end{itemize}

\noindent
\textit{This detailed procedure reproduces the physics outlined in our earlier derivations while being \textbf{mesh-independent} (any segment length) and fully compatible with AMPS’s adaptive Monte-Carlo framework.}

\section*{Pitch-Angle--Free Variant of the Monte-Carlo Algorithm}

Below is the \textbf{``pitch-angle--free'' variant} of the algorithm posted earlier. Everything that once relied on an individual particle’s pitch-angle cosine $\mu$ is now replaced by \textbf{Fickian streaming} that comes \textit{directly} from the Parker equation’s spatial-diffusion term. Particles no longer carry $\mu$, but you still obtain the net streaming $S_\pm(k)$ that drives Alfvén-wave growth/damping.

\section*{0.\quad State Carried by the Code}

\begin{tabular}{|l|p{10cm}|}
\hline
\textbf{Object} & \textbf{Comment} \\
\hline
\texttt{Field-line mesh} (B, $\rho$, U, A) & Exactly as before (piecewise-linear segments). \\
\texttt{Wave spectra} $W_{\text{plus}}[j][m]$, $W_{\text{minus}}[j][m]$ & $m =$ log-$k$ bins per cell $j$. \\
\texttt{Monte-Carlo particles} & Only three variables per particle: position $s$, speed $v$, statistical weight $w$. \textbf{No $\mu$.} \\
\texttt{Parker coefficients} per cell & Parallel diffusion $\kappa_\parallel(p,s,t)$ and flow divergence $\nabla \cdot U$. \\
\texttt{Shock object} (pos, speed, $\eta$, $r$) & Same as before (for ramp injection). \\
\hline
\end{tabular}

\section*{1.\quad Parker-Equation Monte-Carlo Analogue}

\textit{(Isotropic advection + spatial diffusion)}

For each particle during a global step $\Delta t$:

\begin{itemize}
\item Adiabatic cooling/heating:
\[
\frac{dv}{dt} = \frac{1}{3} \, v \, (\nabla \cdot U)_{\text{cell}}
\]
\[
v \leftarrow v + \frac{dv}{dt} \, \Delta t
\]
\item Spatial transport (convection + random walk):
\[
\Delta s = U_{\text{cell}} \, \Delta t + \sqrt{2 \, \kappa_\parallel(v, \text{cell}) \, \Delta t} \, N(0,1)
\]
\[
s \leftarrow s + \Delta s
\]
\end{itemize}

No random pitch-angle is drawn; $\kappa_\parallel$ encodes the effect of all scattering.

\section*{2.\quad Cell-Wise Streaming Flux from Particles}

Because each stochastic step already implements Fick’s law, you obtain the \textbf{net flux} in cell $j$ by simple counting:

\[
S_{\text{cell}} = \frac{1}{\Delta t \, A_j} \sum_i w_i (s_i^{\text{new}} - s_i^{\text{old}})
\]

\begin{itemize}
\item Positive if the population drifts sunward, negative otherwise.
\item Units: particles $\cdot \text{m}^{-2} \cdot \text{s}^{-1} \cdot (\text{dp})^{-1}$.
\end{itemize}

(This is identical to evaluating $S = -\kappa_\parallel \frac{\partial f}{\partial s}$ on the Eulerian grid but avoids a noisy finite difference.)

\section*{3.\quad Project That Streaming onto the Two Wave Senses}

For an \textbf{isotropic} distribution, the pitch-angle of particles contributing to a given resonant $k$ is fixed by the resonance condition with the \textbf{average} cosine:
\[
\mu_{\text{res}} = \pm \frac{V_A}{v}.
\tag{1}
\]

Thus, the local streaming apportioned to senses ``+'' and ``--'' is simply:

\[
\boxed{
\begin{aligned}
S_{+}(k) &= \frac{1}{2} \, S \, H(k, k_{\text{res}}^{+}) \\
S_{-}(k) &= \frac{1}{2} \, S \, H(k, k_{\text{res}}^{-})
\end{aligned}
}
\qquad
k_{\text{res}}^{\pm} = \frac{\Omega}{v \mu_{\text{res}} \mp V_A}
\tag{2}
\]

where $H(k, k_{\text{res}})$ deposits the flux into whichever $k$-bin contains $k_{\text{res}}$ (nearest-bin or CIC weighting). The factor $1/2$ comes from averaging over $\mu$.

\section*{4.\quad Growth / Damping Rate Without $\mu$-Dependence}

Insert (2) into the standard quasilinear formula:

\[
\boxed{ \gamma_\pm(k) = C V_A \frac{S_\pm(k)}{k B_0^{2}} }
\qquad
C = \frac{\pi^{2} e^{2}}{c}.
\tag{3}
\]

\begin{itemize}
\item $S > 0$ (net sunward streaming) $\Rightarrow$ amplifies inward waves $W_{-}$, damps outward $W_{+}$.
\item $S < 0$ $\Rightarrow$ the opposite.
\end{itemize}

\section*{5.\quad Update the Wave Field (Per Cell, Per $k$)}

Use the same finite-volume equation as before:

\[
\frac{\partial W_\pm}{\partial t} + (V_A \mp U) \frac{\partial W_\pm}{\partial s}
= 2 \gamma_\pm W_\pm - \frac{W_\pm}{\tau_{\text{cas}}} + Q_{\text{shock}}.
\tag{4}
\]

\section*{6.\quad Coupled Loop for One Global Step $\Delta t$}

\begin{itemize}
\item[1.] Propagate shock $\rightarrow$ jump $U$, $\rho$, $B$ (if it crosses a vertex).
\item[2.] For every particle:
    \begin{itemize}
    \item Deterministic $dv$ (adiabatic).
    \item Random spatial step ($\kappa_\parallel$).
    \end{itemize}
\item[3.] Tally $S_{\text{cell}}$ from $\sum w \, \Delta s$.
\item[4.] For each $k$-bin:
    \begin{itemize}
    \item Compute $k_{\text{res}}^{\pm}$ (Eq.~2).
    \item Deposit $S_\pm(k)$.
    \item Compute $\gamma_\pm(k)$ via Eq.~3.
    \item Advance $W_\pm$ with Eq.~4 (upwind + RK2).
    \end{itemize}
\item[5.] Recompute $\kappa_\parallel(p,s)$ from new $W_\pm$ (isotropic CR formula).
\item[6.] Proceed to next $\Delta t$.
\end{itemize}

\section*{7.\quad What Changed Relative to the $\mu$-Carrying Version?}

\begin{tabular}{|l|p{10cm}|}
\hline
\textbf{Old Step} & \textbf{Replacement in Isotropic Scheme} \\
\hline
Draw random $\mu$ per particle & \textbf{No $\mu$ stored.} Streaming supplied by $\sum w \Delta s / (A \Delta t)$. \\
Resonant $k$ via $\Omega / (v \mu \pm V_A)$ & Use average $\mu_{\text{res}} = \pm V_A / v$. \\
$S_\pm$ built from $\sum (w v \mu)$ & $S_\pm = \frac{1}{2} S_{\text{cell}}$ onto $k_{\text{res}}$ bin. \\
Energy exchange $\Delta E \propto \Delta \mu$ & Dissipative term included implicitly via $\gamma < 0$ when $S$ has the wrong sign. \\
\hline
\end{tabular}

Everything else—adiabatic cooling, shock injection, Kolmogorov cascade—remains unchanged.

\section*{References (Pitch-Angle--Free Implementations)}

\begin{itemize}
\item \textbf{Ng \& Reames} (1994), ApJ \textbf{424}, 1032 — Parker + QLT growth with Fickian flux.
\item \textbf{Vainio \& Laitinen} (2007), ApJ \textbf{658}, 622 — two-moment ($f_0$, $f_1$) comparison; shows isotropic limit works once gradient streaming is used.
\item \textbf{Zank, Rice \& Wu} (2000), JGR \textbf{105}, 25079 — Monte-Carlo particles without $\mu$; flux from step displacement.
\item \textbf{Afanasiev et al.} (2015), ApJ \textbf{799}, 80 — Practical SEP+wave code employing this isotropic-flux method.
\end{itemize}

\section*{Take-Away}

Even in a \textbf{pitch-angle--free Parker framework}, the spatial gradient of the isotropic intensity produces a deterministic streaming flux. Insert that flux into the quasilinear growth expression and you get physically correct \textbf{wave amplification in foreshocks} and \textbf{damping elsewhere}—without ever storing $\mu$ per particle.

\section*{Why the ``$\Omega / 0$'' Never Appears in a Real Code \& How to Avoid It Explicitly}

The apparent singularity creeps in only if you \textbf{blindly} substitute the average resonance cosine
\[
\mu_{\text{res}} = \pm \frac{V_A}{v} \tag{1}
\]
into
\[
k_{\text{res}}^{\pm} = \frac{\Omega}{v \mu_{\text{res}} \mp V_A}. \tag{2}
\]

For the \textbf{same sign} choice ($+$ into $+$, $-$ into $-$), the denominator indeed becomes $v (V_A / v) - V_A = 0$. 

But in the actual quasilinear picture—and in every working SEP code—\textbf{that situation never occurs}, because:

\begin{table}[h!]
\centering
\begin{tabular}{|p{5cm}|p{9cm}|}
\hline
\textbf{Fact} & \textbf{Why the division-by-zero cannot appear} \\
\hline
\textbf{1. Physical resonance requires $v \mu \neq \pm V_A$.} & The wave’s phase speed is $V_A$. Particles with a parallel speed $v \mu = V_A$ stay in the wave frame forever; the QLT integral shows they contribute \textbf{zero power} (they sit exactly at the $\delta$-function edge). \\
\hline
\textbf{2. A finite $k$-grid never contains $k = \infty$.} & The singular denominator corresponds to $k \to \infty$ (wavelength $\to 0$), far beyond any turbulence range you model. \\
\hline
\textbf{3. Codes map a \textit{band} of $\mu$ to every finite $k$-bin.} & You integrate over the $\delta$-function numerically; the resonant $\mu$ in each bin is $\mu = (\Omega / k \pm V_A)/v$, which is \textit{never} exactly $V_A/v$ for a finite $k$. \\
\hline
\end{tabular}
\end{table}

\section*{How You Implement It Safely}

\subsection*{Step 1: Pick the \textbf{wave sense} first, then the $\mu$ that resonates with your \textit{finite} $k$-bin}

\begin{lstlisting}[language=C++, basicstyle=\ttfamily\small]
// given: particle speed v, Alfvén speed VA, gyrofrequency Omega
for (int m = 0; m < Nk; ++m) {
    double kbin = k[m];                       // centre of bin m
    double muRes_plus  = (Omega / kbin + VA) / v;  // for outward waves
    double muRes_minus = (Omega / kbin - VA) / v;  // for inward waves
    // accept only if |muRes| <= 1
}
\end{lstlisting}

\textit{Because $k_{\text{bin}}$ is finite, the denominator in (2) is finite.}

\subsection*{Step 2: Weight the Particle Flux with \textbf{Fick’s Law}}

For each momentum bin $p$,
\[
S(p) = -\kappa_\parallel \frac{\partial f_0}{\partial s}.
\]
Split that scalar flux \textbf{equally} into the two resonant $\mu$-roots that fall inside $[-1,+1]$.

\begin{lstlisting}[language=C++, basicstyle=\ttfamily\small]
double fluxShare = 0.5 * S(p_bin);    // half to +, half to - sense
Splus [j][m] += fluxShare;
Sminus[j][m] += fluxShare;
\end{lstlisting}

If only one sense has a valid root (e.g., $v < V_A$), put the \textbf{full} flux into that sense and zero into the other.

\subsection*{Step 3: Compute the Growth / Damping Coefficient}

\[
\gamma_\pm(k) = \frac{\pi^2 e^2 V_A}{c B_0^2} \frac{S_\pm(k)}{k}. \tag{3}
\]

No singularity appears because $k$ is finite and $S_\pm(k)$ is finite.

\section*{Edge Cases \& Numerical Tips}

\begin{table}[h!]
\centering
\begin{tabular}{|p{5cm}|p{9cm}|}
\hline
\textbf{Situation} & \textbf{What the Code Does} \\
\hline
$v < V_A$ (thermal ions) & Only \textbf{one} of $k_{\text{res}}^{\pm}$ is physical. Put the full scalar flux $S$ into that wave sense; the other sense gets zero -- that is correct QLT. \\
\hline
$k$-bin centre formally gives $\mu_{\text{res}} > 1$ & Skip that bin; the particle cannot resonate with that $k$. \\
\hline
Very high $k$ near the upper grid limit & The resonant $\mu$ becomes extremely small; the particle flux there is negligible compared with damping, so $\gamma \to 0$ automatically. \\
\hline
Want a single ``average'' $\mu$ for a whole $k$-bin to avoid per-particle roots & Use $\mu_{\text{eff}} = \Omega / (v k_{\text{centre}})$ \textbf{with the opposite sign in Eq. (2)}, so the denominator is $\pm 2 V_A$, never zero. \\
\hline
\end{tabular}
\end{table}

\section*{References That Use Exactly This Trick}

\begin{itemize}
\item \textbf{Ng \& Reames} (1994), \textit{ApJ} \textbf{424}, 1032 -- finite-difference Parker + wave code; Appendix A shows how $\mu$-roots are binned, always avoiding $k \to \infty$.
\item \textbf{Zank, Rice \& Wu} (2000), \textit{JGR} \textbf{105}, 25079 -- Monte-Carlo particles without $\mu$; Section 3 derives the same split-flux method.
\item \textbf{Afanasiev et al.} (2015), \textit{ApJ} \textbf{799}, 80 -- production code; Table 2 lists grid limits guaranteeing $k$ finite.
\end{itemize}

\section*{Bottom-Line Rule}

\textbf{Never plug the $\pm$ sign from Eq. (1) into the same $\pm$ sign in Eq. (2).}

Instead:
\begin{enumerate}
\item For each finite $k$-bin choose the wave sense,
\item Solve for its resonant $\mu$,
\item Reject roots outside $[-1,1]$,
\item Distribute the Parker flux accordingly.
\end{enumerate}

Follow that sequence and the ``$\Omega / 0$'' can neither appear mathematically nor numerically.


\section*{1.\quad The Resonance Equation and the ``Singularity''}

For a particle of speed $v$ interacting with parallel Alfvén waves, the exact cyclotron-resonance condition is
\[
k_\parallel = \frac{\Omega}{v \mu \mp V_A},
\qquad
\Bigl(\text{upper sign = outward } W_{+};\quad \text{lower = inward } W_{-}\Bigr).
\tag{1}
\]

The \textbf{apparent singularity} arises when you try to solve Eq.~(1) for the pitch-angle cosine $\mu$:
\[
\mu_{\text{res}}^{(\pm)}(k) = \frac{\Omega}{v k} \pm \frac{V_A}{v}.
\tag{2}
\]

For the ``same'' sign choice (e.g., using $+$ in both places), the two terms cancel at $\Omega/(v k) = V_A/v$, giving a denominator $v \mu - V_A = 0$ in Eq.~(1) and the illusion of $\Omega/0$.

\section*{2.\quad Why the Real Monte-Carlo Never Hits That Point}

\begin{enumerate}
\item \textbf{Finite $k$-bins:} In a code, the spectrum is stored in bins of \textit{finite} width $\Delta k$.
The resonance condition only requires $\mu$ to fall \textbf{within} a bin, never exactly on a mathematical point.
\item \textbf{Physical range:} The wavelengths where $v|\mu| = V_A$ correspond to $k \to \infty$ (sub-proton scales) or $|\mu| > 1$; both lie outside the MHD/gyro-QLT domain you model.
\item \textbf{Opposite-sign rule:} A particle resonates with the wave sense travelling \textbf{against} its own parallel motion.
Numerically that means: always combine the ``$+$'' wave with the ``$-$'' sign in Eq.~(2) and vice versa.
Then the sum in Eq.~(2) cannot be zero.
\end{enumerate}

\section*{3.\quad Robust $\mu$-Sampling Algorithm}

Below is the practical sequence used in Monte-Carlo foreshock codes such as \textbf{Zank \& Rice (2000)} and \textbf{Afanasiev et al. (2015)}.

\begin{lstlisting}[language=C++, basicstyle=\ttfamily\small]
// inputs: particle speed v, local VA, gyro-frequency Omega, log-k grid k[m]
for (int m = 0; m < Nk; ++m) {
    double kbin = k[m];

    // ----- outward (+) waves -----
    double mu_plus  = (Omega / (v * kbin)) + (VA / v);   // NB: opposite sign!
    if (std::fabs(mu_plus) <= 1.0) {
        // deposit half of scalar flux S into this bin
        Splus[j][m] += 0.5 * S_scalar;
    }

    // ----- inward (-) waves -----
    double mu_minus = (Omega / (v * kbin)) - (VA / v);
    if (std::fabs(mu_minus) <= 1.0) {
        Sminus[j][m] += 0.5 * S_scalar;    // the other half
    }
}
\end{lstlisting}

\textbf{Key Points:}
\begin{itemize}
\item $+$ wave uses the \textbf{+$V_A/v$} branch, $-$ wave uses \textbf{--$V_A/v$}.
The apparent $\Omega/0$ never materializes.
\item If $v < V_A$, only \textbf{one} of the two roots lies inside $[-1, 1]$; put the \textbf{full} scalar flux into that sense and set the other to zero.
\item Near the upper end of the $k$-grid, the value $\Omega/(v k)$ becomes so small that $\mu_{\text{res}}$ is dominated by $\pm V_A/v$; the code still works because the bin width keeps the denominator finite.
\end{itemize}

\section*{4.\quad Handling the Extreme Corner Cases}

\begin{table}[h!]
\centering
\begin{tabular}{|p{5cm}|p{9cm}|}
\hline
\textbf{Situation} & \textbf{Safe Handling Rule} \\
\hline
$\mu_{\text{res}} > 1$ & No resonance in this bin $\rightarrow$ skip. \\
\hline
$v \approx V_A$ and outward sense & Only inward wave has a physical root; assign full flux to $S_{-}$. \\
\hline
Want a single ``average $k$'' per particle & Solve $k_{\text{res}} = \Omega / \left[v (V_A/v \pm \mu_{\text{eff}})\right]$ with a small safety shift $\epsilon = 10^{-6}$ if needed. \\
\hline
\end{tabular}
\end{table}

\section*{5.\quad References for the Procedure}

\begin{itemize}
\item \textbf{Jokipii, J.R.} 1966, ApJ \textbf{146}, 480.
\item \textbf{Ng \& Reames} 1994, ApJ \textbf{424}, 1032 --- Appendix A (bin-based $\mu$-root).
\item \textbf{Zank, Rice \& Wu} 2000, JGR \textbf{105}, 25079 --- Section 3.2 (half--flux rule).
\item \textbf{Afanasiev et al.} 2015, ApJ \textbf{799}, 80 --- Code implementation details, Table 2.
\end{itemize}

\section*{Bottom Line}

\textit{In a Monte-Carlo solver you \textbf{never} divide by zero if you:}
\begin{enumerate}
\item Map particles to \textbf{finite} $k$-bins,
\item Pair each wave sense with the \textbf{opposite sign} in the $\mu$-formula,
\item Discard roots that fall outside $|\mu| \leq 1$ or beyond your $k$-range.
\end{enumerate}

With those three rules, the ``$\Omega/0$'' singularity is purely a textbook artefact and cannot appear in real computations.


\section*{Fully Explicit Monte-Carlo Recipe}

Below is a \textbf{fully explicit Monte-Carlo recipe} that satisfies all three requirements you gave:
\begin{itemize}
\item \textbf{(1) Parker equation only} – the code advances the \textit{isotropic} phase–space density $f_0(p,s,t)$; no pitch-angle variable is stored per particle.
\item \textbf{(2) Piece-wise linear field line} – the guiding centre moves through consecutive straight segments of arbitrary length $L_j$.
\item \textbf{(3) Two-way coupling} – the Parker solver and the finite-volume wave‐transport solver exchange the scattering coefficient $\kappa_\parallel$ and the growth/damping rate $\gamma_\pm(k)$ every global step.
\end{itemize}

Everything is given in \textbf{algorithmic order} so you can translate it to C / C++ / Fortran directly.

\section*{0.\quad Data You Must Hold in Memory}

\begin{tabular}{|p{4cm}|p{4cm}|p{7cm}|}
\hline
\textbf{Object} & \textbf{Dimensions} & \textbf{Notes} \\
\hline
\textbf{Vertices} $k=0..N_v$ & $s_k,\,B_k,\,\rho_k,\,U_k,\,A_k$ & $s_k$ = cumulative arc-length; $A_k\propto 1/B_k$ (flux conservation). \\
\hline
\textbf{Segments} $j=0..N_v-1$ & length $L_j$ & $L_j = s_{k+1} - s_k$. \\
\hline
\textbf{Wave spectra} & $W_{\pm}[j][m]$ & $m=0..N_k-1$ logarithmic $k$-bins. \\
\hline
\textbf{Diffusion coefficient} & $\kappa_\parallel(p_m, s_j)$ & Updated each global step. \\
\hline
\textbf{Particles} & $\{j, \xi, v, w\}$ & Segment ID $j$; local coordinate $0\leq\xi\leq L_j$; speed $v$; weight $w$. \\
\hline
\textbf{Shock object} & $\{s_{\text{sh}}, V_{\text{sh}}, \eta, r\}$ & Position, speed, wave-injection efficiency, compression ratio. \\
\hline
\end{tabular}

No particle carries a pitch-angle cosine.

\section*{1.\quad One Global Time Step $\Delta t$ (Top-Level Flow Chart)}

\begin{enumerate}
\item Move the shock: \( s_{\text{sh}} \leftarrow s_{\text{sh}} + V_{\text{sh}} \, \Delta t \)
\item Jump \( U \), \( \rho \), \( B \) behind the shock (Rankine–Hugoniot)
\item Monte-Carlo sweep over all particles:
    \begin{itemize}
        \item deterministic adiabatic \( \frac{dv}{dt} \)
        \item spatial convection + spatial diffusion step
        \item tally scalar streaming \( S_j(p) \) in every cell
    \end{itemize}
\item Deposit streaming into wave bins \( \rightarrow S_{\pm}(k) \) (see §4)
\item Compute \( \gamma_{\pm}(k) \) from \( S_{\pm}(k) \) (see §5)
\item Finite-volume wave update (see §6)
\item Recompute \( \kappa_{\parallel} \) from the new wave power (see §7)
\item Loop to next \( \Delta t \)
\end{enumerate}


\section*{2.\quad Particle Evolution in One Segment}

\subsection*{2.1\quad Adiabatic Cooling/Heating}

\[
\frac{dv}{dt} = \frac{1}{3} v \left( \frac{U}{B} \frac{dB}{ds} - \frac{dU}{ds} \right)_j
\]
\[
\text{div}U = \frac{U_{j+1} A_{j+1} - U_j A_j}{0.5 (A_j + A_{j+1}) L_j}
\]
\[
v \leftarrow v \times \exp\left( \frac{1}{3} \, \text{div}U \, \Delta t \right)
\]

\subsection*{2.2\quad Spatial Convection + Diffusion (Fickian Step)}

\[
\Delta s = U_j\,\Delta t + \sqrt{2\kappa_\parallel(p, v, s_j)\, \Delta t}\; G(0,1)
\]
where $G(0,1)$ is a standard Gaussian deviate.

\section*{3.\quad Streaming Flux Produced by the Particles}

For each cell $j$ and each momentum bin $p_m$:
\[
S_j(p_m) = \frac{1}{\Delta t \, A_j} \sum_i w_i (s_i^{\text{new}} - s_i^{\text{old}})
\]
This is exactly the Monte-Carlo realisation of the Parker flux $S = -\kappa_\parallel \partial f_0 / \partial s$.

\section*{4.\quad Split Scalar Flux into the Two Wave Senses}

For every $k$-bin center $k_m$:
\[
\begin{aligned}
\mu_{\text{res}}^{+} &= \frac{\Omega}{v k_m} + \frac{V_A}{v}, \quad \text{(outward waves } +\text{)} \\
\mu_{\text{res}}^{-} &= \frac{\Omega}{v k_m} - \frac{V_A}{v}, \quad \text{(inward waves } -\text{)}
\end{aligned}
\]
\[
\text{If } |\mu_{\text{res}}^{+}| \leq 1, \quad S_{+}[j][m] \mathrel{+}= 0.5 \times S_j(p_{\text{bin}})
\]
\[
\text{If } |\mu_{\text{res}}^{-}| \leq 1, \quad S_{-}[j][m] \mathrel{+}= 0.5 \times S_j(p_{\text{bin}})
\]

If only one root lies inside $[-1,1]$, assign \textbf{all} the scalar flux to that sense.

\section*{5.\quad Growth / Damping Rate in Each Bin}

\[
\boxed{
\gamma_\pm(j,m) =
\frac{\pi^2 e^2 V_{A,j}}{c B_{0,j}^2}
\frac{S_\pm(j,m)}{k_m}
}
\quad [\text{s}^{-1}]
\]
If $S_\pm > 0$ → growth; if $S_\pm < 0$ → damping.

\section*{6.\quad Finite-Volume Wave Transport}

\[
W_\pm^{n+1} = W_\pm^{n}
- \frac{\Delta t}{\Delta s} (F_{i+1/2} - F_{i-1/2})
+ 2\gamma_\pm W_\pm^{n} \Delta t
- \frac{W_\pm^{n} \Delta t}{\tau_{\text{cas}}}
+ Q_{\text{shock}}(j, m)
\]
\begin{itemize}
\item Upwind flux $F$ with speed $V_A \mp U$.
\item $Q_{\text{shock}}$ is the ramp injection: $\Delta W = F_{\text{sh}} (f_j \Delta t) / \Delta s_j$, with $F_{\text{sh}} = \tfrac{1}{2} \eta \rho_1 (V_{\text{sh}} - U_1)^3$.
\end{itemize}

\section*{7.\quad New Parallel Mean-Free Path and Diffusion Coefficient}

\[
\lambda_\parallel(p, s_j)^{-1} =
\frac{\pi e^2}{B_{0,j}^2}
\left( W_{+}^{\text{int}} + W_{-}^{\text{int}} \right)
\left( 1 - \mu_{\text{res}}^2 \right),
\qquad
\mu_{\text{res}} = \frac{V_{A,j}}{v}.
\]
Set $\kappa_\parallel = v \lambda_\parallel / 3$ and store for the next global step.

\section*{8.\quad Scattering Energy Gain / Loss (Optional)}

If you keep a particle energy ledger, the net energy transfer per cell is already in the FV update:
\[
\Delta E_{\text{waves}} = 2\gamma_\pm W_\pm \Delta t A_j \Delta s_j.
\]
Update the cell’s particle energy by $-\Delta E_{\text{waves}}$ to conserve total energy; no per-particle kick is needed.

\section*{9.\quad Numerical Stability \& Accuracy Checklist}

\begin{tabular}{|p{4cm}|p{10cm}|}
\hline
\textbf{Item} & \textbf{Rule of Thumb} \\
\hline
\textbf{CFL (wave advection)} & $(V_A + U) \Delta t < 0.5\,\min(\Delta s_j)$. \\
\hline
\textbf{Diffusion step} & Keep $2 \kappa_\parallel \Delta t < (0.4 \Delta s_{\min})^2$. \\
\hline
\textbf{Random seeds} & One \texttt{std::mt19937\_64} per thread. \\
\hline
\textbf{k-grid} & 20–40 log bins from $k_{\min} = 10^{-6}$ to $k_{\max} = 10^{-1}$ rad m$^{-1}$. \\
\hline
\textbf{Cascade sink} & $\tau_{\text{cas}}^{-1} = C_K \sqrt{2W/N_k} / L_\perp$, with $C_K \approx 0.2$. \\
\hline
\end{tabular}

\section*{Key References Behind Each Step}

\begin{itemize}
\item Jokipii 1966 (growth formula)
\item Ng \& Reames 1994 (Parker + streaming coupling)
\item Zank, Rice \& Wu 2000 (MC without $\mu$, flux from displacement)
\item Afanasiev et al. 2015 (shock-injection and FV wave transport)
\end{itemize}

\bigskip

Follow these nine numbered sections line-for-line and you have a \textbf{resolution-independent, pitch-angle–free Monte-Carlo SEP solver} that self-consistently grows and damps Alfvénic turbulence along any piece-wise linear heliospheric field line.


\section*{What exactly are \( s_i^{\text{old}} \) and \( s_i^{\text{new}} \)?}

\begin{table}[h!]
\centering
\begin{tabular}{|l|p{10.5cm}|c|}
\hline
\textbf{Symbol} & \textbf{Precise meaning} & \textbf{Units} \\
\hline
\texttt{\textbf{s\_i\_old}} & The particle’s \textbf{arc-length position} \emph{along the magnetic field line} \textbf{at the start of the global step} \( \Delta t \). In code, this is the cumulative distance from the line’s origin to the particle’s location \textbf{before} you do advection + diffusion. & metres \\
\hline
\texttt{\textbf{s\_i\_new}} & The particle’s \textbf{arc-length position at the end of the same global step} \( \Delta t \), after you have applied:
\begin{itemize}
    \item deterministic convection \( U\,\Delta t \),
    \item stochastic diffusion step \( +\sqrt{2\kappa_\parallel \Delta t}\,G(0,1) \),
    \item all segment crossings.
\end{itemize}
& metres \\
\hline
\end{tabular}
\end{table}

Hence
\[
\boxed{ \; s_i^{\text{new}} - s_i^{\text{old}} = \text{net distance the $i$-th particle travelled \textbf{along} $B$ during the step} \; },
\]
\noindent
\textbf{signed}: positive if the displacement is in the $+$ field-line direction (usually anti-sunward) and negative otherwise.

\bigskip

\section*{How it fits into the streaming flux formula}

For one spatial cell \( j \) and one momentum bin \( p_m \):
\[
S_j(p_m) \;=\;
\frac{1}{\Delta t\,A_j}\;
\sum_{i\in(j,p_m)}
w_i\,\bigl(s_i^{\text{new}}-s_i^{\text{old}}\bigr),
\]
\begin{itemize}
\item \( w_i \) = statistical weight of that Monte-Carlo particle (real particles it represents).
\item The sum runs over \textbf{all pseudo-particles whose \emph{start–position} is inside cell \( j \) and whose speed falls in the momentum bin \( p_m \)}.
\end{itemize}

The quotient converts total displaced “charge” into a physical \textbf{number flux} (particles \( \cdot \, \text{m}^{-2} \, \text{s}^{-1} \, (\Delta p)^{-1} \)).

\bigskip

\section*{Crossing more than one cell in one step}

If a particle crosses a cell boundary during \( \Delta t \), you \textbf{split the displacement} so each cell receives only the distance travelled \emph{inside that cell}:
\begin{lstlisting}[language=C++, basicstyle=\ttfamily\small]
double delta_s = total_displacement;
while (delta_s != 0.0) {
    double step = (delta_s > 0)
                  ? std::min(L_j - xi, delta_s)    // distance until right face
                  : std::max(-xi, delta_s);        // distance until left face
    // Accumulate streaming for current cell j
    streaming[j][p_bin] += w_i * step;
    // Advance local coords
    xi  += step;
    delta_s -= step;
    if (xi == L_j) { j++; xi = 0; }           // crossed right face
    if (xi == 0 && delta_s < 0) { j--; xi = L_j; } // crossed left face
}
\end{lstlisting}

Finally divide by \( (\Delta t\,A_j) \) when you convert the per-cell accumulator into \( S_j(p_m) \).

\bigskip

\section*{Why this definition gives the correct Parker streaming}

Fick’s law for the Parker equation reads
\[
S(p,s,t) = -\,\kappa_{\parallel}(p,s,t)\,
           \frac{\partial f_0}{\partial s}.
\]

A Monte-Carlo random-walk with variance \( 2\kappa_\parallel\Delta t \) reproduces the same ensemble-average flux when you measure it exactly as “distance travelled per time” — which is what the
\[
s_i^{\text{new}} - s_i^{\text{old}}
\]
term captures. Summing over all particles in a cell therefore yields the \textbf{same \( S \)} that appears in the quasilinear growth/damping coefficient.

\section*{Do you \textbf{have} to momentum-bin the Monte-Carlo particles?}

Strictly speaking, \textbf{no}—the growth (or damping) rate in each \(k\)-bin depends only on the \textbf{total resonant streaming}
\[
S_\pm(k) = \int dp\, S(p)\, \delta\left(k - \frac{\Omega}{v\mu \mp V_A}\right),
\tag{1}
\]
and one can evaluate Eq.~(1) by summing every particle individually.
\textbf{Practically}, however, almost every production‐quality SEP+wave code \textbf{bins particles in \(p\)} (or \(v\)) for four reasons:

\begin{table}[h!]
\centering
\begin{tabular}{|l|p{11.5cm}|}
\hline
\textbf{Reason} & \textbf{What momentum-binning gives you} \\
\hline
\textbf{1. Noise suppression} & Particles are Poisson-sampled; the raw \( S_\pm(k) \propto \sum w_i v_i \) fluctuates as \( \sqrt{N} \). Grouping \( N \gg 1 \) particles per \(p\)-bin averages the shot noise before you project onto \(k\). \\
\hline
\textbf{2. Consistent \( \kappa_\parallel \) update} & After each wave step you must recompute \( \kappa_\parallel(p,s) \). A discrete \( p_m \)-grid (same as used for Parker’s finite-difference solver) lets you tabulate \( \lambda_\parallel(p_m) \) once and reuse it for \emph{all} particles in that bin. \\
\hline
\textbf{3. Energy-selective output} & Spacecraft SEP data are reported in energy channels (e.g., 0.5–1 MeV, 1–5 MeV). Bins let you form synthetic intensities or fluences that map one-to-one onto observations. \\
\hline
\textbf{4. Resonance bookkeeping} & The \( \delta \)-function in Eq.~(1) ties every \(p\)-bin to exactly \textbf{one} outward and \textbf{one} inward \(k\)-bin, making the splitting algorithm (half‐flux rule) a two-index array update instead of a per-particle search. \\
\hline
\end{tabular}
\end{table}

\noindent\textbf{References}
\begin{itemize}
\item Ng \& Reames 1994 (ApJ 424, 1032) – uses 20 logarithmic \(p\)-bins to feed the Lee growth term.
\item Zank, Rice \& Wu 2000 (JGR 105, 25079) – Monte-Carlo code with \(p\)-bins; Appendix B shows the noise scaling.
\item Afanasiev et al. 2015 (ApJ 799, 80) – tests bin widths and shows convergence of \( \gamma(k) \).
\end{itemize}

\section*{Where the \(p\)-bins enter the algorithm}

\begin{enumerate}
\item \textbf{Streaming accumulator}
\begin{verbatim}
for each p-bin m:
    S_cell(m) = Σ_i∈bin w_i Δs_i / (Δt A_cell)
\end{verbatim}
The scalar flux \( S(p_m) \) is now a smooth function of \( p_m \).

\item \textbf{Projection onto \(k\)-bins} (half-flux rule)

For each \(k\)-bin centre \(k_l\) and \(p_m\):
\[
S_{+,l} \mathrel{+}= \frac{1}{2}\,S(m) \quad
\text{if } |\mu_{\text{res}}^{(+)}(k_l,p_m)| \le 1,
\]
and similarly for the “–” sense.

\item \textbf{Growth / damping}
\[
\gamma_\pm(k_l) =
\frac{\pi^2 e^{2} V_A}{c B_0^2}\,
\frac{S_\pm(k_l)}{k_l}.
\]
Because \( S_\pm(k_l) \) already carries units “flux per \(p\)-bin,” no further integral is required.

\item \textbf{Update of \( \kappa_\parallel \)}
\[
\lambda_\parallel^{-1}(p_m) =
\frac{\pi e^{2}}{B_0^{2}}
\int dk\, \frac{W_{+} + W_{-}}{k}\,
\left(1 - \mu_{\text{res}}^2\right),
\]
is computed once per \(p_m\) and reused for every particle contained in that momentum bin during the next Monte-Carlo step.
\end{enumerate}

\section*{Bin width choice}
\begin{itemize}
\item \textbf{10–15 bins per decade} in \(p\) is enough to keep the resonance mapping \emph{one-to-one} (Ng \& Reames 1994).
\item Make the \(p\)-grid align with the finite-difference grid used in the isotropic Parker step—then the same array serves both the Eulerian solver \emph{and} the Monte-Carlo tally.
\end{itemize}

\section*{What if you skip bins?}
\emph{The code still works}—but
\begin{itemize}
\item instantaneous \( \gamma(k) \) becomes noisy (needs more particles),
\item \( \kappa_\parallel \) has to be recomputed \textbf{per particle} (expensive),
\item output energy channels must be histogrammed after the run.
\end{itemize}

Hence almost every operational SEP model (SOLPENCO, Shock-Particle-Transport Code, iPATH, etc.) chooses the modest memory overhead of 50–100 \(p\)-bins to gain speed, stability, and direct comparability with data.

\bigskip

\noindent\textbf{Bottom line:} \\
Momentum binning is \textbf{not mathematically mandatory}, yet it is the standard engineering solution that keeps streaming, growth/damping, and diffusion-coefficient updates numerically robust and observation-ready.


\section*{``With a fixed Kolmogorov spectrum the wave field is only one scalar per cell – how do I still use the Parker-flux $S(p)$ to obtain growth and damping?''}

\subsection*{1. What “Kolmogorov spectrum” really means in transport codes}

Most SEP/wave solvers do \textbf{not} store \( W(k) \) in many bins when they assume an inertial-range Kolmogorov shape; instead they keep \textbf{one amplitude}
\[
A(s,t) \quad \text{such that} \quad 
W(k,s,t) = A(s,t)\,k^{-5/3}, \qquad k_{\min} \leq k \leq k_{\max}. \tag{1}
\]
The total magnetic wave–energy density in the cell is
\[
W_{\text{tot}}(s,t) = A(s,t)\,N_k,
\qquad
N_k = \int_{k_{\min}}^{k_{\max}} k^{-5/3}\,dk
     = \frac{3}{2} \frac{k_{\max}^{-2/3} - k_{\min}^{-2/3}}{k_{\min}^{-5/3}}. \tag{2}
\]
Thus, \textbf{you evolve only \(A\)}; the \(k^{-5/3}\) shape is frozen in.

\subsection*{2. How \(S(p)\) couples to that single amplitude}

Write the QLT growth/damping term (Jokipii 1966) with the Kolmogorov \(W(k)\):
\[
\frac{dW_{\pm}}{dt}
      = 2 \int_{k_{\min}}^{k_{\max}} \gamma_\pm(k)\,W(k)\,dk
      = \frac{2\pi^{2}e^{2}V_A}{cB_0^{2}}\, A \int_{k_{\min}}^{k_{\max}} \frac{S_\pm(k)}{k^{8/3}}\,dk. \tag{3}
\]
Evaluate \(S_\pm(k)\) with the \(\delta\)-function resonance:
\[
S_\pm(k) = \int dp\, S(p)\,
          \delta\left(k - \frac{\Omega}{v\mu \mp V_A}\right)
       \xrightarrow[\text{isotropic}]{\mu = \pm V_A/v}
       \frac{1}{v} S(p) \left| \frac{dk}{dp} \right|^{-1}_{p = p_{\rm res}}. \tag{4}
\]
Combine (3)–(4) and change variables from \(k\) to resonant momentum \(p\). The result is \textbf{an analytic, positive kernel}:
\[
\boxed{\, \frac{dA}{dt} = C
               \, A(s,t)
               \sum_{p_m} 
               \frac{S_{\text{sign}}(p_m,s,t)}
                    {v_m^{8/3}}\,
               k_{\text{res}}(p_m)^{-2/3}\, \Delta p \,}
\qquad
C = \frac{\pi^{2} e^{2} V_A}{c B_0^{2} N_k}. \tag{5}
\]
\begin{itemize}
\item ``sign'' = choose \(+\) or \(-\) according to the wave sense that resonates.
\item No division by zero occurs because \(k_{\text{res}}\) is always finite for any finite \(p_m\).
\end{itemize}

\subsection*{3. Full finite-difference update for the amplitude}
\[
\boxed{\, A^{n+1} = A^{n} \left[ 1 + 2 \Gamma_{\pm} \Delta t
                   - \frac{\Delta t}{\tau_{\text{cas}}} \right]
          + Q_{\text{shock}}\, \Delta t \,}. \tag{6}
\]
\begin{itemize}
\item \( \Gamma_{\pm} \) is the boxed sum in (5).
\item \( \tau_{\text{cas}} \) – Kolmogorov cascade sink time
\[
\tau_{\text{cas}}^{-1} = C_K \sqrt{ \frac{2 A N_k}{\rho} } / L_\perp, \quad C_K \approx 0.2.
\]
\item \( Q_{\text{shock}} \) – ramp injection:
\[
Q_{\text{shock}} = \frac{ \eta \rho (V_{\text{sh}} - U)^3 }{ 2 N_k }.
\]
\end{itemize}

\subsection*{4. Particle algorithm is unchanged – only the wave step is simpler}
\begin{tabbing}
\hspace{2em} \= \texttt{for each global } $\Delta t$: \\
\> \hspace{2em} \= \texttt{\# Monte-Carlo: ballistic + } $\sqrt{2\kappa\Delta t}$ \texttt{ step $\rightarrow$ scalar Parker flux } $S(p)$ \\
\> \texttt{compute } $\Gamma_{\pm}$ \texttt{ via Eq. (5)} \\
\> \texttt{update amplitude } $A$ \texttt{ with Eq. (6)} \\
\> \texttt{recompute } $\lambda_\parallel(p) = \dfrac{3v}{\pi e^2 A / B_0^2 \cdot \left( k_{\min}^{-2/3} - k_{\max}^{-2/3} \right)}$ \\
\> \texttt{compute } $\kappa_\parallel = \dfrac{v \lambda_\parallel}{3}$
\end{tabbing}
Because \(A\) multiplies the Kolmogorov kernel explicitly, the spatial diffusion coefficient returns to the familiar analytic form \( \kappa_\parallel \propto A^{-1} \).

\subsection*{5. Why binning in \(p\) is still handy}

Even though \(\gamma\) is now \textbf{an integral over \(p\) only}, you still benefit from momentum bins because:
\begin{itemize}
\item You need \(S(p)\) on a grid to compute the discrete sum in (5).
\item The updated \(\kappa_\parallel(p)\) is piece-wise constant in those same bins.
\item You can output intensity spectra channel-by-channel.
\end{itemize}
But you do \textbf{not} need any \(k\)-bin array; the turbulence solver carries just the \emph{scalar} \(A(s,t)\).

\subsection*{References for ``one-amplitude Kolmogorov'' implementations}
\begin{itemize}
\item Lee, M.~A. 2005, \textit{ApJS} 158, 38 — derives Eq.~(5) for a \(k\)-power-law.
\item Zank, Rice \& Wu 2000, \textit{JGR} 105, 25079 — Section 3.3, ``single-amplitude turbulence''.
\item Ng, Reames \& Tylka 2012, \textit{A\&A} 543, A68 — uses exactly the update (6).
\end{itemize}

\bigskip

\noindent These papers show that once the Kolmogorov shape is fixed, you only carry one amplitude per spatial cell; streaming from Parker’s scalar flux is still all that is needed to grow or damp that amplitude self-consistently.


\section*{Change of Transport Equation When Switching to a Moving Mesh}

\subsection*{Yes — the transport equation \emph{changes in form but not in physics} when you switch from a fixed (Eulerian) field-line mesh to a mesh that rides with the solar-wind bulk flow (Lagrangian).}

The difference is entirely in the \textbf{advection term}; all source/sink terms (growth, damping, cascade, shock‐ramp injection) stay the same.

\subsection*{1. Starting Point – Eulerian (Static) Field-Line Mesh}

For Alfvén waves that propagate parallel (+) or anti-parallel (–) to the mean field, the one-dimensional transport equation used in most SEP codes is:

\begin{equation}
\boxed{%
\frac{\partial W_\pm}{\partial t}
+\left(U\;\pm\;V_A\right)\frac{\partial W_\pm}{\partial s}
=2\gamma_\pm W_\pm
-\frac{W_\pm}{\tau_{\rm cas}}
+Q_{\text{shock}} } \tag{E}
\end{equation}

\begin{itemize}
\item $U(s)$ — solar-wind speed \textbf{along} the field line.
\item $\pm V_A$ — group speed of Alfvén waves \textbf{relative to the plasma}.
\item $2\gamma_\pm W_\pm$ — growth or damping driven by SEP streaming.
\item $W/\tau_{\text{cas}}$ — Kolmogorov cascade sink.
\item $Q_{\text{shock}}$ — broadband power injected in the first downstream cell.
\end{itemize}

\subsection*{2. Transform to a Mesh That Moves With the Wind}

Define a comoving (material) derivative:

\[
\frac{d}{dt}\equiv
\frac{\partial}{\partial t}
+U\frac{\partial}{\partial s},
\]

and a new spatial coordinate labeling a fixed plasma element:

\[
s' = s - \int^t U\,dt.
\]

Insert:

\[
\frac{\partial}{\partial t} = \frac{d}{dt} - U\frac{\partial}{\partial s}
\]

into (E):

\[
\frac{dW_\pm}{dt}
+\underbrace{\left(U\pm V_A - U\right)}_{= \pm V_A}\,
  \frac{\partial W_\pm}{\partial s}
=2\gamma_\pm W_\pm-\frac{W_\pm}{\tau_{\rm cas}}+Q_{\rm shock}.
\]

Because the mesh already follows $U$, the \textbf{wind speed cancels out}. Writing $s$ again for the comoving arc-length coordinate, we obtain the \textbf{Lagrangian form}:

\begin{equation}
\boxed{%
\frac{dW_\pm}{dt}
\;\pm\;V_A\,\frac{\partial W_\pm}{\partial s}
=2\gamma_\pm W_\pm
-\frac{W_\pm}{\tau_{\rm cas}}
+Q_{\text{shock}} } \tag{L}
\end{equation}

\begin{itemize}
\item \textit{The only change is that the advection speed is now just} $\pm V_A$.
\end{itemize}

\subsection*{3. Practical Implementation Differences}

\begin{table}[h!]
\centering
\begin{tabular}{|p{4cm}|p{5.5cm}|p{5.5cm}|}
\hline
\textbf{Numerical Step} & \textbf{Eulerian Mesh (Eq.~E)} & \textbf{Advecting Mesh (Eq.~L)} \\
\hline
\textbf{Vertex array} & Fixed & Move every vertex by $+U\Delta t$ each step \\
\hline
\textbf{Particle drift} & Add $+U\Delta t$ to each particle’s $s$ & \textit{No} $U\Delta t$ term — only diffusion step \\
\hline
\textbf{Wave flux term in FV solver} & Upwind with speed $U\pm V_A$ & Upwind with speed $\pm V_A$ \\
\hline
\textbf{CFL limit} & $\left(U + V_A\right)\Delta t < \Delta s_{\min}$ & $V_A\,\Delta t < \Delta s_{\min}$ (larger stable $\Delta t$) \\
\hline
\end{tabular}
\end{table}

\smallskip

Everything else — growth $2\gamma_\pm W_\pm$, Kolmogorov cascade, shock-ramp injection, $\kappa_\parallel$ update — uses identical formulas in both frameworks.

\subsection*{4. Why the Physics is Identical}

\begin{itemize}
\item The plasma and its frozen-in field line really do travel outward at $U$.
\item In an Eulerian description you let the \textbf{waves and particles} carry that motion.
\item In a Lagrangian description you let the \textbf{mesh} carry it, so waves only need their intrinsic $\pm V_A$ propagation to move relative to that mesh.
\end{itemize}

Both yield the same lab-frame wave power $W_\pm(s,t)$; the difference is purely a frame choice.

\subsection*{Key Papers Showing Each Form}

\begin{itemize}
\item \textbf{Eq. (E)} (static mesh): Ng \& Reames 1994; Dröge 2010.
\item \textbf{Eq. (L)} (moving mesh): Zank, Rice \& Wu 2000; Afanasiev et al. 2015.
\end{itemize}

\bigskip

Switching from (E) to (L) therefore means: \textbf{remove the $+U\Delta t$ drift from every particle and drop $U$ from the wave advection speed}—the rest of your coupled Monte-Carlo Parker + turbulence solver stays the same.


\section{Init Alfven turbulence wave energy density}

Below is a compact “rule-of-thumb” profile for the \textbf{magnetic-fluctuation level of Alfvénic turbulence} in the heliosphere, expressed as the \textbf{r.m.s. amplitude normalized to the local mean field}

\[
\frac{\delta B_{\mathrm{rms}}}{B_0}
     \equiv
     \frac{\sqrt{\langle (B-\langle B\rangle)^2\rangle}}{\langle B\rangle}.
\]

\begin{tabular}{|l|c|l|}
\hline
\textbf{Typical distance (fast-wind values)} & $\delta B_{\mathrm{rms}}/B_0$ & \textbf{Main in-situ evidence} \\
\hline
\textbf{27--60 $R_\odot$ (0.05--0.13 au)} & 0.15 -- 0.30 & Parker Solar Probe E4/E5 statistics ($\delta$B/B panel in Fig. 2 of \textbf{Matteini et al. 2021}) \href{https://www.aanda.org/articles/aa/full_html/2021/06/aa39872-20/aa39872-20.html}{(A\&A)} \\
\hline
\textbf{0.3 au} & $\approx 0.05$ & Helios 2 triple-stream study; radial decay law $\delta B \propto r^{-1.5}$ for fast wind (Fig. 1 of \textbf{Sorriso-Valvo et al. 2023}) \href{https://www.aanda.org/articles/aa/full_html/2023/04/aa44889-22/aa44889-22.html}{(A\&A)} \\
\hline
\textbf{0.7 au} & $\approx 0.03$ & Same Helios dataset; extrapolation of the $r^{-1.5}$ trend \href{https://www.aanda.org/articles/aa/full_html/2023/04/aa44889-22/aa44889-22.html}{(A\&A)} \\
\hline
\textbf{1 au} & $0.02 \pm 0.01$ & ACE/Wind long-term survey; Bruno \& Carbone 2013 review (Living Reviews \S 4.2) \href{https://link.springer.com/article/10.12942/lrsp-2013-2}{(Living Reviews)} \\
\hline
\textbf{3--5 au} & 0.004 -- 0.008 & Ulysses polar passes and Voyager-2; outer-heliosphere turbulence review (Fig. 5 of \textbf{Oughton \& Engelbrecht 2022}) \href{https://link.springer.com/article/10.1007/s11214-022-00914-2}{(Space Sci. Rev.)} \\
\hline
\end{tabular}

\section*{Empirical radial law}

Fast, Alfvénic wind streams follow approximately

\[
\boxed{
\frac{\delta B_{\mathrm{rms}}}{B_0}(r) \simeq 0.05
\left( \frac{r}{0.3\,\mathrm{au}} \right)^{-0.5}
\quad (0.05\!-\!5\,\mathrm{au})
}
\]

\begin{itemize}
\item Derived by fitting the Helios and PSP points to the Voyager/Ulysses values (slope $\simeq -0.48$).
\item Slow wind decays more gently, $\approx r^{-0.3}$ (Tu \& Marsch 1995; not shown).
\end{itemize}

\section*{Converting to \textbf{Elsässer energy}}

Because
\[
\frac{\delta B_{\mathrm{rms}}}{B_0} \simeq \frac{\langle |{\bf z}^{\pm}|^2 \rangle^{1/2}}{V_A}
\]
for highly Alfvénic intervals, the same scaling implies

\[
Z_\pm^2 \propto r^{-1.5\pm0.2}
\]

--- the textbook result originally found by \textbf{Marsch \& Tu 1990} for 0.3--1 au fast wind (confirmed by Helios re-analysis) \href{https://www.aanda.org/articles/aa/full_html/2025/01/aa51686-24/aa51686-24.html}{(A\&A)}.

\section*{How to use these numbers}

\begin{itemize}
\item \textbf{Initial condition for SEP/wave simulations} -- take $\delta B/B_0 \approx 0.05$ at 0.3 au, scale with $r^{-1/2}$ outward, and adjust inward with PSP’s 0.1 au value.
\item \textbf{Check of self-generated waves} -- if your code’s $\delta B/B_0$ exceeds the table by an order of magnitude upstream of a shock, you are firmly in the wave-growth regime; if it stays below, damping dominates.
\item \textbf{Energy density} -- multiply $\delta B_{\mathrm{rms}}^2 / (2\mu_0)$ by the cross-section to get J m$^{-1}$.
\end{itemize}

\section*{Key References}

\begin{enumerate}
\item \textbf{Matteini et al. 2021}, A\&A 398 A72 -- PSP turbulence at 27 $R_\odot$. \href{https://www.aanda.org/articles/aa/full_html/2021/06/aa39872-20/aa39872-20.html}{(link)}
\item \textbf{Sorriso-Valvo et al. 2023}, A\&A 672 A13 -- Helios 0.3--0.9 au decay law. \href{https://www.aanda.org/articles/aa/full_html/2023/04/aa44889-22/aa44889-22.html}{(link)}
\item \textbf{Bruno \& Carbone 2013}, Living Rev. Sol. Phys. 10, 2 -- 1 au statistics. \href{https://link.springer.com/article/10.12942/lrsp-2013-2}{(link)}
\item \textbf{Oughton \& Engelbrecht 2022}, Space Sci. Rev. 218, 29 -- outer-heliosphere review. \href{https://link.springer.com/article/10.1007/s11214-022-00914-2}{(link)}
\item \textbf{Marsch \& Tu 1990}, JGR 95, 11945 -- original $r^{-1.5}$ scaling (re-analysed in modern studies). \href{https://www.aanda.org/articles/aa/full_html/2025/01/aa51686-24/aa51686-24.html}{(link)}
\end{enumerate}

These studies collectively map the “typical” Alfvén-turbulence amplitude from the corona to the outer heliosphere.



======


\section*{Algorithm: Splitting the Scalar Parker‐Flux $S(p)$ into the Two Resonant Wave-Fluxes $S_{+}(k)$ and $S_{-}(k)$}

\subsection*{Inputs (per spatial cell $j$)}

\begin{tabular}{ll}
\textbf{Symbol}       & \textbf{Description} \\
\hline
$S(p_m)$              & Scalar Parker flux in momentum bin $p_m$\quad [particles m$^{-2}$ s$^{-1}$ ($\Delta$p)$^{-1}$] \\
$k_\ell$              & Centres of the logarithmic $k$-bins ($\ell = 0 \dots N_k{-}1$) \\
$B_0,\ \rho$          & Mean field strength and density in the cell \\
Constants             & Proton charge $e$, mass $m_p$, Alfvén speed $V_A = B_0 / \sqrt{\mu_0 \rho}$ \\
\end{tabular}

\subsection*{Step 1: Pre-compute Resonance Lookup Tables}

For every momentum bin $p_m$:

\begin{align*}
v_m       &= \frac{p_m}{m_p} \quad \text{(non-relativistic speed)} \\
\Omega_m  &= \frac{e B_0}{m_p} \quad \text{(gyro-frequency)} \\
\mu_{\rm res} &= \frac{V_A}{v_m} \quad \text{(resonant pitch angle for isotropic Parker flux)} \\
k^{+}_{\rm res}(m) &= \frac{\Omega_m}{v_m \mu_{\rm res} + V_A} \quad \text{(outward wave)} \\
k^{-}_{\rm res}(m) &= \frac{\Omega_m}{v_m (-\mu_{\rm res}) + V_A} \quad \text{(inward wave)}
\end{align*}

\textit{(If $\lvert \mu_{\rm res} \rvert > 1$ the particle cannot resonate with that sense; flag it invalid.)}

\subsection*{Step 2: Zero the Wave-Flux Accumulators}

\[
\begin{aligned}
\text{for } \ell &= 0 \dots N_k{-}1: \\
&\quad S_{+}[\ell] \leftarrow 0 \\
&\quad S_{-}[\ell] \leftarrow 0
\end{aligned}
\]


\subsection*{Step 3: Loop Over Momentum Bins – Deposit Half-Flux into Matching $k$-Bin}

\[
\begin{aligned}
&\text{for each } m \text{ (momentum bin):} \\
&\quad S_{\text{scalar}} \leftarrow S(p_m) \quad \text{(signed scalar flux)} \\
\\
&\quad \text{\# --- outward (+) wave ---} \\
&\quad \text{if } k_{\min} \leq k^{+}_{\text{res}}(m) \leq k_{\max} \text{ and } |\mu_{\text{res}}| \leq 1: \\
&\qquad \ell \leftarrow \text{nearest\_k\_bin}(k^{+}_{\text{res}}(m)) \\
&\qquad S_{+}[\ell] \mathrel{+}= 0.5 \cdot S_{\text{scalar}} \\
\\
&\quad \text{\# --- inward (–) wave ---} \\
&\quad \text{if } k_{\min} \leq k^{-}_{\text{res}}(m) \leq k_{\max} \text{ and } |\mu_{\text{res}}| \leq 1: \\
&\qquad \ell \leftarrow \text{nearest\_k\_bin}(k^{-}_{\text{res}}(m)) \\
&\qquad S_{-}[\ell] \mathrel{+}= 0.5 \cdot S_{\text{scalar}}
\end{aligned}
\]


\textit{If only one wave sense has a valid resonance, deposit the \textbf{full} flux into that sense and skip the other.}

\subsection*{Step 4: (Optional) CIC Weighting Instead of Nearest-Bin}

To reduce bin-edge noise, replace \texttt{nearest\_k\_bin} with a simple cloud-in-cell scheme:

\[
\begin{aligned}
\ell_0,\ \ell_1 &\leftarrow \text{bounding $k$ bins} \\
w_1 &\leftarrow \frac{k_{\text{res}} - k_{\ell_0}}{k_{\ell_1} - k_{\ell_0}} \\
S_{+}[\ell_0] &\mathrel{+}= (1 - w_1) \cdot \text{flux} \\
S_{+}[\ell_1] &\mathrel{+}= w_1 \cdot \text{flux}
\end{aligned}
\]


Do likewise for $S_{-}$.

\subsection*{Outputs}

Arrays $S_{+}(k_\ell)$ and $S_{-}(k_\ell)$ in units of particles m$^{-2}$ s$^{-1}$ ($Delta$p)$^{-1}$, which go directly into the quasilinear growth/damping rate:

\[
\gamma_\pm(k_\ell) = \frac{\pi^{2} e^{2} V_A}{c B_0^{2}} \cdot \frac{S_\pm(k_\ell)}{k_\ell}
\]

\subsection*{Why the Half-Flux Rule is Correct}

For an isotropic distribution, the $\delta$-function resonance picks two equal-magnitude pitch-angle roots ($\pm \mu_{\rm res}$).  
Each root interacts with the wave propagating \textit{opposite} to the particle’s streaming direction.  
Splitting the scalar Parker flux evenly between the two senses reproduces the exact QLT streaming integral (Jokipii 1966; Ng \& Reames 1994).


=======


\section*{Self-Contained Algorithm: Scalar Parker Flux to Wave-Sense Flux Arrays}

Below is a \textbf{single, self-contained algorithm} that converts the \textbf{scalar Parker flux} $S(p)$ (one value per momentum bin) into the two \textbf{wave-sense flux arrays} $S_{+}(k)$ and $S_{-}(k)$ that enter the quasilinear growth/damping rate.

\subsection*{A. Inputs (per spatial cell $j$)}

\begin{center}
\renewcommand{\arraystretch}{1.2}
\begin{tabular}{|l|l|l|l|}
\hline
\textbf{Symbol} & \textbf{Type / Size} & \textbf{Units} & \textbf{Comment} \\
\hline
\texttt{p[m]} ($m = 0 \dots N_p{-}1$) & momentum-grid centres & kg m s$^{-1}$ & log-spaced is best \\
\texttt{S[m]} & scalar Parker flux in bin $m$ & particles m$^{-2}$ s$^{-1}$ ($\Delta$p)$^{-1}$ & signed (+ anti-Sun, – Sun) \\
\texttt{k[l]} ($l = 0 \dots N_k{-}1$) & centres of log-$k$ grid & rad m$^{-1}$ & \\
\texttt{B0} & mean magnetic field in cell & tesla & \\
\texttt{rho} & plasma mass density & kg m$^{-3}$ & \\
Physical constants & $e$, $m_p$, $\mu_0$, $c$, $\Omega_\odot$ & & \\
\hline
\end{tabular}
\end{center}

\subsection*{B. Pre-compute Auxiliary Numbers (Once per Cell)}

\[
\begin{aligned}
V_A &= \frac{B_0}{\sqrt{\mu_0 \rho}} \quad \text{(Alfvén speed)} \\
\Omega &= \frac{e B_0}{m_p} \quad \text{(proton gyrofrequency)} \\
k_{\min} &= k[0], \quad k_{\max} = k[N_k{-}1]
\end{aligned}
\]

\subsection*{C. Initialise Output Arrays}

\[
\begin{aligned}
\text{for } l &= 0 \dots N_k{-}1: \\
&\quad S_{+}[l] \leftarrow 0 \quad \text{(outward-propagating waves)} \\
&\quad S_{-}[l] \leftarrow 0 \quad \text{(inward-propagating waves)}
\end{aligned}
\]

\subsection*{D. Momentum Loop – Deposit Half (or All) Flux into the Matching $k$-Bin}

\[
\begin{aligned}
\text{for } m &= 0 \dots N_p{-}1: \\
v &\leftarrow \frac{p[m]}{m_p} \\
\text{if } v &< 10^{-10}: \quad \text{continue (avoid division by zero)} \\
\mu_{\text{res}} &\leftarrow \frac{V_A}{v} \quad \text{(isotropic Parker)} \\
\\
\text{\# outward (+) resonance:} \\
k_{+} &\leftarrow \frac{\Omega}{v \mu_{\text{res}} + V_A} \\
\text{if } \mu_{\text{res}} \leq 1 \text{ and } k_{\min} \leq k_{+} \leq k_{\max}: \\
&\quad l \leftarrow \text{nearest\_logk\_bin}(k_{+}) \\
&\quad S_{+}[l] \mathrel{+}= 0.5 \cdot S[m] \\
\\
\text{\# inward (–) resonance:} \\
k_{-} &\leftarrow \frac{\Omega}{-v \mu_{\text{res}} + V_A} \\
\text{if } \mu_{\text{res}} \leq 1 \text{ and } k_{\min} \leq k_{-} \leq k_{\max}: \\
&\quad l \leftarrow \text{nearest\_logk\_bin}(k_{-}) \\
&\quad S_{-}[l] \mathrel{+}= 0.5 \cdot S[m] \\
\\
\text{\# if } v < V_A: \quad \text{(only one valid resonance)} \\
\text{if } k_{+} \text{ in range: } &\quad S_{+}[l] \mathrel{+}= 0.5 \cdot S[m] \\
\text{else if } k_{-} \text{ in range: } &\quad S_{-}[l] \mathrel{+}= 0.5 \cdot S[m]
\end{aligned}
\]

\subsubsection*{Nearest Bin Finder (Log-Grid)}
\[
\begin{aligned}
l &\leftarrow \text{round}\left( \frac{\log k - \log k_{\min}}{\Delta \log k} \right) \\
l &\leftarrow \min\left( \max(l, 0), N_k{-}1 \right)
\end{aligned}
\]

\emph{Optional: replace with CIC weighting for reduced bin-edge noise.}

\subsection*{E. Outputs}

Arrays \texttt{Splus[l]}, \texttt{Sminus[l]} — signed wave-sense fluxes in each $k$-bin  
(units: particles m$^{-2}$ s$^{-1}$ ($\Delta$p)$^{-1}$), which directly enter the quasilinear growth/damping expression:

\[
\gamma_\pm(k_l) = \frac{\pi^2 e^2 V_A}{c B_0^2} \cdot \frac{S_\pm(k_l)}{k_l}
\]

\subsection*{Notes and Justification}

\begin{enumerate}
\item \textbf{Half-flux rule:} For an \emph{isotropic} distribution, the $\delta$-function resonance produces two equal pitch-angle roots; each couples to a wave propagating in the opposite direction to particle streaming (Jokipii 1966; Ng \& Reames 1994).

\item \textbf{If $v < V_A$:} One of the roots violates $|\mu| \leq 1$ and is invalid; the full scalar flux is deposited in the single valid wave sense (Lee 1983).

\item \textbf{Log-$p$ and log-$k$ grids:} Enable accurate integration of power-law spectra using only modest bin counts (12–20 bins per decade).
\end{enumerate}

\medskip

\noindent This compact algorithm is the standard approach in modern Monte Carlo SEP + wave models (e.g., Zank et al. 2000; Afanasiev et al. 2015) for converting Parker flux arrays into resonant streaming that drives Alfvén-wave growth and damping.



===============================================================


\section*{Removing Energy from Particles When the Wave Field Gains Energy}

\noindent
\emph{(i.e., when $\Delta W > 0$ in the amplitude step)}

\subsection*{1. Decide Which Particles Pay the Bill}

Give the energy loss only to the particles that actually \textbf{drove} the wave change—the same ones whose pitch-angle or streaming weight $G_i$ you used for growth:

\[
G_i = w_i\,|p_{\parallel i}| \quad\Longrightarrow\quad \Delta E_i \propto G_i.
\]

\subsection*{2. Compute the Total Energy That Must Be Removed}

\[
E_{\rm wave\,gain}
   = \left( 2\gamma_\pm A_\pm + Q_{\rm shock} - \frac{A_\pm}{\tau_{\rm cas}} \right)
     \Delta t \cdot V_{\rm cell},
\]
where the cell volume is
\[
V_{\rm cell} = \frac{L_j}{3} \left( A_{\rm L} + \sqrt{A_{\rm L} A_{\rm R}} + A_{\rm R} \right).
\]

\subsection*{3. Distribute the Loss Proportionally to $G_i$}

\begin{align*}
&\text{Total streaming weight (resonant group only):} \\
G_{\text{tot}} &= \sum_i G_i \\
\\
&\text{Energy per unit } G: \\
\Delta E_{\text{per } G} &= \frac{E_{\text{wave gain}}}{G_{\text{tot}}} \\
\\
&\text{For each particle } i \text{ in the resonant set:} \\
\Delta E_i &= \Delta E_{\text{per } G} \cdot G_i \\
v_i^2 &\leftarrow v_i^2 - \frac{2 \Delta E_i}{m_p} \\
\\
&\text{If } v_i^2 < v_{\text{floor}}^2 \text{ (too much energy removed):} \\
\Delta E_i &\leftarrow \tfrac{1}{2} m_p \left(v_i^2 - v_{\text{floor}}^2\right) \\
v_i^2 &\leftarrow v_{\text{floor}}^2 \\
\\
&\text{Update speed:} \\
v_i &\leftarrow \sqrt{v_i^2} \\
\\
&\text{Accumulate total energy removed:} \\
E_{\text{removed}} &\leftarrow E_{\text{removed}} + \Delta E_i
\end{align*}

\begin{itemize}
    \item \texttt{v\_floor} — a safety floor, e.g., the ambient thermal speed:
    \[
    v_{\rm th} = \sqrt{\frac{2kT}{m_p}} \sim 50\;{\rm km\,s^{-1}} \text{ at 1~AU},
    \]
    or a low numerical floor such as $10^3\;{\rm m\,s^{-1}}$.
    \item If a particle hits the floor, the loop keeps a running tally \texttt{energyRemoved}. Any shortfall is redistributed among remaining particles in a second pass, ensuring:
    \[
    \sum \Delta E_i = E_{\rm wave\,gain}.
    \]
\end{itemize}

\subsection*{4. Guarantee Non-negative Speeds}

Always verify:
\[
v^2_{\rm new} = v^2_{\rm old} - \frac{2\Delta E_i}{m_p} > 0.
\]
If this expression would go negative, clip $v_i$ to \texttt{v\_floor} and redistribute the remaining energy.

\subsection*{5. Dependence on Particle Energy}

The fractional energy loss per particle is:
\[
\frac{\Delta E_i}{E_{k,i}}
= \frac{G_i}{E_{k,i}} \cdot \frac{E_{\rm wave\,gain}}{G_{\rm tot}}
= \frac{|p_{\parallel i}|}{\frac{1}{2} m_p v_i^2}
  \cdot \frac{w_i \cdot E_{\rm wave\,gain}}{G_{\rm tot}}.
\]

Thus, \textbf{higher-energy particles lose less fractionally}, but if their streaming weight $w_i |p_{\parallel i}|$ is large, they still lose a significant absolute amount.

\subsection*{6. Summary Algorithm}

\begin{enumerate}
    \item Build the resonant list (same test used to form $S_\pm$).
    \item Compute $E_{\rm wave\,gain}$.
    \item First pass: subtract $\Delta E_i \propto G_i$ but do \textbf{not} allow $v_i < v_{\rm floor}$; keep a tally of energy removed.
    \item If the tally $< E_{\rm wave\,gain}$, do a \textbf{second pass} over particles still above the floor and redistribute remaining energy equally or proportionally.
    \item Confirm:
    \[
    \sum_i \Delta E_i = E_{\rm wave\,gain} \quad \text{(to machine precision)}.
    \]
\end{enumerate}

\bigskip

\noindent
This procedure preserves energy, prevents non-physical speeds, and ensures that energy is removed in proportion to each particle’s parallel momentum—precisely mirroring the resonance condition that produced wave growth in the first place.


=======================================

\section*{Finite-volume Advection Step for Kolmogorov Turbulence Amplitudes}

\noindent
\emph{(One scalar per propagation sense, one value per segment)}

\subsection*{Definitions}

\begin{center}
\renewcommand{\arraystretch}{1.3}
\begin{tabular}{@{}ll@{}}
\toprule
\textbf{Symbol} & \textbf{Meaning} \\
\midrule
$A_{+,j},\,A_{-,j}$ & Kolmogorov amplitudes in segment $j$ (J m$^{-2/3}$) \\
$U_j$ & Solar wind bulk speed in segment $j$ \\
$V_{A,j}$ & Alfvén speed in segment $j$ \\
$c_{+,j} = V_{A,j} - U_j$ & Upwind speed for \textbf{outward} waves \\
$c_{-,j} = -V_{A,j} - U_j$ & Upwind speed for \textbf{inward} waves \\
$L_j$ & Segment length \\
$\Delta t$ & Global time step \\
\bottomrule
\end{tabular}
\end{center}

\medskip

We advance the pure advection equation using a conservative upwind finite-volume scheme:
\begin{equation}
\frac{\partial A_\pm}{\partial t}
\;+\;
c_\pm\,\frac{\partial A_\pm}{\partial s} = 0
\end{equation}

\subsection*{1. Compute Face-centred Speeds}
\begin{align*}
c_+ &= V_{A,j} - U_j \\
c_- &= -V_{A,j} - U_j
\end{align*}

\subsection*{2. Upwind Fluxes at Left and Right Faces}

\textbf{Left face (between $j-1$ and $j$):}
\begin{align*}
F_{+,L} &= 
\begin{cases}
c_+ \cdot A_{+,j-1}, & c_+ > 0 \\
c_+ \cdot A_{+,j},   & c_+ \leq 0
\end{cases} \\
F_{-,L} &= 
\begin{cases}
c_- \cdot A_{-,j-1}, & c_- > 0 \\
c_- \cdot A_{-,j},   & c_- \leq 0
\end{cases}
\end{align*}

\textbf{Right face (between $j$ and $j+1$):}
\begin{align*}
F_{+,R} &= 
\begin{cases}
c_+ \cdot A_{+,j},   & c_+ > 0 \\
c_+ \cdot A_{+,j+1}, & c_+ \leq 0
\end{cases} \\
F_{-,R} &= 
\begin{cases}
c_- \cdot A_{-,j},   & c_- > 0 \\
c_- \cdot A_{-,j+1}, & c_- \leq 0
\end{cases}
\end{align*}

\textit{Note:} At boundaries ($j = 0$ or $j = N - 1$), supply boundary values or set the flux to zero.

\subsection*{3. Conservative Update}

Assuming uniform cross-section:
\[
\text{invVol} = \frac{1}{L_j}
\]
\begin{align*}
A_{+,j}^{n+1} &= A_{+,j}^{n} + \left(F_{+,L} - F_{+,R}\right) \cdot \Delta t \cdot \text{invVol} \\
A_{-,j}^{n+1} &= A_{-,j}^{n} + \left(F_{-,L} - F_{-,R}\right) \cdot \Delta t \cdot \text{invVol}
\end{align*}

\subsection*{4. Stability Condition}

CFL criterion for first-order upwind:
\[
\max |c_\pm| \cdot \Delta t < 0.5 \, L_{\min}
\]

\subsection*{5. Complete Operator-split Step}
\begin{itemize}
    \item (1) Advection (above)
    \item (2) Growth/Damping: \quad $A \mathrel{+}= 2\gamma A \cdot \Delta t$
    \item (3) Kolmogorov Cascade: \quad $A \mathrel{-}= \frac{A}{\tau_{\text{cas}}} \cdot \Delta t$
    \item (4) Shock Injection: \quad $A \mathrel{+}= Q_{\text{shock}} \cdot \Delta t$
\end{itemize}

All steps are first-order in time. To achieve second-order accuracy, alternate the sequence at each step (Strang splitting), although most SEP codes use the simpler one-way operator split shown above.

\subsection*{6. Boundary Options}

\begin{itemize}
    \item \textbf{Sunward (inner) boundary:} Reflective for $A_-$, outflow for $A_+$
    \item \textbf{Outer boundary (e.g., 5 AU):} Fixed value: $A_\pm = A_{\rm BG}$
    \item \textbf{Shock cell:} Advection uses \emph{post-growth} values of $A_\pm$ to ensure ramp-injected turbulence is immediately propagated downstream.
\end{itemize}

\medskip

\noindent
This procedure transports Kolmogorov turbulence energy densities along the field line in a manner consistent with both the solar-wind flow and the Alfvén group velocity.


==========================


\section*{Finite-volume Advection on a Lagrangian Field-line Mesh}

\noindent
This is a \textbf{finite-volume advection routine} for a \emph{Lagrangian} field-line mesh—i.e., one in which every vertex is convected outward by the local solar-wind speed $U$.

Because the mesh itself moves with the wind, only the Alfvén group velocity relative to the plasma ($\pm V_A$) transports wave energy between cells. The routine also accounts for geometric stretching of each frustum-shaped cell as the field line expands.

\subsection*{1. Preliminaries and Notation}

\begin{center}
\renewcommand{\arraystretch}{1.3}
\begin{tabular}{@{}ll@{}}
\toprule
\textbf{Symbol} & \textbf{Meaning (at time level $n$)} \\
\midrule
$A_{\pm,j}^n$ & Kolmogorov amplitudes in segment $j$ (J m$^{-2/3}$) \\
$V_j^n$ & Cell volume: $\displaystyle V_j^n = \frac{L_j}{3} \left(A_L + \sqrt{A_L A_R} + A_R\right)$ \\
$V_A^n$ & Alfvén speed at cell center \\
$c_{+,j} = +V_A^n$,\quad $c_{-,j} = -V_A^n$ & Wave speeds relative to the moving plasma \\
$\Delta t$ & Global time step \\
\bottomrule
\end{tabular}
\end{center}

\smallskip
\noindent
Because the mesh moves with $U$, \textbf{no solar-wind speed $U$ appears in the flux speed}.

\subsection*{2. Lagrangian Update Sequence (One Global Step)}

\subsubsection*{Step 0: Move Vertices and Recompute Geometry}

\begin{align*}
\text{For each vertex } k: \quad
&r_k \leftarrow r_k + U_k \cdot \Delta t \quad \text{(radial convection)} \\
&B_k,\; \rho_k \propto r_k^{-2}, \quad A_k = \frac{B_0}{B_k} \quad \text{(cross-section update)}
\\[1em]
\text{For each segment } j: \quad
&L_j = \left\| \mathbf{R}_{j+1} - \mathbf{R}_j \right\| \quad \text{(segment length)} \\
&A_L = A_{\text{left face}}, \quad A_R = A_{\text{right face}} \\
&V_j^{n+1} = \frac{L_j}{3} \left( A_L + \sqrt{A_L A_R} + A_R \right) \quad \text{(frustum volume)} \\
&V_{A,j}^{n+1} = \frac{B_{\text{mid}}}{\sqrt{\mu_0 \rho_{\text{mid}}}} \quad \text{(Alfvén speed)}
\end{align*}

\textbf{Stretch factor:}
\[
S_j = \frac{V_j^{n+1}}{V_j^n}
\]
will be used to conserve energy as the cell expands.

\subsubsection*{Step 1: Upwind Advection (Plasma-frame Group Velocity)}

For each segment $j$:

\begin{verbatim}
double cPlus  =  VA_j;
double cMinus = -VA_j;

// LEFT FACE fluxes (j-1 to j)
double Fp_L = (cPlus > 0) ? cPlus * A_plus[j-1]  : cPlus * A_plus[j];
double Fm_L = (cMinus> 0) ? cMinus* A_minus[j-1] : cMinus* A_minus[j];

// RIGHT FACE fluxes (j to j+1)
double Fp_R = (cPlus > 0) ? cPlus * A_plus[j]    : cPlus * A_plus[j+1];
double Fm_R = (cMinus> 0) ? cMinus* A_minus[j]   : cMinus* A_minus[j+1];

\begin{align*}
\text{(in old volume } V_j^n\text{):} \quad
\Delta A_{+,j} &= \frac{(F_{+,L} - F_{+,R}) \cdot \Delta t}{V_j^n} \\
\Delta A_{-,j} &= \frac{(F_{-,L} - F_{-,R}) \cdot \Delta t}{V_j^n}
\end{align*}

A_plus[j]  += dAp;
A_minus[j] += dAm;
\end{verbatim}

\subsubsection*{Step 2: Geometric Stretching (Adiabatic Dilution)}

\begin{verbatim}
double Sstretch = V_j^n / V_j^{n+1};     // < 1 because V increases
A_plus [j] *= Sstretch;
A_minus[j] *= Sstretch;
\end{verbatim}

If amplitudes are stored per unit \textbf{magnetic flux} ($A/B$), this step can be omitted.

\subsubsection*{Step 3: Add Local Source/Sink Terms}

\begin{align*}
A_{+,j}^{n+1} &= A_{+,j}^{n+1} 
+ \left( 2 \gamma_{+,j} A_{+,j}^{n+1} 
       - \frac{A_{+,j}^{n+1}}{\tau_{\text{cas}}} 
       + Q_{\text{shock},+}[j] \right) \cdot \Delta t \\
\\
A_{-,j}^{n+1} &= A_{-,j}^{n+1} 
+ \left( 2 \gamma_{-,j} A_{-,j}^{n+1} 
       - \frac{A_{-,j}^{n+1}}{\tau_{\text{cas}}} 
       + Q_{\text{shock},-}[j] \right) \cdot \Delta t
\end{align*}

Growth/damping, cascade, and shock injection are local to each cell and independent of mesh motion.

\subsection*{3. Stability and Accuracy}

\begin{itemize}
    \item \textbf{CFL (Lagrangian mesh):}
    \[
    |V_A| \cdot \Delta t < 0.5 \, L_{\min}
    \]
    This is usually less restrictive than in Eulerian schemes since $|U|$ does not appear.
    
    \item \textbf{Stretching constraint:}
    If $S_j \ll 1$ (strong expansion), sub-cycle the rescaling or limit $\Delta t$ so that:
    \[
    S_j > 0.8
    \]
\end{itemize}

\subsection*{4. Energy Conservation Check (Per Cell)}

\[
E_{\text{wave}}^{n+1}
= S_{\text{stretch}} \left( E_{\text{wave}}^n
   + (F_{\text{in}} - F_{\text{out}})\, \Delta t
   + \left( 2\gamma A - \frac{A}{\tau_{\text{cas}}} + Q_{\text{shock}} \right) \Delta t \cdot V_j^n \right)
\]

This matches exactly the bookkeeping in the algorithm, with all global error limited to machine precision.

\subsection*{5. Full Update in Code (Per Cell $j$)}

\begin{align*}
\intertext{\textbf{(0) Update geometry and $V_A$:}}
V_{\text{old}} &= V_j^n \quad \text{(stored from previous step)} \\
V_{\text{new}} &= \text{computeFrustumVolume}(j) \\
S_{\text{stretch}} &= \frac{V_{\text{old}}}{V_{\text{new}}} \\
c_+ &= V_{A,j} \\
c_- &= -V_{A,j} \\
%
\intertext{\textbf{(1) Upwind advection:}}
F_{+,L} &= 
\begin{cases}
c_+ \cdot A_{+,j-1}, & \text{if } c_+ > 0 \\
c_+ \cdot A_{+,j},   & \text{otherwise}
\end{cases} \\
F_{+,R} &= 
\begin{cases}
c_+ \cdot A_{+,j},   & \text{if } c_+ > 0 \\
c_+ \cdot A_{+,j+1}, & \text{otherwise}
\end{cases} \\
F_{-,L} &= 
\begin{cases}
c_- \cdot A_{-,j-1}, & \text{if } c_- > 0 \\
c_- \cdot A_{-,j},   & \text{otherwise}
\end{cases} \\
F_{-,R} &= 
\begin{cases}
c_- \cdot A_{-,j},   & \text{if } c_- > 0 \\
c_- \cdot A_{-,j+1}, & \text{otherwise}
\end{cases} \\
A_{+,j}^{n+1} &= A_{+,j}^n + \frac{(F_{+,L} - F_{+,R}) \, \Delta t}{V_{\text{old}}} \\
A_{-,j}^{n+1} &= A_{-,j}^n + \frac{(F_{-,L} - F_{-,R}) \, \Delta t}{V_{\text{old}}} \\
%
\intertext{\textbf{(2) Geometric dilution:}}
A_{+,j}^{n+1} &\leftarrow A_{+,j}^{n+1} \cdot S_{\text{stretch}} \\
A_{-,j}^{n+1} &\leftarrow A_{-,j}^{n+1} \cdot S_{\text{stretch}} \\
%
\intertext{\textbf{(3) Local source/sink terms:}}
A_{+,j}^{n+1} &\leftarrow A_{+,j}^{n+1} + \left( 2 \gamma_+ A_{+,j}^{n+1} - \frac{A_{+,j}^{n+1}}{\tau_{\text{cas}}} + Q_{\text{shock},+}[j] \right) \cdot \Delta t \\
A_{-,j}^{n+1} &\leftarrow A_{-,j}^{n+1} + \left( 2 \gamma_- A_{-,j}^{n+1} - \frac{A_{-,j}^{n+1}}{\tau_{\text{cas}}} + Q_{\text{shock},-}[j] \right) \cdot \Delta t
\end{align*}


\bigskip

\noindent
This is the \textbf{complete advection-plus-stretch update} for turbulence energy density along a \emph{moving} (Lagrangian) Parker-spiral field line.

