

\section{Turbulence}


\subsection{Wave Transport Equation (with Kolmogorov Spectrum) for Bidirectional Alfvén Waves}

Assuming Alfvén waves propagate along the magnetic field (both {\it toward} and {\it away} from the Sun), the {\it wave energy density} $W_\pm(k, z, t)$ at wavenumber $k$ and position $z$ evolves according to the wave transport equation:

$$
\frac{\partial W_\pm(k, z, t)}{\partial t}
+ \left( V_A \mp U \right) \frac{\partial W_\pm}{\partial z}
= \Gamma_\pm(k) W_\pm(k) - \mathcal{D}_\pm(k) W_\pm(k) - \frac{\partial}{\partial k} \left[ D_{kk}^\pm \frac{\partial W_\pm}{\partial k} \right]
$$
Where: $W_+(k)$ = waves propagating away from the Sun, 
 $W_-(k)$ = waves propagating toward the Sun, 
 $V_A$ = local Alfvén speed, 
 $U$ = solar wind speed (assumed radial), 
 $\Gamma_\pm(k)$ = growth rate from streaming SEPs, 
 $\mathcal{D}_\pm(k)$ = damping rate (e.g., turbulence cascade or cyclotron damping), 
 $D_{kk}^\pm$ = spectral diffusion coefficient for nonlinear cascading (e.g., Kolmogorov-type). 


\subsection{Alfvén Wave Growth/Damping Rate ($\gamma$)}

For self-generated Alfvén waves due to streaming particles:

$$
\gamma(k) = \frac{\pi^2 e^2 v_A}{c B_0^2} \, p v \left[ \frac{\partial f(p, \mu)}{\partial \mu} \right]_{\mu = v_A/v}
$$
Where:
 $\gamma(k)$ = growth (if positive) or damping (if negative) rate for wave number $k$,
 $e$ = elementary charge,
 $v_A$ = Alfvén speed,
 $c$ = speed of light,
 $B_0$ = background magnetic field strength,
 $p$ = particle momentum,
 $v$ = particle speed,
 $f(p, \mu)$ = particle distribution function,
 $\mu$ = pitch-angle cosine.

This equation applies for {\it resonant interactions}, where:

$$
k = \frac{\Omega}{v \mu - v_A}
$$

with $\Omega$ being the particle gyrofrequency.



If you’re referring to {\it Kolmogorov-type turbulence} or cascading effects (sometimes loosely visualized as "frothy" structures in plasma), that's a different discussion involving spectral energy transfer.

If you're asking about the **overall Alfvén wave growth/damping rate** generated by the **entire particle population** (rather than for a single particle resonance), here's how it's typically formulated.



\subsection{Problem in Monte Carlo: Derivative of $f$}

In Monte Carlo models, computing $\partial f / \partial \mu$ is {\it noisy} and expensive due to statistical fluctuations. To avoid this:
 Use {\it streaming-based approximations}:
$$
\gamma(k) \propto \frac{S(k)}{W(k)}
$$



1. Identify resonant particles for each $k$ using:

$$
k = \frac{\Omega}{v \mu - v_A}
$$

2. Compute net streaming $S(k)$ of resonant particles.
3. Estimate wave growth/damping:

$$
\gamma(k) = C \cdot S(k)
$$

where $C$ includes constants like $e^2 v_A / B_0^2$.

4. Update wave energy density**:

$$
\frac{dW(k)}{dt} = 2 \gamma(k) W(k) - \text{damping terms} + \text{cascade}
$$







\subsection{Wave Growth/Damping Rates from SEPs (Modeled via Monte Carlo Particles)}

In Monte Carlo, SEPs are simulated as individual particles with positions, momenta, and pitch angles, so we must **extract macroscopic quantities** (like streaming) from these particles to compute $\Gamma_\pm(k)$:

\textbf{Growth rate (streaming-based approximation):}

$$
\Gamma_\pm(k) = \frac{\pi^2 e^2 v_A}{c B_0^2} \cdot \frac{1}{k} \cdot S_\pm(k)
$$

Where:

* $S_\pm(k)$ = **resonant streaming** of particles with wavenumber $k$, moving in direction $\pm$
* Computed from particle ensemble satisfying the resonance condition:

$$
k_{\text{res}} = \frac{\Omega}{v \mu \mp V_A}
$$

* The streaming $S(k)$ is estimated by summing over particles:

$$
S(k) = \sum_i w_i v_i \mu_i \, \delta\left(k - k_{\text{res},i}\right)
$$

where $w_i$ is the weight of particle $i$

\textbf{ Damping rate:}

Can be assumed constant or modeled as part of a Kolmogorov cascade:

$$
\mathcal{D}_\pm(k) \sim C_d \, k^{\alpha}
$$

Where $\alpha = 1/3$ for Kolmogorov scaling, or empirically determined.



\textbf{ 3. Coupling Wave Transport and Monte Carlo SEPs}


\textbf{ At each time step:}

1. Particles (Monte Carlo):

   * Advance SEPs via guiding center or Parker equation.  
   
   * At each step, compute local **pitch-angle scattering rate** $\nu(k)$ from $W(k)$, e.g.:

     $$
     D_{\mu\mu} \sim \frac{\pi^2 \Omega^2}{B_0^2} \left(1 - \mu^2\right) \frac{W(k)}{k}
     $$
     
   * Sample pitch-angle deflections accordingly.

2. Wave Growth:

   * From the **instantaneous SEP population**, compute streaming $S_\pm(k)$
   * Calculate $\Gamma_\pm(k)$, then update wave energy densities:

     $$
     \frac{dW_\pm}{dt} = \left[ \Gamma_\pm(k) - \mathcal{D}_\pm(k) \right] W_\pm + \text{cascade}
     $$

3. Wave Update:

   Integrate wave transport equation to update $W_\pm(k, z, t)$ spatially and spectrally.

4. Feedback Loop:

   New $W(k)$ updates scattering coefficients $D_{\mu\mu}$ used in the next Monte Carlo particle update.

This coupling ensures **self-consistent evolution** of both the turbulence spectrum and SEP distribution.







Because {\it growth} $\Gamma$ and {\t damping} $\mathcal D$ occur {\it locally in wavenumber space} (each scale $k$ exchanges energy with the particles and with neighbouring scales independently). If you decide to track only a {\it single, total} wave–energy quantity,

$$
\mathcal W_\pm(z,t)\;=\;\int_{k_1}^{k_2} W_\pm(k,z,t)\,dk ,
$$

you must still compute that total’s time-derivative from an integral of the $k$-dependent source terms:

$$
\frac{\partial \mathcal W_\pm}{\partial t}
\;+\;(V_A\!\mp\!U)\,\frac{\partial \mathcal W_\pm}{\partial z}
=\;2\!\int_{k_1}^{k_2}\!\Gamma_\pm(k)\,W_\pm(k)\,dk
\;-\;2\!\int_{k_1}^{k_2}\!\mathcal D_\pm(k)\,W_\pm(k)\,dk 
$$

Inside the integrals the {\it $k$-dependence remains}, because:

1. {\it Resonance is $k$-selective:}
   A particle with speed $v$ and pitch angle $\mu$ resonates only with
   $k_{\text{res}} = \Omega /(v\mu \mp V_A)$.
   The streaming that drives wave growth therefore differs from scale to scale.

2. {\it Non-linear cascade and physical dissipation are scale-selective:}
   Kolmogorov transfer, cyclotron damping, Landau damping, etc., each act most strongly at particular $k$.

3. {\it Closure still requires a spectrum:}
   To evaluate the integrals in (A) you must know (or assume) how the *total* energy $\mathcal W_\pm$ is distributed across $k$.
   A common closure is

   $$
   W_\pm(k) \;=\; \frac{\mathcal W_\pm}{N}\;k^{-5/3},
   \quad
   N=\!\!\int_{k_1}^{k_2}\!k^{-5/3}\,dk ,
   $$

   but the kernels $\Gamma(k)$ and $\mathcal D(k)$ still enter the integrals explicitly.

\subsection{Putting it into a Monte-Carlo coupling loop}

1. {\it Monte-Carlo step} – advance SEPs, tally the resonant streaming $S_\pm(k)$ in logarithmic $k$-bins.

2. {\it Growth rate per bin} 

   $$
   \Gamma_\pm(k)=\frac{\pi^2 e^2 V_A}{cB_0^2}\,\frac{S_\pm(k)}{k}.
   $$

3. {\it If you store the full spectrum} $W_\pm(k)$
   – update it with the $k$-dependent transport equation.
   {\it If you store only $\mathcal W_\pm$}
   – use your assumed spectral shape to evaluate the integrals in (A) and update the single variable $\mathcal W_\pm$.





\subsection{Coupling loop (wave and particle)}
for each global time step $\Delta t$:

     1. advance particles: 
    move particles along field, include focusing \& adiabatic deceleration.
    scatter each particle with local $D_{\mu\mu}$ built from current W(k) :
        \[
D_{\mu\mu}(\mu_i) = \frac{\pi \, \Omega_i^2}{B_0^2} (1 - \mu_i^2) \frac{W(k_{\text{res}})}{k_{\text{res}}}
\]
    update positions, $\mu_i$, energies

     2. accumulate streaming $S_{\pm}(k)$ from updated particles  

     3. compute $\Gamma_{\pm}(k)$ with eq. 

     4. integrate wave eq. (1) in (k,z) mesh for $\Delta t$
         e.g. finite-volume in z, Crank-Nicolson in k

     5. go to next step





\subsection{1.  Discretise wavenumber space}

\begin{tcolorbox}[colframe=black, colback=white]
\textbf{Log-$k$ grid:} \quad \( k_1, k_2, \ldots, k_N \) \\
Typically 30--60 bins per decade. \\
\[
\Delta k_j = k_{j+1} - k_j
\]
\end{tcolorbox}



\subsection{2.  Loop over Monte-Carlo particles and tally resonant streaming}

\begin{tcolorbox}[colframe=black, colback=white, title=Wave Accumulator Initialization and Update]
\textbf{Initialise accumulators for this spatial cell:}
\begin{align*}
S_{+}[1 \ldots N] &= 0 \quad \text{(waves moving anti-sunward)} \\
S_{-}[1 \ldots N] &= 0 \quad \text{(waves moving sunward)}
\end{align*}

\textbf{For each Monte-Carlo particle \( i \):}
\begin{align*}
v_i &= \frac{|\mathbf{p}_i|}{\gamma_i m_i} \\
\mu_i &= \cos(\text{pitch angle in local field frame}) \\
\Omega_i &= \frac{q_i B_0}{\gamma_i m_i} \quad \text{(gyro-frequency)}
\end{align*}

\textbf{Resonant \( k \) for each propagation sense:}
\begin{align*}
k_{\text{res},+} &= \frac{\Omega_i}{v_i \mu_i - V_A} \\
k_{\text{res},-} &= \frac{\Omega_i}{v_i \mu_i + V_A}
\end{align*}

\textbf{Choose the valid one (sign of denominator must match direction):}
\begin{itemize}
  \item If \( k_{\text{res},+} \) lies inside grid:
  \begin{itemize}
    \item Bin \( j = \text{index}(k_{\text{res},+}) \)
    \item \( S_{+}[j] \mathrel{+}= w_i v_i \mu_i \) \quad (where \( w_i \) is particle weight)
  \end{itemize}
  \item If \( k_{\text{res},-} \) lies inside grid:
  \begin{itemize}
    \item Bin \( j = \text{index}(k_{\text{res},-}) \)
    \item \( S_{-}[j] \mathrel{+}= w_i v_i \mu_i \)
  \end{itemize}
\end{itemize}
\end{tcolorbox}

\begin{tcolorbox}[colframe=black, colback=white, title=Units]
\( S \) is the \textbf{number flux}, with units:
\[
\text{m}^{-2} \, \text{s}^{-1}
\]
\end{tcolorbox}



\section*{3. Compute \textbf{growth rates} from streaming (avoids \( \partial f / \partial \mu \))}

For each bin \( j \) (centre \( k_j \)):

\[
\boxed{
\Gamma_{\pm}(k_j) = \frac{\pi^{2} e^{2} V_A}{c B_0^{2}} \, \frac{S_{\pm}(k_j)}{k_j}
}
\tag{1}
\]

\begin{itemize}
    \item \textbf{Positive} streaming in the same direction as the wave \textbf{amplifies} it.
    \item Opposite-sign streaming \textbf{damps} it.
\end{itemize}

If desired, smooth \( S(k) \) across neighbouring bins to suppress Monte-Carlo noise.



\begin{tabular}{@{}lll@{}}
\toprule
\textbf{Damping term} & \textbf{Expression} & \textbf{When to use} \\
\midrule
Kolmogorov cascade & 
\(
\mathcal{D}_{\pm}(k) = C_K^{-1} \epsilon^{1/3} k^{2/3}
\) (with \( C_K \approx 2 \)) &
If you assume an inertial-range cascade with cascade rate \( \epsilon \) (often set to keep background \( \delta B \) level) \\
\addlinespace
Cyclotron / Landau &
\(
\mathcal{D}_{\pm}(k) = \alpha_c k^2
\) (empirical) &
To remove energy at high \( k \) beyond the proton cyclotron scale \\
\bottomrule
\end{tabular}

\vspace{1em}

You can combine them:
\[
\mathcal{D} = \mathcal{D}_{\text{cascade}} + \mathcal{D}_{\text{phys}}.
\]

\subsection{ 5.  Update the wave spectrum}

Using an implicit or Crank-Nicolson step for stability:

$$
\frac{W_\pm^{n+1}-W_\pm^{n}}{\Delta t}
= 2\bigl[\Gamma_\pm-\mathcal{D}_\pm\bigr]\,\frac{W_\pm^{n+1}+W_\pm^{n}}{2}
-\frac{\partial}{\partial k}\!\Bigl(D_{kk}\,\partial_k W_\pm \Bigr)^{n+1/2}.
$$

$D_{kk}$ is the Kolmogorov diffusion coefficient
$D_{kk}=C_K\,\epsilon^{1/3}\,k^{7/3}$.

\subsection{Feed new $W(k)$ back to the particles}

For each particle i at its new pitch angle $\mu_i$:

$$
k_{\text{res}}  = \frac{\Omega_i}{v_i \mu_i \mp V_A},\quad
D_{\mu\mu}(\mu_i) = \frac{\pi\,\Omega_i^{2}}{B_0^{2}}\,(1-\mu_i^{2})\,
\frac{W_\pm(k_{\text{res}})}{k_{\text{res}}}.
$$

Draw a random pitch-angle kick from a Gaussian with variance $2D_{\mu\mu}\Delta t$.

---

\begin{tcolorbox}[colframe=black, colback=white, title=Time-Stepping Loop]
\begin{itemize}
    \item \textbf{while} \( t < t_{\text{end}} \):
    \begin{itemize}
        \item advance particles and accumulate \( S(k) \)
        \item \( \Gamma(k) \leftarrow \) eq.~(1)
        \item \( \mathcal{D}(k) \leftarrow \) chosen law
        \item update \( W(k) \) (implicit solve)
        \item update \( D_{\mu\mu} \) and scatter particles
        \item \( t \leftarrow t + \Delta t \)
    \end{itemize}
\end{itemize}
\end{tcolorbox}



\subsection{ 1.  Fix the spectral shape, keep only one amplitude}

Assume an inertial-range Kolmogorov spectrum between $k_{\min }$ and $k_{\max }$:

$$
\boxed{\,W_\pm(k,z,t)\;=\;A_\pm(z,t)\;k^{-5/3}},\qquad
N = \int_{k_{\min}}^{k_{\max}} k^{-5/3}\,dk
$$

Only the scalar **amplitude** $A_\pm(z,t)$ is evolved; the $k^{-5/3}$ factor is static.

Total wave energy density in the cell:

$$
\mathcal W_\pm(z,t)=A_\pm(z,t)\,N .
$$


\subsection{ 2.  Still tally resonant streaming per $k$-bin}

You must keep the {\it $k$}-resolved streaming** because resonance is scale-selective.

```pseudo

\begin{tcolorbox}[colframe=black, colback=white, title=Streaming Accumulation per Particle]
\begin{itemize}
    \item \textbf{for each particle} \( i \) \textbf{in the cell}:
    \begin{itemize}
        \item \( k_{\text{res},+} = \dfrac{\Omega_i}{v_i \mu_i - V_A} \) \quad \text{(anti-sunward waves)}
        \item \( k_{\text{res},-} = \dfrac{\Omega_i}{v_i \mu_i + V_A} \) \quad \text{(sunward waves)}
        \item \textbf{if} \( k_{\text{res},+} \in [k_{\text{min}}, k_{\text{max}}] \):
        \begin{itemize}
            \item \( j = \text{index}(k_{\text{res},+}) \)
            \item \( S_{+}[j] \mathrel{+}= w_i v_i \mu_i \)
        \end{itemize}
        \item \textbf{if} \( k_{\text{res},-} \in [k_{\text{min}}, k_{\text{max}}] \):
        \begin{itemize}
            \item \( j = \text{index}(k_{\text{res},-}) \)
            \item \( S_{-}[j] \mathrel{+}= w_i v_i \mu_i \)
        \end{itemize}
    \end{itemize}
\end{itemize}
\end{tcolorbox}



\subsection{ 3.  Growth rate per bin (unchanged)}

$$
\Gamma_\pm(k_j)=\frac{\pi^{2}e^{2}V_A}{cB_0^{2}}\,
\frac{S_\pm(k_j)}{k_j}.
$$



\subsection{ 4.  Collapse to one scalar growth term}

The {\it net} growth of the total energy is the spectrum-weighted integral:

$$
\frac{d\mathcal W_\pm}{dt}
=\;2\!\int_{k_{\min}}^{k_{\max}}\!\Gamma_\pm(k)\,W_\pm(k)\,dk
-\;2\!\int_{k_{\min}}^{k_{\max}}\!\mathcal D_\pm(k)\,W_\pm(k)\,dk .
$$

Insert $W_\pm(k)=A_\pm k^{-5/3}$:

$$
\boxed{\;
\frac{dA_\pm}{dt}=2\,A_\pm
\bigl[\langle\Gamma_\pm\rangle - \langle\mathcal D_\pm\rangle\bigr]},
\qquad
\langle X\rangle=
\frac{\displaystyle\int_{k_{\min}}^{k_{\max}}X(k)\,k^{-5/3}\,dk}
     {\displaystyle\int_{k_{\min}}^{k_{\max}}k^{-5/3}\,dk } .
$$

\begin{tcolorbox}[colframe=black, colback=white, title=Weighted Average Growth Rate]
\begin{itemize}
    \item Initialize:
    \[
    \text{num} = 0 \quad \text{(numerator for } \langle \Gamma \rangle \text{)}
    \]
    \[
    \text{den} = 0 \quad \text{(normalisation } N \text{, same each step, pre-compute)}
    \]
    \item \textbf{for each} \( k \)-bin \( j \):
    \begin{itemize}
        \item \( \text{weight} = k_j^{-5/3} \, \Delta k_j \)
        \item \( \text{num} \mathrel{+}= \Gamma_{+}[j] \times \text{weight} \quad \) (or \( \Gamma_{-}[j] \) for other sense)
        \item \( \text{den} \mathrel{+}= \text{weight} \quad \) (den \( = N \))
    \end{itemize}
    \item Compute:
    \[
    \langle \Gamma \rangle = \frac{\text{num}}{\text{den}}
    \]
\end{itemize}
\end{tcolorbox}

Do the same loop with $\mathcal D(k)$ to get $\langle\mathcal D\rangle$.

\subsection{5.  Advance the single amplitude}

\begin{tcolorbox}[colframe=black, colback=white, title=Amplitude Update]
\[
A_{\text{new}} = A_{\text{old}} \times \exp\left\{ 2 \Delta t \left( \langle \Gamma \rangle - \langle \mathcal{D} \rangle \right) \right\}
\]
\textit{(exact for constant rates in \( \Delta t \))}
\end{tcolorbox}

or, second-order Euler:


\begin{tcolorbox}[colframe=black, colback=white, title=Amplitude Update (Linearized)]
\[
A_{\text{new}} = A_{\text{old}} + \Delta t \times 2 A_{\text{old}} \left( \langle \Gamma \rangle - \langle \mathcal{D} \rangle \right)
\]
\end{tcolorbox}



\subsection{6.  Feed back to the particles}

Whenever you need the scattering coefficient for a particle with pitch-angle $\mu$:

$$
k_{\text{res}}=\frac{\Omega}{v\mu\mp V_A},\qquad
D_{\mu\mu}= \frac{\pi \Omega^{2}}{B_0^{2}}\,(1-\mu^{2})\,
\frac{A_\pm\,k_{\text{res}}^{-5/3}}{k_{\text{res}}}.
$$

Thus $D_{\mu\mu}\propto A_\pm$, so every time you update $A_\pm$ you automatically update the scattering strength everywhere in $k$.

\begin{tcolorbox}[colframe=black, colback=white, title=Why \(k\)-Resolved Streaming is Still Needed]
Although only one number \( A_{\pm} \) is stored, the averages \( \langle \Gamma \rangle \) and \( \langle \mathcal{D} \rangle \) in (2) require:
\begin{itemize}
    \item \( \Gamma(k) \) — from streaming in each bin, eq.~(1)
    \item \( \mathcal{D}(k) \) — from your cascade/dissipation law
\end{itemize}

Hence, the Monte-Carlo procedure still loops over \( k \)-bins, but the memory footprint is tiny (just two 1-D arrays for \( S_{\pm} \)), and only one scalar per direction is time-advanced.
\end{tcolorbox}

#### Bottom line

* **Kolmogorov assumption** → fixes the spectral *shape*.
* **Streaming tally per $k$-bin** → captures the correct resonance physics.
* **Equation (2)** → collapses those $k$-dependent rates to a single growth (or damping) rate for the total wave-energy amplitude that your Monte-Carlo code can evolve with one line of algebra.


Below is a compact, self-contained **physics specification** for a Monte-Carlo (MC) module that couples energetic charged particles to a background of field-aligned Alfvén turbulence whose spectrum is fixed to the inertial-range Kolmogorov form
$W(k)=A\,k^{-5/3}$ between $k_{\min }$ and $k_{\max }$.



\subsection{ 1.  Probability that the particle is scattered in $\Delta t$}

For gyro-resonant, small-angle pitch-angle diffusion the 1-D Fokker–Planck coefficient is

$$
D_{\mu\mu}(\mu_i)=\frac{\pi \Omega_i^{2}}{B_0^{2}}\bigl(1-\mu_i^{2}\bigr)
\frac{W\!\bigl(k_{\text{res}}^{\pm}\bigr)}{\,\lvert k_{\text{res}}^{\pm}\rvert\,}.
$$

Because $W(k)=A\,k^{-5/3}$,

$$
D_{\mu\mu}= \frac{\pi \Omega_i^{2}}{B_0^{2}}\bigl(1-\mu_i^{2}\bigr)
A\;|k_{\text{res}}^{\pm}|^{-8/3}.
$$

Pitch-angle diffusion is a homogeneous Wiener process, so the **survival probability** (no scatter in $\Delta t$) is

$$
P_{\rm surv}=e^{-2D_{\mu\mu}\,\Delta t},
\qquad
P_{\rm scatt}=1-P_{\rm surv}\;\simeq\;2D_{\mu\mu}\,\Delta t\;\;(\text{if }2D_{\mu\mu}\Delta t\ll1).
$$

In code: draw a uniform random number $\mathcal U\in[0,1]$; scatter if $\mathcal U<P_{\rm scatt}$.

\subsection{Scattering angle in the frame co-moving with the wave}

Work in the de-Hoffmann–Teller frame of the resonant wave, i.e. a frame translating with $V_A$ **along the field** (sign depends on wave sense).
In this frame the magnetic perturbation is time-stationary and pitch-angle diffusion is isotropic around the local field.

* Draw a Gaussian deviate $\Delta \mu$ with
  $\displaystyle \langle\Delta \mu\rangle=0,\quad
  \langle(\Delta \mu)^{2}\rangle = 2D_{\mu\mu}\,\Delta t.$

* Update the pitch angle: $\mu^{\prime}=\mu_i+\Delta \mu$.
  If $|\mu^{\prime}|>1$ reflect specularly to keep $|\mu|\le1$.

* Transform back to the plasma frame.
  Because $V_A\ll v_i$ for SEPs, the Lorentz transformation changes the magnitude of $\mu$ only by $\mathcal O(V_A/v_i)$; to first order you may keep $\mu^{\prime}$ unchanged.

**Angular deflection**

$$
\Delta\theta \;=\;\arccos μ^{\prime}-\arccos μ_i
\;\approx\;
-\frac{\Delta μ}{\sqrt{1-μ_i^{2}}}.
\tag{5}
$$

(This linearised form is what MC codes normally record for diagnostics.)

\subsection{Wave-energy increment produced by the scattering}

Each scattering event exchanges energy between particle and wave.
In QLT the **energy transferred to waves in bin $k_{\text{res}}$** is the work done by the induced electric field $E_\parallel$ over the interaction:

$$
\delta\!W(k_{\text{res}}) \;=\;
-\,\delta p_{\parallel}\,V_A/V_{\rm ph},
\qquad V_{\rm ph}\simeq V_A ,
$$

so that (sign convention: positive $\delta W$ = wave growth)

$$
\delta W(k_{\text{res}})\;=\;
-\,V_A\,\Delta p_{\parallel}.
$$

For small-angle scattering in the wave frame,

$$
\Delta p_{\parallel}=p_i\,(\mu^{\prime}-\mu_i)=p_i\,\Delta \mu .
$$

**Insert (8) into (7)** and distribute over the $k$-bin volume $\Delta k\Delta z$:

$$
\boxed{\;
\delta W(k_j)
= -\,p_i V_A \,\Delta μ \;
\bigl/(\Delta k\,\Delta z)\;}
$$

Summing (9) over all particles gives the **net spectral energy change**.
Dividing by $W(k_j)\,\Delta t$ recovers the familiar growth/damping rate

$$
\Gamma(k_j)\;=\;\frac{\pi^{2}e^{2}V_A}{cB_0^{2}}\,
\frac{S(k_j)}{k_j},
$$
where $S(k_j)=\sum_i w_i v_i \mu_i$ is the resonant streaming accumulated *before* scattering.
Consistency is automatic: equations (7)–(10) ensure that the *same* $D_{\mu\mu}$ that scatters particles also sets the wave-growth term in the turbulence transport equation.

---


\begin{tcolorbox}[colframe=black, colback=white, title=Summary of the Algorithm]
\textbf{For every particle each step:}
\begin{itemize}
    \item Evaluate \( k_{\text{res}} \) via (1);
    \item Compute \( D_{\mu\mu} \) with (3);
    \item Draw a scatter/no-scatter decision with (4);
    \item If scattered, draw \( \Delta \mu \) (Gaussian) and update \( \mu \);
    \item Add \( \delta W(k_{\text{res}}) \) from (9) to the wave-energy accumulator.
\end{itemize}

\textbf{After all particles are processed for the cell:}
\begin{itemize}
    \item Update the wave amplitude \( A \) (or full \( W(k) \) if you track it) with the accumulated \( \sum \delta W \);
    \item Proceed to the next Monte-Carlo time step.
\end{itemize}


With these three derived elements you have a physically closed Monte-Carlo model that:
\begin{itemize}
    \item reproduces Kolmogorov scattering statistics,
    \item conserves energy between particles and waves event-by-event, and
    \item yields the same macroscopic growth rate (10) as standard quasilinear theory.
\end{itemize}
\end{tcolorbox}

\subsection{Handling out-of-range pitch-angle cosine after a diffusion step}

When you update a particle’s pitch-angle cosine with

$$
\mu'=\mu+\Delta\mu ,\qquad \Delta\mu\sim\mathcal N\bigl(0,2D_{\mu\mu}\,\Delta t\bigr),
$$
the Gaussian kick can push $\mu'$ outside the **physical interval** $[-1,1]$.  In quasilinear theory the correct boundary condition is **zero probability flux** at $\mu=\pm1$;($\partial f/\partial\mu=0$).
Numerically this is enforced with **specular reflection**—exactly the rule used for a 1-D Brownian particle between two perfectly reflecting walls.

\subsubsection{ Reflection algorithm (vectorised-safe)}

\begin{tcolorbox}[colframe=black, colback=white, title=Pitch-Angle Reflection at Boundaries]
\textbf{If} \( \mu' > 1 \):
\begin{itemize}
    \item \( \mu' = 2 - \mu' \) \quad (reflect about \( \mu = +1 \))
    \item \textbf{If} \( \mu' < -1 \): \quad (overshot both walls in an extreme kick)
    \begin{itemize}
        \item \( \mu' = -2 - \mu' \) \quad (second reflection about \( \mu = -1 \))
    \end{itemize}
\end{itemize}
\textbf{Else if} \( \mu' < -1 \):
\begin{itemize}
    \item \( \mu' = -2 - \mu' \) \quad (reflect about \( \mu = -1 \))
    \item \textbf{If} \( \mu' > 1 \):
    \begin{itemize}
        \item \( \mu' = 2 - \mu' \) \quad (second reflection about \( \mu = +1 \))
    \end{itemize}
\end{itemize}
\end{tcolorbox}

\begin{tcolorbox}[colframe=black, colback=white, title=Why Reflection (and not Truncation or Redraw?)]

You may wrap the two reflections in a \textbf{while} \( |\mu'| > 1 \) loop; with realistic \( \Delta t \), two reflections are enough in practice.

The operation preserves the step's magnitude \( |\Delta \mu| \) and satisfies \( \frac{\partial f}{\partial \mu} = 0 \) at the boundaries.

\bigskip

\textbf{Why reflection (and not truncation or re-draw)?}
\begin{itemize}
    \item \textbf{Detailed balance:} Reflection conserves the diffusive probability current; truncation artificially piles up particles at \( \mu = \pm 1 \).
    \item \textbf{Energy conservation:} In the wave frame, scattering is elastic; reflecting the overshoot keeps \( |v| \) unchanged.
    \item \textbf{Computationally cheap:} Just a pair of arithmetic operations; no extra random draw.
\end{itemize}

\bigskip

\textbf{Compact formula:}

You can write the double reflection in one line:
\[
\mu' =
\begin{cases}
2 - \mu', & \text{if } \mu' > 1, \\
-2 - \mu', & \text{if } \mu' < -1.
\end{cases}
\]
Then test once more: if \( |\mu'| > 1 \), repeat (rare).

\bigskip

After this adjustment, you are guaranteed \( -1 \leq \mu' \leq 1 \), and the Monte-Carlo simulation remains faithful to quasilinear diffusion physics.

\end{tcolorbox}

\begin{tcolorbox}[colframe=black, colback=white, title=Self-Contained Monte-Carlo Recipe for Solar-Wind Ions]

The goal is to couple:
\begin{itemize}
    \item a Parker-type particle transport solved with Monte-Carlo pseudo-particles whose distribution is isotropic in the local frame, and
    \item a single, time-dependent wave–energy amplitude \( A_{\pm}(r, t) \) in the fixed spectrum:
\end{itemize}
\[
W_{\pm}(k) = A_{\pm}(r, t) \, k^{-5/3}, \quad k_{\min} \leq k \leq k_{\max}.
\]
No high-speed (\( v \gg V_A \)) approximation is made; all formulae remain valid down to \( v \sim V_A \).

\bigskip

\textbf{1. Gyro-resonant condition (valid for all \( v \))}

For each particle \( i \) with speed \( v_i \) and pitch-angle cosine \( \mu_i \):
\[
k_{\text{res},\pm} = \frac{\Omega_i}{v_i \mu_i \mp V_A}, \quad \Omega_i = \frac{q_i B_0}{m_i \gamma_i}.
\tag{1}
\]
A resonance exists only when \( |v_i \mu_i| > V_A \).

\bigskip

\textbf{2. Pitch–angle diffusion coefficient \( D_{\mu\mu} \)}
\[
D_{\mu\mu}(\mu_i, v_i) = \frac{\pi \Omega_i^2}{B_0^2} (1 - \mu_i^2) A_{\pm} |k_{\text{res},\pm}|^{-8/3}.
\tag{2}
\]

\bigskip

\textbf{3. Scattering probability in a Monte-Carlo time-step \( \Delta t \)}
\[
P_{\text{scatt}} = 1 - \exp\left( -2 D_{\mu\mu} \Delta t \right) \quad \rightarrow \quad 2 D_{\mu\mu} \Delta t \ll 1 \quad \text{(approximation)}.
\tag{3}
\]
Draw a uniform deviate \( U \in [0, 1] \); scatter if \( U < P_{\text{scatt}} \).

\bigskip

\textbf{4. Pitch-angle update with reflections}

If scattered, draw a Gaussian kick:
\[
\Delta \mu \sim \mathcal{N}(0, \, 2 D_{\mu\mu} \Delta t), \quad \mu' = \mu_i + \Delta \mu.
\]
Apply successive specular reflections until \( |\mu'| \leq 1 \):
\[
\mu' =
\begin{cases}
2 - \mu', & \text{if } \mu' > 1, \\
-2 - \mu', & \text{if } \mu' < -1.
\end{cases}
\tag{4}
\]

\bigskip

\end{tcolorbox}
\begin{tcolorbox}[colframe=black, colback=white, title=Self-Contained Monte-Carlo Recipe for Solar-Wind Ions]

\textbf{5. Energy (velocity-magnitude) diffusion — ion heating}

Quasilinear perpendicular velocity diffusion coefficient:
\[
D_{v_{\perp} v_{\perp}}(v_i, \mu_i) = \frac{\pi \Omega_i^2 V_A^2}{B_0^2} A_{\pm} |k_{\text{res},\pm}|^{-8/3}.
\tag{5}
\]
For an isotropic distribution, convert to speed diffusion:
\[
D_{vv} = \frac{1 - \mu_i^2}{2} D_{v_{\perp} v_{\perp}}.
\tag{6}
\]
Update the speed:
\[
v'_i = v_i + \sqrt{2 D_{vv} \Delta t} \, \xi, \quad \xi \sim \mathcal{N}(0, 1).
\tag{7}
\]

\bigskip

\textbf{6. Wave–energy change from each diffusive kick}

Energy conservation gives:
\[
\delta W(k_{\text{res},\pm}) = -\frac{m_i}{V_A} \left( v'_i - v_i + v_i \Delta \mu \right).
\tag{8}
\]
Accumulate \( \sum \delta W \) in the resonant \( k \)-bin.

\bigskip

\textbf{7. Single-amplitude wave update}

The net spectral energy change in the cell:
\[
\Delta W_{\pm} = \sum_{\text{particles}} \delta W, \quad W_{\pm} = A_{\pm} N, \quad N = \int_{k_{\min}}^{k_{\max}} k^{-5/3} \, \mathrm{d}k.
\]
Advance the Kolmogorov amplitude:
\[
A_{\pm}^{\text{new}} = \max \left( 0, \, A_{\pm}^{\text{old}} - \frac{\Delta W_{\pm}}{N} \right).
\tag{9}
\]
(No growth term appears because net streaming is zero for isotropic \( f \).)

\bigskip

\end{tcolorbox}
\begin{tcolorbox}[colframe=black, colback=white, title=Self-Contained Monte-Carlo Recipe for Solar-Wind Ions]

\textbf{8. Algorithm in One Glance}

\begin{itemize}
    \item \textbf{for each time-step} \( \Delta t \):
    \begin{itemize}
        \item \textbf{Loop over particles in cell}:
        \begin{itemize}
            \item \textbf{If} \( |v_i \mu_i| \leq V_A \): continue (non-resonant).
            \item Calculate \( k_{\text{res}} = \Omega_i / (v_i \mu_i \mp V_A) \).
            \item \textbf{Pitch-angle scattering}:
            \[
            D_{\mu\mu} = \text{eq.~(2)}, \quad P_{\text{scatt}} = 1 - \exp(-2 D_{\mu\mu} \Delta t).
            \]
            \item Draw random; if \( \text{random}() < P_{\text{scatt}} \):
            \[
            \Delta \mu \sim \mathcal{N}(0, \sqrt{2 D_{\mu\mu} \Delta t}), \quad \mu_{\text{new}} = \text{reflect}(\mu_i + \Delta \mu).
            \]
            Else:
            \[
            \mu_{\text{new}} = \mu_i, \quad \Delta \mu = 0.
            \]
            \item \textbf{Speed diffusion (ion heating)}:
            \[
            D_{vv} = \frac{1 - \mu_{\text{new}}^2}{2} \left( \frac{\pi \Omega_i^2 V_A^2}{B_0^2} \right) A k_{\text{res}}^{-8/3}.
            \]
            \[
            v_{\text{new}} = v_i + \text{gaussian}(0, \sqrt{2 D_{vv} \Delta t}), \quad \Delta v = v_{\text{new}} - v_i.
            \]
            \item \textbf{Energy exchange with wave}:
            \[
            \delta W = -m_i ( \Delta v + v_i \Delta \mu ) / V_A.
            \]
            Accumulate \( \delta W \) in \( \text{bin}(k_{\text{res}}) \).
            \item \textbf{Store new phase-space coordinates}:
            \[
            v_i, \mu_i = v_{\text{new}}, \mu_{\text{new}}.
            \]
        \end{itemize}
        \item \textbf{Update wave amplitude from all} \( \delta W \):
        \[
        A_{\pm} \leftarrow \max\left( 0, A_{\pm} - \frac{\sum \delta W}{N} \right).
        \]
    \end{itemize}
\end{itemize}

\bigskip

\textbf{Why it Works}
\begin{itemize}
    \item \textbf{No high-speed assumption} — all \( v \) appear explicitly; resonance switches off when \( v \mu \) falls below \( V_A \).
    \item \textbf{Parker-equation compatibility} — isotropy is preserved because pitch-angle kicks are symmetric and large \( D_{\mu\mu} \) quickly erases momentary anisotropy.
    \item \textbf{Self-consistent heating} — Eqs.~(5–9) conserve energy between particles and waves event-by-event and reduce to standard cyclotron damping expressions when averaged over a Maxwellian.
\end{itemize}

\end{tcolorbox}


\section*{Parker-equation–compatible Monte-Carlo Recipe (No \(\mu\))}

Below is a \textbf{Parker-equation–compatible Monte-Carlo recipe} that \textbf{never stores \( \mu \)}.
All pitch-angle information is handled statistically, so the particle ensemble remains \textbf{strictly isotropic} at every step.

\bigskip
\hrule
\bigskip

\section*{0. Set-up and Fixed Kolmogorov Spectrum}

\begin{itemize}
    \item \textbf{Wave spectrum} (one amplitude per propagation sense \( \pm \)):
    \[
    W_\pm(k, r, t) = A_\pm(r, t)\; k^{-5/3}, \qquad k_{\min} \leq k \leq k_{\max}.
    \]
    
    \item Particles are represented by pseudo-particles that carry only \textbf{speed} \( v \) and \textbf{position} \( r \).
    
    \item Their directions are redrawn isotropically whenever scattering occurs.
    
    \item Local background: \( B_0(r), \; V_A(r) \).
\end{itemize}

\bigskip
\hrule
\bigskip

\section*{1. Isotropic Pitch-Angle Diffusion \(\rightarrow\) One \textbf{Scattering Frequency}}

Start from the quasilinear coefficient that \emph{would} hold for a given pitch cosine \( \mu \):
\[
D_{\mu\mu}(\mu, v) = \frac{\pi \Omega^2}{B_0^2} (1 - \mu^2) A_\pm \left| k_{\text{res}}^\pm \right|^{-8/3}, \qquad
k_{\text{res}}^\pm = \frac{\Omega}{v\mu \mp V_A}.
\]

Average over an \textbf{isotropic} distribution \( P(\mu) = \frac{1}{2} \):
\[
\boxed{ \bar{D}_{\mu\mu}(v) = \frac{1}{2} \int_{-1}^{+1} D_{\mu\mu}(\mu, v) \, d\mu. }
\tag{1}
\]

Carry out the integral analytically for \( v > V_A \) (or numerically tabulate once):
\[
\bar{D}_{\mu\mu}(v) = \frac{\pi \Omega^2 A_\pm}{B_0^2} \; \mathcal{I}\left( \frac{v}{V_A} \right),
\]
where
\[
\mathcal{I}(x) = \frac{1}{2} \int_{-1}^{+1} (1 - \mu^2) \left| x\mu \mp 1 \right|^{-8/3} \, d\mu.
\]

\emph{If \( v \leq V_A \), the resonance band does not exist and \( \bar{D}_{\mu\mu} = 0 \).}

\bigskip

\textbf{Scattering frequency}
\[
\boxed{ \nu_{\text{sc}}(v) = 2 \bar{D}_{\mu\mu}(v). }
\tag{2}
\]
\emph{This single number replaces the \( \mu \)-dependent coefficient used in pitch-angle-resolved models.}

\bigskip
\hrule
\bigskip

\section*{2. Monte-Carlo Step – Isotropic Scattering}

For each particle during a time-step \( \Delta t \):
\begin{tcolorbox}[colframe=black, colback=white, title=Isotropic Scattering Step]
\begin{itemize}
    \item \( \nu = \nu_{\text{sc}}(v) \) \hfill \texttt{\# eq.~(2)}
    \item \( P_{\text{sc}} = 1 - \exp(-\nu \, \Delta t) \) \hfill \texttt{\# probability to scatter}
    \item \textbf{if} \texttt{random()} \( < P_{\text{sc}} \): \hfill \texttt{\# scatter event happens}
    \begin{itemize}
        \item \( \hat{v} = \texttt{isotropic\_unit\_vector()} \) \hfill \texttt{\# completely new random direction}
    \end{itemize}
\end{itemize}
\end{tcolorbox}

\emph{No angle bookkeeping is needed; the velocity magnitude stays \( v \) for the moment.}

\bigskip
\hrule
\bigskip

\section*{3. Energy Diffusion (Ion Heating)}

The quasilinear perpendicular-velocity coefficient, averaged over isotropy, becomes
\[
\boxed{ \bar{D}_{vv}(v) = \frac{1}{3} V_A^2 \, \nu_{\text{sc}}(v). }
\tag{3}
\]
\emph{(This follows from \( \langle 1 - \mu^2 \rangle = 2/3 \).)}

\bigskip

\textbf{Speed update}
\[
\boxed{ v' = v + \sqrt{2 \bar{D}_{vv} \, \Delta t} \; \xi, \qquad \xi \sim \mathcal{N}(0, 1). }
\tag{4}
\]
\emph{Set \( v' \geq 0 \); draw a new isotropic direction \textbf{after} the speed change so the distribution remains isotropic.}

\bigskip
\hrule
\bigskip

\section*{4. Wave-Energy Change Associated with One Particle}

The particle’s kinetic-energy increment is
\[
\boxed{ \Delta E_p = \frac{1}{2} m (v'^{2} - v^2). }
\tag{5}
\]

Energy conservation (wave frame speed \( V_A \)) gives the wave change:
\[
\boxed{ \Delta \mathcal{W}_\pm = - \Delta E_p. }
\tag{6}
\]
\emph{Because the ensemble is isotropic there is no net streaming, so waves only \textbf{damp}. Growth stays zero.}

Accumulate \( \sum \Delta \mathcal{W}_\pm \) for the cell during the step.

\bigskip
\hrule
\bigskip

\section*{5. Update the Single Kolmogorov Amplitude}

Total spectral energy in the cell:
\[
\mathcal{W}_\pm = A_\pm N, \qquad N = \int_{k_{\min}}^{k_{\max}} k^{-5/3} \, dk.
\]

After the step:
\[
\boxed{ A_\pm^{\text{new}} = \max\left( 0, \; A_\pm^{\text{old}} - \frac{\sum \Delta \mathcal{W}_\pm}{N} \right). }
\tag{7}
\]

\bigskip
\hrule
\bigskip

\section*{6. Complete Algorithm (Pseudo-code)}

\begin{lstlisting}[mathescape=true]
# pre-tabulate ν_sc(v) and Dvv(v) on a v-grid for speed.

for each global step Δt at radius r:

    E_wave_change_plus  = 0
    E_wave_change_minus = 0          # two propagation senses if both kept

    for each pseudo-particle:
        # (a) SCATTERING DECISION
        ν   = ν_sc(v)
        if random() < 1 - exp(-ν Δt):
            v_dir = isotropic_unit_vector()

        # (b) SPEED (HEATING) UPDATE
        Dvv = (1/3) * V_A^2 * ν
        v_new = v + gaussian(0, sqrt(2 Dvv Δt))
        v_new = max(v_new, 0)

        # (c) ENERGY EXCHANGE
        ΔE = 0.5 * m * (v_new**2 - v**2)
        E_wave_change_plus  -= ΔE     # choose + or – sense; here use “+”
        
        # (d) STORE NEW SPEED
        v = v_new

    # (e) WAVE AMPLITUDE UPDATE
    A_plus  = max( 0,  A_plus  - E_wave_change_plus / N )
    # repeat for A_minus if needed
\end{lstlisting}

\bigskip
\hrule
\bigskip

\section*{Key Points}
\begin{itemize}
    \item \textbf{No \( \mu \) anywhere:} only the \emph{scalar} scattering frequency \( \nu_{\text{sc}}(v) \) and speed-diffusion coefficient \( \bar{D}_{vv}(v) \) are required.
    \item \textbf{Isotropisation:} enforced by drawing a \emph{completely new} random direction at each scatter; between scatters the particle simply convects.
    \item \textbf{Heating of solar-wind ions:} obtained naturally from the speed-diffusion step (4).
    \item \textbf{Wave damping:} the negative of particle heating, ensuring energy conservation.
\end{itemize}



\begin{tcolorbox}[colframe=black, colback=white, title=Step-by-Step Coupling Scheme]

Below is a \textbf{step-by-step coupling scheme} that marries the \textbf{isotropic-Parker Monte-Carlo module} (previous reply) to a \textbf{transport equation for the turbulence–energy density}. The scheme keeps only \textbf{one Kolmogorov amplitude per propagation sense} \( A_\pm(r, t) \) yet still:
\begin{itemize}
    \item advects Alfvén-wave energy with the solar wind,
    \item accounts for nonlinear (Kolmogorov) cascade,
    \item damps/feeds the waves through \textbf{exact energy exchange with the particles}.
\end{itemize}

\section*{1. Wave-Energy Transport Equation (Single Kolmogorov Amplitude)}

Integrate the 1-D spectral wave equation over \( k \) using \( W_\pm(k) = A_\pm k^{-5/3} \). The result for the \emph{total} energy density \( \mathcal{W}_\pm = A_\pm N \) (with \( N = \int_{k_{\min}}^{k_{\max}} k^{-5/3} \, dk \)) is:
\[
\boxed{
\frac{\partial \mathcal{W}_\pm}{\partial t}
+ (V_A \mp U) \frac{\partial \mathcal{W}_\pm}{\partial r}
= - \underbrace{\frac{\mathcal{W}_\pm}{\tau_{\text{cas}}}}_{\text{Kolmogorov cascade}}
- \underbrace{Q_{\text{ion}}}_{\text{particle damping}}
+ S_{\text{inj}}(r, t)
}
\tag{1}
\]
\begin{itemize}
    \item \textbf{Advection speed:} group velocity \( V_A \mp U \).
    \item \textbf{Cascade term:} e.g., \( \tau_{\text{cas}}^{-1} = \frac{C_K}{L_\perp} \sqrt{ \delta B^2 / (4 \pi \rho) } \).
    \item \textbf{Particle term:} 
    \[
    Q_{\text{ion}} = \sum_{\text{all MC particles in cell}} \frac{-\Delta E_p}{\Delta r}
    \]
    with \( \Delta E_p \) from Eq.~(5) in the previous answer.
    \item \textbf{Source term:} \( S_{\text{inj}} \), optional Alfvén-wave source (e.g., at the coronal base).
\end{itemize}
Since \( \mathcal{W}_\pm = A_\pm N \) and \( N \) is constant, you can update either \( \mathcal{W}_\pm \) or \( A_\pm \); below we update \( A_\pm \).

\end{tcolorbox}
\begin{tcolorbox}[colframe=black, colback=white, title=Step-by-Step Coupling Scheme]

\section*{2. Discretisation in a Radial Mesh}

\begin{itemize}
    \item \textbf{Radial grid:} \( r_j, \quad j = 0 \dots J \) (log-spacing is convenient).
    \item \textbf{Time step:} \( \Delta t \) limited by CFL condition:
    \[
    \Delta t < \min_j \left\{ \frac{ \Delta r_j }{ | V_A \mp U | } \right\}.
    \]
\end{itemize}

\textbf{Finite-volume update for \( A_\pm \)}:
\[
A_\pm^{n+1}(j) = A_\pm^{n}(j) - \frac{ \Delta t }{ \Delta r_j } \left[ F_{j+1/2} - F_{j-1/2} \right]
- \frac{ \Delta t }{ N } \left[ \frac{A_\pm N}{\tau_{\text{cas}}} + Q_{\text{ion}} - S_{\text{inj}} \right]_j
\tag{2}
\]
\begin{itemize}
    \item \textbf{Upwind fluxes:}
    \[
    F_{j+1/2} = (V_A \mp U)_{j+1/2} \, A_\pm^{\text{up}}
    \]
    Use van-Leer or piecewise-linear reconstruction for second-order accuracy.
    \item Cascade, particle, and source terms are cell-centred.
\end{itemize}


\end{tcolorbox}

\begin{tcolorbox}[colframe=black, colback=white, title=Step-by-Step Coupling Scheme]

\section*{3. Monte-Carlo \(\leftrightarrow\) Turbulence Coupling at Each Global Step}

\begin{lstlisting}[mathescape=true]
============================================
(1)  Advection / Cascade sub-step for $A_\pm$
============================================
  for each radial cell j:
        # 1a. build upwind fluxes  F
        # 1b. update $A_\pm$ with eq.(2)   (no particle term yet)

============================================
(2)  Particle sub-step in each cell j
============================================
  initialise   Q_ion(j) = 0

  loop over Monte-Carlo particles:
\begin{itemize}
    \item \textbf{2a.} Scattering decision with $\nu_{\text{sc}}(v) $
    \item \textbf{2b.} Speed diffusion $v \rightarrow v'$
    \item \textbf{2c.} Energy change $
    \Delta E_p = \frac{1}{2} m (v'^2 - v^2) $
    
    \item \textbf{2d.} Update particle position by Parker drifts
\end{itemize}
============================================
(3) Particle damping $\leftrightarrow$ wave update
============================================
\text{for each radial cell } j:

$A_\pm(j) \leftarrow A_\pm(j) - \frac{\Delta t}{N} \cdot Q_{\text{ion}}(j) 
\quad \texttt{\# Eq.~(2) term}
$
$
A_\pm(j) = \max(A_\pm(j), 0) 
\quad \texttt{\# maintain positivity}
$
\end{lstlisting}

\emph{Step-ordering:} split the advection/cascade and particle feedback for clarity; operator-splitting error is \( \mathcal{O}(\Delta t) \) and can be reduced with Strang splitting.

\end{tcolorbox}
\begin{tcolorbox}[colframe=black, colback=white, title=Step-by-Step Coupling Scheme]

\section*{4. What the Coupling Achieves}


\noindent
\begin{tabularx}{\textwidth}{@{}lXl@{}}
\toprule
\textbf{Module} & \textbf{Uses} & \textbf{Provides to the other} \\
\midrule
\textbf{Monte-Carlo Parker solver} & local \( A_\pm(r, t) \) to compute \( \nu_{\text{sc}} \), \( D_{vv} \) & energy sink \( Q_{\text{ion}} \) (heating) \\
\textbf{Wave-transport solver} & \( Q_{\text{ion}} \) to damp \( \mathcal{W}_\pm \); cascade law; boundary injection & updated \( A_\pm(r, t + \Delta t) \) \\
\bottomrule
\end{tabularx}


\medskip

\emph{Energy is conserved} cell-by-cell because the exact negative of each particle’s \( \Delta E_p \) is removed from \( A_\pm N \).

\section*{5. Boundary and Initial Conditions}

\begin{itemize}
    \item At \( r = r_0 \) (coronal base) prescribe an \textbf{injected amplitude} \( A_{+,0} \) (outward) and optionally a reflection coefficient to seed \( A_- \).
    \item At the outer boundary \( r = r_{\max} \), use free-outflow for \( A_+ \) and impose a small inflowing \( A_- \) if interstellar turbulence is required.
\end{itemize}

Initial \( A_\pm(r, 0) \) can be set from empirical \( \delta B / B \propto r^{-3/2} \) or any magnetogram-based model.

\section*{6. Practical Diagnostics}

\begin{itemize}
    \item \textbf{Ion heating rate:} \( Q_{\text{ion}}(r) \).
    \item \textbf{Wave energy flux:} \( F_\pm = A_\pm N (V_A \mp U) \).
    \item \textbf{Residual energy ratio:} 
    \[
    \frac{ \delta B^2 }{ B_0^2 } = \frac{ 2 \int W_\pm(k) \, dk }{ B_0^2 } = \frac{ 2 A_\pm N }{ B_0^2 }.
    \]
\end{itemize}

\emph{Consistent trends:} outward \( A_+ \) decays by cascade and ion heating; inward \( A_- \) is generated by reflection or parametric decay—reproducing standard fast-wind observations.

\bigskip
\hrule
\bigskip

\section*{In One Sentence}

\begin{quote}
\textbf{Advance \( A_\pm \) with an upwind-cascade equation, subtract exactly the power \( Q_{\text{ion}} \) that your isotropic Parker Monte-Carlo adds to the particles, and feed the updated \( A_\pm \) back into the next step’s scattering frequency.}
\end{quote}

That closes the loop between turbulence transport and Parker-equation particle evolution \textbf{without ever tracking \( \mu \) explicitly}.

\end{tcolorbox}


\begin{tcolorbox}[colback=white, colframe=black, title={From Parker‐Equation Transport Coefficient $D_{xx}$ to a Monte-Carlo Scattering Probability}, fonttitle=\bfseries]

In an \textbf{isotropic} Parker equation the spatial diffusion coefficient parallel to the mean field is
\begin{equation}
D_{xx} = \frac{1}{3} v \lambda,
\tag{1}
\end{equation}
where
\begin{itemize}
  \item $v$ – particle speed in the plasma frame,
  \item $\lambda$ – \textit{scattering mean free path} (average distance travelled between independent pitch-angle randomisations).
\end{itemize}

\medskip

\textbf{1. Scattering frequency}

Solve Eq.~(1) for $\lambda$:
\[
\lambda = \frac{3 D_{xx}}{v}.
\]

A particle that moves at speed $v$ crosses one mean free path in the \textbf{scattering time}
\begin{equation}
\tau_{\text{sc}} = \frac{\lambda}{v} = \frac{3 D_{xx}}{v^2}.
\tag{2}
\end{equation}

Hence the \textbf{scattering frequency}
\begin{equation}
\nu_{\text{sc}} = \tau_{\text{sc}}^{-1} = \frac{v^2}{3 D_{xx}}.
\tag{3}
\end{equation}

\medskip

\textbf{2. Probability of at least one scattering in a time step $\Delta t$}

Scattering events are Poisson-distributed with rate $\nu_{\text{sc}}$. Therefore
\begin{equation}
\boxed{
P_{\text{scatt}}(\Delta t) = 1 - \exp\!\left[-\nu_{\text{sc}} \Delta t\right] = 1 - \exp\!\left[-\frac{v^2 \Delta t}{3 D_{xx}}\right]}.
\tag{4}
\end{equation}

\begin{itemize}
  \item \textbf{Small-step limit:} If $\Delta t \ll \tau_{\text{sc}}$ (typical Monte-Carlo choice),
  \[
  P_{\text{scatt}} \approx \frac{v^2}{3 D_{xx}} \Delta t.
  \]
  
  \item \textbf{Relativistic extension:} Replace $v$ with $v = \beta c$ and, if desired, scale $D_{xx}$ with the particle’s rigidity $R$; the form of Eq.~(4) is unchanged.
\end{itemize}

\medskip

\textbf{3. Implementation snippet}
\begin{lstlisting}[mathescape=true, basicstyle=\ttfamily\small]
# inputs: $v$, $D_{xx}$, $\Delta t$
$\nu_{\text{sc}} = \dfrac{v \times v}{3.0 \times D_{xx}}$
$P_{\text{sc}} = 1.0 - \exp(-\nu_{\text{sc}} \times \Delta t)$
if random() < $P_{\text{sc}}$:
    # perform isotropic re-orientation (full scattering)
\end{lstlisting}


Choose $\Delta t$ so that $P_{\text{scatt}} \lesssim 0.3$ for numerical stability; larger probabilities can be handled but require multi-scatter logic.

\medskip

\textbf{Result:}

Using only the Parker-equation coefficient $D_{xx}$, the probability that a particle undergoes an Alfvén-wave pitch-angle scattering during the next interval $\Delta t$ is given by Eq.~(4).

\end{tcolorbox}


\begin{tcolorbox}[colback=white, colframe=black, title={Setup}]
\begin{itemize}
\item Two counter-propagating Alfvén-wave populations: 
$W_{+}(k)$ – outward (anti-Sunward) waves, and $W_{-}(k)$ – inward (Sunward) waves.
\item A particle of speed $v$, pitch-angle cosine $\mu$ resonates with a single wavenumber in each population:
\[
k_{\text{res}}^{\,+} = \frac{\Omega}{v\mu - V_A}, \qquad
k_{\text{res}}^{\,-} = \frac{\Omega}{v\mu + V_A}.
\]
\item The \textbf{pitch-angle diffusion coefficient} is the sum of the two contributions:
\[
D_{\mu\mu}(\mu) = D_{\mu\mu}^{\,+}(\mu) + D_{\mu\mu}^{\,-}(\mu),
\]
where (quasi-linear theory)
\[
D_{\mu\mu}^{\,\pm}(\mu) = \frac{\pi \Omega^{2}}{B_0^{2}} (1 - \mu^{2}) 
\frac{W_{\pm}\!\left(k_{\text{res}}^{\pm}\right)}{\left|k_{\text{res}}^{\pm}\right|}.
\]
\end{itemize}

\medskip

\textbf{1. Probability that \emph{a} scattering occurs in the next interval $\Delta t$}

Each wave population supplies an independent Poisson process with rate 
$\nu_{\pm} = 2 D_{\mu\mu}^{\,\pm}(\mu)$. The total rate is
\[
\nu_{\text{tot}} = 2\left(D_{\mu\mu}^{\,+} + D_{\mu\mu}^{\,-}\right) \equiv 2 D_{\mu\mu}.
\]
Hence
\begin{equation}
\boxed{
P_{\text{scatt}}(\Delta t) = 1 - \exp\left[-\nu_{\text{tot}} \Delta t\right] = 1 - \exp\left[-2 D_{\mu\mu} \Delta t\right]
}.
\tag{1}
\end{equation}

Small-step approximation ($2 D_{\mu\mu} \Delta t \ll 1$):
\[
P_{\text{scatt}} \simeq 2 D_{\mu\mu} \Delta t.
\]

\medskip

\end{tcolorbox}
\begin{tcolorbox}[colback=white, colframe=black, title={Setup}]

\textbf{2. Probability that the scattering comes from a \emph{particular} wave band}

Divide each population into narrow $k$-bins (or ``bands'') labelled by index $j$. For band $j$ in the $+$ population let
\[
\nu_{j}^{\,+} = 2 D_{\mu\mu,j}^{\,+} 
= \frac{2\pi \Omega^{2}}{B_0^{2}} (1 - \mu^{2}) \frac{W_{+,j}}{k_{j}},
\]
and analogously $\nu_{j}^{\,-} = 2 D_{\mu\mu,j}^{\,-}$.

\medskip

\textbf{2-a Absolute probability in the next $\Delta t$}

Because the individual Poisson processes are independent,
\begin{equation}
P_{j}^{\pm}(\Delta t) = 1 - \exp\left[-\nu_{j}^{\pm} \Delta t\right]
\approx \nu_{j}^{\pm} \Delta t = 2 D_{\mu\mu,j}^{\pm} \Delta t.
\tag{2}
\end{equation}

\medskip

\textbf{2-b Conditional probability given that a scattering \emph{will} occur}

If you first decide whether any scattering happens (probability $P_{\text{scatt}}$), then choose which band caused it, the branching ratio is
\begin{equation}
\boxed{
\mathcal{P}_{j}^{\pm} = 
\frac{\nu_{j}^{\pm}}{\nu_{\text{tot}}} = 
\frac{D_{\mu\mu,j}^{\pm}}{D_{\mu\mu}^{\,+} + D_{\mu\mu}^{\,-}}
}.
\tag{3}
\end{equation}
so that $\sum_{j} \left( \mathcal{P}_{j}^{+} + \mathcal{P}_{j}^{-} \right) = 1$.

\medskip

\textbf{Monte-Carlo algorithm (per particle per step)}
\begin{lstlisting}[mathescape=true, basicstyle=\ttfamily\small]
$\Delta t$ chosen so that $\nu_{\text{tot}} \times \Delta t \lesssim 0.3$
$\nu_{j}^{+}  = 2 \times D_{\mu\mu,j}^{+}$      # for all k-bins that resonate
$\nu_{j}^{-}  = 2 \times D_{\mu\mu,j}^{-}$
$\nu_{\text{tot}} = \sum \left( \nu_{j}^{+} + \nu_{j}^{-} \right )$

# 1. decide if any scatter
if random() < 1 - exp($-\nu_{\text{tot}} \times \Delta t$):      # Eq. (1)
    # 2. pick band and direction
    r = random() $\times \nu_{\text{tot}}$                      # roulette wheel
    cumulative = 0
    for each j:
        cumulative += $\nu_{j}^{+}$
        if r < cumulative: choose band j, direction +
        break
        cumulative += $\nu_{j}^{-}$
        if r < cumulative: choose band j, direction -
        break
    # 3. draw Gaussian kick $\Delta \mu$ with variance $2 D_{\mu\mu,j}^{\pm} \times \Delta t$
\end{lstlisting}

This procedure exactly realizes the probabilities (1)–(3).

\end{tcolorbox}

\begin{tcolorbox}[colback=white, colframe=black, title={Pitch–Angle Cosine After a Single Monte-Carlo Scattering Step}]

\textbf{1. Frames of reference}
\begin{itemize}
\item \textbf{Plasma frame (lab)} – the Parker equation and your Monte-Carlo particles are expressed here; pitch-angle cosine is
\[
\mu \equiv \cos\theta = \frac{\mathbf{v} \cdot \mathbf{B}_0}{v B_0}.
\]
\item \textbf{Wave frame} – the frame translating along the mean field at the Alfvén speed $V_A$ in the direction of the resonant wave packet ($+$ for anti-Sunward, $-$ for Sunward waves).
In this frame the electric field of the Alfvén wave vanishes and the gyro-resonant diffusion coefficient $D_{\mu\mu}$ is derived.
\end{itemize}

You \emph{apply} the stochastic pitch-angle kick in the \textbf{wave frame}, then convert the new cosine back to the plasma frame.

\medskip

\textbf{2. Diffusive kick in the wave frame}

For a time step $\Delta t$, the quasilinear theory gives
\[
\langle(\Delta \mu_{\text{wf}})^2\rangle = 2 D_{\mu\mu}^{\text{(res)}} \Delta t.
\]
Draw one Gaussian deviate $\xi \sim \mathcal{N}(0,1)$ and set
\begin{equation}
\boxed{
\mu_{\text{wf}}' = \mu_{\text{wf}} + \sqrt{2 D_{\mu\mu}^{\text{(res)}} \Delta t} \, \xi
}
\tag{1}
\end{equation}
where $\mu_{\text{wf}}$ is the particle’s current pitch-angle cosine measured in the wave frame:
\[
\mu_{\text{wf}} = \frac{\mu - \sigma \beta_A}{1 - \sigma \beta_A \mu},
\qquad
\beta_A \equiv \frac{V_A}{c}, \quad
\sigma = \begin{cases}
+1 & (\text{outward wave})\\[3pt]
-1 & (\text{inward wave}).
\end{cases}
\]

\medskip

\textbf{3. Boundary reflection}

If $|\mu_{\text{wf}}'| > 1$ after Eq.~(1), use specular reflection to keep it inside $[-1,1]$:
\[
\mu_{\text{wf}}' \gets
\begin{cases}
2 - \mu_{\text{wf}}', & \mu_{\text{wf}}' > 1, \\[2pt]
-2 - \mu_{\text{wf}}', & \mu_{\text{wf}}' < -1.
\end{cases}
\]
(Iterate once more if an extreme kick overshoots both walls.)

\medskip

\textbf{4. Transform back to the plasma frame}

A Lorentz boost of velocity components parallel to $B_0$ gives
\begin{equation}
\boxed{
\mu' = \frac{\mu_{\text{wf}}' + \sigma \beta_A}{1 + \sigma \beta_A \mu_{\text{wf}}'}
}
\tag{2}
\end{equation}
with the same sign $\sigma$ used in step 2.

\medskip

\textit{Non-relativistic or SEP limit $v \gg V_A$:}
\[
\beta_A \ll 1 \quad \Rightarrow \quad \mu' \simeq \mu_{\text{wf}}'
\]
to better than $V_A / v$.

\medskip

\end{tcolorbox}
\begin{tcolorbox}[colback=white, colframe=black, title={Pitch–Angle Cosine After a Single Monte-Carlo Scattering Step}]

\textbf{5. Summary algorithm (per particle)}
\begin{lstlisting}[mathescape=true, basicstyle=\ttfamily\small]
# inputs: $\mu$ (plasma frame), sign $\sigma$ (wave direction), $D_{\mu\mu}^{\text{(res)}}$, $\Delta t$
$\beta_A$   = $V_A / c$
$\mu_{\text{wf}}$ = ( $\mu - \sigma \beta_A$ ) / ( $1 - \sigma \beta_A \mu$ )

# Gaussian kick in wave frame
$\Delta \mu$ = sqrt( $2 D_{\mu\mu}^{\text{(res)}} \Delta t$ ) * normal_rand()
$\mu_{\text{wf}}^{\text{new}}$ = $\mu_{\text{wf}} + \Delta \mu$
$\mu_{\text{wf}}^{\text{new}}$ = reflect( $\mu_{\text{wf}}^{\text{new}}$ )      # keep in [-1,1]

# transform back to plasma frame
$\mu^{\text{new}}$ = ( $\mu_{\text{wf}}^{\text{new}} + \sigma \beta_A$ ) / ( $1 + \sigma \beta_A \mu_{\text{wf}}^{\text{new}}$ )
\end{lstlisting}


\medskip

\textbf{Result:}

Equation (1) gives the stochastic update of pitch angle \emph{in the wave frame}; Equation (2) returns the new cosine $\mu'$ that your Monte-Carlo solver stores in the plasma (Parker) frame.

\end{tcolorbox}


\begin{tcolorbox}[colback=white, colframe=black, title={Why Does the Frame‐Change Formula Use $V_A/c$ Instead of Just $V_A$?}]

\textbf{1. Two frames and one boost}

\medskip

\begin{tabular}{@{}p{0.35\linewidth} p{0.25\linewidth} p{0.35\linewidth}@{}}
\toprule
\textbf{Frame} & \textbf{Symbol} & \textbf{Velocity of the boost} \\
\midrule
Plasma (lab) frame & unprimed quantities $(\mu, v_\parallel, v)$ & -- \\
Wave frame (co-moving with an Alfvén wave packet) & ``wf'' quantities $(\mu_{\rm wf})$ & $+V_A$ along $\mathbf{B}_0$ for outward waves ($\sigma = +1$); $-V_A$ for inward waves ($\sigma = -1$) \\
\bottomrule
\end{tabular}

\medskip

To move from the plasma frame to the wave frame you perform a \textbf{Lorentz boost} of speed $\sigma V_A$ along the magnetic-field direction.

\medskip

\textbf{2. Lorentz transformation of a velocity component}

\medskip

For any velocity component parallel to the boost, special relativity gives
\[
v_\parallel^{\rm(wf)} = \frac{v_\parallel - \sigma V_A}{1 - \sigma \dfrac{v_\parallel V_A}{c^2}},
\]
where $c$ is the \textbf{speed of light in vacuum}.

\medskip

\textbf{3. Expressing the result as a pitch-angle cosine}

\medskip

Divide by the particle speed $v$ to get pitch-angle cosine:
\[
\mu_{\rm wf}
= \frac{v_\parallel^{\rm(wf)}}{v}
= \frac{\mu - \sigma \beta_A}{1 - \sigma \beta_A \mu}, 
\qquad 
\beta_A \equiv \frac{V_A}{c}.
\]

Hence the appearance of the $\boldsymbol{V_A/c}$ ratio.

\medskip

\textbf{4. Why use the full relativistic formula even for non-relativistic SEPs?}

\begin{itemize}
    \item \textbf{Correct for all energies} – Some Monte-Carlo particles may reach relativistic speeds; the exact boost keeps the algorithm valid without extra branches.
    \item \textbf{Consistency across modules} – Wave growth and damping formulas already rely on relativistic gyro-frequency $\Omega = qB/(\gamma m)$; using the same kinematics avoids subtle bookkeeping errors.
    \item \textbf{Negligible overhead} – Computing with $\beta_A = V_A/c \lesssim 10^{-3}$ adds one multiply and divide per particle — insignificant compared to random-number generation.
\end{itemize}

\medskip

In the common SEP limit where $v \gg V_A$ (e.g., $v \sim 30{,}000\,\text{km/s}$ vs $V_A \sim 50\,\text{km/s}$),
$\beta_A \ll 1$ and the formula reduces to the intuitive Galilean shift:
\[
\mu_{\rm wf} \simeq \mu - \sigma \frac{V_A}{v},
\]
but writing it with $V_A/c$ ensures exactness whenever it matters.

\medskip

\end{tcolorbox}
\begin{tcolorbox}[colback=white, colframe=black, title={Why Does the Frame‐Change Formula Use $V_A/c$ Instead of Just $V_A$?}]

\textbf{5. What is $c$ here?}

\medskip

$c$ is the fundamental constant 
\[
c = 2.9979 \times 10^{8}\;\text{m\,s}^{-1}.
\]
Its role is purely kinematic: it scales the boost velocity $V_A$ into a dimensionless $\beta_A = V_A/c$ so that the relativistic velocity‐addition law can be applied.

\end{tcolorbox}


\begin{tcolorbox}[colback=white, colframe=black, title={Detailed Derivation of the Pitch–Angle Cosine in the \textbf{Wave Frame}}]

\[
\boxed{
\mu_{\text{wf}} = \frac{\mu - \sigma \beta_A}{1 - \sigma \beta_A \mu}
},
\qquad
\beta_A \equiv \frac{V_A}{c}, \quad
\sigma = 
\begin{cases}
+1 & \text{(outward wave)} \\ 
-1 & \text{(inward wave)}
\end{cases}
\]

\medskip

\textbf{Step 0: Set the geometry}

\begin{itemize}
\item Choose the $z$-axis along the mean magnetic field $\vb{B}_0$.
\item \textbf{Plasma (lab) frame:} particle velocity components $\vb{v} = (v_\perp, v_\parallel)$ with $\mu = v_\parallel / v$.
\item \textbf{Wave frame:} moves at speed $\sigma V_A$ along $+z$ (Alfvén speed). Denote wave-frame quantities with subscript ``wf''.
\end{itemize}

\medskip

\textbf{Step 1: Lorentz boost along the field}

For a boost of speed $u = \sigma V_A$ parallel to $z$, the exact special-relativistic velocity–addition law gives
\begin{equation}
v_\parallel^{\text{(wf)}} = \frac{v_\parallel - u}{1 - \dfrac{u v_\parallel}{c^2}}.
\tag{1}
\end{equation}

The perpendicular component transforms as $v_\perp^{\text{(wf)}} = v_\perp / [\gamma_u (1 - uv_\parallel/c^2)]$ with $\gamma_u = (1 - u^2/c^2)^{-1/2}$, but we will not need it explicitly.

\medskip

\textbf{Step 2: Write everything in terms of $\mu$}

Insert $u = \sigma V_A$ and $v_\parallel = \mu v$ into Eq.~(1):
\begin{equation}
v_\parallel^{\text{(wf)}} 
= \frac{\mu v - \sigma V_A}{1 - \sigma \dfrac{V_A}{c^2} \mu v}
= \frac{\mu v - \sigma V_A}{1 - \sigma \beta_A \mu \dfrac{v}{c}}.
\tag{2}
\end{equation}

Define $\beta \equiv v/c$ and $\beta_A \equiv V_A/c$:
\begin{equation}
v_\parallel^{\text{(wf)}}
= \frac{c(\beta \mu - \sigma \beta_A)}{1 - \sigma \beta_A \beta \mu}.
\tag{3}
\end{equation}

\medskip

\textbf{Step 3: Transform total speed $v$}

The total speed in the wave frame is
\begin{equation}
v^{\text{(wf)}} = \frac{v}{\gamma_u (1 - uv_\parallel / c^2)} = \frac{v}{\gamma_A (1 - \sigma \beta_A \beta \mu)},
\quad
\gamma_A = (1 - \beta_A^2)^{-1/2}.
\tag{4}
\end{equation}

\medskip

\end{tcolorbox}
\begin{tcolorbox}[colback=white, colframe=black, title={Detailed Derivation of the Pitch–Angle Cosine in the \textbf{Wave Frame}}]

\textbf{Step 4: Form the pitch–angle cosine in the wave frame}

By definition,
\begin{equation}
\mu_{\text{wf}} = \frac{v_\parallel^{\text{(wf)}}}{v^{\text{(wf)}}}.
\tag{5}
\end{equation}

Insert Eqs.~(3) and (4):
\[
\mu_{\text{wf}} = \frac{c(\beta \mu - \sigma \beta_A)}{1 - \sigma \beta_A \beta \mu}
\div
\frac{v}{\gamma_A (1 - \sigma \beta_A \beta \mu)}
= \frac{\beta \mu - \sigma \beta_A}{\beta}
= \frac{\mu - \sigma \beta_A}{1 - \sigma \beta_A \mu}.
\tag{6}
\]
(*Multiply numerator and denominator by $(1 - \sigma \beta_A \mu)$ and cancel $\beta$.*)

Thus we recover the compact formula:
\[
\boxed{
\mu_{\text{wf}} = \frac{\mu - \sigma \beta_A}{1 - \sigma \beta_A \mu}
},
\qquad
\beta_A = \frac{V_A}{c}.
\]

\medskip

\textbf{Step 5: Non-relativistic limit for SEPs}

For solar-wind Alfvén speeds $V_A \sim 50$--$100\,\text{km\,s}^{-1}$ and SEP speeds $v \gtrsim 10^4\,\text{km\,s}^{-1}$:
\[
\beta_A = \frac{V_A}{c} \ll 1, \qquad
\beta = \frac{v}{c} \lesssim 0.1,
\]
so the denominator differs from unity by at most $\mathcal{O}(\beta_A)$. To first order:
\[
\mu_{\text{wf}} \simeq \mu - \sigma \frac{V_A}{v},
\]
but using the full relativistic form ensures exactness for any particle energy and keeps the code consistent with relativistic gyro-frequency formulas.

\medskip

\textbf{What is $c$ here?}

\[
c = 2.998 \times 10^8\,\text{m\,s}^{-1}
\]
is the speed of light in vacuum. It converts the boost velocity $V_A$ into the dimensionless parameter $\beta_A = V_A/c$ required by Lorentz (not Galilean) velocity addition.

\end{tcolorbox}

\begin{tcolorbox}[colback=white, colframe=black, title={Step-by-Step Derivation of Pitch–Angle Cosine in the Wave Frame}]

\textbf{Goal}
\[
\boxed{
\mu_{\text{wf}} = \frac{\mu - \sigma \beta_A}{1 - \sigma \beta_A \mu}
},
\qquad
\beta_A \equiv \frac{V_A}{c}, \quad
\sigma =
\begin{cases}
+1 & \text{outward (anti-Sunward) wave},\\
-1 & \text{inward (Sunward) wave}.
\end{cases}
\]

\medskip

\textbf{0. Geometry and notation}

\begin{itemize}
\item Field–aligned $z$-axis ($\vb{B}_0 \parallel \vu{z}$).
\item Plasma–frame velocity $\vb{v} = (v_\perp, v_\parallel)$ with pitch-angle cosine $\mu = \dfrac{v_\parallel}{v}$.
\item Wave frame moves at speed $\sigma V_A$ along $+\vu{z}$.
\end{itemize}

\medskip

\textbf{1. Lorentz boost of the parallel component}

For a boost velocity $u = \sigma V_A$ along $z$,
\begin{equation}
v_\parallel^{\text{(wf)}} = 
\frac{v_\parallel - u}{1 - \dfrac{u v_\parallel}{c^2}}.
\tag{1}
\end{equation}

\medskip

\textbf{2. Substitute $u$ and $v_\parallel$}

Let $\beta = v/c$ and $\beta_A = V_A / c$.

Insert $u = \sigma V_A$ and $v_\parallel = \mu v = \mu \beta c$:
\begin{equation}
v_\parallel^{\text{(wf)}}
= \frac{\mu \beta c - \sigma \beta_A c}{1 - \sigma \beta_A \beta \mu}
= c \frac{\beta \mu - \sigma \beta_A}{1 - \sigma \beta_A \beta \mu}.
\tag{2}
\end{equation}

\medskip

\textbf{3. Lorentz boost of the total speed}

The total speed in the wave frame is:
\begin{equation}
v^{\text{(wf)}} =
\frac{v}{\gamma_A (1 - \sigma \beta_A \beta \mu)},
\qquad
\gamma_A = \frac{1}{\sqrt{1 - \beta_A^2}}.
\tag{3}
\end{equation}

\medskip

\textbf{4. Form the cosine in the wave frame}

\[
\mu_{\text{wf}}
= \frac{v_\parallel^{\text{(wf)}}}{v^{\text{(wf)}}}
= \frac{c(\beta \mu - \sigma \beta_A)}{1 - \sigma \beta_A \beta \mu}
\times
\frac{\gamma_A (1 - \sigma \beta_A \beta \mu)}{v}.
\]

Because $v = \beta c$, the $c$ and $(1 - \sigma \beta_A \beta \mu)$ terms cancel:
\begin{equation}
\mu_{\text{wf}}
= \frac{\beta \mu - \sigma \beta_A}{\beta}
= \frac{\mu - \sigma \beta_A}{1 - \sigma \beta_A \mu}.
\tag{4}
\end{equation}

This is exactly the boxed result.

\medskip

\textbf{5. Non-relativistic (SEP) limit}

If $v \gg V_A$, then $\beta_A \ll 1$ and
\[
\mu_{\text{wf}} \simeq \mu - \sigma \frac{V_A}{v},
\]
the familiar Galilean correction, obtained as the small-$\beta_A$ limit of the fully relativistic expression.



\end{tcolorbox}

\newpage

\subsection{Probability that a particle will be scatted on Alfven}

Sure! Let’s proceed carefully.

You want to assume:
\begin{itemize}
    \item \textbf{Kolmogorov turbulence} for Alfvén waves,
    \item \textbf{Isotropic SEP distribution} described by the \textbf{Parker equation},
    \item and derive the \textbf{probability that a particle will scatter} due to interaction with the Alfvénic turbulence during a time step $\Delta t$,
    \item assuming pitch-angle diffusion governed by quasi-linear theory (QLT),
    \item with no explicit dependence on $\mu$ (since the Parker equation describes the \emph{isotropic part} of $f$).
\end{itemize}

\hrulefill

\section*{\texorpdfstring{\textbf{Derivation}}{}}

\hrulefill

\subsection*{1. \textbf{Start with pitch-angle diffusion in QLT}}

In quasi-linear theory for Alfvén wave turbulence, the \textbf{pitch-angle diffusion coefficient} for a particle of speed $v$, pitch-angle cosine $\mu$, and charge $q$ is:
\begin{equation}
D_{\mu\mu}(\mu, v) = \frac{\pi \Omega^2}{B_0^2} (1 - \mu^2) \frac{W(k_{\text{res}})}{|k_{\text{res}}|}
\tag{1}
\end{equation}
where:
\begin{itemize}
    \item $\Omega = \frac{q B_0}{\gamma m}$ is the relativistic gyrofrequency,
    \item $k_{\text{res}}$ is the resonant wavenumber:
\end{itemize}
\begin{equation}
k_{\text{res}} = \frac{\Omega}{v \mu \mp V_A}
\end{equation}
with $V_A$ being the Alfvén speed,

\begin{itemize}
    \item and $W(k)$ is the spectral energy density of Alfvén waves at wavenumber $k$.
\end{itemize}

\hrulefill

\subsection*{2. \textbf{Kolmogorov spectrum for Alfvén waves}}

Assume the turbulence follows a \textbf{Kolmogorov inertial-range} scaling:
\begin{equation}
W(k) = A\,k^{-5/3}
\tag{2}
\end{equation}
where $A$ is the amplitude of the spectrum.

Insert into (1):
\begin{equation}
D_{\mu\mu}(\mu, v) = \frac{\pi \Omega^2}{B_0^2} (1 - \mu^2) A |k_{\text{res}}|^{-8/3}
\tag{3}
\end{equation}

\hrulefill

\subsection*{3. \textbf{Average over isotropic pitch-angle distribution}}

For an \textbf{isotropic distribution}, $f(\mu) \approx \text{const}$, the \textbf{average scattering rate} (pitch-angle decorrelation rate) is obtained by integrating $D_{\mu\mu}$ over $\mu \in [-1,1]$ with uniform probability:
\begin{equation}
\bar{D}_{\mu\mu}(v) = \frac{1}{2} \int_{-1}^{+1} D_{\mu\mu}(\mu, v) \, d\mu
\tag{4}
\end{equation}
Substituting (3):
\begin{equation}
\bar{D}_{\mu\mu}(v) = \frac{\pi \Omega^2 A}{2 B_0^2} \int_{-1}^{+1} (1 - \mu^2) \left| \frac{v \mu \mp V_A}{\Omega} \right|^{8/3} \, d\mu
\tag{5}
\end{equation}
Simplify:
\begin{equation}
\bar{D}_{\mu\mu}(v) = \frac{\pi A}{2 B_0^2} \Omega^{-2/3} v^{-8/3} \int_{-1}^{+1} (1 - \mu^2) |\mu \mp V_A/v|^{-8/3} \, d\mu
\tag{6}
\end{equation}
This integral, call it $I(v)$, is usually calculated numerically or approximated for $v \gg V_A$, but remains well-defined.

\hrulefill

\subsection*{4. \textbf{Define scattering frequency}}

The \textbf{scattering frequency} $\nu_{\text{sc}}(v)$ is twice the averaged diffusion coefficient (from standard diffusion theory):
\begin{equation}
\nu_{\text{sc}}(v) = 2 \bar{D}_{\mu\mu}(v)
\tag{7}
\end{equation}
Thus:
\begin{equation}
\nu_{\text{sc}}(v) = \frac{\pi A}{B_0^2} \Omega^{-2/3} v^{-8/3} I(v)
\tag{8}
\end{equation}

\hrulefill

\subsection*{5. \textbf{Probability of scattering in time $\Delta t$}}

The probability that a particle undergoes a scattering event in a short time interval $\Delta t$ is:
\begin{equation}
P_{\text{scatt}} = 1 - e^{-\nu_{\text{sc}}(v) \Delta t}
\tag{9}
\end{equation}
For small $\Delta t$ (i.e., $\nu_{\text{sc}} \Delta t \ll 1$), this reduces to:
\begin{equation}
P_{\text{scatt}} \approx \nu_{\text{sc}}(v) \Delta t
\tag{10}
\end{equation}

\hrulefill

\section*{\texorpdfstring{ \textbf{References}}{}}

This derivation is based on well-established quasi-linear theory and Parker transport theory for SEPs:
\begin{enumerate}
    \item \textbf{Jokipii, J. R. (1966)} --- \textit{Cosmic-Ray Propagation I. Charged Particles in a Random Magnetic Field} [ApJ, 146, 480].
    \item \textbf{Skilling, J. (1975)} --- \textit{Cosmic ray streaming. I. Effect of Alfvén waves on particles} [MNRAS, 172, 557--566].
    \item \textbf{Schlickeiser, R. (1989)} --- \textit{Cosmic-ray transport and acceleration I. Derivation of the kinetic equation and application to cosmic rays in static cold media} [ApJ, 336, 243].
    \item \textbf{Bieber, J. W. et al. (1994)} --- \textit{Proton and electron mean free paths: The Palmer consensus revisited} [ApJ, 420, 294].
    \item \textbf{Zank, G. P. et al. (1998)} --- \textit{Particle acceleration at perpendicular shocks} [JGR, 103, 2085].
    \item \textbf{Shalchi, A. (2009)} --- \textit{Nonlinear Cosmic Ray Diffusion Theories} (Springer).
    \item \textbf{Zank, G. P., Rice, W. K. M., \& Wu, C. C. (2000)} --- \textit{Particle acceleration and coronal mass ejection driven shocks: A theoretical model} [JGR, 105, 25079].
    \item \textbf{Matthaeus, W. H. et al. (2003)} --- \textit{Nonlinear collisionless damping of noncompressive MHD turbulence} [ApJ, 590, L53].
    \item \textbf{Qin, G., Matthaeus, W. H., \& Bieber, J. W. (2002)} --- \textit{Subdiffusive transport of charged particles perpendicular to the large scale magnetic field} [GRL, 29, 1048].
    \item \textbf{Zhang, M. (2006)} --- \textit{The radial dependence of the parallel mean free path of solar energetic particles} [ApJ, 645, 1421].
\end{enumerate}

\hrulefill

\section*{\texorpdfstring{ \textbf{Final Summary Formula}}{}}

The \textbf{probability} that a particle will scatter on Kolmogorov Alfvén-wave turbulence in a time step $\Delta t$ is:
\begin{equation}
\boxed{P_{\text{scatt}}(v) = 1 - e^{-\nu_{\text{sc}}(v)\,\Delta t}}
\end{equation}
where
\begin{equation}
\boxed{\nu_{\text{sc}}(v) = \frac{\pi A}{B_0^2} \Omega^{-2/3} v^{-8/3} I(v)}
\end{equation}
with $I(v)$ being the isotropic integral over pitch angles.

\hrulefill

Would you like me to proceed further and:
\begin{itemize}
    \item Give a \textit{numerical example} assuming solar-wind conditions?
    \item Approximate $I(v)$ for the case where $v \gg V_A$ (common for SEPs)?
\end{itemize}


\subsection{Derivation of Scattering Efficiency}




You’re asking:
\begin{itemize}
    \item After a scattering event off Alfvén wave turbulence, \textbf{what happens to the particle’s velocity}?
    \item \textbf{How much energy is exchanged} between the particle and the turbulence?
\end{itemize}

We’ll proceed step-by-step, and \textbf{support with references}.

\hrulefill

\section*{\texorpdfstring{\textbf{1. Particle Velocity After Scattering}}{}}

In \textbf{quasilinear theory (QLT)}:
\begin{itemize}
    \item The particle undergoes \textbf{pitch-angle diffusion} without a large change in energy.
    \item \textbf{Elastic scattering} occurs in the \textbf{wave frame} (the frame moving with the wave) --- typically at Alfvén speed $V_A$ along the magnetic field.
\end{itemize}

\subsection*{1.1. \textbf{Frame of Reference}}

We consider:
\begin{itemize}
    \item \textbf{Plasma frame}: the frame of the background magnetic field $\mathbf{B}_0$.
    \item \textbf{Wave frame}: moving at $\pm V_A$ along $\mathbf{B}_0$, depending on wave propagation direction.
\end{itemize}

In the \textbf{wave frame}, the particle’s total energy is approximately conserved.

\subsection*{1.2. \textbf{Energy Conservation in the Wave Frame}}

The \textbf{kinetic energy} of the particle in the wave frame is:
\begin{equation}
E' = \gamma' m c^2 - m c^2
\end{equation}
where $\gamma'$ is the Lorentz factor in the wave frame:
\begin{equation}
\gamma' = \frac{1}{\sqrt{1 - (v'_{\parallel} - V_A)^2/c^2 - v_{\perp}^2/c^2}}
\end{equation}

However, for \textbf{SEP energies} ($v \sim 0.1c$) and \textbf{solar wind turbulence} ($V_A \sim 10^{-4}c$), $V_A \ll v$, so the Lorentz transformation simplifies.

In the \textbf{non-relativistic limit} ($v \ll c$):
\begin{itemize}
    \item Scattering is approximately \textbf{elastic in the wave frame}.
    \item The particle’s energy changes only because of the \textbf{change of frames}.
\end{itemize}

\subsection*{1.3. \textbf{Velocity After Scattering}}

Since scattering is mainly \textbf{pitch-angle diffusion}:
\begin{itemize}
    \item The \textbf{speed $v$} of the particle remains approximately the \textbf{same} in the plasma frame.
    \item Only the \textbf{pitch-angle cosine} $\mu$ changes due to the scattering process.
\end{itemize}

Thus, after scattering:
\begin{equation}
\boxed{v_{\text{after}} \approx v_{\text{before}}}
\end{equation}
but the pitch-angle distribution becomes more isotropic.

\medskip
\noindent
\textbf{References}:
\begin{enumerate}
    \item Jokipii (1966) --- \textit{Cosmic-Ray Propagation I} [ApJ 146, 480].
    \item Schlickeiser (2002) --- \textit{Cosmic Ray Astrophysics}, Chapter 4 (Elastic scattering).
    \item Melrose (1980) --- \textit{Plasma Astrophysics}, Vol. 1.
\end{enumerate}

\hrulefill

\section*{\texorpdfstring{ \textbf{2. Energy Exchange with Alfvénic Turbulence}}{}}

Although the scattering is elastic in the wave frame, in the \textbf{plasma frame}, the particle \textbf{gains or loses energy} relative to the plasma due to the Doppler shift from the moving wave frame.

This leads to \textbf{energy exchange} between particles and the wave field.

\subsection*{2.1. \textbf{Energy Change Per Scattering}}

The average \textbf{energy gain or loss} per scattering can be written (for non-relativistic speeds) as:
\begin{equation}
\Delta E = - p_{\parallel} \Delta v_{\parallel}
\end{equation}
where:
\begin{itemize}
    \item $p_{\parallel} = m v_{\parallel}$ is the parallel momentum,
    \item $\Delta v_{\parallel}$ is the pitch-angle scattering increment in $v_{\parallel}$.
\end{itemize}

Using quasilinear theory, the \textbf{rate of energy change} for a single particle is:
\begin{equation}
\frac{dE}{dt} = - m V_A \frac{d v_{\parallel}}{dt}
\end{equation}
because scattering acts to isotropize the distribution by changing $v_{\parallel}$.

This is often written in terms of the \textbf{pitch-angle diffusion coefficient}:
\begin{equation}
\frac{dE}{dt} = - m V_A v \left\langle \frac{d\mu}{dt} \right\rangle
\tag{1}
\end{equation}
where $\left\langle \frac{d\mu}{dt} \right\rangle$ is linked to $D_{\mu\mu}$.

\textbf{Key point}:
\begin{itemize}
    \item Particles interacting with Alfvén waves tend to:
    \begin{itemize}
        \item \textbf{Gain energy} if moving \textbf{opposite} to the wave ($\mu < 0$),
        \item \textbf{Lose energy} if moving \textbf{with} the wave ($\mu > 0$).
    \end{itemize}
\end{itemize}

Thus, overall, the wave and particle can \textbf{exchange energy}.

\subsection*{2.2. \textbf{Energy Transfer to the Waves}}

Due to the action-reaction principle (conservation of total energy):
\begin{itemize}
    \item The energy change of the \textbf{wave} is \textbf{opposite} to the particle’s energy change.
\end{itemize}

Thus:
\begin{equation}
\boxed{\Delta W_{\text{wave}} = - \Delta E_{\text{particle}}}
\tag{2}
\end{equation}

In a Monte-Carlo model, for each scattering event where a particle changes its pitch angle, the energy exchange with the wave field is calculated as:
\begin{equation}
\Delta W = - m V_A v \Delta \mu
\tag{3}
\end{equation}
for each particle.

\subsection*{2.3. \textbf{Interpretation}}

Even though the particle speed stays the same, there is \textbf{energy transfer} between the particle and the turbulence because of the \textbf{moving frame} of the Alfvén waves.

\medskip
\noindent
\textbf{References}:
\begin{enumerate}
    \setcounter{enumi}{3}
    \item Earl (1974) --- \textit{The Diffusion of Cosmic Rays across a Magnetic Field} [ApJ 193, 231].
    \item Skilling (1975) --- \textit{Cosmic ray streaming I: effect of Alfvén waves on particles} [MNRAS 172, 557].
    \item Völk (1975) --- \textit{Transport Processes of Cosmic Rays} [Rev. Geophys. Space Phys. 13, 547].
    \item McKenzie \& Völk (1982) --- \textit{Dynamical Effects of Cosmic Rays in the Galaxy} [A\&A 116, 191].
    \item Schlickeiser (1989) --- \textit{Cosmic-ray transport and acceleration I}.
    \item Malkov \& Drury (2001) --- \textit{Nonlinear Theory of Diffusive Acceleration} [Rep. Prog. Phys. 64, 429].
    \item Shalchi (2009) --- \textit{Nonlinear Cosmic Ray Diffusion Theories}, Chapter 3.
\end{enumerate}

\hrulefill

\section*{\texorpdfstring{ \textbf{Summary}}{}}

\begin{center}
\begin{tabular}{@{}lll@{}}
\toprule
\textbf{Quantity} & \textbf{Result} \\
\midrule
\textbf{Particle speed after scattering} & $v_{\text{after}} \approx v_{\text{before}}$ --- no significant change (elastic in wave frame) \\
\textbf{Energy change} & $\Delta E_{\text{particle}} = - m V_A v \Delta \mu$ \\
\textbf{Wave energy change} & $\Delta W_{\text{wave}} = - \Delta E_{\text{particle}}$ --- action-reaction principle \\
\bottomrule
\end{tabular}
\end{center}

Particles gain or lose energy \textbf{depending on their pitch angle}, and the \textbf{waves gain or lose energy accordingly}.

\hrulefill

Would you like me to show:
\begin{itemize}
    \item The \textit{full ensemble average} energy transfer rate (e.g., how turbulence damps by particle scattering)?
    \item \textit{Numerical examples} of the size of energy exchanged for SEPs at 1 AU?
\end{itemize}



\subsection{\textbf{General Derivation: Energy Exchange During Scattering on Alfvén Turbulence}}

\hrulefill

\subsection*{1. \textbf{Resonant Condition}}

For a particle interacting with Alfvén waves:
\begin{itemize}
    \item Cyclotron resonance condition:
\end{itemize}
\begin{equation}
\omega - k_\parallel v \mu = \pm \Omega
\tag{1}
\end{equation}
For \textbf{Alfvén waves} ($\omega = \pm V_A k_\parallel$), we get:
\begin{equation}
k_\parallel = \frac{\Omega}{v \mu \mp V_A}
\tag{2}
\end{equation}
where:
\begin{itemize}
    \item $\Omega = \frac{q B_0}{\gamma m}$ --- relativistic gyrofrequency,
    \item $V_A$ --- Alfvén speed,
    \item $\mu \equiv \cos\theta$ --- pitch-angle cosine.
\end{itemize}

\noindent
\textbf{References}:
\begin{enumerate}
    \item Schlickeiser (1989),
    \item Kulsrud (2005), \textit{Plasma Physics for Astrophysics}.
\end{enumerate}

\hrulefill

\subsection*{2. \textbf{Frame Transformations}}

Let’s consider:
\begin{itemize}
    \item \textbf{Plasma frame}: Rest frame of solar wind background.
    \item \textbf{Wave frame}: Frame moving at $\pm V_A$ along $B_0$.
\end{itemize}

Define:
\begin{itemize}
    \item In plasma frame: parallel velocity $v_\parallel = v \mu$,
    \item In wave frame:
\end{itemize}
\begin{equation}
v'_\parallel = v_\parallel \mp V_A
\tag{3}
\end{equation}

The \textbf{total energy} in the wave frame is:
\[
E' = \gamma' m c^2,
\]
with:
\[
\gamma' = \frac{1}{\sqrt{1 - \frac{(v'_\parallel)^2 + v_\perp^2}{c^2}}}.
\]

If scattering is \textbf{elastic in the wave frame}, $E'$ is conserved during the pitch-angle diffusion process:
\begin{equation}
\Delta E' = 0.
\tag{4}
\end{equation}

\hrulefill

\subsection*{3. \textbf{Energy Change in the Plasma Frame}}

The particle’s energy in the \textbf{plasma frame} is:
\[
E = \gamma m c^2.
\]

The connection between frames:
\begin{equation}
E = \gamma_w ( E' + p'_\parallel V_A )
\tag{5}
\end{equation}
where:
\begin{itemize}
    \item $\gamma_w = \frac{1}{\sqrt{1 - V_A^2/c^2}}$ --- Lorentz factor of the wave frame relative to plasma frame,
    \item $p'_\parallel = \gamma' m v'_\parallel$.
\end{itemize}

Now, differentiate to find the energy change in the plasma frame due to small changes in pitch-angle scattering:
\begin{equation}
dE = \gamma_w \left( d p'_\parallel \right) V_A
\tag{6}
\end{equation}
since $\Delta E' = 0$, and $\gamma_w$ and $V_A$ are constants.

Now:
\[
p'_\parallel = \gamma' m (v \mu \mp V_A),
\]
so for small changes $d\mu$ in pitch-angle:
\begin{equation}
d p'_\parallel = \gamma' m v \, d\mu + m (v \mu \mp V_A) \, d\gamma'.
\end{equation}
But for small-angle scattering, the dominant term is:
\begin{equation}
d p'_\parallel \approx \gamma' m v \, d\mu
\tag{7}
\end{equation}
because $d\gamma'$ is second-order small in $d\mu$.

Thus:
\begin{equation}
dE = \gamma_w \gamma' m v V_A \, d\mu
\tag{8}
\end{equation}

\hrulefill

\subsection*{4. \textbf{Energy Change Per Scattering}}

For a finite change $\Delta \mu$, the total energy change per scattering event is:
\begin{equation}
\boxed{ \Delta E = \gamma_w \gamma' m v V_A \, \Delta \mu }
\tag{9}
\end{equation}

This equation works for:
\begin{itemize}
    \item \textbf{Thermal ions} ($v \sim V_A$),
    \item \textbf{SEPs} ($v \gg V_A$),
    \item \textbf{Any intermediate case}.
\end{itemize}

No assumptions have been made about $v \gg V_A$ or $v \ll V_A$.

\noindent
\textbf{References}:
\begin{enumerate}
    \setcounter{enumi}{2}
    \item Earl (1974) --- \textit{The Diffusion of Cosmic Rays across a Magnetic Field} [ApJ 193, 231],
    \item Skilling (1975),
    \item Schlickeiser (2002), \textit{Cosmic Ray Astrophysics}.
\end{enumerate}

\hrulefill

\subsection*{5. \textbf{Energy Exchange with the Wave Field}}

\textbf{Action--reaction}: The energy gained or lost by the particle is lost or gained by the wave field.

Thus, for the wave field:
\begin{equation}
\boxed{ \Delta W_{\text{wave}} = - \Delta E }
\tag{10}
\end{equation}

Every scattering event transfers a small amount of energy between particles and waves.
\begin{itemize}
    \item If $\Delta \mu > 0$ (particle moving toward larger pitch angles), it gains energy,
    \item If $\Delta \mu < 0$ (particle moves toward smaller pitch angles), it loses energy.
\end{itemize}

\noindent
\textbf{References}:
\begin{enumerate}
    \setcounter{enumi}{5}
    \item Völk (1975),
    \item Melrose (1980), \textit{Plasma Astrophysics} Vol. 1,
    \item McKenzie \& Völk (1982).
\end{enumerate}

\hrulefill

\subsection*{6. \textbf{Interpretation}}

\textbf{For Thermal Ions}:
\begin{itemize}
    \item $v \sim V_A$,
    \item The energy change per scattering is \textbf{significant},
    \item The scattering leads to \textbf{heating} of thermal plasma --- important in solar wind.
\end{itemize}

\textbf{For SEPs}:
\begin{itemize}
    \item $v \gg V_A$,
    \item The energy change per scattering is \textbf{small} (energy transfer efficiency $\sim V_A/v$),
    \item Scattering primarily leads to \textbf{pitch-angle isotropization}, with only minor energy exchange.
\end{itemize}

This explains why:
\begin{itemize}
    \item \textbf{Alfvén turbulence can heat the solar wind} (thermal ions),
    \item \textbf{SEPs scatter off turbulence} with little energy loss.
\end{itemize}

\hrulefill

\section*{\texorpdfstring{ \textbf{Supporting References (full list)}}{}}

\begin{enumerate}
    \item \textbf{Jokipii, J. R. (1966)} --- \textit{Cosmic-Ray Propagation I} [ApJ, 146, 480].
    \item \textbf{Schlickeiser, R. (1989)} --- \textit{Cosmic-ray transport and acceleration} [ApJ, 336, 243].
    \item \textbf{Kulsrud, R. M. (2005)} --- \textit{Plasma Physics for Astrophysics} (Princeton).
    \item \textbf{Earl, J. A. (1974)} --- \textit{Diffusion of Cosmic Rays Across Magnetic Fields} [ApJ, 193, 231].
    \item \textbf{Skilling, J. (1975)} --- \textit{Cosmic Ray Streaming I} [MNRAS, 172, 557].
    \item \textbf{Völk, H. J. (1975)} --- \textit{Transport Processes of Cosmic Rays} [Rev. Geophys., 13, 547].
    \item \textbf{Melrose, D. B. (1980)} --- \textit{Plasma Astrophysics}, Vol. 1 (Gordon and Breach).
    \item \textbf{McKenzie, J. F. \& Völk, H. J. (1982)} --- \textit{Dynamical Effects of Cosmic Rays in the Galaxy} [A\&A, 116, 191].
    \item \textbf{Malkov, M. A. \& Drury, L. O'C. (2001)} --- \textit{Nonlinear theory of diffusive acceleration} [Rep. Prog. Phys., 64, 429].
    \item \textbf{Schlickeiser, R. (2002)} --- \textit{Cosmic Ray Astrophysics} (Springer).
\end{enumerate}

\hrulefill

\section*{\texorpdfstring{ \textbf{Final Boxed Results}}{}}

\begin{equation}
\boxed{ \Delta E = \gamma_w \gamma' m v V_A \, \Delta \mu }
\end{equation}

\begin{equation}
\boxed{ \Delta W_{\text{wave}} = - \Delta E }
\end{equation}

No assumptions about $v \gg V_A$ or $v \ll V_A$.



\textbf{Assumptions}:
\begin{enumerate}
    \item \textbf{Isotropic SEP or thermal ion distribution}: no explicit pitch-angle $\mu$ --- just speed $v$ and position $r$.
    \item \textbf{Kolmogorov turbulence}:
    \[
    W(k) = A(r, t)\,k^{-5/3}
    \]
    between $k_{\min}$ and $k_{\max}$ --- \textbf{only the amplitude} $A(r, t)$ evolves.
    \item \textbf{Energy exchange}: Particles scatter on turbulence, transferring energy via pitch-angle diffusion; turbulence energy is updated accordingly.
\end{enumerate}

\hrulefill

\section*{\texorpdfstring{ \textbf{Monte Carlo Algorithm}}{}}

\hrulefill

\subsection*{0. \textbf{Preliminaries}}

\begin{center}
\begin{tabular}{@{}ll@{}}
\toprule
\textbf{Variable} & \textbf{Meaning} \\
\midrule
$r$ & Radial distance along field line (can generalize to 1D Parker spiral) \\
$v$ & Particle speed \\
$w_i$ & Weight of particle $i$ (real particles per MC particle) \\
$A(r,t)$ & Kolmogorov turbulence amplitude at position $r$ and time $t$ \\
$\nu_{\text{sc}}(v,r,t)$ & Scattering frequency \\
$\Delta t$ & Global time step \\
$\Delta r$ & Radial cell size \\
$N_k$ & Normalizing integral over $k$: $N_k = \int_{k_{\min}}^{k_{\max}} k^{-5/3} \, dk$ \\
\bottomrule
\end{tabular}
\end{center}

Turbulence energy density per unit volume:
\begin{equation}
\mathcal{W}(r,t) = A(r,t)\,N_k
\tag{1}
\end{equation}

Alfvén wave group speed:
\[
V_{\text{g}} = V_A - U(r),
\]
where $U(r)$ is the solar wind speed.

\hrulefill

\subsection*{1. \textbf{Initialization}}

\begin{verbatim}
Initialize:
    - Grid in r: r_j, j = 0..N_r
    - A(r, t=0) from some initial model (e.g., $\sim$ r^(-3/2))
    - Particle ensemble {r_i, v_i, w_i}
    - $Delta$t satisfying CFL condition for advection
    - Precompute N_k = (k_max^(-2/3) - k_min^(-2/3)) / (2/3)
\end{verbatim}

\hrulefill

\subsection*{2. \textbf{Per Global Time Step $\Delta t$}}

\begin{lstlisting}[language={}, mathescape=true]
for each time step t:
    1. Wave advection and cascade (no particle coupling yet)
    --------------------------------------------------------
    for each cell j:
        * Compute advection flux:
          $F_j = (V_A - U(r_j)) \times A(r_j)$
        
        * Update $A(r_j)$ via upwind finite difference:
          $A(r_j) \leftarrow A(r_j) - \frac{\Delta t}{\Delta r} \left( F_j - F_{j-1} \right)$
        
        * Apply Kolmogorov cascade sink:
          $\tau_{\text{cas}} = \frac{L_{\perp}(r_j)}{C_k \sqrt{ \frac{2 A(r_j) N_k}{\rho(r_j)} }}$
          
          $P_{\text{cas}} = \frac{A(r_j) N_k}{\tau_{\text{cas}}}$
          
          $A(r_j) \leftarrow A(r_j) - \frac{\Delta t}{N_k} \times P_{\text{cas}}$

    2. Monte Carlo particle advance
    --------------------------------
    Initialize array $P_{ion}(j) = 0$ for all cells j
    for each particle i:
        - Get local cell j for $r_i$
        * Compute gyrofrequency:
          $\Omega_i = \frac{q B(r_j)}{\gamma_i m}$
        
        * Compute scattering frequency:
          $\nu_{\text{sc}}(v_i, r_j) = \left( \frac{\pi A(r_j)}{B(r_j)^2} \right) \times \Omega_i^{-2/3} \times v_i^{-8/3} \times I\left( \frac{v_i}{V_A} \right)$
        
        * Compute probability of scattering:
          $P_{\text{scat}} = 1 - \exp\left( -\nu_{\text{sc}} \times \Delta t \right)$
        
        * Generate random number $\xi \in [0, 1]$.
        
        * If $\xi < P_{\text{scat}}$:
            - Isotropically randomize velocity direction (keep $v_i$ constant).
            - Draw Gaussian increment:
              $\Delta \mu \sim \mathcal{N}\left(0, \sqrt{2 D_{\mu \mu} \Delta t} \right)$
            - Compute particle energy change:
              $\Delta E = \gamma_w \gamma'_i m v_i V_A \Delta \mu$
            - Update energy (optional, for monitoring).
            - Update particle-wave energy exchange:
              $P_{\text{ion}}(j) \leftarrow P_{\text{ion}}(j) - \frac{w_i \times \Delta E}{\Delta t}$
        
        * Move particle (radial drift):
          $r_i \leftarrow r_i + \left( U(r_j) + v_i \mu' \right) \times \Delta t$

    3. Particle-Wave Energy Exchange
    --------------------------------
    For each cell j:
        * Update turbulence amplitude:
          $A(r_j) \leftarrow A(r_j) - \frac{\Delta t}{N_k} \times P_{\text{ion}}(j)$
        * Ensure positivity:
          $A(r_j) \leftarrow \max\left( A(r_j), 0 \right)$

    4. Boundary conditions
    ----------------------
    - Reflect or absorb particles at inner/outer boundaries
    - Inject new turbulence if needed at r = $r_0$ (base)
    
    5. Diagnostics (optional)
    -------------------------
    - Compute particle density n(r), energy density
    - Monitor total wave energy
\end{lstlisting}

\hrulefill

\subsection*{3. \textbf{Important Notes}}

\begin{itemize}
    \item \textbf{Scattering frequency} $\nu_{\text{sc}}(v, r)$ is evaluated \textbf{from the local turbulence amplitude} $A(r)$.
    \item \textbf{Energy conservation} is \textbf{exact}:
    The energy gained by particles is exactly removed from the turbulence $A(r) \times N_k$.
    \item \textbf{Kolmogorov spectrum} is maintained by assuming that the \textbf{shape} $W(k) \propto k^{-5/3}$ stays fixed --- only the amplitude $A(r, t)$ changes.
    \item \textbf{Scattering is isotropic}: no explicit $\mu$ tracking; only speed $v$ and position $r$ are evolved.
\end{itemize}

\hrulefill

\section*{\texorpdfstring{ \textbf{Summary Diagram}}{}}
\[
\begin{array}{c@{\qquad}c}
\texttt{+------------------------+} & \texttt{+--------------------------+} \\
\texttt{|  Turbulence Amplitude  |} & \texttt{|  Monte Carlo Particles   |} \\
\texttt{|      A(r, t)           |} & \texttt{|  \{r\_i, v\_i, w\_i\}     |} \\
\texttt{+------------------------+} & \texttt{+--------------------------+} \\
\multicolumn{2}{c}{\texttt{                    |}} \\
\multicolumn{2}{c}{\texttt{                    |}} \\
\multicolumn{2}{c}{\texttt{         Advection + Kolmogorov Cascade}} \\
\multicolumn{2}{c}{\texttt{              Isotropic Scattering}} \\
\multicolumn{2}{c}{\texttt{                    |}} \\
\multicolumn{2}{c}{\texttt{    +-------------------------------------------+}} \\
\multicolumn{2}{c}{\texttt{            Coupled Energy Exchange}} \\
\end{array}
\]

\section*{\texorpdfstring{ \textbf{Key References}}{}}

\begin{enumerate}
    \item Jokipii (1966) --- \textit{Cosmic-Ray Propagation}.
    \item Schlickeiser (1989) --- \textit{Cosmic-ray transport and acceleration}.
    \item Earl (1974) --- \textit{Cosmic-ray diffusion}.
    \item Shalchi (2009) --- \textit{Nonlinear Cosmic Ray Diffusion Theories}.
    \item Tu \& Marsch (1995) --- \textit{MHD Turbulence in the Solar Wind}.
    \item Bruno \& Carbone (2013) --- \textit{Solar Wind Turbulence}.
    \item Skilling (1975) --- \textit{Cosmic ray streaming I}.
    \item Matthaeus et al. (1999) --- \textit{Evolution of Turbulence in the Solar Wind}.
    \item Malkov \& Drury (2001) --- \textit{Nonlinear Diffusive Shock Acceleration}.
    \item Zank et al. (1998) --- \textit{Particle acceleration at shocks}.
\end{enumerate}

\hrulefill



\subsection{Why Split into $W_+$ and $W_-$?}

\begin{center}
\begin{tabular}{@{}lll@{}}
\toprule
& $W_+$ & $W_-$ \\
\midrule
Propagation direction    & Outward Alfvén waves (away from Sun) & Inward Alfvén waves (toward Sun) \\
Group velocity           & $+V_A$                               & $-V_A$                           \\
Resonance with particles & $\mu > 0$                            & $\mu < 0$                        \\
\bottomrule
\end{tabular}
\end{center}

\begin{itemize}
    \item \textbf{Alfvén waves} are \textbf{non-compressive}, and they can \textbf{propagate only along or against} the magnetic field $\mathbf{B}_0$.
    \item \textbf{SEPs} or \textbf{solar wind ions} interact differently with $W_+$ and $W_-$ because:
    \begin{itemize}
        \item \textbf{Scattering} depends on wave direction relative to particle motion.
        \item \textbf{Wave growth/damping} depends on the net streaming relative to the waves.
    \end{itemize}
\end{itemize}

In quasilinear theory, \textbf{resonant interactions} occur \textbf{only with the wave population moving in the opposite direction} to the particle's parallel motion.

\hrulefill

\subsection*{Formal Definition}

You model:
\[
W_+(k, r, t) \quad \text{(outward, anti-sunward Alfvén waves)},
\]
\[
W_-(k, r, t) \quad \text{(inward, sunward Alfvén waves)}.
\]

Each obeys its own \textbf{transport equation}:
\begin{equation}
\frac{\partial W_\pm}{\partial t} + (V_A \mp U) \frac{\partial W_\pm}{\partial r}
= \text{Growth}_\pm - \text{Cascade}_\pm - \text{Damping}_\pm.
\tag{1}
\end{equation}

\hrulefill

\section*{\texorpdfstring{ \textbf{Why Splitting Matters}}{}}

\begin{enumerate}
    \item \textbf{Anisotropic wave field:} The solar wind typically has much more \textbf{outward} wave power $W_+ \gg W_-$ near the Sun, but backscattering and reflection gradually build $W_-$.
    
    \item \textbf{Particle scattering depends on wave sense:} In QLT, the \textbf{pitch-angle diffusion coefficient} $D_{\mu\mu}$ has contributions:
    \[
    D_{\mu\mu}(\mu) \propto (1 - \mu^2) \left[
    \frac{W_+\bigl(k_{\text{res}}^+\bigr)}{|k_{\text{res}}^+|}
    +
    \frac{W_-\bigl(k_{\text{res}}^-\bigr)}{|k_{\text{res}}^-|}
    \right].
    \tag{2}
    \]
    where
    \[
    k_{\text{res}}^\pm = \frac{\Omega}{v\mu \mp V_A}.
    \]
    
    Each wave field $W_\pm$ resonates at different $k$ depending on the particle's direction of motion.
    
    \item \textbf{Energy exchange (growth/damping) is direction-dependent:} The particles \textbf{gain/lose energy} differently depending on their interaction with $W_+$ or $W_-$. The turbulence damps/grows accordingly.
    
    \item \textbf{Alfvén wave advection:}
    \[
    W_+ \text{ is carried outward faster (group speed } V_A + U\text{)},
    \]
    \[
    W_- \text{ can be convected inward relative to the wind (group speed } V_A - U\text{)}.
    \]
\end{enumerate}

\hrulefill

\section*{\texorpdfstring{ \textbf{How to Adapt the Monte Carlo Algorithm}}{}}

You must track:
\begin{itemize}
    \item $A_+(r, t)$ — amplitude of outward waves.
    \item $A_-(r, t)$ — amplitude of inward waves.
\end{itemize}

Each amplitude evolves by:

\begin{verbatim}
# For W_+
Advection at V_A - U
- Cascade sink
- Energy change from particle scattering (P_ion_plus)

# For W_-
Advection at -V_A - U
- Cascade sink
- Energy change from particle scattering (P_ion_minus)
\end{verbatim}

And the \textbf{scattering frequency} for each particle is now:
\[
\nu_{\text{sc}}(v, r) = \nu_+(v, r) + \nu_-(v, r),
\]
where
\[
\nu_\pm(v, r) = \frac{\pi A_\pm(r)}{B_0(r)^2} \Omega^{-2/3} v^{-8/3} I_\pm(v).
\]
$I_\pm(v)$ are pitch-angle integrals that weight the contribution of $W_+$ and $W_-$.

\hrulefill

\section*{\texorpdfstring{ \textbf{Key References Supporting Split $W_+, W_-$}}{}}

\begin{enumerate}
    \item Jokipii (1966) --- \textit{Cosmic-Ray Propagation I}.
    \item Skilling (1975) --- \textit{Cosmic ray streaming and Alfvén waves}.
    \item Schlickeiser (1989) --- \textit{Cosmic-ray transport}.
    \item Matthaeus et al. (1990) --- \textit{Turbulence transport equations}.
    \item Bieber et al. (1994) --- \textit{Proton and electron mean free paths}.
    \item Zank et al. (1996) --- \textit{Turbulence evolution in the solar wind}.
    \item Shalchi (2009) --- \textit{Nonlinear Cosmic Ray Diffusion Theories}.
    \item Tu \& Marsch (1995) --- \textit{MHD turbulence in the solar wind}.
    \item Ruffolo et al. (2012) --- \textit{Turbulence transport modeling}.
    \item Zank et al. (2012) --- \textit{Transport of Turbulence in the Outer Heliosphere}.
\end{enumerate}

\hrulefill

\section*{\texorpdfstring{ \textbf{Final Boxed Answer}}{}}

\[
\boxed{
W(k, r, t) = W_+(k, r, t) + W_-(k, r, t) \quad \text{(must track both)}
}
\]

and in your Monte Carlo + turbulence coupling:
\begin{itemize}
    \item Track both $A_+(r, t)$ and $A_-(r, t)$ separately,
    \item Calculate scattering and energy transfer for each wave direction,
    \item Update turbulence fields accordingly.
\end{itemize}

\section*{1. Governing Equation (per Propagation Sense)}

For a single wavenumber $k$ bin, or for the Kolmogorov amplitude $A_\pm \propto W_\pm$,
\begin{equation}
\boxed{
\frac{\partial W_\pm}{\partial t}
+ \underbrace{(V_A \mp U_\parallel) \frac{\partial W_\pm}{\partial s}}_{\text{advection}}
= \underbrace{S_\pm}_{\substack{\text{wave–particle} \\ \text{growth/damping}}}
- \underbrace{D_\pm}_{\substack{\text{nonlinear} \\ \text{cascade sink}}}.
}
\tag{1}
\end{equation}

\begin{itemize}
    \item $s$ = arc-length along the magnetic field.
    \item $U_\parallel(s)$ = solar-wind speed \textbf{along} the field.
    \item $V_A(s) = B / \sqrt{\mu_0 \rho}$ = Alfvén speed.
    \item $S_\pm$ and $D_\pm$ are cell-centred source/sink terms (from SEP streaming and Kolmogorov cascade).
\end{itemize}

\hrulefill

\section*{2. Mesh and Notation}

\begin{itemize}
    \item $i = 0, \ldots, N$ (cell centres)
    \item $s_i$ = arc length of centre $i$
    \item $\Delta s_i$ = cell length
    \item $F_{i+1/2}$ = flux through right face
\end{itemize}

Define the \textbf{cell-average} wave energy:
\[
\langle W_\pm \rangle_i(t) = \frac{1}{\Delta s_i} \int_{s_{i-1/2}}^{s_{i+1/2}} W_\pm(s, t)\, ds.
\]

\hrulefill

\section*{3. Semi-Discrete FV Form}

\begin{equation}
\boxed{
\frac{d}{dt} \langle W_\pm \rangle_i
= -\frac{F_{i+1/2} - F_{i-1/2}}{\Delta s_i}
+ S_{\pm, i} - D_{\pm, i}.
}
\tag{2}
\end{equation}

\hrulefill

\section*{4. Numerical Flux $F_{i+1/2}$}

Because advection speed $a_\pm = V_A \mp U_\parallel$ is \textbf{known and sign-definite} at a face, use an \textbf{upwind flux}:

\begin{equation}
F_{i+1/2} =
\begin{cases}
a_{i+1/2}\; W_{\pm,i}^{(\text{L})} & a_{i+1/2} > 0, \\
a_{i+1/2}\; W_{\pm,i+1}^{(\text{R})} & a_{i+1/2} < 0,
\end{cases}
\tag{3}
\end{equation}
where $W^{(\text{L})}$ and $W^{(\text{R})}$ are left/right reconstructions:
\begin{itemize}
    \item \textbf{First-order (robust):} $W^{(\text{L})} = W_i$, $W^{(\text{R})} = W_{i+1}$.
    \item \textbf{Second-order TVD:}
    \[
    W^{(\text{L})} = W_i + \frac{1}{2} \phi(\theta_i)(W_i - W_{i-1}),
    \]
    with limiter (minmod, van Leer, etc.).
\end{itemize}

\textbf{CFL}: $|a_\pm|\, \Delta t < \min(\Delta s)$ for explicit update.

\hrulefill

\section*{5. Source/Sink Discretisation}

\begin{itemize}
    \item \textbf{Wave–particle term} (from the Monte Carlo step):
    \[
    S_{\pm, i} = \frac{\Delta E_{\pm, i}^{(\text{MC})}}{\Delta s_i},
    \]
    where $\Delta E_{\pm, i}^{(\text{MC})}$ is the energy gained/lost by the waves in cell $i$ during $\Delta t$.

    \item \textbf{Cascade sink} for a Kolmogorov inertial range:
    \[
    D_{\pm, i} = \frac{W_{\pm, i}}{\tau_{\text{cas}, i}},
    \]
    with
    \[
    \tau_{\text{cas}}^{-1} = C_K \frac{\sqrt{W_{\pm, i}}}{L_\perp}.
    \]
    (choose $C_K \simeq 0.2$, $L_\perp(s)$ from geometry).
\end{itemize}

Both terms are \textbf{local} and added directly after the flux divergence.

\hrulefill

\section*{6. Time Integration}

\textbf{Second-order TVD Runge–Kutta (RK2):}
\begin{align*}
W^*      &= W^n + \Delta t \cdot \text{RHS}(W^n), \\
W^{n+1}  &= \frac{1}{2} W^n + \frac{1}{2} (W^* + \Delta t \cdot \text{RHS}(W^*)),
\end{align*}
where \texttt{RHS} implements Eq. (2) with fluxes (3) and sources.

\hrulefill

\section*{7. Complete Update Cycle (Pseudo-Algorithm)}

\begin{lstlisting}[language={}, mathescape=true]
for every global step $\Delta t$:
    # 1. compute face velocities $a_{i+1/2} = V_A \pm U_{\text{parallel}}$ (geometry)
    # 2. compute upwind fluxes $F_{i+1/2}$ (Eq. 3 with limiter)
    # 3. Monte-Carlo loop $\rightarrow$ obtain $\Delta E^{\pm}_i$ (wave-particle coupling)
    # 4. compute source arrays   $S_i = \Delta E^{\pm}_i / \Delta s_i$
    # 5. compute cascade sinks   $D_i = W_i / \tau_{\text{cas},i}$
    # 6. perform RK2 finite-volume update (Eq. 2)
    # 7. boundary conditions:
         $\bullet$ inner: prescribed driver or symmetry
         $\bullet$ outer: zero-gradient (outflow) for $W_+$
                   small inflow or reflection for $W_-$ if needed
\end{lstlisting}

\hrulefill

\section*{8. Typical Accuracy and Stability}

\begin{itemize}
    \item \textbf{First-order upwind:} very stable, diffusive; fine for coarse tests.
    \item \textbf{TVD-second-order:} good shock-like front resolution (e.g., abrupt driver).
    \item \textbf{CFL safety factor} 0.4–0.6 keeps RK2 explicit scheme stable.
\end{itemize}

\hrulefill

\section*{9. Core References for Finite-Volume Turbulence Transport}

\begin{enumerate}
    \item \textbf{Zank, G. P. et al.} 1996, \textit{JGR} 101, 457: 1-D turbulence transport in the solar wind.
    \item \textbf{Matthaeus, W. H. \& Velli, M.} 1999, \textit{Space Sci. Rev.} 87, 269: description of turbulence advection/decay equations.
    \item \textbf{Oughton, S. et al.} 2011, \textit{ApJ} 731, 73: FV discretisation of $W_\pm$ equations.
    \item \textbf{Usmanov, A. V. et al.} 2014, \textit{ApJ} 788, 43: TVD upwind scheme for Alfvénic turbulence in 1-D flux tubes.
\end{enumerate}

\hrulefill

\section*{Take-away}

\textit{Integrate Eq. (1) in conservative form (2); use upwind-TVD fluxes for advection, add local Monte-Carlo energy exchange and Kolmogorov sink as centred source terms; advance with an explicit RK2 or RK3 method under a CFL condition determined by $|V_A \mp U_\parallel|$.}
