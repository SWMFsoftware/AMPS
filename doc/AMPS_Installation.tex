\chapter{Installation and execution of the code}

\section{Stand-alone mode}

\subsection{CVS installation}
Full command for AMPS checkout:\\\\
{\tt cvs -d username@herot.engin.umich.edu:/CVS/FRAMEWORK checkout AMPS}
\subsection{SPICE installation}
Go to the following URL to download the newest version of SPICE (cspice)\\\\
\url{http://naif.jpl.nasa.gov/naif/toolkit_C.html}\\\\
choose the proper version and download these two files
\begin{itemize}
 \item cspice.tar.Z
 \item importCSpice.csh
\end{itemize}
Copy these two files into a folder where you want to have you SPICE installation, then\\\\
{\tt /bin/csh importCSPice.csh}\\\\
for installation



\subsection {Installation command}
\begin{enumerate}
\item Add current directory '.' to your PATH, i.e. put  {\tt PATH=\$PATH:.} in your\\
.bashrc file (.bash\_profile for OSX)
\item Make sure that CVS is installed on your computer. 
\item Then go to a folder you wish to install AMPS in via terminal. Then type: \\
{\tt CVS checkout AMPS }\\
AMPS is being set up into a subfolder called AMPS.
\item Now move into the AMPS folder:\\
 {\tt cd AMPS}
\item Now choose an application you wish to use and install the code regarding your interest.
In case you want to do a simulation of the Moon, type the following into the command line:\\
{\tt Config.pl -install -application=moon}\\
You may choose one of the following existing applications:\\ 
{\tt  bullet}, {\tt CG}, {\tt CouplerTest}, {\tt Enceladus}, {\tt Europa}, {\tt Hartley2},
{\tt Hartley2RotationBody}, {\tt Interface}, {\tt Mercury}, {\tt Moon}, {\tt Rosetta}, {\tt Shock}\\
ToDo: SHORT DESCRIPTION OF ALL THESE PRE-DEFINED CASES
\item Set the path for SPICE and for SPICE Kernels by typing the following two lines:\\
{\tt Config.pl -spice-path=/Users/Username/SPICE/cspice }\\
{\tt Config.pl -spice-kernels=/Users/Username/SPICE/Kernels }\\\\
If you do not need any SPICE functions for you simulation it can be turned off by\\\\
{\tt Config.pl -spice-kernels=nospice}
\end{enumerate}
Now the code is ready to compile. Type {\tt make} into the terminal and compile the code. An executable file called ''{\tt amps}'' appears in the AMPS folder. You may run the code now with the executable or submit it via a job-file on a supercomputer. It takes 6-8 hours to run on a single core. The output files are generated under the AMPS folder and data outputs are in the PT/plots folder.

\subsection{Arguments for {\tt Config.pl}}
When using the Config.pl for setting up the code we give values to its arguments by using the {\tt =} sign, for example:\\
{\tt Config.pl -install -application=moon}\\
passes the value {\tt moon} for the {\tt application} argument.\\
The possible arguments are:
\begin{itemize}
\item {\tt Config.pl -application } when setting up the code before compiling it we can choose the object of our interest: {\tt  bullet}, {\tt CG }, {\tt CouplerTest }, {\tt Enceladus }, {\tt Europa }, {\tt Hartley2 }, {\tt Hartley2RotationBody }, {\tt Interface }, {\tt Mercury }, {\tt Moon }, {\tt Rosetta }, {\tt Shock}\\
\item {\tt Config.pl -help } self-explanatory.\\
\item {\tt Config.pl -ices-path } in case we use the code in stand-alone mode, we have to use a pair of input files generaded by BATSRUS: icesCellCenterCoordinates.MHD.dat and ices.data.dat. We have to set the path for these files with this argument. \\
\item {\tt Config.pl -show} shows the current settings of Config.pl and AMPS, showing the paths and application we currently use.\\
\item {\tt Config.pl -spice-kernels } sets the path for SPICE Kernels.\\
\item {\tt Config.pl -spice-path } sets the path for SPICE.\\
\end{itemize}



\section {Component of SWMF}
\subsection {Passing arguments from SWMF's Config.pl}
\subsection {Reading SWMF's PARAM.in file}
\subsection {Debugging}




