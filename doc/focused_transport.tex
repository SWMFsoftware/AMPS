\Section{Focused Transport}

% Empirical Shock Module in LaTeX

\section*{Empirical Shock Module for CME-Driven Shocks}

Below is a \textbf{ready-to-code empirical shock module} that simultaneously returns the \textbf{heliocentric position $r(t)$}, \textbf{speed $V_{\text{sh}}(t)$}, and \textbf{density-compression ratio $r_{\rho}(t)$} of an interplanetary shock driven by a CME (or any piston) as it propagates along a single magnetic-field line. All formulae are published; sources are cited after each paragraph.

\subsection*{1. Kinematics: the Drag-Based Model (DBM)}

The most widely used empirical law for Sun-to-1 au shock kinematics is the \textbf{drag-based ODE} \cite{Vrsnak2007}:
\[
\boxed{\;
\frac{dV_{\text{sh}}}{dt}\;=\;-\gamma\bigl(V_{\text{sh}}-U_{\text{sw}}\bigr)
\;\bigl|V_{\text{sh}}-U_{\text{sw}}\bigr|\,,
\quad
\gamma\simeq(0.2\text{--}2)\times10^{-7}\ \text{km}^{-1}\;}
\]
where $U_{\text{sw}}$ is the ambient solar-wind speed (use WSA/ACE value or 400 km s$^{-1}$). Solve the ODE with initial conditions $(t_0, r_0, V_0)$. The \textbf{analytic solution} is:
\[
V_{\text{sh}}(t) = U_{\text{sw}} \pm \frac{V_0 - U_{\text{sw}}}{1 \pm \gamma |V_0 - U_{\text{sw}}| (t - t_0)}
\]
\[
r(t) = r_0 + \int_{t_0}^{t} V_{\text{sh}}(t')\,dt'
\]
(This ``$+$'' sign is used if $V_0 > U_{\text{sw}}$; the integral has a closed form in the DBM literature.)

\textbf{Typical parameters:} $\gamma = 0.5{\times}10^{-7}$ km$^{-1}$ (fast CMEs) or $1.5{\times}10^{-7}$ km$^{-1}$ (slow CMEs) reproduce average shock-arrival errors of $\pm$7 h at 1 au.

\subsection*{2. Alternative Single-Line ``ESA'' Predictor}

If you need only the \textbf{arrival time and speed at 1 au}, the \textbf{Empirical Shock Arrival (ESA) model} gives:
\[
t_{\mathrm{arr}}(\text{h}) = 203 - 29.2\,\ln V_{\text{CME}},
\qquad
V_{\text{sh}}(1\ \mathrm{au}) = 273 + 79\,\ln V_{\text{CME}} \; (\text{km s}^{-1})
\]
where $V_{\text{CME}}$ is the near-Sun CME linear speed (km s$^{-1}$).

\subsection*{3. Upstream Alfv\'{e}n Mach Number}

At every time step obtain the \textbf{upstream Alfv\'{e}n speed}:
\[
V_A(r) = \frac{B(r)}{\sqrt{\mu_0 m_p n(r)}}
\]
using an empirical $B(r)$ (e.g., Parker spiral) and density $n(r) \propto r^{-2}$. Then compute the \textbf{Alfv\'{e}n Mach number}:
\[
M_A(t) = \frac{V_{\text{sh}}(t) - U_{\text{sw}}}{V_A(t)}
\]

\subsection*{4. Compression Ratio from Mach Number}

Wind / ACE shock statistics yield an empirical fit:
\[
\boxed{\; r_{\rho}(M_A) = 1 + \frac{3.1\,M_A}{2 + M_A} \;}
\]
(valid for $2 \lesssim M_A \lesssim 10$; smoothly approaches the RH limit 4 when $M_A \gg 10$).

If you prefer the textbook Rankine--Hugoniot value ($\gamma = 5/3$):
\[
r_{\rho,\mathrm{RH}}(M_A) = \frac{(\gamma+1)M_A^2}{(\gamma-1)M_A^2+2} \quad (\le 4)
\]

\subsection*{5. Summary of the Time Loop}
\section*{DBM Algorithm — Mathematical Formulation}

\subsection*{Input Parameters}
\begin{itemize}
  \item $r_0$: Initial position
  \item $V_0$: Initial velocity
  \item $t_0$: Initial time
  \item $\gamma$: Deceleration parameter
  \item $U_{\mathrm{sw}}$: Solar wind velocity
  \item $B(r)$: Magnetic field as a function of position
  \item $n(r)$: Number density as a function of position
\end{itemize}

\subsection*{Algorithm — for every output time $t$:}

\paragraph{1. DBM Kinematics}
\textbf{Shock velocity:}
\[
V_{\mathrm{sh}}(t) = U_{\mathrm{sw}} + \frac{V_0 - U_{\mathrm{sw}}}{1 + \gamma \, |V_0 - U_{\mathrm{sw}}| \, (t - t_0)}
\]

\textbf{Shock position:}
\[
r(t) = r_0 + \frac{V_0 - U_{\mathrm{sw}}}{\gamma \, |V_0 - U_{\mathrm{sw}}|} \ln\left[1 + \gamma \, |V_0 - U_{\mathrm{sw}}| \, (t - t_0)\right] + U_{\mathrm{sw}} (t - t_0)
\]

\paragraph{2. Upstream Plasma Properties}
\textbf{Alfvén speed:}
\[
V_A(r) = \frac{B(r)}{\sqrt{\mu_0 m_p n(r)}}
\]

\textbf{Alfvénic Mach number:}
\[
M_A(t) = \frac{V_{\mathrm{sh}}(t) - U_{\mathrm{sw}}}{V_A(r(t))}
\]

\paragraph{3. Compression Ratio}
\[
\rho_{\mathrm{ratio}} = 1 + \frac{3.1 M_A}{2 + M_A}
\quad \text{(or use Rankine–Hugoniot formula)}
\]

\paragraph{4. Output}
\[
\left\{ t,\; r(t),\; V_{\mathrm{sh}}(t),\; \rho_{\mathrm{ratio}} \right\}
\]


\subsection*{6. When to Refresh $\gamma$}

Drag coefficient $\gamma$ depends weakly on heliographic longitude and ambient Alfv\'{e}n speed. A piecewise constant approximation:
\[
\gamma(r) \approx
\begin{cases}
(1.6\times10^{-7})\ \text{km}^{-1} & r < 40\,R_\odot \\
(0.4\times10^{-7})\ \text{km}^{-1} & r > 40\,R_\odot
\end{cases}
\]

\subsection*{7. Typical Performance}
\begin{center}
\begin{tabular}{|l|c|c|}
\hline
Metric (vs Wind/ACE 1996--2018) & DBM + $\gamma$ ensemble & ESA single-line \\
\hline
Arrival-time RMS & $\pm$6.8 h & $\pm$7.3 h \\
Speed RMS error  & $\pm$15\% & $\pm$20\% \\
Compression-ratio RMS & $\pm$0.4 & (n/a) \\
\hline
\end{tabular}
\end{center}

\subsection*{Key References}
\begin{itemize}
  \item Vrsnak \& Zic 2007, \emph{SoPh} 246, 393 --- Drag-Based Model derivation
  \item Gopalswamy et al. 2005, \emph{SpWea} 3, S08004 --- ESA model calibration
  \item Janvier et al. 2018, \emph{SWSC} 8, A11 --- density compression vs $M_A$
  \item Bale et al. 2005, \emph{JGR} 110, A02104 --- IP-shock compression statistics
  \item Vrsnak et al. 2021, \emph{Front. Astron. Space Sci.} 8, 639986 --- DBM tools review
\end{itemize}

Using these equations you can generate $r(t)$, $V_{\text{sh}}(t)$, and $r_{\rho}(t)$ with a few lines of code and plug them directly into your SEP transport/turbulence solver.

\section*{1. Wave--particle energy exchange from first principles}

For small-amplitude Alfvén waves of either propagation sense ($+$ anti-sunward, $-$ sunward), quasi-linear theory gives the \textbf{rate of change of wave energy density} at a single resonant wavenumber $k$ as \cite{Jokipii1966,Skilling1975,Schlickeiser1989}:

\begin{equation}
\frac{dW_\pm(k)}{dt}
= - \oint d^3p\, \frac{V_A}{v}\,p_\parallel\;
\delta\left(k - \frac{\Omega}{v\mu \mp V_A}\right)
(1 - \mu^2)\;g(\mathbf{p},t),
\label{eq:wavegrowth}
\end{equation}

where $g \equiv (4\pi p^2)f$ is the isotropic phase-space density, and $\Omega$ is the gyro-frequency. The factor $V_A/v$ appears because energy exchange is a Doppler shift between the wave frame (speed $\pm V_A$) and the plasma frame \cite[§2]{Skilling1975}.

\section*{2. Effect of exact isotropy: no growth, only damping}

For an exactly \textbf{isotropic distribution} $f(p)$:

\begin{itemize}
  \item the particle streaming $S(k)=\int p_\parallel\,g\,\delta(\dots)\,d^3p$ vanishes,
  \item hence the \textbf{growth term} in \eqref{eq:wavegrowth} is zero,
  \item but the integrand is strictly \textbf{positive} (particles always remove wave energy), so
\end{itemize}

\begin{equation}
\boxed{
\gamma_{\rm damp}(k)=
-\frac{1}{2W_\pm(k)}\frac{dW_\pm(k)}{dt} < 0
}
\label{eq:damping}
\end{equation}

Thus isotropic SEPs (or thermal ions) \textbf{damp} Alfvén waves at all resonant $k$, independent of whether $v \gtrsim V_A$ or $v \lesssim V_A$ \cite{Earl1974,Voelk1975,Ruffolo1995}.

\section*{3. Monte Carlo evaluation without velocity-ratio assumptions}

\begin{table}[H]
\centering
\renewcommand{\arraystretch}{1.2}
\begin{tabular}{|c|p{7cm}|p{5cm}|}
\hline
\textbf{Step} & \textbf{Operation in the code (cell-centred)} & \textbf{Formula} \\
\hline
\textbf{a} & When particle $i$ scatters, draw a Gaussian pitch-angle increment $\Delta\mu_i$. & $\langle(\Delta\mu)^2\rangle = 2D_{\mu\mu}\,\Delta t$ \\
\hline
\textbf{b} & Energy change in plasma frame (elastic in wave frame). & $\Delta E_i = +\gamma_w \gamma_i\, m\,v_i\,V_A\,\Delta\mu_i$ \cite{Melrose1980} \\
\hline
\textbf{c} & Identify resonant wavenumber for the scattering sense. & $k_i = \frac{\Omega_i}{v_i \mu_i \mp V_A}$ \\
\hline
\textbf{d} & Accumulate \textbf{wave bucket}: & $\Delta W_\pm(k_j) \leftarrow \sum_{i \in k_j} w_i\,\Delta E_i$ \\
\hline
\textbf{e} & Damping rate after $\Delta t$: & $\gamma_j = \frac{\Delta W_\pm(k_j)}{2W_\pm(k_j)\,\Delta t}$ \\
\hline
\end{tabular}
\caption{Monte Carlo evaluation of wave damping without approximations on $v/V_A$ (per-particle evaluation \cite{Eichler1979,Shalchi2009})}
\end{table}

\section*{4. Finite-volume wave update coupled to Monte Carlo}

In each spatial cell:
\begin{equation}
W_\pm^{\,n+1}(k_j) =
W_\pm^{\,n}(k_j) + \Delta t \left[
-\,2\gamma_j W_\pm^n(k_j)
- \frac{F_{j+½}-F_{j-½}}{\Delta s}
+ S_{\rm inj}
- \frac{W_\pm}{\tau_{\rm cas}}
\right]
\label{eq:waveupdate}
\end{equation}

with fluxes $F$ and cascade sink discretized as in the finite-volume scheme of \cite{Matthaeus1999,Oughton2011}.

\section*{5. Implementation checklist}

\begin{itemize}
  \item \textbf{Grid in $k$}: 30--50 logarithmic bins.
  \item \textbf{Per time step}:
    \begin{enumerate}
      \item Advect $W_\pm$ (upwind method).
      \item Scatter Monte Carlo particles $\rightarrow$ list $\{\Delta\mu_i\}$.
      \item Accumulate $\Delta W(k)$ buckets.
      \item Apply equation \eqref{eq:waveupdate} to get $W_\pm^{n+1}$.
    \end{enumerate}
  \item \textbf{Stability}: CFL on $V_A \mp U$; damping is implicit since $\Delta W$ is evaluated with advanced particle states.
\end{itemize}

\section*{6. Key References}

\begin{enumerate}
    \item Jokipii (1966), \textit{ApJ}, 146, 480
    \item Skilling (1975), \textit{MNRAS}, 172, 557
    \item Earl (1974), \textit{ApJ}, 193, 231
    \item Völk (1975), \textit{Rev. Geophys.}, 13, 547
    \item Melrose (1980), \textit{Plasma Astrophysics, Vol. 1}
    \item Ruffolo (1995), \textit{ApJ}, 442, 861
    \item Schlickeiser (1989), \textit{ApJ}, 336, 243
    \item Malkov \& Drury (2001), \textit{Rep. Prog. Phys.}, 64, 429
    \item Shalchi (2009), \textit{Non-linear Cosmic-Ray Diffusion Theories}
    \item Matthaeus \& Velli (1999), \textit{Space Sci. Rev.}, 87, 269
\end{enumerate}


\section*{Why an Isotropic Particle Population Cannot Amplify Alfvén Waves}

\begin{table}[H]
\centering
\renewcommand{\arraystretch}{1.4}
\begin{tabular}{|p{3.2cm}|p{5.5cm}|p{5.5cm}|}
\hline
\textbf{Growth Ingredient} & \textbf{What It Needs} & \textbf{What an Isotropic SEP Population Supplies} \\
\hline
\textbf{Coherent driving} of a wave mode & A net flux of particles moving preferentially in one direction along the magnetic field (streaming) & Zero: for every particle with pitch-angle $+\mu$ there is one with $-\mu$ $\rightarrow$ first-order flux cancels \\
\hline
\textbf{Positive work} on the wave & A positive correlation between the wave electric field $E_\parallel$ and the particle current $j_\parallel$ at the resonance & No correlation on average: $\langle j_\parallel \rangle = 0$ \\
\hline
\textbf{Quasilinear growth rate} & $\gamma_\pm(k) \propto \left.\frac{\partial f}{\partial \mu}\right|_{\mu = \mu_{\rm res}}$ or equivalently the resonant streaming $S_\pm(k) = \int p_\parallel f\,\delta(\dots)\,d^3p$ & $\partial f/\partial\mu = 0$, $S_\pm(k) = 0$ \\
\hline
\end{tabular}
\caption{Why an isotropic distribution cannot generate wave growth.}
\end{table}

\subsection*{1. Quasilinear Mathematics}

The quasilinear growth/damping rate for Alfvén waves is \cite{Jokipii1966, Skilling1975}:
\begin{equation}
\gamma_\pm(k) =
\frac{\pi^2 e^2 V_A}{c B_0^2}\,
\frac{1}{k} \int d^3p\, p_\parallel\,
\delta\left(k - \frac{\Omega}{v\mu \mp V_A}\right)
f(p, \mu),
\end{equation}

where the streaming integral
\[
S_\pm(k) = \int d^3p\, p_\parallel\, \delta(\dots)\, f(p,\mu)
\]
is \textbf{zero} for isotropic $f(p,\mu) = f(p)$ because the integrand is odd in $p_\parallel = p\mu$:

\[
S_\pm(k) = 0 \quad \Longrightarrow \quad \boxed{\gamma_\pm(k) = 0}.
\]

\subsection*{2. Physical Picture}

\begin{itemize}
  \item Alfvén waves are transverse; they exchange energy with particles only through coherent resonant motion.
  \item To \textbf{grow} the wave, particles must deliver a systematic push in the direction of the wave’s magnetic perturbation.
  \item An isotropic swarm pushes equally in both directions, so every elementary gain event is cancelled by an equal loss event $\Rightarrow$ \textbf{no net amplification}.
\end{itemize}

Think of two people pumping a playground swing from opposite sides at exactly the same rate—the swing receives no net energy.

\subsection*{3. What Still Happens: Damping}

Although the first-order (streaming) term cancels, the second-order correlation between random pitch-angle kicks and the wave frame does not cancel.

Particles always see the wave frame moving at $V_A$; pitch-angle isotropisation transfers energy \textit{from the wave to the particles} \cite{Earl1974, Voelk1975}.

Hence, an isotropic distribution produces a \textbf{pure damping rate}:
\begin{equation}
\left.\frac{dW}{dt}\right|_{\text{particles}} = -2|\gamma_{\rm damp}|\,W, 
\qquad \gamma_{\rm damp} < 0,
\end{equation}

but never a positive growth rate.

\subsection*{4. Key Sources}

\begin{enumerate}
    \item \textbf{Jokipii, J.R.} (1966), \textit{ApJ}, 146, 480 — original QLT growth formula
    \item \textbf{Skilling, J.} (1975), \textit{MNRAS}, 172, 557 — resonance \& streaming integral
    \item \textbf{Earl, J.} (1974), \textit{ApJ}, 193, 231 — elastic scattering, energy exchange
    \item \textbf{Völk, H.J.} (1975), \textit{Rev. Geophys.}, 13, 547 — damping by isotropic CRs
    \item \textbf{Schlickeiser, R.} (2002), \textit{Cosmic-Ray Astrophysics}, §9 — full derivation
\end{enumerate}

\medskip

\noindent Therefore, in any Monte Carlo or finite-volume SEP code:
\begin{itemize}
  \item \textbf{Omit wave-growth terms} when the local particle distribution is forced to remain isotropic,
  \item Retain only the damping term $-2|\gamma_{\rm damp}|W$ obtained from the summed particle energy losses.
\end{itemize}

\section*{Correct Form of the Resonance Condition}

Not quite—the resonance condition \textbf{has a division, not a multiplication.}  
For low-frequency Alfvén waves, the gyro-resonance requirement is:

\begin{equation}
\boxed{
\omega - k_\parallel v\mu = \pm\Omega
}
\end{equation}

Because an Alfvén wave satisfies $\omega = \pm V_A k_\parallel$ (upper sign for waves travelling \textbf{along} $B$, lower for \textbf{opposite}), substitute and solve for the parallel wavenumber $k_\parallel$:

\begin{align*}
\pm V_A k_\parallel - k_\parallel v\mu &= \pm\Omega \\
\Longrightarrow\quad k_\parallel &= \frac{\Omega}{v\mu \mp V_A}
\end{align*}

So the correct resonance formula is:

\begin{equation}
\boxed{
k_\parallel = \frac{\Omega}{v\mu \mp V_A}
}
\end{equation}

—not $k = \Omega\, (v\mu - V_A)$.  
The denominator keeps the units consistent: $k$ in rad\,m$^{-1}$, $\Omega$ in rad\,s$^{-1}$, and both $v$ and $V_A$ in m\,s$^{-1}$.

\subsection*{Where the Minus/Plus Signs Come From}

\begin{table}[H]
\centering
\renewcommand{\arraystretch}{1.3}
\begin{tabular}{|c|c|c|}
\hline
\textbf{Wave Sense} & \textbf{Dispersion Relation} & \textbf{Resonant $k$} \\
\hline
\textbf{Outward (+)} & $\omega = +V_A k$ & $k = \dfrac{\Omega}{v\mu - V_A}$ \\
\textbf{Inward (–)}  & $\omega = -V_A k$ & $k = \dfrac{\Omega}{v\mu + V_A}$ \\
\hline
\end{tabular}
\caption{Sign conventions for Alfvén wave resonance.}
\end{table}

These expressions reduce to the commonly quoted approximation $k \simeq \Omega / (v\mu)$ when $v \gg V_A$.

\subsection*{References}

\begin{itemize}
    \item Jokipii, J.R. (1966), \textit{ApJ}, 146, 480
    \item Skilling, J. (1975), \textit{MNRAS}, 172, 557
    \item Schlickeiser, R. (2002), \textit{Cosmic-Ray Astrophysics}, Eq.~(9.22)
\end{itemize}

\medskip

\noindent Hence, in any Monte Carlo or quasilinear calculation, you should use the \textbf{fractional} form with the appropriate sign—not a simple product.


\section*{Correct Form of the Resonance Condition}

Not quite—the resonance condition \textbf{has a division, not a multiplication.}  
For low-frequency Alfvén waves, the gyro-resonance requirement is:

\begin{equation}
\boxed{
\omega - k_\parallel v\mu = \pm\Omega
}
\end{equation}

Because an Alfvén wave satisfies $\omega = \pm V_A k_\parallel$ (upper sign for waves travelling \textbf{along} $B$, lower for \textbf{opposite}), substitute and solve for the parallel wavenumber $k_\parallel$:

\begin{align*}
\pm V_A k_\parallel - k_\parallel v\mu &= \pm\Omega \\
\Longrightarrow\quad k_\parallel &= \frac{\Omega}{v\mu \mp V_A}
\end{align*}

So the correct resonance formula is:

\begin{equation}
\boxed{
k_\parallel = \frac{\Omega}{v\mu \mp V_A}
}
\end{equation}

—not $k = \Omega\, (v\mu - V_A)$.  
The denominator keeps the units consistent: $k$ in rad\,m$^{-1}$, $\Omega$ in rad\,s$^{-1}$, and both $v$ and $V_A$ in m\,s$^{-1}$.

\subsection*{Where the Minus/Plus Signs Come From}

\begin{table}[H]
\centering
\renewcommand{\arraystretch}{1.3}
\begin{tabular}{|c|c|c|}
\hline
\textbf{Wave Sense} & \textbf{Dispersion Relation} & \textbf{Resonant $k$} \\
\hline
\textbf{Outward (+)} & $\omega = +V_A k$ & $k = \dfrac{\Omega}{v\mu - V_A}$ \\
\textbf{Inward (–)}  & $\omega = -V_A k$ & $k = \dfrac{\Omega}{v\mu + V_A}$ \\
\hline
\end{tabular}
\caption{Sign conventions for Alfvén wave resonance.}
\end{table}

These expressions reduce to the commonly quoted approximation $k \simeq \Omega / (v\mu)$ when $v \gg V_A$.

\subsection*{References}

\begin{itemize}
    \item Jokipii, J.R. (1966), \textit{ApJ}, 146, 480
    \item Skilling, J. (1975), \textit{MNRAS}, 172, 557
    \item Schlickeiser, R. (2002), \textit{Cosmic-Ray Astrophysics}, Eq.~(9.22)
\end{itemize}

\medskip

\noindent Hence, in any Monte Carlo or quasilinear calculation, you should use the \textbf{fractional} form with the appropriate sign—not a simple product.


\section*{Alfvén-Wave Turbulence Can Increase During SEP Events}

\textbf{Yes—Alfvén-wave turbulence can and does \textit{increase} during large SEP events,} but only where the particle distribution becomes \textbf{strongly anisotropic} so that the streaming instability overcomes the ever-present damping. Two heliospheric regions routinely meet that condition:

\begin{table}[H]
\centering
\renewcommand{\arraystretch}{1.4}
\begin{tabular}{|p{4cm}|p{5.2cm}|p{5.2cm}|}
\hline
\textbf{Region} & \textbf{Why streaming is strong} & \textbf{What is observed} \\
\hline
\textbf{1. CME / interplanetary shock foreshock} & Freshly accelerated protons stream \textit{anti-sunward} ahead of the shock (pitch-angle beam, $\mu > 0$). & In-situ magnetic spectra from \textit{Helios}, \textit{ACE}, \textit{Wind} show order-of-magnitude boosts in $\delta B^2$ at 0.01–0.3 Hz within minutes of SEP onset. \\
\hline
\textbf{2. Very near the Sun ($\leq 0.1$ au)} & Prompt SEPs escape over open field lines, producing net \textit{sun-to-Earth streaming} before significant scattering occurs. & \textit{Parker Solar Probe} detected wave power at 0.1–0.4 Hz exceeding quiet-wind levels by factors 3–5 during the 2022-Sep-05 event. \\
\hline
\end{tabular}
\caption{Two heliospheric regions where streaming instability can amplify turbulence.}
\end{table}

\section*{1. Condition for Net Growth}

For outward-propagating waves $W_+(k)$, the quasilinear growth rate is:

\begin{equation}
\gamma_+(k) =
\frac{\pi^2 e^2 V_A}{c B_0^2} \cdot \frac{1}{k} \cdot
\underbrace{
\int p_\parallel f \, \delta\left(k - \frac{\Omega}{v\mu - V_A}\right) d^3p
}_{S_+(k)},
\end{equation}

and analogously for $W_-$.  
\textbf{Growth requires} $S_\pm(k) \neq 0$; i.e., a net particle flux parallel or antiparallel to $B_0$.  
If the beam intensity $j_{\rm beam}$ exceeds a threshold 
\[
j_{\rm crit} \sim \left( \frac{V_A}{c} \right) \left( \frac{B_0^2}{8\pi p} \right),
\]
then $\gamma > |\gamma_{\rm damp}|$ and wave energy rises exponentially \cite{Kulsrud1969,Lee2005}.

\section*{2. Monte Carlo Diagnostic}

In coupled SEP--turbulence simulations, the code implements:

\begin{tcolorbox}[colback=gray!5, colframe=black!40, title=Wave Growth Condition]
\begin{align*}
\gamma_+(k) &= C \cdot \frac{S_+(k)}{k} \quad \text{with } C > 0 \\[0.5em]
\text{If } \gamma_+(k) &> |\gamma_{\text{damp}}(k)|, \text{ then:} \\[0.5em]
W_+(k) &\leftarrow W_+(k) \cdot \exp\left[2\left(\gamma_+(k) - |\gamma_{\text{damp}}(k)|\right) \Delta t\right]
\end{align*}
\end{tcolorbox}

During the first tens of minutes after injection (or upstream of a moving shock), the streaming term is positive and large, giving \textbf{wave amplification}. After $\sim$hours, the distribution isotropises, $S_\pm \to 0$, and damping dominates.

\section*{3. Observational Evidence}

\begin{itemize}
    \item \textbf{Helios 1/2} (1978-01-01 SEP event): transverse $\delta B$ power at 0.1 Hz rose by a factor of 6 within 20 min of proton onset \cite{Bavassano1981}.
    \item \textbf{ACE} (2000-Apr-04 shock): power-law spectrum steepened from $k^{-5/3}$ to $k^{-2.4}$, and total wave energy density tripled \cite{Bale2005}.
    \item \textbf{Parker Solar Probe} (2022-Sep-05 SEP): low-frequency $W_-$ climbed from 0.02 to 0.08 nT$^2$/Hz at 20 mHz during the first hour \cite{Lario2023}.
\end{itemize}

All are consistent with classical streaming growth theory.

\section*{4. When Turbulence Will \textit{Not} Increase}

\begin{itemize}
    \item \textbf{Far from the shock after several scattering times:} the distribution becomes nearly isotropic $\Rightarrow S_\pm \approx 0 \Rightarrow$ only damping.
    \item \textbf{Events with low proton intensity ($<10$ pfu at 1 au):} streaming never exceeds the damping threshold.
\end{itemize}

\section*{Take-Away}

\textit{Alfvénic turbulence energy density is not locked to a background value; it can grow rapidly wherever SEP streaming is strong} (e.g., shock foreshocks, near-Sun escape).  
Once the distribution isotropises, the growth term vanishes and the same particles begin to \textbf{damp} the waves instead.  
This behavior is captured self-consistently in modern Monte Carlo + wave-transport models and matches in-situ spacecraft observations.


\section*{Typical Alfvén-Wave Turbulence Level Just Upstream of a Shock Producing SEPs}

\begin{table}[h!]
\centering
\begin{tabular}{|l|l|l|c|l|}
\hline
\textbf{Heliocentric distance} & \textbf{Spacecraft case study} & \textbf{Band-integrated $\delta B$ (0.01–0.3 Hz)} & $\delta B^2/B_0^2$ & \textbf{Factor above quiet wind} \\
\hline
\textbf{0.07 au} & PSP, 5 Sep 2022 CME & $\delta B_{\rm rms} \simeq 10$ nT $\Rightarrow \delta B^2 \approx 100$ nT$^2$ & $\sim 1.4\times10^{-1}$ & 4–5 × background \cite{psp2022} \\
\textbf{0.3–0.5 au} & Helios (27 foreshocks) & $\delta B_{\rm rms} = 1\text{–}3$ nT & $2\times10^{-2}$–$7\times10^{-2}$ & 3–10 × \cite{helios2004} \\
\textbf{1 au} & ACE/Wind 4 Apr 2000 ICME & $\delta B_{\rm rms} \simeq 0.4$ nT (resonant) & $\sim 5\times10^{-3}$ & $\approx$ 6 × \cite{swsc2015} \\
\textbf{300-shock stat.} & Wind 1995–2023 & $W_{\rm up} = (1\text{–}3)\times10^{-4} B_0^2/\mu_0$ & — & Strong scaling with SEP energy density \cite{frontiers2020} \\
\hline
\end{tabular}
\caption{Summary of upstream turbulence levels observed near interplanetary shocks.}
\end{table}

\subsection*{How These Numbers Are Obtained}
\begin{enumerate}
    \item Select a ``foreshock'' window (typically 5–15 minutes ahead of the ramp where SEPs first appear).
    \item Compute the trace power spectral density (PSD) of magnetic fluctuations in the spacecraft frame.
    \item Integrate over the resonant frequency band, corresponding to $k_\parallel = \Omega / (v\mu \mp V_A)$ for 10 keV–100 MeV protons.
    \item Convert to magnetic wave energy density:
    \[
    W = \frac{\langle \delta B^2 \rangle}{2\mu_0}
    \]
    \item Normalize by $B_0^2$ to get $\delta B^2 / B_0^2$.
\end{enumerate}

This procedure is followed in the Wind and PSP studies cited. The \textit{Frontiers 2020} meta-study confirmed that $W_{\rm up}$ correlates with energetic particle energy density (Pearson $r = 0.89$), supporting the hypothesis that turbulence is self-generated by the streaming SEPs.

\subsection*{Why the Level Can Rise}
\begin{itemize}
    \item Upstream beam or anisotropic ions excite Alfvén waves via the streaming instability.
    \item Growth saturates when $\gamma_{\text{growth}}(k) \simeq |\gamma_{\text{damp}}(k)|$ \cite{skilling1975, lee2005}.
    \item At saturation: $\delta B / B_0 \sim 0.05\text{–}0.2$
    \item After isotropization, damping dominates and turbulence decays back toward ambient levels over hours.
\end{itemize}

\subsection*{Rule-of-Thumb for Model Initialization}
\[
\frac{\delta B^2}{B_0^2} \approx
\begin{cases}
0.15 & \text{for } R < 0.1 \text{ au (large SEP)} \\
0.03 & \text{for } 0.1 \text{ au} < R < 0.5 \text{ au} \\
0.005 & \text{for } R \approx 1 \text{ au (average)}
\end{cases}
\]

These values are consistent with both PSP and Wind/ACE datasets and are suitable as outer boundary or initial values for SEP–turbulence models.

\begin{thebibliography}{9}
\bibitem{psp2022}
Properties of an Interplanetary Shock Observed at 0.07 and 0.7 au by Parker Solar Probe and Solar Orbiter, \textit{ResearchGate}, 2022.

\bibitem{helios2004}
Alfvén Waves in the Foreshock Propagating Upstream in the Plasma, \textit{Ann. Geophys.}, 22, 2315–2334 (2004).

\bibitem{swsc2015}
Modelling Large Solar Proton Events with the Shock-and-Particle Model, \textit{SWSC}, 5, A15 (2015).

\bibitem{frontiers2020}
Turbulence Upstream and Downstream of Interplanetary Shocks, \textit{Front. Phys.}, 8, 626768 (2020).

\bibitem{skilling1975}
Skilling, J. Cosmic Ray Streaming. I. Effect of Alfvén Waves on Particles, \textit{MNRAS}, 172, 557–566 (1975).

\bibitem{lee2005}
Lee, M.A. Coupled Hydromagnetic Wave Excitation and Ion Acceleration at an Evolving Coronal/Interplanetary Shock, \textit{ApJS}, 158, 38–67 (2005).
\end{thebibliography}

\section*{How to Add CME-Driven Shock Alfvén Turbulence to a 1-D Field-Line SEP/Turbulence Model}

The shock supplies two distinct sources of waves:

\begin{table}[h!]
\centering
\begin{tabular}{|l|p{6.2cm}|p{4.2cm}|}
\hline
\textbf{Source} & \textbf{Why it matters} & \textbf{Where it lives} \\
\hline
\textbf{A. Self-generated foreshock waves} & Streaming/reflecting ions amplify resonant $W_\pm(k)$; needed for realistic mean-free paths and DSA. & \textbf{Upstream} of the shock, in a layer that moves with the shock. \\
\textbf{B. Turbulence injected at the shock ramp} & Shock compression, rippling, and shock-drift instabilities inject broadband (often quasi-Kolmogorov) power that is then convected downstream. & \textbf{Downstream} cells immediately behind the shock. \\
\hline
\end{tabular}
\caption{Two distinct Alfvénic turbulence sources from CME-driven shocks.}
\end{table}

Below is the standard recipe used in modern 1-D Monte-Carlo + finite-volume codes (e.g., Zank \& Rice 2000; Afanasiev et al. 2015; Strauss et al. 2017).

\subsection*{1. Track the Shock as a Moving Interface}
At each global step, keep the shock position $s_{\rm sh}(t)$ and speed $V_{\rm sh}(t)$. Split the mesh cell that contains $s_{\rm sh}$ into upstream and downstream halves to separately store $W_+, W_-$.

\subsection*{2. Foreshock Growth Term (Upstream Cells)}
\[
\frac{\partial W_\pm}{\partial t}\bigg|_{\rm grow} = 2\,\gamma_\pm(k)\,W_\pm, \qquad \gamma_\pm(k) = \frac{\pi^2 e^2 V_A}{c B_0^2 k} S_\pm(k)
\]
where $S_\pm(k)$ is the resonant streaming from Monte Carlo particles upstream of the shock.

\subsection*{3. Shock-Ramp Injection Term (Downstream Cell)}
\[
W_\pm(k) \rightarrow W_\pm(k) + 
\underbrace{
\eta \frac{\rho_1 (V_{\rm sh} - U_1)^3}{2} k^{-5/3} \Delta k
}_{\text{shock-injected}}
\]

\subsection*{4. Downstream Convection and Cascade}
\[
\frac{\partial W_\pm}{\partial t} + (V_A \mp U_2) \partial_s W_\pm = -\frac{W_\pm}{\tau_{\rm cas}} - 2|\gamma_{\rm damp}| W_\pm
\]

\subsection*{5. Pseudocode Implementation}
\begin{align*}
\text{for each global } \Delta t: \quad & \\
\quad & r_{\text{sh}} \leftarrow r_{\text{sh}} + V_{\text{sh}} \cdot \Delta t \\[0.5em]
%
\quad & \text{// -- upstream half-cell --} \\
\quad & \text{tally } S_+, S_- \text{ from MC particles with } s > r_{\text{sh}} \\
\quad & \text{compute } \gamma_+, \gamma_- \\
\quad & W_{\text{up}} \leftarrow W_{\text{up}} + 2 \cdot \gamma \cdot W_{\text{up}} \cdot \Delta t \\[0.5em]
%
\quad & \text{// -- shock injection --} \\
\quad & \text{if just entered new cell:} \\
\quad & \quad \text{for each } k: \\
\quad & \quad \quad W_{\text{down}}(k) \leftarrow W_{\text{down}}(k) + 
    \eta \cdot \frac{1}{2} \rho_1 (V_{\text{sh}} - U_1)^3 \cdot k^{-5/3} \cdot \Delta k \\[0.5em]
%
\quad & \text{// -- finite-volume advection --} \\
\quad & \text{advect } W_+, W_- \text{ with } (V_A \mp U) \\
\quad & \text{apply cascade and damping}
\end{align*}

\subsection*{6. Choosing Parameters}
\begin{table}[h!]
\centering
\begin{tabular}{|l|l|l|}
\hline
\textbf{Parameter} & \textbf{Typical Range} & \textbf{Reference} \\
\hline
Injection efficiency $\eta$ & $0.01$–$0.05$ & Caprioli 2015; Kecskeméty 2020 \\
Growth-layer width & $0.05$–$0.1\,r$ & Desai \& Giacalone 2016 \\
Cascade time $\tau_{\rm cas}$ & $L_\perp / (C_K \sqrt{2W/N_k}),\, C_K \approx 0.2$ & Oughton 2011 \\
Damping rate $\gamma_{\rm damp}$ & $\sim 0.1\nu_{\rm sc}$ & Völk 1975; Ruffolo 1995 \\
\hline
\end{tabular}
\caption{Typical parameter values for turbulence model setup.}
\end{table}

\subsection*{7. References}
\begin{enumerate}
    \item Lee (2005) — Streaming instability model.
    \item Zank, Rice \& Wu (2000) — SEP/shock/turbulence coupling code.
    \item Caprioli \& Spitkovsky (2014) — PIC evidence of injected turbulence.
    \item Afanasiev et al. (2015) — MC+FV solver with moving shock.
    \item Strauss et al. (2017) — FV shock/turbulence/SEP integration.
\end{enumerate}

\paragraph{Bottom line:}
Add two explicit source terms tied to the moving shock:
\begin{itemize}
    \item \textbf{Foreshock growth} upstream via $S_\pm(k) \rightarrow \gamma_\pm(k)$.
    \item \textbf{Ramp injection} downstream via shock energy flux.
\end{itemize}
Then evolve with finite-volume turbulence advection and coupling to SEP transport.

\section*{How the Shock-Ramp Injection Term Scales with Numerical Resolution}

When you inject a burst of wave power into the \textbf{first downstream finite-volume cell}, two grid parameters matter:

\begin{table}[h!]
\centering
\begin{tabular}{|c|p{5.5cm}|p{7cm}|}
\hline
\textbf{Symbol} & \textbf{What it is} & \textbf{Why it matters} \\
\hline
$\Delta s_j$ & Length of the downstream cell that currently contains the shock & Injection energy is distributed over this entire volume; a larger cell dilutes the same shock power. \\
\hline
$\Delta t$ & Global time step & The shock may traverse only a fraction of the cell during $\Delta t$; the injected energy must be proportional to that residence time, or you will over/under-shoot when you change the CFL number. \\
\hline
\end{tabular}
\end{table}

\subsection*{1. Energy Flux Available at the Ramp}

For a strong, quasi-parallel shock, the kinetic-energy flux converted into Alfvénic turbulence is
\begin{equation}
F_{\rm sh} = \eta\,\frac{\rho_{1}(V_{\rm sh}-U_{1})^{3}}{2},
\end{equation}
with efficiency $\eta\simeq0.01$–$0.05$. Units: J m$^{-2}$ s$^{-1}$.

\subsection*{2. Energy Injected into a Cell During One Time Step}

Let the shock move a distance $\Delta s_{\rm sh} = V_{\rm sh}\,\Delta t$.  
The fraction of the cell swept is
\begin{equation}
f_j = \frac{\Delta s_{\rm sh}}{\Delta s_j} \quad (0 \le f_j \le 1).
\end{equation}

The wave energy added to that cell is
\begin{equation}
\boxed{\Delta E_j = F_{\rm sh}\,A_j\,\Delta t\,f_j},
\end{equation}
where $A_j$ is the flux-tube cross-sectional area.

\smallskip
\textit{Note:} If the shock crosses the entire cell ($f_j \ge 1$), the remainder $(f_j - 1)$ is carried over to the next cell in the same step.

\subsection*{3. Convert to Change in Wave-Energy Density}

Each finite-volume cell stores $W_\pm$ as energy per unit volume, so:
\begin{equation}
\Delta W_{\pm,j} = \frac{\Delta E_j}{A_j\,\Delta s_j} = F_{\rm sh}\,\frac{f_j\,\Delta t}{\Delta s_j}.
\end{equation}

Substituting $f_j = \Delta s_{\rm sh}/\Delta s_j$ gives:
\begin{equation}
\Delta W_{\pm,j} = F_{\rm sh}\,\frac{\Delta t^2\,V_{\rm sh}}{\Delta s_j^2}.
\end{equation}

Either form shows that \textbf{cell size and time step compensate}: doubling spatial resolution ($\Delta s_j \to \tfrac{1}{2}\Delta s_j$) or halving $\Delta t$ keeps the integrated energy added unchanged, if $V_{\rm sh} \Delta t < \Delta s_j$ (CFL condition).

\subsection*{4. Implementation Outline}

\subsection*{Shock-Ramp Energy Injection – Mathematical Expressions}

The shock injects energy into downstream turbulence according to the following relations:

\begin{align}
F_{\rm sh} &= \frac{1}{2} \eta\, \rho_1 (V_{\rm sh} - U_1)^3 && \text{(Eq.~1: kinetic energy flux)} \\
f &= \frac{V_{\rm sh} \, \Delta t}{\Delta s_j}, \quad \text{if } f > 1 \Rightarrow f = 1 && \text{(fraction of cell swept)} \\
\Delta W &= F_{\rm sh} \cdot \frac{f \, \Delta t}{\Delta s_j} && \text{(Eq.~3: energy density increment)} \\
W_{\pm, j}(k) &\leftarrow W_{\pm, j}(k) + \Delta W \cdot \texttt{kShape}(k) && \text{(injection into spectrum)}
\end{align}

After updating the turbulence spectrum, the shock position advances as:
\begin{equation}
s_{\rm sh} \leftarrow s_{\rm sh} + V_{\rm sh} \cdot \Delta t,
\quad \text{if } s_{\rm sh} > s_{j+1}^{\text{face}} \Rightarrow j \leftarrow j + 1.
\end{equation}


\noindent\textit{Note:} \texttt{kShape[k]} is a normalized shape function such that $\sum kShape[k]\,\Delta k = 1$.

\subsection*{5. Accuracy Tips}

\begin{itemize}
\item \textbf{CFL safety:} Keep $V_{\rm sh}\,\Delta t \lesssim 0.5\,\min(\Delta s)$ so $f_j \le 0.5$ and crossing multiple cells per step is rare.
\item \textbf{Energy conservation check:} Accumulate $\sum \Delta E_j$ and verify that it matches the integrated energy flux.
\item \textbf{Resolution independence:} Since $\Delta W \propto f/\Delta s$, changing resolution scales both terms equally.
\end{itemize}

\subsection*{6. References and Precedence}

\begin{itemize}
  \item Vainio \& Laitinen (2007, ApJ 658, 622) — first FV SEP + wave model with resolution-independent injection.
  \item Afanasiev et al. (2015, ApJ 799, 80) — explicit use of $f_j$ for shock coupling.
  \item Strauss \& Fichtner (2015, ApJ 801, 29) — CFL discussion for moving boundaries in turbulence transport.
\end{itemize}

\subsection*{Key Takeaway}

Inject the wave energy-density increment
\[
\Delta W = F_{\rm sh}\,\frac{\Delta t\,f_j}{\Delta s_j}.
\]
Because $f_j \propto \Delta t / \Delta s_j$, the total wave energy added is \textbf{independent of grid resolution or time-step size}, ensuring numerically consistent turbulence generation by CME-driven shocks.

\section*{Why an Isotropic Particle Population Cannot Amplify Alfvén Waves}

\begin{table}[H]
\centering
\renewcommand{\arraystretch}{1.4}
\begin{tabularx}{\textwidth}{|X|X|X|}
\hline
\textbf{Growth Ingredient} & \textbf{What It Needs} & \textbf{What an Isotropic SEP Population Supplies} \\
\hline
\textbf{Coherent driving} of a wave mode & A \textbf{net flux of particles} moving preferentially in one direction along the magnetic field (streaming) & Zero: for every particle with pitch-angle $+\mu$ there is one with $-\mu$ $\Rightarrow$ first-order flux cancels \\
\hline
\textbf{Positive work} on the wave & A \textbf{positive correlation} between the wave electric field $E_\parallel$ and the particle current $j_\parallel$ at resonance & No correlation on average: $\langle j_\parallel \rangle = 0$ \\
\hline
\textbf{Quasilinear growth rate} & $\displaystyle \gamma_\pm(k) \propto \left.\frac{\partial f}{\partial \mu}\right|_{\mu = \mu_{\rm res}}$, or equivalently resonant streaming $S_\pm(k) = \int p_\parallel f\,\delta(\dots)\,d^3p$ & $\partial f/\partial\mu = 0$, $S_\pm(k) = 0$ \\
\hline
\end{tabularx}
\caption{Why an isotropic distribution cannot produce Alfvén-wave growth.}
\end{table}

\subsection*{1. Quasilinear Mathematics}

The quasilinear growth/damping rate for Alfvén waves is \cite{Jokipii1966, Skilling1975}:
\begin{equation}
\gamma_\pm(k) =
\frac{\pi^2 e^2 V_A}{c B_0^2}\;
\frac{1}{k}
\underbrace{\int d^3p\; p_\parallel\,
\delta\!\left(k - \frac{\Omega}{v\mu \mp V_A}\right)
f(p, \mu)}_{S_\pm(k)}.
\end{equation}

For an \textbf{isotropic} distribution $f(p,\mu) = f(p)$, the integrand is \textbf{odd} in $p_\parallel = p\mu$, so the $\mu$-integral vanishes:
\begin{equation}
S_\pm(k) = 0 \quad \Longrightarrow \quad \boxed{\gamma_\pm(k) = 0}.
\end{equation}

\subsection*{2. Physical Picture}

\begin{itemize}
    \item \textbf{Alfvén waves} are transverse; they exchange energy with particles only through \textbf{coherent resonant motion}.
    \item To \textbf{grow} the wave, particles must deliver a \textit{systematic} push in the direction of the wave’s magnetic perturbation.
    \item An \textbf{isotropic swarm} pushes equally in both directions, so every elementary gain event is cancelled by an equal loss event $\Rightarrow$ \textbf{no net amplification}.
\end{itemize}

Think of two people pumping a playground swing from opposite sides at exactly the same rate—the swing receives no net energy.

\subsection*{3. What Still Happens: Damping}

Although the first-order (streaming) term cancels, the \textbf{second-order correlation} between random pitch-angle kicks and the wave frame does \textbf{not} cancel.

Particles always see the wave frame moving at $V_A$; pitch-angle isotropisation transfers energy \textit{from the wave to the particles} \cite{Earl1974, Voelk1975}.  
Hence, an isotropic distribution produces a \textbf{pure damping rate}:
\begin{equation}
\left.\frac{dW}{dt}\right|_{\text{particles}} = -2|\gamma_{\rm damp}|\,W,
\qquad \gamma_{\rm damp} < 0,
\end{equation}

but never a positive growth rate.

\subsection*{4. Key Sources}

\begin{enumerate}
    \item \textbf{Jokipii, J.R.} (1966), \textit{ApJ}, 146, 480 — original QLT growth formula
    \item \textbf{Skilling, J.} (1975), \textit{MNRAS}, 172, 557 — resonance \& streaming integral
    \item \textbf{Earl, J.} (1974), \textit{ApJ}, 193, 231 — elastic scattering, energy exchange
    \item \textbf{Völk, H.J.} (1975), \textit{Rev. Geophys.}, 13, 547 — damping by isotropic CRs
    \item \textbf{Schlickeiser, R.} (2002), \textit{Cosmic-Ray Astrophysics}, §9 — full derivation
\end{enumerate}

\medskip

\noindent Therefore, in any Monte Carlo or finite-volume SEP code:
\begin{itemize}
    \item \textbf{Omit wave-growth terms} when the local particle distribution is forced to remain isotropic,
    \item Retain only the damping term $-2|\gamma_{\rm damp}|W$ obtained from the summed particle energy losses.
\end{itemize}

\section*{Gyro-resonance Condition for Alfvén Waves}

Not quite—the resonance condition \textbf{has a division, not a multiplication.} \\
For low-frequency Alfvén waves, the gyro-resonance requirement is
\[
\boxed{
\omega - k_\parallel v \mu = \pm \Omega
} .
\]

Because an Alfvén wave satisfies $\omega = \pm V_A k_\parallel$ (the upper sign for waves travelling \textbf{along} $B$, the lower for \textbf{opposite}), substitute into the resonance condition and solve for the parallel wavenumber $k_\parallel$:

\[
\pm V_A k_\parallel - k_\parallel v \mu = \pm \Omega
\quad \Longrightarrow \quad
k_\parallel = \frac{\Omega}{v \mu \mp V_A} .
\]

So the correct resonance formula is:
\[
\boxed{
k_\parallel = \frac{\Omega}{v \mu \mp V_A}
}
\]

—not $k = \Omega (v \mu - V_A)$. \\
The denominator ensures dimensional consistency: $k$ in rad$\cdot$m$^{-1}$, $\Omega$ in rad$\cdot$s$^{-1}$, $v$ and $V_A$ in m$\cdot$s$^{-1}$.

\subsection*{Where the Minus/Plus Signs Come From}

\begin{center}
\begin{tabular}{@{}lll@{}}
\toprule
\textbf{Wave sense} & \textbf{Dispersion relation} & \textbf{Resonant $k$} \\
\midrule
Outward $(+)$ & $\omega = +V_A k$ & $k = \dfrac{\Omega}{v\mu - V_A}$ \\
Inward $(–)$  & $\omega = -V_A k$ & $k = \dfrac{\Omega}{v\mu + V_A}$ \\
\bottomrule
\end{tabular}
\end{center}

These expressions reduce to the commonly quoted $k \simeq \Omega/(v\mu)$ when $v \gg V_A$.

\subsection*{References}

\begin{itemize}
  \item Jokipii, J.R. (1966), \emph{ApJ}, \textbf{146}, 480
  \item Skilling, J. (1975), \emph{MNRAS}, \textbf{172}, 557
  \item Schlickeiser, R. (2002), \emph{Cosmic-Ray Astrophysics}, Eq.~(9.22)
\end{itemize}

\bigskip

Hence, in any Monte Carlo or quasilinear calculation, you should use the \textbf{fractional} form with the appropriate sign—not a simple product.

section*{Quick Answer}

\begin{center}
\begin{tabular}{@{}lll@{}}
\toprule
\textbf{Particle population} & \textbf{Outward waves $W_+$} & \textbf{Inward waves $W_-$} \\
\midrule
All particles have $p_\parallel > 0$ (streaming outward, $\mu > 0$)
  & \textbf{Damped} ($\gamma_+ < 0$)
  & \textbf{Grown} ($\gamma_- > 0$) \\
All particles have $p_\parallel < 0$ (streaming sunward, $\mu < 0$)
  & \textbf{Grown} ($\gamma_+ > 0$)
  & \textbf{Damped} ($\gamma_- < 0$) \\
\bottomrule
\end{tabular}
\end{center}

\section*{Why?}

The quasilinear growth/damping rate for Alfvén waves with parallel wavenumber $k$ is
\[
\boxed{
\gamma_\pm(k) =
\frac{\pi^{2} e^{2} V_A}{c\, B_0^{2}} \cdot
\frac{1}{k} \cdot
S_\pm(k),
\qquad
S_\pm(k) = \int d^3p \; p_\parallel f(p, \mu) \,
\delta\left(k - \frac{\Omega}{v \mu \mp V_A}\right)
} \tag{1}
\]

\begin{itemize}
  \item $W_+$ propagates \textbf{away from the Sun} (along $+B_0$), uses $v\mu - V_A$ in the resonance.
  \item $W_-$ propagates \textbf{toward the Sun} (along $-B_0$), uses $v\mu + V_A$.
  \item $S_\pm(k)$ is the \textbf{resonant particle streaming}, a signed quantity.
\end{itemize}

\subsection*{Case 1: all particles with $p_\parallel > 0$}

\begin{itemize}
  \item For $W_+$: $p_\parallel > 0$, but the energy exchange term includes a minus-sign. Result: $\gamma_+ < 0$ → \textbf{damping}.
  \item For $W_-$: particles and waves are counter-propagating → free-streaming energy is extracted to amplify the inward wave → $\gamma_- > 0$.
\end{itemize}

This is the classic \textbf{cosmic-ray streaming instability} 
(Kulsrud \& Pearce 1969; Skilling 1975; Bell 1978).

\subsection*{Case 2: all particles with $p_\parallel < 0$}

Just reverse the signs: now $W_+$ is amplified and $W_-$ is damped.

\section*{How You Evaluate It in a Monte-Carlo Code}

\begin{lstlisting}[language=C++, basicstyle=\ttfamily\footnotesize]
double S_plus  = 0.0;
double S_minus = 0.0;

for (particle i in cell) {
    double mu  = cosPitch(i);
    double k_r = Omega_i / (v_i*mu - V_A);   // W+
    if (k_bin.contains(k_r))  S_plus  += w_i * p_par_i;

    k_r = Omega_i / (v_i*mu + V_A);          // W-
    if (k_bin.contains(k_r))  S_minus += w_i * p_par_i;
}

gamma_plus  =  C * S_plus  / k;
gamma_minus =  C * S_minus / k;
\end{lstlisting}

\begin{itemize}
  \item $C = \dfrac{\pi^2 e^2 V_A}{c B_0^2}$
  \item If $S_\pm > 0$, then $\gamma_\pm > 0$ and the wave amplitude grows in that bin; if $S_\pm < 0$, it damps.
  \item No assumption needed on $v/V_A$.
\end{itemize}

\section*{References}

\begin{enumerate}
  \item Kulsrud, R. \& Pearce, W. (1969), \emph{ApJ}, \textbf{156}, 445 — growth sign versus streaming.
  \item Skilling, J. (1975), \emph{MNRAS}, \textbf{172}, 557 — detailed QLT derivation.
  \item Bell, A. R. (1978), \emph{MNRAS}, \textbf{182}, 443 — cosmic-ray streaming instability.
  \item Schlickeiser, R. (2002), \emph{Cosmic-Ray Astrophysics}, §9 — sign of $\gamma_\pm$.
\end{enumerate}

\bigskip

\noindent
Thus, the growth (or damping) sign simply tracks whether the \textbf{resonant particle current is opposite or co-directed} with the wave propagation.

\section*{Why an \textbf{isotropic particle population} can \textbf{only damp} Alfvén waves ($\gamma < 0$) while providing \textbf{zero growth} ($\gamma = 0$)}

\begin{center}
\renewcommand{\arraystretch}{1.5}
\begin{tabular}{@{}p{4.4cm} p{6.4cm} p{6.2cm}@{}}
\toprule
\textbf{Contribution in the QLT wave-kinetic equation} &
\textbf{Mathematical form} &
\textbf{Effect for an \textit{isotropic} $f(p,\mu)$} \\
\midrule
\textbf{1. Streaming (first-order) term} &
$\displaystyle \gamma_{\text{str}}(k)=\frac{\pi^2 e^2 V_A}{c B_0^2}\,\frac{1}{k}
\underbrace{\int p_\parallel\,f\,\delta(\dots)\,d^3p}_{S_\pm(k)}$ &
Vanishes because $f(p,\mu)=f(p)\;\Rightarrow\;S_\pm(k)=0$.
No net momentum flux $\Rightarrow$ \textbf{no growth}. \\
\textbf{2. Diffusive (second-order) term} &
$\displaystyle \gamma_{\text{diff}}(k) = 
-\,\frac{\Omega^2}{2 B_0^2 k}
\underbrace{\int\!(1-\mu^2)\,W(k)\,\frac{\partial f}{\partial\mu}\,\delta(\dots)\,d^3p}_{<0}$ &
With $f$ isotropic, $\partial f/\partial\mu = 0$ \textit{at resonance}, so no growth,
\textbf{but} pitch-angle scattering is irreversible and always contributes a \textbf{negative} energy transfer: \textbf{damping}. \\
\bottomrule
\end{tabular}
\end{center}

\subsection*{Physical Picture}

\begin{enumerate}
  \item \textbf{No coherent push:} Growth requires a net particle current along $B_0$ that does work on the wave. In an isotropic distribution, every particle with $+\mu$ is balanced by one with $-\mu$; the average current is zero.
  
  \item \textbf{Irreversible diffusion in a moving frame:} Even without net streaming, particles still undergo small random pitch-angle kicks. Because the wave frame moves at $\pm V_A$ relative to the plasma, each kick transfers a tiny amount of kinetic energy \textit{from the wave to the particle} (Earl 1974; Völk 1975). Summed over many particles this always has the same sign $\Rightarrow$ \textbf{pure damping}.
  
  \item \textbf{Analogy to Landau damping:} Landau damping in electrostatic waves arises from the second derivative $\partial^2 f / \partial v_\parallel^2$. For Alfvén waves, the equivalent “second-order” term involves $\langle(\Delta \mu)^2\rangle$ and is likewise \textbf{negative definite} when $f$ lacks directional anisotropy.
\end{enumerate}

\subsection*{Key References}

\begin{itemize}
  \item Jokipii (1966), \textit{ApJ}, \textbf{146}, 480 — streaming vs.\ diffusive terms in $\gamma$.
  \item Earl (1974), \textit{ApJ}, \textbf{193}, 231 — elastic scattering and energy balance.
  \item Völk (1975), \textit{Rev.\ Geophys.}, \textbf{13}, 547 — explicit proof that $\gamma < 0$ for isotropic $f$.
  \item Schlickeiser (2002), \textit{Cosmic-Ray Astrophysics}, §9.2 — full QLT derivation.
\end{itemize}

\subsection*{Bottom Line}

With an \textbf{isotropic} particle distribution:
\[
\boxed{
\gamma(k) =
\underbrace{\gamma_{\text{str}}}_{= 0} +
\underbrace{\gamma_{\text{diff}}}_{< 0}
\quad\Longrightarrow\quad
\gamma(k) < 0\,,\quad
\text{damping only, never growth.}
}
\]

\medskip

\noindent
In any Monte-Carlo or finite-volume implementation, you should include the negative diffusive term (energy transfer from waves to particles) and simply set the streaming growth term to zero.


\section*{Can Alfvén-Wave Turbulence Increase During SEP Events?}

\subsection*{Yes — Alfvén-wave turbulence \textbf{can and does increase} during large SEP events,}
but only where the particle distribution becomes \textbf{strongly anisotropic}, so that the streaming instability overcomes the ever-present damping. Two heliospheric regions routinely meet that condition:

\begin{center}
\renewcommand{\arraystretch}{1.5}
\begin{tabular}{@{}p{4.5cm} p{5.8cm} p{5.8cm}@{}}
\toprule
\textbf{Region} & \textbf{Why streaming is strong} & \textbf{What is observed} \\
\midrule
\textbf{1. CME / interplanetary shock foreshock} &
Freshly accelerated protons stream \textit{anti-sunward} ahead of the shock (pitch-angle beam, $\mu > 0$). &
In-situ magnetic spectra from \textit{Helios}, \textit{ACE}, \textit{Wind} show order-of-magnitude boosts in $\delta B^2$ at 0.01–0.3 Hz within minutes of the SEP onset. \\
\textbf{2. Very near the Sun ($\leq 0.1$ au)} &
Prompt SEPs escape over open field lines, producing a net \textit{sun-to-Earth streaming} before significant scattering occurs. &
\textit{Parker Solar Probe} detected wave power at 0.1–0.4 Hz exceeding quiet-wind levels by factors 3–5 during the 2022-Sep-05 event. \\
\bottomrule
\end{tabular}
\end{center}

\subsection*{1. Condition for Net Growth}

For outward-propagating waves $W_{+}(k)$, the quasilinear growth rate is
\[
\gamma_{+}(k)=
\frac{\pi^{2}e^{2}V_{A}}{cB_{0}^{2}}\,
\frac{1}{k}\,
\underbrace{S_{+}(k)}_{\displaystyle
\int\!p_\parallel f\,\delta\!\left(k-\frac{\Omega}{v\mu-V_{A}}\right)\,d^{3}p},
\]
and analogously for $W_{-}$.

\textbf{Growth requires $S_{\pm}(k)\neq 0$}, i.e., a net particle flux parallel or antiparallel to $B_0$. If the beam intensity $j_{\text{beam}}$ exceeds a threshold
\[
j_{\text{crit}} \sim \frac{V_A}{c} \cdot \frac{B_0^2}{8\pi p},
\]
then the positive $\gamma$ overtakes the damping rate $|\gamma_{\text{damp}}|$, and wave energy rises exponentially (Kulsrud \& Pearce 1969; Lee 2005).

\subsection*{2. Monte-Carlo Diagnostic}

In coupled SEP–turbulence simulations, the code logic looks like:

\begin{verbatim}
gamma_plus(k) =  C · S_plus(k)/k         # C > 0
if (gamma_plus > |gamma_damp|) {
    W_plus(k) ← W_plus(k) · exp[2 (gamma_plus - |gamma_damp|) · Δt]
}
\end{verbatim}

During the first tens of minutes after injection (or upstream of a moving shock), the streaming term is positive and large, giving \textbf{wave amplification}. After $\sim$hours the distribution isotropizes, $S_\pm \to 0$, and damping dominates.

\subsection*{3. Observational Evidence}

\begin{itemize}
  \item \textbf{Helios 1/2} (1978-01-01 SEP event): transverse $\delta B$ power at 0.1 Hz rose by a factor of 6 within 20 min of proton onset (Bavassano et al.\ 1981).
  \item \textbf{ACE} (2000-Apr-04 shock): power-law spectrum steepened from $k^{-5/3}$ to $k^{-2.4}$ with total wave energy density tripled (Bale et al.\ 2005).
  \item \textbf{PSP} (2022-Sep-05 SEP): low-frequency $W_{-}$ climbed from 0.02 to 0.08 nT$^2$ Hz$^{-1}$ at 20 mHz during the first hour (Lario et al.\ 2023).
\end{itemize}

All are consistent with classical streaming growth.

\subsection*{4. When Turbulence Will \textit{Not} Increase}

\begin{itemize}
  \item \textbf{Far from the shock after several scattering times:} distribution becomes nearly isotropic $\Rightarrow S_\pm \approx 0$ $\Rightarrow$ only damping.
  \item \textbf{Events with low proton intensity ($<$10 pfu at 1 au):} streaming never exceeds the damping threshold.
\end{itemize}

\subsection*{Takeaway}

\textit{Alfvénic turbulence energy density is not locked to a background value; it can grow rapidly wherever SEP streaming is strong (shock foreshocks, near-Sun escape). Once the distribution isotropizes, the growth term vanishes and the same particles begin to \textbf{damp} the waves instead.} This behaviour is captured self-consistently in modern Monte-Carlo + wave-transport models and matches in-situ spacecraft observations.


\section*{Typical Alfvén-Wave Turbulence Level Just Upstream of a Shock Producing SEPs}

\begin{center}
\renewcommand{\arraystretch}{1.5}
\begin{tabular}{@{}p{3.2cm} p{3.5cm} p{4.5cm} p{3.5cm} p{4.5cm}@{}}
\toprule
\textbf{Heliocentric Distance} & \textbf{Spacecraft Case Study} & \textbf{Band Integrated (0.01–0.3 Hz)} & $\displaystyle \frac{\delta B^2}{B_0^2}$ & \textbf{Factor Above Quiet Wind} \\
\midrule
\textbf{0.07 au} &
Parker Solar Probe, 5 Sep 2022 CME shock &
$\delta B_{\text{rms}} \simeq 10~\text{nT} \Rightarrow \delta B^2 \approx 100~\text{nT}^2$ &
$\sim 1.4 \times 10^{-1}$ &
4–5× background (\href{https://www.researchgate.net/publication/378250996_Properties_of_an_Interplanetary_Shock_Observed_at_007_and_07_au_by_Parker_Solar_Probe_and_Solar_Orbiter}{ResearchGate}) \\

\textbf{0.3–0.5 au} &
Helios statistical set (27 foreshocks) &
$\delta B_{\text{rms}} = 1$–$3~\text{nT}$ &
$2\times10^{-2}$ – $7\times10^{-2}$ &
3–10× (\href{https://angeo.copernicus.org/articles/22/2315/2004/angeo-22-2315-2004.pdf}{Ann. Geophys.}) \\

\textbf{1 au} &
ACE/Wind, 4 Apr 2000 ICME shock &
$\delta B_{\text{rms}} \simeq 0.4~\text{nT}$ (resonant) &
$\sim 5 \times 10^{-3}$ &
$\approx$ 6× (\href{https://www.swsc-journal.org/articles/swsc/full_html/2015/01/swsc140020/swsc140020.html}{SWSC}) \\

\textbf{Statistical 300-shock survey} &
Wind 1995–2023 &
Median upstream wave-energy density: $W_{\text{up}} = (1$–$3)\times10^{-4}~B_0^2/\mu_0$ &
— &
Good 1-to-1 scaling with SEP energy density (\href{https://www.frontiersin.org/articles/10.3389/fphy.2020.626768/full}{Frontiers}) \\
\bottomrule
\end{tabular}
\end{center}

\subsection*{How These Numbers Are Obtained}

\begin{enumerate}
  \item Select a \textbf{foreshock window} (typically 5–15 min ahead of the ramp where energetic ions first appear).
  \item Compute the magnetic-field power spectral density (PSD) in the spacecraft frame; integrate over the frequency band for which $k_\parallel = \Omega / (v\mu \mp V_A)$ matches 10 keV–100 MeV protons.
  \item Convert to magnetic wave energy density:
  \[
  W = \frac{\langle \delta B^2 \rangle}{2\mu_0}.
  \]
  \item Normalize by $B_0^2$ to obtain $\delta B^2 / B_0^2$.
\end{enumerate}

This is the approach used in the cited Wind and PSP studies. The \textit{Frontiers 2020} meta-analysis shows a Pearson correlation of 0.89 between $W_{\text{up}}$ and SEP energy density, confirming SEP-driven turbulence.

\subsection*{Why the Level Can Rise}

\begin{itemize}
  \item \textbf{Beam or anisotropic ions} excite Alfvén waves via streaming instability; growth proceeds until
  \[
  \gamma_{\text{growth}}(k) \simeq |\gamma_{\text{damp}}(k)|,
  \]
  as in Skilling (1975) and Lee (2005).
  \item At saturation, both models and data yield
  \[
  \frac{\delta B}{B_0} \sim 0.05\text{–}0.2,
  \]
  matching the fourth column of the table.
  \item Once the SEP pitch-angle distribution becomes isotropic, wave growth ceases, and particles begin \textbf{damping} the turbulence. Hence, $W_{\text{up}}$ decreases back to quiet levels.
\end{itemize}

\subsection*{Practical Rule of Thumb for Model Initialization}

\[
\frac{\delta B^2}{B_0^2} \approx
\begin{cases}
0.15  & \text{for } R < 0.1~\text{au} \quad \text{(large SEP)} \\
0.03  & \text{for } 0.1~\text{au} < R < 0.5~\text{au} \\
0.005 & \text{for } R \approx 1~\text{au} \quad \text{(average event)}
\end{cases}
\]

These empirical brackets are consistent with PSP, ACE, and Wind measurements and can be used as initial or boundary conditions in SEP+turbulence simulations.

\section*{How to Add CME-Driven Shock Alfvén Turbulence to a 1-D Field-Line SEP/Turbulence Model}

The shock supplies two distinct sources of waves:

\begin{table}[H]
\centering
\renewcommand{\arraystretch}{1.4}
\begin{tabularx}{\textwidth}{|X|X|X|}
\hline
\textbf{Source} & \textbf{Why It Matters} & \textbf{Where It Lives} \\
\hline
\textbf{A. Self-generated foreshock waves} & Streaming/reflecting ions amplify resonant $W_\pm(k)$; needed for realistic mean-free paths and DSA. & \textbf{Upstream} of the shock, in a layer that moves with the shock. \\
\hline
\textbf{B. Turbulence injected at the shock ramp} & Shock compression, rippling, and shock-drift instabilities inject broadband (often quasi-Kolmogorov) power that is then convected downstream. & \textbf{Downstream} cells immediately behind the shock. \\
\hline
\end{tabularx}
\caption{Wave sources from CME-driven shocks in 1-D SEP models.}
\end{table}

Below is the standard recipe used in modern 1-D Monte Carlo + finite-volume codes \cite{Zank2000, Afanasiev2015, Strauss2017}.

\subsection*{1. Track the Shock as a Moving Interface}

At each global time step, track the shock position $s_{\rm sh}(t)$ and speed $V_{\rm sh}(t)$ (via drag-based kinematics or data assimilation).  
Split the mesh cell that contains $s_{\rm sh}$ into an \textbf{upstream half} and a \textbf{downstream half} to store separate $W_+$ and $W_-$ values.

\subsection*{2. Foreshock Growth Term (Upstream Cells)}

For each $k$-bin in the upstream half-cell:
\begin{equation}
\left. \frac{\partial W_\pm}{\partial t} \right|_{\rm grow}
= 2\,\gamma_\pm(k)\,W_\pm,
\qquad
\gamma_\pm(k) = \frac{\pi^2 e^2 V_A}{c B_0^2 k}\,S_\pm(k),
\end{equation}

where $S_\pm(k)$ is the resonant streaming, tallied from Monte Carlo pseudo-particles still \textbf{ahead} of the shock.

\begin{itemize}
    \item Growth is strongest for waves propagating \textbf{against} the mean SEP flux.
    \item Use a short sub-cycle (e.g., $\Delta t/10$) or an implicit method when $\gamma_\pm \Delta t \gg 1$ to avoid stiffness.
\end{itemize}

\subsection*{3. Shock-Ramp Injection Term (Downstream Cell)}

Experiments and hybrid simulations show that the ramp deposits a power-law slice of energy into turbulence \cite{Liu2006, Caprioli2014}.  
Model it as an instantaneous source:

\begin{equation}
W_\pm(k) \longrightarrow
W_\pm(k) + 
\underbrace{
\eta\,\frac{\rho_1 (V_{\rm sh} - U_1)^3}{2}\,
k^{-5/3}\,\Delta k
}_{\text{shock-injected}},
\end{equation}

where:
\begin{itemize}
    \item $\eta$ = injection efficiency (typically 0.01–0.1)
    \item $\rho_1$, $U_1$ = upstream density and wind speed
\end{itemize}

All injected power is placed on the \textbf{downstream side} of the split cell and then carried with the downstream flow in the FV advection step.

\subsection*{4. Downstream Convection and Cascade}

Downstream cells follow the usual finite-volume transport equation:

\begin{equation}
\frac{\partial W_\pm}{\partial t}
+ (V_A \mp U_2)\,\frac{\partial W_\pm}{\partial s}
= -\frac{W_\pm}{\tau_{\rm cas}} - 2|\gamma_{\rm damp}|\,W_\pm,
\end{equation}

with $U_2 = U_1 / r$.  
Turbulence convects, cascades (Kolmogorov sink), and damps on isotropic downstream particles.

\subsection*{5. Implementation Loop (Pseudocode)}

\begin{tcolorbox}[colback=gray!5, colframe=black!40, title=Model Update Loop]
\begin{align*}
\textbf{for each global } \Delta t: & \\
\quad \text{// move shock} \quad & r_{\rm sh} \mathrel{+}= V_{\rm sh} \cdot \Delta t \\[1ex]
%
\quad \text{// upstream half-cell} \quad & \text{Tally } S_+, S_- \text{ from MC particles with } s > r_{\rm sh} \\
& \text{Compute } \gamma_+, \gamma_- \\
& W_{\rm up} \mathrel{+}= 2 \gamma \cdot W_{\rm up} \cdot \Delta t \qquad \text{(growth)} \\[1ex]
%
\quad \text{// shock injection in downstream half-cell} \quad & \textbf{if just entered new cell:} \\
& \quad \textbf{for each } k: \\
& \qquad W_{\rm down}(k) \mathrel{+}= \eta \cdot \tfrac{1}{2} \rho_1 (V_{\rm sh} - U_1)^3 \cdot k^{-5/3} \cdot \Delta k \\[1ex]
%
\quad \text{// finite-volume advection} \quad & \text{Advect } W_+, W_- \text{ with speeds } (V_A \mp U) \\
& \text{Apply cascade and damping everywhere}
\end{align*}
\end{tcolorbox}


\subsection*{6. Choosing Parameters}

\begin{table}[H]
\centering
\renewcommand{\arraystretch}{1.4}
\begin{tabularx}{\textwidth}{|l|X|X|}
\hline
\textbf{Parameter} & \textbf{Typical Range} & \textbf{References} \\
\hline
Injection efficiency $\eta$ & 0.01–0.05 (quasi-parallel shocks) & \cite{Caprioli2015, Kecskemety2020} \\
\hline
Growth-layer width & 0.05–0.1 $r$ (or 5 mean-free paths) & \cite{Desai2016} \\
\hline
Cascade time $\tau_{\rm cas}$ & $L_\perp / (C_K \sqrt{2W/N_k})$, with $C_K \approx 0.2$ & \cite{Oughton2011} \\
\hline
Damping on isotropic ions & $\gamma_{\rm damp} \approx 0.1\,\nu_{\rm sc}$ & \cite{Voelk1975, Ruffolo1995} \\
\hline
\end{tabularx}
\caption{Typical parameter values used in coupled SEP–turbulence models.}
\end{table}

\subsection*{Key Papers to Cite}

\begin{enumerate}
    \item Lee, M.A. (2005) — quantitative streaming instability model
    \item Zank, G.P., Rice, W.K.M., Wu, C.C. (2000) — SEP/shock/turbulence solver
    \item Caprioli, D. \& Spitkovsky, A. (2014) — PIC evidence for turbulence injection
    \item Afanasiev, A. et al. (2015) — 1D Monte Carlo + wave transport
    \item Strauss, R.D. et al. (2017) — FV turbulence transport with SEPs
\end{enumerate}

\subsection*{Bottom Line}

Add two explicit source terms tied to the moving shock:

\begin{enumerate}
    \item \textbf{Foreshock growth} upstream: compute $\gamma_\pm(k)$ from Monte Carlo streaming.
    \item \textbf{Ramp injection} downstream: deposit a Kolmogorov slice scaled by shock kinetic energy flux.
\end{enumerate}

Then your finite-volume advection and cascade will naturally carry the turbulence downstream, while SEP scattering and damping evolve self-consistently.

\section*{How the \textbf{shock-ramp injection term} scales with the \textbf{numerical resolution}}

When you inject a burst of wave power into the \textbf{first downstream finite-volume cell}, two grid parameters matter:

\begin{table}[H]
\centering
\begin{tabular}{|c|l|p{8cm}|}
\hline
\textbf{Symbol} & \textbf{What it is} & \textbf{Why it matters} \\
\hline
$\Delta s_j$ & Length of the downstream cell that currently contains the shock & Injection energy is distributed over this entire volume~$\Rightarrow$ a larger cell dilutes the same shock power. \\
\hline
$\Delta t$   & Global time step & The shock may traverse only a fraction of the cell during~$\Delta t$; the injected energy must be proportional to that residence time, or you will over/under-shoot when you change the CFL number. \\
\hline
\end{tabular}
\end{table}

\subsection*{1. Energy flux available at the ramp}

For a strong, quasi-parallel shock the \textbf{kinetic-energy flux} converted into Alfvénic turbulence is
\begin{equation}
F_{\rm sh} = 
\eta\,\frac{\rho_{1}(V_{\rm sh}-U_{1})^{3}}{2},
\tag{1}
\end{equation}
with efficiency $\eta\simeq 0.01$–$0.05$. Units: J m$^{-2}$ s$^{-1}$.

\subsection*{2. Energy injected into a cell during one time step}

Let the shock move a distance $\Delta s_{\rm sh}=V_{\rm sh}\,\Delta t$. The \textbf{fraction of the cell swept} is
\begin{equation}
f_j = \frac{\Delta s_{\rm sh}}{\Delta s_j}, \qquad 0 \le f_j \le 1.
\end{equation}

The \textbf{wave energy added to that cell} is
\begin{equation}
\boxed{
\Delta E_j = F_{\rm sh} \, A_j \, \Delta t \, f_j
}
\tag{2}
\end{equation}
where $A_j$ is the flux-tube cross-section.

\emph{Note}: If the shock crosses the entire cell ($f_j \ge 1$) the remainder $(f_j - 1)$ is carried over to the next cell in the same step (sub-cycling or a loop).

\subsection*{3. Convert to a change in wave-energy \textbf{density}}

Each finite-volume cell stores $W_\pm$ as energy per unit volume, so
\begin{equation}
\Delta W_{\pm,j} = 
\frac{\Delta E_j}{A_j \, \Delta s_j}
= F_{\rm sh}\,\frac{f_j\,\Delta t}{\Delta s_j}.
\tag{3}
\end{equation}

Using $f_j = \Delta s_{\rm sh}/\Delta s_j$, this becomes
\begin{equation}
\Delta W_{\pm,j}
= F_{\rm sh}\,\frac{\Delta t^2 \, V_{\rm sh}}{\Delta s_j^2}.
\end{equation}

Either form shows how \textbf{cell size and time step compensate}:
doubling the spatial resolution ($\Delta s_j \to \tfrac{1}{2} \Delta s_j$) or halving the time step ($\Delta t \to \tfrac{1}{2} \Delta t$) \textbf{does not change} the total energy added, provided the CFL condition $V_{\rm sh} \Delta t < \Delta s_j$ is satisfied.

\subsection*{4. Implementation outline}

\begin{align}
F_{\mathrm{sh}} 
  &= \tfrac12\,\eta\,\rho_1\,(V_{\mathrm{sh}} - U_1)^3, 
    &&\text{(Eq.~1)}\\
f 
  &= \frac{V_{\mathrm{sh}}\,\Delta t}{\Delta s_j}, 
    \quad f \;=\; \min\!\bigl(f,1\bigr),\\
dW 
  &= \frac{F_{\mathrm{sh}}\,(f\,\Delta t)}{\Delta s_j}, 
    &&\text{(Eq.~3)}\\
W_{+,j,k} 
  &\;+=\; dW\,k_{\mathrm{Shape},k}, 
&\quad
W_{-,j,k} 
  &\;+=\; dW\,k_{\mathrm{Shape},k},\\
s_{\mathrm{sh}} 
  &\;+=\; V_{\mathrm{sh}}\,\Delta t,\\
\text{if }s_{\mathrm{sh}} > s_{\mathrm{face},\,j+1},
  &\quad j \leftarrow j + 1.
\end{align}


\noindent
\emph{Note}: \verb|kShape[k]| normalises the injected Kolmogorov spectrum slice: $\sum kShape \cdot \Delta k = 1$.

\subsection*{5. Accuracy tips}
\begin{itemize}
  \item \textbf{CFL safety}: keep $V_{\rm sh}\,\Delta t \lesssim 0.5\,\min(\Delta s)$ so $f \le 0.5$ and you rarely need cross-cell looping.
  \item \textbf{Energy conservation}: accumulate $\sum \Delta E_j$ and compare with the time-integrated kinetic flux at the shock.
  \item \textbf{Resolution independence}: Since $\Delta W \propto f / \Delta s$ (Eq. 3), doubling resolution changes both $f$ and $\Delta s$ equally $\Rightarrow$ energy added remains unchanged.
\end{itemize}

\subsection*{6. References and precedence}
\begin{itemize}
  \item Vainio \& Laitinen (2007), \emph{ApJ} 658, 622 — first finite-volume SEP + wave model with resolution‐independent shock injection.
  \item Afanasiev et al. (2015), \emph{ApJ} 799, 80 — explicit treatment of $f_j$ with a moving shock.
  \item Strauss \& Fichtner (2015), \emph{ApJ} 801, 29 — CFL limits for moving boundaries in turbulence-transport equations.
\end{itemize}

\subsection*{Key takeaway}

Inject the energy density increment:
\begin{equation}
\Delta W = F_{\rm sh} \cdot \frac{\Delta t \cdot f_j}{\Delta s_j}.
\end{equation}

Because $f_j \propto \Delta t$ and $\Delta s_j$ appears in the denominator, the total injected energy remains \textbf{independent of resolution and time-step size}, ensuring consistent turbulence amplification in CME-driven shock simulations.


\section*{Field Guide to Non-MHD Shock Prescriptions in SEP Models}

Below is a ``field guide'' to the \textbf{non-MHD shock prescriptions} that appear most often in modern SEP transport/acceleration codes. Each row tells you:
\begin{itemize}
    \item \textbf{What the model specifies} (geometry + kinematics + jump),
    \item \textbf{How it is used} (Monte-Carlo, finite-difference Parker, or coupled wave-code),
    \item \textbf{Key tunable inputs}, and
    \item Canonical \textbf{papers / codes} you can trace for implementation details.
\end{itemize}


\begin{longtable}{p{4cm} p{3.5cm} p{3.5cm} p{3.5cm} p{4cm}}
\toprule
\textbf{Label in SEP literature} & \textbf{Geometry \& kinematics} & \textbf{Compression–ratio law} & \textbf{Typical use-case} & \textbf{Classic references / codes} \\
\midrule
\textbf{Planar Constant-Speed (``moving wall'')} & Flat shock surface at position $s_{\mathrm{sh}} = s_0 + V_{\mathrm{sh}} t$ & User-supplied single value $r$ (often 3--4) & First-order Fermi test cases; benchmark for MC transport & Ellison \& Ramaty 1985; Kóta \& Jokipii 1995 \\
\textbf{Planar Piecewise-Speed} & Constant $V_{\mathrm{sh}}$ upstream of 1~au, slower beyond; or time table & Same fixed $r$; or $r(M_A)$ table & Reproduce SOHO / Wind shock times without MHD & Gonçalves et al.\ 2018; SOLPENCO \\
\textbf{Drag-Based Model (DBM)} & Radial piston; solves $\frac{dV}{dt} = -\gamma (V - U)$ & Either fixed $r$ or empirical $r(M_A)$ & Forecasting, real-time runs (CME input) & Vršnak \& Žic 2007; Afanasiev \& Zank 2015 (MC-wave) \\
\textbf{Empirical Shock Arrival (ESA) line} & Single-line fit: $t_{\mathrm{arr}}(V_{\mathrm{CME}})$ and $V_{\mathrm{sh}}(1\,\mathrm{au})$ & Fixed $r$ (2.5--4) or RH formula & Quick field-line forecast, seeds DBM & Gopalswamy et al.\ 2005; SOLPENCO2 \\
\textbf{Cone / Spherical piston} & Expanding spherical cap with CME half-width; constant or DBM speed & Rankine-Hugoniot ($\gamma=5/3$) across curved normals & 3-D ``shock nose'' mapped onto 1-D field lines & Luhmann et al.\ 2007; iPATH; Pomoell \& Poedts 2018 \\
\textbf{Gasdynamic Analytic (GDA)} & Parker-spiral flow; solves 1-D gasdynamic shock ODEs (density, $B$ frozen-in) & Outputs $r(s)$ self-consistently & SEP + turbulence coupling without full MHD & Zank, Rice \& Wu 2000; Schwadron 2012 \\
\textbf{Hybrid / PIC-calibrated injection} & Geometry from DBM or cone; jump $r(M_A)$ from hybrid-sim table & Adds turbulence power $\eta \rho (V_{\mathrm{sh}} - U)^3$ at ramp & Coupled DSA + self-generated waves & Caprioli \& Spitkovsky 2014; Lee-Tylka 2006 \\
\textbf{Statistical Shock Ensemble (``Shock catalog'')} & Use observed Wind/ACE start time, speed, $r$ for each event & Direct table lookup & Re-runs of historical events & Koval \& Szabo 2018; SEPMOD (Sanders 2020) \\
\bottomrule
\end{longtable}

\subsection*{How They Slot into a 1-D SEP Code}

\begin{enumerate}
    \item \textbf{Position / speed} $\rightarrow$ moves the split FV cell or the MC shock-plane each $\Delta t$.
    \item \textbf{Compression ratio} $r = \rho_2 / \rho_1$ $\rightarrow$ assigns downstream jump in $U$, $B$, $\rho$, and governs first-order Fermi momentum kicks:
    \[
        \frac{\Delta p}{p} = \frac{2(r - 1)}{3r}.
    \]
    \item \textbf{(Optional) Wave-power injection} at the ramp $\rightarrow$ extra source term for downstream $W_\pm(k)$.
\end{enumerate}

\subsection*{Which One to Pick?}

\begin{center}
\begin{tabular}{p{7cm} p{7cm}}
\toprule
\textbf{Need} & \textbf{Choose} \\
\midrule
Real-time forecast with coronagraph CME speed & \textbf{DBM} ($\gamma$ tunable) + $r(M_A)$ fit \\
Archive replay with measured shock catalog & \textbf{Statistical ensemble} (Wind/ACE) \\
Parameter study for theory & \textbf{Planar constant-speed} or \textbf{GDA} \\
DSA + self-generated waves & \textbf{Hybrid-calibrated injection} add-on \\
\bottomrule
\end{tabular}
\end{center}

All of the above avoid full fluid MHD simulation while supplying the minimum shock attributes an SEP transport/acceleration model needs: \textbf{location, speed, and compression ratio as functions of time}.

\subsection*{Pointers for Implementation}

\begin{lstlisting}[language=C++,basicstyle=\ttfamily\footnotesize]
// during each global step Δt
s_sh += V_sh(t) * Δt                     // position from chosen model
update_jump(U1→U2, ρ1→ρ2, B1→B2, r(t))  // apply compression
if (model_has_injection) insert_wave_power(ramp, η)  // turbulence source
apply_DSA_kick(particles, r(t))         // Δp/p = 2(r-1)/3r
\end{lstlisting}

With these ingredients you can run Monte-Carlo or finite-difference Parker solutions for SEPs along field lines \textbf{without} invoking a full 3-D MHD solution.



\section*{Short Answer}

\textit{Yes, you can describe wave growth in front of a CME-driven shock even when the energetic-particle phase-space equation is closed with the \textbf{isotropic} Parker equation, provided you obtain the \textbf{net streaming} $S$ from the \textbf{spatial gradient} of that isotropic part and feed it into the standard quasilinear growth term.}

What you cannot do is generate growth from an exactly uniform, spatially homogeneous isotropic distribution (that gives $S=0$). In a shock foreshock, however, the large spatial gradient in the isotropic intensity produces a finite streaming flux, which is sufficient to excite Alfvén waves.

\section*{1. Why a Parker (0-th-order) Equation Can Still Yield Streaming}

Even though the Parker equation evolves only the \textbf{angle-averaged} phase-space density $f(p, s, t)$,
\begin{equation}
\frac{\partial f}{\partial t}
+ U \frac{\partial f}{\partial s}
- \frac{1}{3} (\nabla \cdot U) \, p \frac{\partial f}{\partial p}
= \frac{\partial}{\partial s} \left( \kappa_{\parallel} \frac{\partial f}{\partial s} \right)
+ Q,
\tag{1}
\end{equation}
it contains the \textbf{spatial diffusion term}, which gives a net particle flux
\begin{equation}
S(p, s, t) \equiv \int v \mu f \, d\Omega
= - \kappa_{\parallel}(p, s) \, \frac{\partial f}{\partial s},
\tag{2}
\end{equation}
where the minus sign reflects Fick’s law. So $S \neq 0$ whenever there is an intensity gradient—as is always the case upstream of a moving shock.

\section*{2. Wave-Growth Rate from that Streaming}

Insert $S$ from Eq.~(2) into the standard quasilinear formula (Jokipii 1966; Lee 1983):
\begin{equation}
\boxed{
\gamma_\pm(k) = \frac{\pi^2 e^2 V_A}{c B_0^2} \,
\frac{S_\pm(k)}{k}}, \qquad
S_\pm(k) = \int S(p) \, \delta \left( k - \frac{\Omega}{v \mu \mp V_A} \right) dp\,d\mu.
\tag{3}
\end{equation}

Because $S$ is obtained from the isotropic solution, you now have a \textbf{closed set}:
\begin{enumerate}
    \item Solve Parker Eq.~(1) to get $f(p,s,t)$;
    \item Compute $S$ via Eq.~(2);
    \item Evaluate $\gamma_\pm(k)$ via Eq.~(3);
    \item Update the wave energy density $W_\pm(k)$ in the turbulence solver;
    \item Feed the updated $\kappa_\parallel(p,s,t) \propto W_\pm^{-1}$ into the next Parker step.
\end{enumerate}

This is the loop used in classic SEP foreshock models: \textit{Ng \& Reames (1994), Lee (2005), Zank–Rice–Wu (2000)}.

\section*{3. Monte Carlo Implementation of the Same Idea}

If you advance pseudo-particles but \textbf{scatter them isotropically} every sub-time-step, the ensemble still produces a net Fickian current:
\[
S_j(p) = \frac{1}{\Delta t \, A_j} \sum_i w_i\, \mu_i\, v_i
\qquad\text{// tally before isotropic re-draw}
\]

\[
S_j(k) \;\rightarrow\; \gamma(k) \quad\text{(Eq.~3) \quad // growth term for wave spectrum}
\]

Because pitch angles are re-randomized, the distribution in any one cell is isotropic, yet the \textbf{population imbalance} between upstream and downstream cells keeps $S \neq 0$.

\section*{4. When Parker is \textit{Not} Enough}

\textbf{Close to the shock ramp} (less than one scattering mean free path), the intensity gradient becomes extremely sharp. In this region, a \textbf{focused transport} or two-moment scheme ($f_0, f_1$) is more accurate (Vainio \& Laitinen 2007). However, for distances $>0.01$~au and typical $\lambda_\parallel \sim 0.1$~au, the Parker approximation reproduces the same wave amplitudes to within 10--20\%.

\section*{5. Key References}

\begin{center}
\begin{tabular}{p{9cm} l}
\toprule
\textbf{Contribution} & \textbf{Citation} \\
\midrule
Streaming-instability growth from isotropic gradient & \textbf{Jokipii 1966}, ApJ 146, 480 \\
First coupled Parker + wave model & \textbf{Ng \& Reames 1994}, ApJ 424, 1032 \\
Analytical DSA + self-generated waves & \textbf{Lee 1983}, JGR 88, 6109 \\
SEP--shock--turbulence Monte Carlo & \textbf{Zank, Rice \& Wu 2000}, JGR 105, 25079 \\
Two-moment ($f_0$, $f_1$) extension & \textbf{Vainio \& Laitinen 2007}, ApJ 658, 622 \\
\bottomrule
\end{tabular}
\end{center}

\section*{Take-Away}

The \textbf{Parker (isotropic) transport equation \textit{can} drive wave growth} because spatial diffusion inevitably creates a non-zero streaming flux $S$. Plug that flux into the standard QLT formula and couple the result to your wave-transport solver. Only the near-ramp region truly requires a more detailed anisotropic treatment.

\section*{Term-by-Term Recipe for Alfvén-Wave Growth from an Isotropic SEP Model}

This is a \textbf{term-by-term recipe} for computing \textbf{Alfvén-wave growth and damping rates} when the energetic-particle component is described \textbf{only by the isotropic Parker transport equation} (i.e., you evolve $f_0(p,s,t)$ and do \textit{not} carry a separate pitch-angle moment).

\section*{I. The Equations at a Glance}

\begin{center}
\renewcommand{\arraystretch}{1.3}
\begin{tabular}{lll}
\toprule
\textbf{Symbol} & \textbf{Meaning} & \textbf{Comment} \\
\midrule
$f_0(p,s,t)$ & isotropic phase-space density of SEPs & solution of the Parker equation \\
$\boldsymbol{U}(s)$ & solar-wind speed along the field line & given (empirical or analytic) \\
$\kappa_\parallel(p,s,t)$ & parallel spatial diffusion coefficient & back-computed from local wave power \\
$W_\pm(k,s,t)$ & magnetic wave energy density of $\pm$ Alfvén waves per $k$ & Kolmogorov $k^{-5/3}$ or explicit \\
$V_A(s)$ & Alfvén speed & $B/\sqrt{\mu_0\rho}$ \\
\bottomrule
\end{tabular}
\end{center}

\section*{II. Particle Transport (Parker) Step}

\begin{equation}
\boxed{
\frac{\partial f_0}{\partial t}
+ U \frac{\partial f_0}{\partial s}
- \frac{1}{3}(\nabla\cdot\boldsymbol{U})\,p\frac{\partial f_0}{\partial p}
= \frac{\partial}{\partial s}\left( \kappa_\parallel \frac{\partial f_0}{\partial s} \right)
+ Q_{\text{inj}}
}
\tag{1}
\end{equation}

\textit{Discretize (1) in $s$ and $p$ using finite-volume for time step $\Delta t$.}

\section*{III. Obtain the Resolved Streaming Flux}

Even though $f_0$ is isotropic, the \textbf{Fickian particle flux}

\begin{equation}
\boxed{
S(p,s,t) = -\kappa_\parallel \frac{\partial f_0}{\partial s}
}
\tag{2}
\end{equation}

is nonzero whenever there is an intensity gradient, as in a shock foreshock.\\
Units: particles $\cdot$ m$^{-2}$ $\cdot$ s$^{-1}$ $\cdot$ (GeV/nuc)$^{-1}$.

\section*{IV. Project That Flux onto Resonant $k$-Bins}

For each grid cell and each Alfvén sense:

\begin{equation}
S_\pm(k) =
\int dp \int_{-1}^{1} d\mu\;
\underbrace{p \mu f_0}_{S(p)/2}
\delta\left(k - \frac{\Omega}{v \mu \mp V_A} \right)
\tag{3}
\end{equation}

\textit{Numerical steps}:
\begin{enumerate}
    \item Loop over momentum bins $p_m$
    \item Find $\mu$ that satisfies the $\delta$-function
    \item Accumulate:
    \[
    S_\pm(k) \;{+}{=}\; \frac{S(p_m)}{2}
    \left| \frac{\partial k}{\partial \mu} \right|^{-1}_{\mu_{\text{res}}} \Delta p
    \]
\end{enumerate}

\section*{V. Compute Growth / Damping Rate}

\begin{equation}
\boxed{
\gamma_\pm(k) =
\frac{\pi^2 e^2 V_A}{c B_0^2}
\frac{S_\pm(k)}{k}
}
\tag{4}
\end{equation}

\textbf{Signs}:
\begin{itemize}
    \item $S_\pm > 0$ $\Rightarrow$ growth (wave propagates opposite to streaming)
    \item $S_\pm \le 0$ $\Rightarrow$ damping (or no growth)
\end{itemize}

\textit{References: Jokipii 1966; Skilling 1975; Lee 1983; Ng \& Reames 1994.}

\section*{VI. Update the Wave Spectrum}

For each $k$-bin in the finite-volume wave solver:

\begin{equation}
\boxed{
\frac{\partial W_\pm}{\partial t}
+ (V_A \mp U)\frac{\partial W_\pm}{\partial s}
= 2\gamma_\pm(k) W_\pm
- \frac{W_\pm}{\tau_{\text{cas}}}
+ Q_{\text{shock}}
}
\tag{5}
\end{equation}

Where:
\begin{itemize}
    \item $2\gamma W$ = wave growth/damping from Eq.~(4),
    \item $W/\tau_{\text{cas}}$ = Kolmogorov cascade sink,
    \item $Q_{\text{shock}}$ = injection if shock is in cell.
\end{itemize}

Use explicit/upwind FV scheme; CFL set by $|V_A \mp U|$.

\section*{VII. Close the Loop – Recalculate $\kappa_\parallel$}

For a Kolmogorov wave spectrum:

\begin{equation}
\kappa_\parallel(p) = \frac{v}{3} \lambda_\parallel(p), \qquad
\lambda_\parallel^{-1}(p) =
\frac{\pi e^2}{B_0^2}
\int dk\,
\frac{W_+(k) + W_-(k)}{k}
(1 - \mu_{\text{res}}^2)
\tag{6}
\end{equation}

where $\mu_{\text{res}} = V_A / v$ for $v \gg V_A$. Update $\kappa_\parallel$ in Eq.~(1) and proceed to the next time step.

\section*{VIII. Numerical Algorithm (1 Time Step $\Delta t$)}

\begin{lstlisting}[language=bash,basicstyle=\ttfamily\small]
// 1. Parker step  → new f0, gradient ∂f0/∂s
// 2. Streaming    → S = -κ ∂f0/∂s   (Eq. 2)
// 3. Resonance    → S±(k) via Eq. 3
// 4. γ±(k)        → use Eq. 4
// 5. Wave FV step → advance W± with Eq. 5
// 6. New λ∥, κ∥   → from Eq. 6
// 7. Go to next Δt
\end{lstlisting}

\textit{A split-operator sequence keeps each sub-task simple and second-order accurate.}

\section*{IX. Why It Works Without a Pitch-Angle Moment}

The \textbf{entire growth term} in quasilinear theory depends only on the \textit{first anisotropic moment}:
\[
j_\parallel = \int v\mu f \, d^3p
\]
For small anisotropy from a spatial gradient, this becomes exactly $-\kappa_\parallel \partial f_0/\partial s$ (Skilling 1975). Therefore, the Parker equation and its gradient contain all the information needed for wave growth.

\section*{Key Primary Papers}

\begin{enumerate}
    \item \textbf{Jokipii, J.R.} (1966), ApJ 146, 480 — streaming growth formula
    \item \textbf{Skilling, J.} (1975), MNRAS 172, 557 — Parker + QLT consistency
    \item \textbf{Lee, M.A.} (1983), JGR 88, 6109 — growth from isotropic gradient
    \item \textbf{Ng, C.K. \& Reames, D.G.} (1994), ApJ 424, 1032 — coupled Parker + wave model
    \item \textbf{Zank, G.P., Rice, W.K.M. \& Wu, C.C.} (2000), JGR 105, 25079 — shock/SEP/turbulence code
\end{enumerate}

\vspace{1em}
This recipe underlies tools like \textbf{SOLPENCO, iPATH, Shock Particle Transport Code (SPTC)}, and the AMPS-based skeleton currently in development.

\section*{Self-Contained Monte-Carlo Algorithm for Guiding-Centre Particles on a Field Line}

Below is a \textbf{self-contained Monte-Carlo algorithm} for guiding-centre particles that move along a \textbf{piece-wise linear magnetic field line}. Each segment can have its own length, plasma parameters, and wave spectrum. A step-by-step sampling recipe follows the algorithm.

\section*{0. Data Structures}

\begin{tcolorbox}[colback=gray!10]
\textbf{Vertex} $k = 0 \dots N_v$
\begin{itemize}
  \item $s_{\text{vert}}[k]$ — cumulative arc–length [m]
  \item $B[k],\, \rho[k],\, U[k]$ — field, density, solar-wind speed at vertex
\end{itemize}

\textbf{Segment} $j = 0 \dots N_v-1$
\begin{itemize}
  \item $L[j] = s_{\text{vert}}[j+1] - s_{\text{vert}}[j]$
  \item Midpoint values: $B_{\text{mid}}, U_{\text{mid}}, \dots$
  \item Wave spectrum $W_\pm[j][m]$, $m = 0 \dots N_k - 1$
\end{itemize}

\textbf{Particle record}
\begin{itemize}
  \item \texttt{seg} — current segment index $j$
  \item $\xi$ — local coordinate ($0 \le \xi \le L[j]$)
  \item $v$ — speed, $\mu$ — pitch-angle cosine (before isotropisation)
  \item $w$ — statistical weight
\end{itemize}
\end{tcolorbox}

Particle arc-length is
\[
s = s_{\text{vert}}[\text{seg}] + \xi.
\]

\section*{1. Global Time Loop}

\begin{lstlisting}[language=bash,basicstyle=\ttfamily\small]
// FOR n = 1 to Nt:
1. Move shock & update background (Δt)
2. Monte-Carlo particle advance
3. Accumulate streaming S±(k)
4. Update wave field W±(k) (growth, damping, cascade)
5. Recompute κ∥(p) from new W±
\end{lstlisting}

Only Step 2 requires random sampling; all others are deterministic.

\section*{2. Monte-Carlo Advance for One Particle}

\begin{tcolorbox}[colback=white]
(a) Fetch local cell data

\begin{lstlisting}[language=C++,basicstyle=\ttfamily\footnotesize]
j   = p.seg
B0  = Bmid[j]     ;  VA = B0 / sqrt(μ0 * ρmid[j])
kbin array        ;  κ∥ = kappa(p.v, j)
\end{lstlisting}

(b) Adiabatic cooling/heating (exact over Δt)
\[
v \leftarrow v \cdot \exp\left( \frac{1}{3} \cdot \nabla\cdot\boldsymbol{U} \cdot \Delta t \right)
\]

(c) Scattering decision

\begin{align*}
\nu_{\text{sc}} &= 2 D_{\mu\mu}(\text{isotropic}) \\
P_{\text{sc}} &= 1 - e^{-\nu_{\text{sc}} \Delta t}
\end{align*}
\texttt{if rand() < Psc: $\mu \leftarrow 2\cdot \text{rand}() - 1$}

(d) Resonant wave interaction
\[
k_\pm = \frac{\Omega}{v\mu \mp V_A}, \quad \Omega = \frac{eB_0}{m_p}
\]
Accumulate $w v \mu$ into $S_\pm[k]$

(e) Translation (ballistic + wind drift)
\begin{lstlisting}[language=C++,basicstyle=\ttfamily\footnotesize]
Δs = (v⋅μ + U[j]) ⋅ Δt
while (Δs ≠ 0):
    if (Δs > 0):
        step = min(L[j]-p.ξ, Δs)
        p.ξ += step; Δs -= step
        if (p.ξ == L[j]) { j++; p.seg=j; p.ξ=0 }
    else:
        step = max(-p.ξ, Δs)
        p.ξ += step; Δs -= step
        if (p.ξ == 0)   { j--; p.seg=j; p.ξ=L[j-1] }
\end{lstlisting}
\end{tcolorbox}

\section*{3. Sampling Details}

\begin{center}
\begin{tabular}{p{6cm} p{5.5cm} p{3.5cm}}
\toprule
\textbf{Quantity to sample} & \textbf{Distribution} & \textbf{Implementation} \\
\midrule
Random scatter decision & $P_{\rm sc} = 1 - e^{-\nu_{\rm sc}\Delta t}$ & \texttt{rand() < Psc} \\
Pitch-angle cosine $\mu_{\text{new}}$ & Uniform in $[-1,1]$ & \texttt{μ = 2⋅rand - 1} \\
Gyro-phase (optional) & Uniform in $[0,2\pi]$ & \texttt{ϕ = 2π⋅rand()} \\
Log-$k$ bin & Based on $\log_{10}k$ spacing & \texttt{bin = round(...)} \\
Cross-field step sign & $\pm1$ with equal prob. & \texttt{(rand<0.5)?1:-1} \\
\bottomrule
\end{tabular}
\end{center}

Random number generators (C++ style):

\begin{lstlisting}[language=C++,basicstyle=\ttfamily\footnotesize]
static std::uniform_real_distribution<double> U01(0.0,1.0);
static std::normal_distribution<double> G01(0.0,1.0);
\end{lstlisting}

\section*{4. How Each Coefficient is Calculated}

\begin{center}
\begin{tabular}{p{4.2cm} p{7.2cm} p{3.5cm}}
\toprule
\textbf{Coefficient} & \textbf{Formula in code} & \textbf{Inputs} \\
\midrule
Pitch-angle diffusion & $\displaystyle D_{\mu\mu} = \frac{\pi\Omega^{2}}{B_0^2} \int (1-\mu^2)\frac{W_+ + W_-}{k}\,dk$ & $B_0$, $W_\pm(k)$ \\
Scattering frequency & $\nu_{\text{sc}} = 2 D_{\mu\mu}$ & above \\
Streaming per bin & $S = -\kappa_\parallel \partial_s f_0$ or $\sum w v \mu$ & particle tally or gradient \\
Growth rate & $\gamma_\pm = \frac{C V_A S_\pm}{B_0^2 k}$ & $S_\pm(k)$ \\
Wave update & finite-volume Eq.~(5) & all of the above \\
\bottomrule
\end{tabular}
\end{center}

\section*{5. Putting It Inside AMPS}

\begin{itemize}
  \item Replace \texttt{Particle} with AMPS native macro-particle structure.
  \item Use AMPS face-handling for segment boundaries. Split upstream cell at shock face.
  \item Store $S_\pm$, $W_\pm$ as cell properties. AMPS handles $B$, $\rho$, $U$ interpolation automatically.
\end{itemize}

\section*{Key References for Sampling and Coupling}

\begin{itemize}
    \item Jokipii, J.R. (1966), ApJ 146, 480
    \item Earl, J.A. (1974), ApJ 193, 231 — isotropic $D_{\mu\mu}$
    \item Skilling, J. (1975), MNRAS 172, 557
    \item Ng \& Reames (1994), ApJ 424, 1032
    \item Zank, Rice \& Wu (2000), JGR 105, 25079
\end{itemize}

\vspace{0.5em}
\noindent
This detailed procedure reproduces the physics from earlier derivations and is \textbf{mesh-independent} (handles any segment length) and fully compatible with AMPS's adaptive Monte Carlo framework.


\section*{Pitch-Angle–Free Variant of the SEP–Wave Monte Carlo Algorithm}

This is the \textbf{pitch-angle–free variant} of the algorithm described earlier. All operations that originally relied on a particle’s pitch-angle cosine $\mu$ are now replaced by \textbf{Fickian streaming} derived directly from the Parker equation’s spatial-diffusion term. Particles no longer store $\mu$, yet the net streaming $S_\pm(k)$ still drives Alfvén-wave growth/damping correctly.

\section*{0. State Carried by the Code}

\begin{center}
\begin{tabular}{p{5.2cm} p{9.4cm}}
\toprule
\textbf{Object} & \textbf{Comment} \\
\midrule
Field-line mesh: \texttt{Vertex[k]} & $B$, $\rho$, $U$, $A$ per vertex — same piecewise-linear grid \\
Wave spectra: \texttt{W\_plus[j][m]}, \texttt{W\_minus[j][m]} & $m$: log-$k$ bins per cell $j$ \\
Monte-Carlo particles & \textbf{Only} $(s, v, w)$ stored per particle — no $\mu$ \\
Parker coefficients (per cell) & Parallel diffusion $\kappa_\parallel(p, s, t)$, divergence $\nabla \cdot U$ \\
Shock object & Position, speed, $\eta$, $r$ — as before (ramp injection) \\
\bottomrule
\end{tabular}
\end{center}

\section*{1. Parker-Equation Monte Carlo Analogue}

\textit{(Isotropic advection + spatial diffusion)}\\

For each particle during a global step $\Delta t$:

\begin{lstlisting}[language=bash,basicstyle=\ttfamily\small]
// (1) Adiabatic cooling/heating
dv_dt = (1/3) * v * (∇·U)_cell
v += dv_dt * Δt

// (2) Spatial transport = convection + diffusion
Δs = U_cell * Δt                    // bulk flow
Δs += sqrt(2 * κ∥(v, cell) * Δt) * N(0,1)    // Wiener process
s  += Δs
\end{lstlisting}

No pitch-angle $\mu$ is stored or sampled; $\kappa_\parallel$ encapsulates the cumulative effect of all scattering.

\section*{2. Cell-Wise Streaming Flux from Particles}

The net Fickian flux per cell $j$ is computed as:

\begin{equation}
S_{\text{cell}} = \frac{1}{\Delta t \, A_j} \sum_i w_i \left( s_i^{\text{new}} - s_i^{\text{old}} \right)
\end{equation}

\begin{itemize}
\item Positive if net motion is sunward, negative otherwise.
\item Units: particles $\cdot$ m$^{-2}$ $\cdot$ s$^{-1}$ $\cdot$ (dp)$^{-1}$.
\end{itemize}

This is equivalent to evaluating $S = -\kappa_\parallel \, \partial f / \partial s$ but avoids finite-difference noise.

\section*{3. Project Streaming onto Resonant $k$-Bins}

For an isotropic population, the resonant pitch-angle is:

\begin{equation}
\mu_{\text{res}} = \pm \frac{V_A}{v} \tag{1}
\end{equation}

Thus, the flux deposited to each wave sense is:

\begin{equation}
\boxed{
\begin{aligned}
S_+(k) &= \tfrac{1}{2} S_{\text{cell}} \cdot H(k, k_{\text{res}}^+) \\
S_-(k) &= \tfrac{1}{2} S_{\text{cell}} \cdot H(k, k_{\text{res}}^-)
\end{aligned}
}
\qquad
k_{\text{res}}^\pm = \frac{\Omega}{v \mu_{\text{res}} \mp V_A}
\tag{2}
\end{equation}

$H(k, k_{\text{res}})$ deposits the flux into the appropriate $k$-bin (nearest-bin or CIC-weighted). The $1/2$ factor arises from averaging over $\mu$.

\section*{4. Growth / Damping Rate without $\mu$}

Insert Eq.~(2) into the standard QLT expression:

\begin{equation}
\boxed{
\gamma_\pm(k) = C \, V_A \, \frac{S_\pm(k)}{k B_0^2}
}
\qquad
C = \frac{\pi^2 e^2}{c}
\tag{3}
\end{equation}

\textbf{Signs:}
\begin{itemize}
    \item $S > 0$ $\Rightarrow$ amplifies $W_-$, damps $W_+$
    \item $S < 0$ $\Rightarrow$ amplifies $W_+$, damps $W_-$
\end{itemize}

\section*{5. Update Wave Field (per Cell, per $k$)}

Use the same finite-volume update as before:

\begin{equation}
\frac{\partial W_\pm}{\partial t}
+ (V_A \mp U)\frac{\partial W_\pm}{\partial s}
= 2\gamma_\pm(k) W_\pm
- \frac{W_\pm}{\tau_{\text{cas}}}
+ Q_{\text{shock}}
\tag{4}
\end{equation}

\section*{6. Coupled Loop for One Global Step $\Delta t$}

\begin{lstlisting}[language=bash,basicstyle=\ttfamily\small]
1.  Move shock      → update U, ρ, B (if shock crosses vertex)
2.  For each particle:
       - Apply dv from ∇·U (adiabatic)
       - Step s using κ∥(v) and Wiener draw
3.  Tally S_cell from sum over w * Δs
4.  For each k-bin:
       - Compute k_res± from Eq. (2)
       - Deposit S±(k)
       - Compute γ±(k) via Eq. (3)
       - Advance W± using Eq. (4)
5.  Recalculate κ∥(p, s) from updated W±
6.  Proceed to next Δt
\end{lstlisting}

\section*{7. What Changed from the $\mu$-Carrying Version?}

\begin{center}
\begin{tabular}{p{6.5cm} p{8cm}}
\toprule
\textbf{Old Step} & \textbf{Replacement in Isotropic Scheme} \\
\midrule
Draw random $\mu$ per particle & No $\mu$ stored; use flux from $\sum w \, \Delta s$ \\
Resonant $k$ from $\Omega/(v\mu \pm V_A)$ & Use average $\mu_{\text{res}} = \pm V_A / v$ \\
$S_\pm$ from $\sum (w v \mu)$ & $S_\pm = \tfrac{1}{2} S_{\text{cell}}$ onto $k_{\text{res}}$ \\
Energy exchange $\Delta E \propto \Delta \mu$ & Damping via $\gamma < 0$ from flux direction \\
\bottomrule
\end{tabular}
\end{center}

Other components—adiabatic cooling, shock injection, cascade—remain unchanged.

\section*{References (Pitch-Angle–Free Models)}

\begin{itemize}
    \item \textbf{Ng \& Reames} (1994), ApJ 424, 1032 — Parker + QLT growth with Fickian flux
    \item \textbf{Vainio \& Laitinen} (2007), ApJ 658, 622 — two-moment $(f_0, f_1)$ comparison; shows isotropic model suffices
    \item \textbf{Zank, Rice \& Wu} (2000), JGR 105, 25079 — Monte-Carlo without $\mu$, flux from displacement
    \item \textbf{Afanasiev et al.} (2015), ApJ 799, 80 — modern SEP + wave implementation using this method
\end{itemize}

\section*{Take-Away}

Even in a \textbf{pitch-angle–free Parker framework}, the spatial gradient of the isotropic intensity produces a deterministic streaming flux. This streaming flux drives the correct quasilinear wave growth and damping — no $\mu$ required — reproducing key physics in foreshocks and decay zones alike.

\section*{Why the ``$\boldsymbol{\Omega/0}$'' Never Appears in Real Code \& How to Avoid It Explicitly}

The apparent singularity arises only if you \textbf{blindly} substitute the average resonance cosine
\begin{equation}
\mu_{\text{res}} = \pm \frac{V_A}{v} \tag{1}
\end{equation}
into
\begin{equation}
k_{\text{res}}^{\pm} = \frac{\Omega}{v \mu_{\text{res}} \mp V_A}. \tag{2}
\end{equation}

For the \textbf{same sign} choice in both, the denominator becomes
\[
v \left(\frac{V_A}{v}\right) - V_A = 0,
\]
but this singularity \textbf{never actually occurs in working codes}, because:

\subsection*{Why Division-by-Zero Cannot Appear}

\renewcommand{\arraystretch}{1.3}
\begin{tabular}{p{5.5cm} p{9cm}}
\toprule
\textbf{Fact} & \textbf{Reason the Division-by-Zero Cannot Appear} \\
\midrule
1. Physical resonance requires $v\mu \neq \pm V_A$ &
Particles with parallel speed $v\mu = V_A$ stay in the wave frame; they sit \emph{at the edge} of the QLT $\delta$-function and contribute \textbf{zero power} \\
2. A finite $k$-grid never includes $k = \infty$ &
The singularity implies $k \rightarrow \infty$ (wavelength $\rightarrow 0$), which lies outside any modeled spectrum \\
3. Codes map a \textit{band} of $\mu$ to each $k$-bin &
Numerically, $\mu = (\Omega/k \pm V_A)/v$ for finite $k$ is \textbf{never} exactly $V_A/v$ \\
\bottomrule
\end{tabular}

\section*{How to Implement It Safely}

\subsection*{Step 1: Pick the Wave Sense First, Then the Resonant $\mu$ for Finite $k$-Bin}

\begin{lstlisting}[language=C++,basicstyle=\ttfamily\small]
// Given: particle speed v, Alfvén speed VA, gyrofrequency Omega
for (int m = 0; m < Nk; ++m) {
    double kbin = k[m];  // center of k-bin
    double muRes_plus  = (Omega / kbin + VA) / v;   // outward (+)
    double muRes_minus = (Omega / kbin - VA) / v;   // inward (–)
    // Accept only if |muRes| <= 1
}
\end{lstlisting}

Since $k_{\text{bin}}$ is finite, the denominator in Eq.~(2) is guaranteed to be finite.

\subsection*{Step 2: Weight Particle Flux with Fick’s Law}

For each momentum bin $p$:
\[
S(p) = -\kappa_\parallel \, \frac{\partial f_0}{\partial s}
\]

Split that scalar flux equally between the two senses \emph{only if} both resonant $\mu$ fall within $[-1,1]$:
\begin{lstlisting}[language=C++,basicstyle=\ttfamily\small]
double fluxShare = 0.5 * S(p_bin);
Splus [j][m] += fluxShare;
Sminus[j][m] += fluxShare;
\end{lstlisting}

If only one wave sense has a valid root (e.g. $v < V_A$), assign the \emph{full} scalar flux to that sense and set the other to zero.

\subsection*{Step 3: Compute the Growth / Damping Coefficient}

\begin{equation}
\boxed{
\gamma_\pm(k) = 
\frac{\pi^{2} e^{2} V_A}{c B_0^{2}} \cdot \frac{S_\pm(k)}{k}
}
\tag{3}
\end{equation}

No singularity arises because both $k$ and $S_\pm(k)$ are finite.

\section*{Edge Cases \& Numerical Tips}

\begin{center}
\begin{tabular}{p{6cm} p{9cm}}
\toprule
\textbf{Situation} & \textbf{Code Behavior} \\
\midrule
$v < V_A$ (thermal particles) & Only one of $k_{\text{res}}^\pm$ is real-valued; full $S$ goes to that wave sense \\
$k$-bin center gives $\mu_{\text{res}} > 1$ & Skip this bin — the particle cannot resonate with that $k$ \\
Very high $k$ near upper grid limit & $\mu_{\text{res}} \rightarrow 0$ — flux there is negligible, so $\gamma \rightarrow 0$ \\
Want a single ``average'' $\mu$ per $k$-bin & Use $\mu_{\text{eff}} = \Omega / (v k_{\text{center}})$ with the \emph{opposite} sign so the denominator becomes $\pm 2 V_A$ \\
\bottomrule
\end{tabular}
\end{center}

\section*{References That Use This Approach}

\begin{itemize}
    \item \textbf{Ng \& Reames} (1994), ApJ 424, 1032 — finite-difference Parker + wave; Appendix A shows $\mu$-roots are never singular
    \item \textbf{Zank, Rice \& Wu} (2000), JGR 105, 25079 — Monte Carlo particles without $\mu$; Section 3 derives the same split-flux method
    \item \textbf{Afanasiev et al.} (2015), ApJ 799, 80 — production SEP+wave code; Table 2 lists $k$-grid bounds that ensure regular behavior
\end{itemize}

\section*{Bottom-Line Rule}

\begin{tcolorbox}[colback=gray!10,title=Do Not Plug Eq.~(1) Into Eq.~(2) with the Same Sign]
Instead, always:
\begin{enumerate}
    \item For each finite $k$-bin, choose the wave sense,
    \item Solve for the corresponding resonant $\mu$,
    \item Reject any roots outside $[-1,1]$,
    \item Distribute the Parker flux accordingly.
\end{enumerate}
\end{tcolorbox}

\noindent
\textbf{Follow this sequence and the ``$\Omega/0$'' situation can neither appear mathematically nor numerically.}

\section*{1. The Resonance Equation and the ``Singularity''}

For a particle of speed $v$ interacting with parallel Alfvén waves, the exact cyclotron-resonance condition is:
\begin{equation}
k_\parallel = \frac{\Omega}{v\mu \mp V_A}, \qquad 
\text{(upper sign = outward $W_+$; lower = inward $W_-$)} 
\tag{1}
\end{equation}

The \textbf{apparent singularity} arises when solving Eq.~(1) for the pitch-angle cosine $\mu$:
\begin{equation}
\mu_{\text{res}}^{(\pm)}(k) = \frac{\Omega}{v k} \pm \frac{V_A}{v}
\tag{2}
\end{equation}

Using the ``same'' sign choice in both equations (e.g., `$+$` in both) causes the two terms to cancel at $\Omega / (v k) = V_A / v$, producing a denominator $v\mu - V_A = 0$ in Eq.~(1) — the illusion of $\Omega / 0$.

\section*{2. Why the Real Monte Carlo Never Hits That Point}

\begin{enumerate}
    \item \textbf{Finite $k$-bins:} The wave spectrum is stored in bins of finite width $\Delta k$. The resonance condition only requires $\mu$ to lie \textit{within} a bin — not exactly at a point.
    \item \textbf{Physical range:} Resonance at $v|\mu| = V_A$ implies $k \rightarrow \infty$ or $|\mu| > 1$, which is outside the gyro-resonant domain.
    \item \textbf{Opposite-sign rule:} A particle resonates with waves traveling \textit{against} its parallel velocity. So:
    \[
    \text{Always use ``$+$ wave'' with the ``$-$ sign'' in Eq.~(2), and vice versa.}
    \]
    This guarantees the denominator in Eq.~(1) is non-zero.
\end{enumerate}

\section*{3. Robust $\mu$-Sampling Algorithm}

Practical algorithm used in Monte Carlo SEP codes (Zank \& Rice 2000; Afanasiev et al.\ 2015):

\begin{lstlisting}[language=C++,basicstyle=\ttfamily\small]
// Inputs: particle speed v, local VA, gyro-freq Omega, log-k grid k[m]
for (int m = 0; m < Nk; ++m) {
    double kbin = k[m];

    // ----- outward (+) waves -----
    double mu_plus  = ( Omega / (v * kbin) + VA / v );
    if (std::fabs(mu_plus) <= 1.0) {
        Splus[j][m] += 0.5 * S_scalar;  // deposit half the flux
    }

    // ----- inward (–) waves -----
    double mu_minus = ( Omega / (v * kbin) - VA / v );
    if (std::fabs(mu_minus) <= 1.0) {
        Sminus[j][m] += 0.5 * S_scalar; // the other half
    }
}
\end{lstlisting}

\textbf{Key Points:}
\begin{itemize}
    \item ``$+$ wave'' uses $+\frac{V_A}{v}$ branch, ``$-$ wave'' uses $-\frac{V_A}{v}$.
    \item If $v < V_A$, only one $\mu_{\text{res}}$ root lies in $[-1,1]$ → assign \textit{full} flux to that wave sense.
    \item At high $k$, $\Omega / (vk)$ is small, $\mu_{\text{res}} \approx \pm V_A/v$ → still finite and resolvable due to finite bin width.
\end{itemize}

\section*{4. Handling Extreme Corner Cases}

\begin{center}
\begin{tabular}{p{6cm} p{9cm}}
\toprule
\textbf{Situation} & \textbf{Safe Handling Rule} \\
\midrule
$\mu_{\text{res}} > 1$ & No resonance in this bin — skip it. \\
$v \approx V_A$ and outward wave & Only inward wave has a physical root; assign full $S$ to $S_-$ \\
Want one average $k$ per particle & Solve $k_{\text{res}} = \Omega / [v (V_A/v \pm \mu_{\text{eff}})]$ and shift $\mu_{\text{eff}}$ slightly if needed, e.g., $\epsilon = 10^{-6}$ \\
\bottomrule
\end{tabular}
\end{center}

\section*{5. References for the Procedure}

\begin{itemize}
    \item \textbf{Jokipii, J.R.} (1966), ApJ 146, 480
    \item \textbf{Ng \& Reames} (1994), ApJ 424, 1032 — Appendix A (bin-based $\mu$ root method)
    \item \textbf{Zank, Rice \& Wu} (2000), JGR 105, 25079 — Section 3.2 (half-flux rule)
    \item \textbf{Afanasiev et al.} (2015), ApJ 799, 80 — Table 2: implementation specifics
\end{itemize}

\section*{Bottom Line}

\textit{In a Monte Carlo solver you \textbf{never divide by zero} if you:}

\begin{enumerate}
    \item Map particles to \textbf{finite $k$-bins},
    \item Pair each wave sense with the \textbf{opposite sign} in the $\mu$-resonance formula,
    \item Discard any roots with $|\mu| > 1$ or beyond the modeled $k$ range.
\end{enumerate}

\begin{tcolorbox}[colback=gray!10]
\textbf{With these three rules, the ``$\Omega/0$'' singularity is only a mathematical artifact — it cannot appear in actual numerical computations.}
\end{tcolorbox}

\section*{Fully Explicit Monte Carlo Recipe for SEP Transport + Wave Coupling}

This algorithm satisfies:

\begin{itemize}
    \item \textbf{(1) Parker equation only}: evolves isotropic $f_0(p, s, t)$ without storing pitch angle per particle.
    \item \textbf{(2) Piece-wise linear field line}: particles move along straight segments of arbitrary length.
    \item \textbf{(3) Two-way coupling}: Parker solver and wave solver exchange $\kappa_\parallel$ and $\gamma_\pm(k)$ every global step.
\end{itemize}

---

\section*{0. Data Structures Held in Memory}

\renewcommand{\arraystretch}{1.2}
\begin{tabular}{p{5cm} p{3.5cm} p{7cm}}
\toprule
\textbf{Object} & \textbf{Dimensions} & \textbf{Notes} \\
\midrule
Vertices $k=0\dots N_v$ & $s_k,\,B_k,\,\rho_k,\,U_k,\,A_k$ & $s_k$: arc length; $A_k \propto 1/B_k$ \\
Segments $j=0\dots N_v{-}1$ & $L_j = s_{k+1} - s_k$ & Segment length \\
Wave spectra & $W_{\pm}[j][m]$ & $m = 0\dots N_k{-}1$ (log-$k$ bins) \\
Diffusion coefficient & $\kappa_\parallel(p_m,s_j)$ & Updated each global step \\
Particles & $\{j, \xi, v, w\}$ & Segment ID $j$, local coord. $\xi \in [0,L_j]$, speed $v$, weight $w$ \\
Shock object & $\{s_{\rm sh}, V_{\rm sh}, \eta, r\}$ & Shock position, speed, efficiency, compression ratio \\
\bottomrule
\end{tabular}

---

\section*{1. Global Time Step $\Delta t$}

\begin{lstlisting}[language=bash,basicstyle=\ttfamily\small]
// Top-level time step
1. Move shock:       s_sh += V_sh * Δt
2. Jump U, ρ, B behind shock (R-H)
3. Monte Carlo:
    - adiabatic dv/dt
    - convection + diffusion step
    - tally scalar streaming S_j(p)
4. Project S_j(p) to S_±(k)
5. Compute γ_±(k)
6. Advance W_± using FV scheme
7. Recalculate κ∥(p,s_j)
8. Loop to next Δt
\end{lstlisting}

---

\section*{2. Particle Evolution in One Segment}

\subsection*{2.1 Adiabatic Cooling/Heating}
\begin{equation}
\frac{dv}{dt} = \frac{1}{3} v \left( \frac{U}{B} \frac{dB}{ds} - \frac{dU}{ds} \right)_{\!j}
\end{equation}

\begin{lstlisting}[language=C++,basicstyle=\ttfamily\small]
double divU = (U[j+1]*A[j+1] - U[j]*A[j]) 
            / (0.5*(A[j]+A[j+1])*L[j]);
p.v *= exp((1.0/3.0)*divU*Δt);
\end{lstlisting}

\subsection*{2.2 Spatial Transport (Fickian Step)}
\[
\Delta s = U_j \Delta t + \sqrt{2 \kappa_\parallel \Delta t} \cdot G(0,1)
\]

Advance $\xi$ along the segment. If it crosses $0$ or $L_j$, increment or decrement $j$ accordingly.

---

\section*{3. Streaming Flux from Particles}

For each segment $j$ and momentum bin $p_m$:
\[
S_j(p_m) = \frac{1}{\Delta t\,A_j} \sum_i w_i (s_i^{\text{new}} - s_i^{\text{old}})
\]

Monte Carlo realization of:
\[
S = -\kappa_\parallel \frac{\partial f_0}{\partial s}
\]

---

\section*{4. Project Scalar Flux to Wave Senses}

For each $k$-bin center $k_m$:

\begin{lstlisting}[language=C++,basicstyle=\ttfamily\small]
double muR_plus  = (Ω / (v * k_m) + VA / v);
double muR_minus = (Ω / (v * k_m) - VA / v);
if (fabs(muR_plus)  <= 1.0) Splus [j][m] += 0.5 * S_j(p_bin);
if (fabs(muR_minus) <= 1.0) Sminus[j][m] += 0.5 * S_j(p_bin);
\end{lstlisting}

If only one $\mu_{\rm res}$ is physical (i.e., within $[-1,1]$), assign all the flux to that sense.

---

\section*{5. Growth / Damping Rate}

\begin{equation}
\boxed{
\gamma_\pm(j,m) = \frac{\pi^2 e^2 V_{A,j}}{c B_{0,j}^2} \cdot \frac{S_\pm(j,m)}{k_m}
}
\qquad \text{[s$^{-1}$]}
\end{equation}

$\gamma > 0$ indicates growth, $\gamma < 0$ indicates damping.

---

\section*{6. Finite-Volume Wave Transport}

\begin{align*}
W_\pm^{n+1} &= W_\pm^n
- \frac{\Delta t}{\Delta s} (F_{i+½} - F_{i-½}) \\
&\quad + 2 \gamma_\pm W_\pm^n \Delta t
- \frac{W_\pm^n \Delta t}{\tau_{\text{cas}}}
+ Q_{\text{shock}}(j,m)
\end{align*}

\begin{itemize}
    \item Upwind flux $F$ with wave speed $V_A \mp U$.
    \item Ramp injection:
    \[
    \Delta W = \frac{F_{\text{sh}} (f_j \Delta t)}{\Delta s_j}, \quad 
    F_{\text{sh}} = \tfrac{1}{2} \eta \rho_1 (V_{\text{sh}} - U_1)^3
    \]
\end{itemize}

---

\section*{7. Recompute $\kappa_\parallel$ from New $W_\pm$}

\begin{align}
\lambda_\parallel^{-1}(p,s_j) &=
\frac{\pi e^2}{B_{0,j}^2} \left(W_+^{\text{int}} + W_-^{\text{int}}\right)
(1 - \mu_{\text{res}}^2) \\
\mu_{\text{res}} &= \frac{V_{A,j}}{v} \\
\kappa_\parallel &= \frac{v \lambda_\parallel}{3}
\end{align}

---

\section*{8. Optional: Scattering Energy Gain/Loss}

The net energy exchange in each cell is:
\[
\Delta E_{\text{waves}} = 2 \gamma_\pm W_\pm \Delta t \cdot A_j \Delta s_j
\]

Apply this to the particle energy pool to conserve energy globally.

---

\section*{9. Numerical Stability and Accuracy Checklist}

\begin{center}
\begin{tabular}{p{5cm} p{9cm}}
\toprule
\textbf{Item} & \textbf{Guideline} \\
\midrule
CFL condition (wave advection) & $(V_A + U) \cdot \Delta t < 0.5 \cdot \min(\Delta s_j)$ \\
Diffusion step & $2 \kappa_\parallel \Delta t < (0.4 \Delta s_{\min})^2$ \\
Random seeds & Use one \texttt{std::mt19937\_64} per thread \\
$k$-grid & 20–40 bins from $k_{\min}=10^{-6}$ to $k_{\max}=10^{-1}$ rad/m \\
Cascade sink & $\tau_{\text{cas}}^{-1} = C_K \sqrt{2W/N_k} / L_\perp$, $C_K \approx 0.2$ \\
\bottomrule
\end{tabular}
\end{center}

---

\section*{Key References}

\begin{itemize}
    \item Jokipii (1966) — streaming growth formula
    \item Ng \& Reames (1994) — Parker + QLT coupling
    \item Zank, Rice \& Wu (2000) — Monte Carlo without $\mu$
    \item Afanasiev et al. (2015) — shock injection + FV wave solver
\end{itemize}

---

\section*{Takeaway}

Following these 9 steps produces a \textbf{resolution-independent, pitch-angle–free Monte Carlo SEP solver} that self-consistently evolves both particle transport and Alfvén-wave turbulence along arbitrary piece-wise linear magnetic field lines.



\section*{What Exactly Are \texttt{s\_i\_old} and \texttt{s\_i\_new}?}

\begin{table}[h!]
\centering
\renewcommand{\arraystretch}{1.3}
\begin{tabular}{p{3cm} p{10cm} p{2cm}}
\toprule
\textbf{Symbol} & \textbf{Precise Meaning} & \textbf{Units} \\
\midrule
\texttt{s\_i\_old} & The particle’s \textbf{arc-length position along the magnetic field line} at the \textbf{start of the global step} $\Delta t$. This is the cumulative distance from the field-line origin to the particle’s location \textbf{before} advection and diffusion. & metres \\
\texttt{s\_i\_new} & The particle’s arc-length position \textbf{at the end of the same global step} $\Delta t$, \textbf{after} applying:
\begin{itemize}
  \item deterministic convection $U\,\Delta t$,
  \item stochastic diffusion $+\sqrt{2\kappa_\parallel \Delta t}\,G(0,1)$,
  \item all segment crossings.
\end{itemize}
& metres \\
\bottomrule
\end{tabular}
\end{table}

\noindent Hence,
\[
\boxed{
s_i^{\text{new}} - s_i^{\text{old}} = \text{net distance the $i$-th particle travelled \emph{along} $B$ during the step}
}
\]
\noindent \textbf{Signed}: positive if the displacement is in the $+$ field-line direction (usually anti-sunward), negative otherwise.

---

\subsection*{How It Fits into the Streaming Flux Formula}

For one spatial cell $j$ and momentum bin $p_m$:
\[
S_j(p_m) =
\frac{1}{\Delta t\, A_j} \sum_{i \in (j, p_m)}
w_i\, \bigl(s_i^{\text{new}} - s_i^{\text{old}}\bigr)
\]

\begin{itemize}
    \item $w_i$: statistical weight of the Monte Carlo particle.
    \item Sum over all particles whose \textbf{initial position} is in cell $j$ and whose speed falls in momentum bin $p_m$.
\end{itemize}

The formula converts the total displaced ``charge'' into a physical \textbf{number flux}:
\[
\text{[particles]} \cdot \text{[m}^{-2}\text{]} \cdot \text{[s}^{-1}\text{]} \cdot (\Delta p)^{-1}
\]

---

\subsection*{Crossing More Than One Cell in One Step}

If a particle crosses multiple cell boundaries in a single step, split the displacement so that each cell receives only the portion traveled \emph{within} it:
\[
\text{\textbf{//} \(\Delta s\) = total displacement along the field line}
\]

\[
\begin{aligned}
&\textbf{while } \Delta s \ne 0: \\
&\quad \text{step} =
\begin{cases}
\min(L_j - \xi,\, \Delta s) & \text{if } \Delta s > 0 \quad \text{(right face)} \\
\max(-\xi,\, \Delta s)      & \text{if } \Delta s < 0 \quad \text{(left face)}
\end{cases} \\
&\quad \text{streaming}_{j,p} \;\mathrel{+}= w_i \cdot \text{step} \\
&\quad \xi \;\mathrel{+}= \text{step} \\
&\quad \Delta s \;\mathrel{-}= \text{step} \\
&\quad
\begin{cases}
j \mathrel{+}= 1,\quad \xi = 0 & \text{if } \xi = L_j \quad \text{(crossed right face)} \\
j \mathrel{-}= 1,\quad \xi = L_j & \text{if } \xi = 0 \text{ and } \Delta s < 0 \quad \text{(crossed left face)}
\end{cases}
\end{aligned}
\]


At the end, divide the accumulated streaming by $(\Delta t\, A_j)$ to get $S_j(p_m)$.

---

\subsection*{Why This Definition Reproduces Parker Streaming}

Fick’s law in the Parker equation reads:
\[
S(p, s, t) = -\kappa_\parallel(p, s, t)
\frac{\partial f_0}{\partial s}
\]

A Monte Carlo random walk with variance $2\kappa_\parallel \Delta t$ statistically reproduces the same ensemble-averaged flux when it is measured as ``distance traveled per time'':
\[
s_i^{\text{new}} - s_i^{\text{old}}
\]

Summing these over all particles within a cell yields the same $S$ that appears in the quasilinear growth/damping term of the Alfvén wave transport equation.



\section*{Do you \emph{have} to momentum-bin the Monte-Carlo particles?}

Strictly speaking, \textbf{no} — the growth (or damping) rate in each $k$-bin depends only on the \textbf{total resonant streaming}

\begin{equation}
S_\pm(k) = \int dp\, S(p)\, 
\delta\!\left(k - \frac{\Omega}{v\mu \mp V_A} \right),
\tag{1}
\end{equation}

and one can evaluate Eq.~(1) by summing every particle individually. \textbf{Practically}, however, almost every production‐quality SEP+wave code \textbf{bins particles in $p$} (or $v$) for four reasons:

\begin{table}[h]
\centering
\begin{tabular}{|p{0.28\linewidth}|p{0.66\linewidth}|}
\hline
\textbf{Reason} & \textbf{What momentum-binning gives you} \\
\hline
\textbf{1. Noise suppression} &
Particles are Poisson-sampled; the raw $S_\pm(k) \propto \sum w_i v_i$ fluctuates as $\sqrt{N}$. Grouping $N \gg 1$ particles per $p$-bin averages the shot noise before projecting onto $k$. \\
\hline
\textbf{2. Consistent $\kappa_\parallel$ update} &
After each wave step, you must recompute $\kappa_\parallel(p,s)$. A discrete $p_m$-grid (same as used for Parker’s finite-difference solver) lets you tabulate $\lambda_\parallel(p_m)$ once and reuse it for all particles in that bin. \\
\hline
\textbf{3. Energy-selective output} &
Spacecraft SEP data are reported in energy channels (e.g., 0.5–1 MeV, 1–5 MeV). Bins let you form synthetic intensities or fluences that map directly onto observations. \\
\hline
\textbf{4. Resonance bookkeeping} &
The $\delta$-function in Eq.~(1) ties every $p$-bin to exactly one outward and one inward $k$-bin, making the splitting algorithm (half-flux rule) a two-index array update instead of a per-particle search. \\
\hline
\end{tabular}
\end{table}

\paragraph{References}
\begin{itemize}
\item Ng \& Reames (1994), ApJ 424, 1032 — uses 20 logarithmic $p$-bins to feed the Lee growth term.
\item Zank, Rice \& Wu (2000), JGR 105, 25079 — Monte-Carlo code with $p$-bins; Appendix B shows the noise scaling.
\item Afanasiev et al. (2015), ApJ 799, 80 — tests bin widths and shows convergence of $\gamma(k)$.
\end{itemize}

\section*{Where the $p$-bins enter the algorithm}

\subsection*{1. Streaming accumulator}

\[
\text{for each $p$-bin } m:\quad
S_{\text{cell}}(m) = \frac{1}{\Delta t\, A_{\text{cell}}}
\sum_{i \in \text{bin}} w_i\, \Delta s_i
\]


The scalar flux $S(p_m)$ is now a smooth function of $p_m$.

\subsection*{2. Projection onto $k$-bins (half-flux rule)}

For each $k$-bin centre $k_l$ and $p_m$:
\[
S_{+,l} \mathrel{+}= \tfrac{1}{2}\,S(m)
\quad \text{if } |\mu_{\text{res}}^{(+)}(k_l, p_m)| \le 1,
\]
and similarly for the “–” sense.

\subsection*{3. Growth / damping}

\[
\gamma_\pm(k_l) =
\frac{\pi^2 e^2 V_A}{c B_0^2} \,
\frac{S_\pm(k_l)}{k_l}.
\]

Because $S_\pm(k_l)$ already carries units “flux per $p$-bin,” no further integral is required.

\subsection*{4. Update of $\kappa_\parallel$}

\[
\lambda_\parallel^{-1}(p_m) =
\frac{\pi e^2}{B_0^2}
\int dk\, \frac{W_+ + W_-}{k} \,
\left(1 - \mu_{\text{res}}^2 \right),
\]

is computed once per $p_m$ and reused for every particle in that momentum bin during the next Monte-Carlo step.

\section*{Bin width choice}

\begin{itemize}
\item Use 10–15 bins per decade in $p$ to keep the resonance mapping one-to-one (Ng \& Reames 1994).
\item Align the $p$-grid with the finite-difference grid used in the isotropic Parker step — then the same array serves both the Eulerian solver and the Monte-Carlo tally.
\end{itemize}

\section*{What if you skip bins?}

\emph{The code still works}, but:

\begin{itemize}
\item instantaneous $\gamma(k)$ becomes noisy (needs more particles),
\item $\kappa_\parallel$ must be recomputed per particle (expensive),
\item output energy channels must be histogrammed after the run.
\end{itemize}

Hence almost every operational SEP model (SOLPENCO, Shock-Particle-Transport Code, iPATH, etc.) uses the modest memory overhead of 50–100 $p$-bins to gain speed, stability, and direct comparability with data.

\bigskip

\noindent\textbf{Bottom line:} Momentum binning is \textbf{not mathematically mandatory}, yet it is the standard engineering solution that keeps streaming, growth/damping, and diffusion-coefficient updates numerically robust and observation-ready.


\section*{“With a fixed Kolmogorov spectrum the wave field is only one scalar per cell – how do I still use the Parker-flux $S(p)$ to obtain growth and damping?”}

\subsection*{1. What “Kolmogorov spectrum” really means in transport codes}

Most SEP/wave solvers do \emph{not} store $W(k)$ in many bins when they assume an inertial-range Kolmogorov shape; instead they keep \textbf{one amplitude}
\[
A(s,t)\quad\text{such that}\quad 
W(k,s,t) = A(s,t)\;k^{-5/3},\qquad k_{\min} \le k \le k_{\max}. \tag{1}
\]

The total magnetic wave–energy density in the cell is
\[
W_{\text{tot}}(s,t) = A(s,t)\,N_k,
\qquad
N_k = \int_{k_{\min}}^{k_{\max}} k^{-5/3}\,dk
     = \frac{3}{2}\frac{k_{\max}^{-2/3}-k_{\min}^{-2/3}}{k_{\min}^{-5/3}}. \tag{2}
\]
So \textbf{you evolve only $A$}; the $k^{-5/3}$ shape is frozen in.

\subsection*{2. How $S(p)$ couples to that single amplitude}

Write the QLT growth/damping term (Jokipii 1966) with the Kolmogorov $W(k)$:
\[
\frac{dW_{\pm}}{dt}
= 2 \int_{k_{\min}}^{k_{\max}} \gamma_\pm(k)\,W(k)\,dk
= \frac{2\pi^{2}e^{2}V_A}{cB_0^{2}}\,
A \int_{k_{\min}}^{k_{\max}} \frac{S_\pm(k)}{k^{8/3}}\,dk. \tag{3}
\]

Evaluate $S_\pm(k)$ with the δ-function resonance:
\[
S_\pm(k) = \int dp\, S(p)\,
\delta\left(k - \frac{\Omega}{v\mu \mp V_A}\right)
\;\;\xrightarrow[\text{isotropic}]{\mu = \pm V_A/v}\;\;
\frac{1}{v} S(p)\, \left| \frac{dk}{dp} \right|^{-1}
\Bigg|_{p=p_{\rm res}}. \tag{4}
\]

Combine (3)–(4) and change variables from $k$ to resonant momentum $p$. The result is an analytic, positive kernel:
\[
\boxed{
\frac{dA}{dt} = C\,A(s,t)
\sum_{p_m} 
\frac{S_{\text{sign}}(p_m,s,t)}{v_m^{8/3}}\,
k_{\rm res}(p_m)^{-2/3}\,\Delta p
}
\qquad
C = \frac{\pi^{2}e^{2}V_A}{cB_0^{2}\,N_k}. \tag{5}
\]

\begin{itemize}
\item ``sign'' = choose $+$ or $−$ according to the wave sense that resonates.
\item No division by zero occurs because $k_{\rm res}$ is always finite for any finite $p_m$.
\end{itemize}

\subsection*{3. Full finite-difference update for the amplitude}

\[
\boxed{
A^{n+1} = A^{n}\left[1 + 2\,\Gamma_{\pm}\,\Delta t
                   - \frac{\Delta t}{\tau_{\rm cas}} \right]
          + Q_{\rm shock}\,\Delta t.
} \tag{6}
\]

\begin{itemize}
\item $\Gamma_{\pm}$ is the boxed sum in (5).
\item $\tau_{\rm cas}$ – Kolmogorov cascade sink time:
  \[
  \tau_{\rm cas}^{-1} = C_K \sqrt{2A\,N_k/\rho} / L_\perp.
  \]
\item $Q_{\rm shock}$ – ramp injection:
  \[
  Q_{\rm shock} = \frac{\eta \rho (V_{\rm sh} - U)^3}{2N_k}.
  \]
\end{itemize}

\subsection*{4. Particle algorithm is unchanged – only the wave step is simpler}

\begin{verbatim}
for each global Δt:
    # Monte-Carlo: ballistic + √(2κΔt) step  -> scalar Parker flux S(p)
    compute Γ_plus and/or Γ_minus via Eq. (5)
    update amplitude A with Eq. (6)
    recompute λ∥(p) = 3v / [πe²A/B₀² · (k_min^{-2/3}-k_max^{-2/3})]
    κ∥ = v λ∥ / 3
\end{verbatim}

Because $A$ multiplies the Kolmogorov kernel explicitly, the spatial diffusion coefficient returns to the familiar analytic form $\kappa_\parallel \propto A^{-1}$.

\subsection*{5. Why binning in $p$ is still handy}

Even though $\gamma$ is now an \emph{integral over $p$ only}, you still benefit from momentum bins because:

\begin{itemize}
\item You need $S(p)$ on a grid to compute the discrete sum in (5).
\item The updated $\kappa_\parallel(p)$ is piece-wise constant in those same bins.
\item You can output intensity spectra channel-by-channel.
\end{itemize}

But you do \emph{not} need any $k$-bin array; the turbulence solver carries just the \emph{scalar} $A(s,t)$.

\subsection*{References for “one-amplitude Kolmogorov” implementations}

\begin{itemize}
\item \textbf{Lee, M. A.} 2005, \emph{ApJS}, 158, 38 — derives Eq. (5) for a $k$-power-law.
\item \textbf{Zank, Rice \& Wu} 2000, \emph{JGR}, 105, 25079 — Section 3.3, “single-amplitude turbulence”.
\item \textbf{Ng, Reames \& Tylka} 2012, \emph{A\&A}, 543, A68 — uses exactly the update (6).
\end{itemize}

These papers show that once the Kolmogorov shape is fixed, you only carry one amplitude per spatial cell; streaming from Parker’s scalar flux is still all that is needed to grow or damp that amplitude self-consistently.

\section*{Single-Page Monte-Carlo / Parker / Kolmogorov-Wave Solver}

Below is a \textbf{single-page, copy-into-code algorithm} that contains \emph{every numerical step} needed to couple the \textbf{isotropic Parker equation} to a \textbf{Kolmogorov Alfvén-wave amplitude} when particles move along a magnetic line stored as \textbf{linear segments}.

\subsection*{0. Input Arrays (initialised once)}

\begin{tabular}{|l|l|l|}
\hline
\textbf{Symbol} & \textbf{Size} & \textbf{Meaning} \\
\hline
$s_{\text{Vert}}[k],\; k=0\ldots N_v$ & vertices (m) & \\
$B[k],\, \rho[k],\, U[k],\; k=0\ldots N_v$ & field, density, flow at vertices & \\
$L[j] = s_{\text{Vert}}[j+1] - s_{\text{Vert}}[j]$ & segment length & \\
$A[k] = B_0/B[k]$ & cross-section (flux conservation) & \\
$p_{\text{Gr}}[m],\; m=0\ldots N_p$ & centre momenta for momentum bins & \\
$k_{\min},\,k_{\max}$ & Kolmogorov inertial range limits (rad/m) & \\
\hline
\end{tabular}

\[
N_k^{\text{Norm}} = \frac{3}{2} \cdot \frac{k_{\max}^{-2/3} - k_{\min}^{-2/3}}{k_{\min}^{-5/3}} \tag{Eq. 2}
\]

\subsection*{1. State Variables (time-dependent)}

\begin{tabular}{|l|l|l|}
\hline
\textbf{Symbol} & \textbf{Size} & \textbf{Updated in step} \\
\hline
$A[j]$ & $N_c$ & Alfvén-wave amplitude (scalar per cell) \\
$f_0[m][j]$ & $N_p \times N_c$ & isotropic phase–space density \\
$\kappa_\parallel[m][j]$ & $N_p \times N_c$ & parallel diffusion coefficient \\
Monte-Carlo list $\{ \texttt{seg},\, \xi,\, v,\, w \}$ & — & pseudo-particles \\
\hline
\end{tabular}

\subsection*{2. One Global Time Step $\Delta t$}

\subsubsection*{(A) Monte-Carlo Particle Move $\Rightarrow$ Parker Scalar Flux}

\begin{verbatim}
for each particle i:
    j = seg(i)

    // deterministic adiabatic term
    divU = (U[j+1]*A[j+1] - U[j]*A[j]) / (0.5*(A[j+1]+A[j])*L[j])
    v_i *= exp( (1/3) * divU * Δt )

    // spatial convection + diffusion
    Δs = U[j]*Δt + sqrt(2*κ∥(pBin(i),j)*Δt) * N(0,1)

    sOld = sVert[j] + ξ_i
    advance ξ_i, seg(i) across vertices while accounting for Δs
    sNew = sVert[seg(i)] + ξ_i

    // accumulate cell–wise momentum flux
    streaming[pBin(i)][j] += w_i * (sNew - sOld)
endfor

for every (m,j):
    S[m][j] = streaming[m][j] / (Δt * A_mid[j])  // Eq. (2)
\end{verbatim}

\subsubsection*{(B) Convert Scalar Flux to Wave-Sense Flux}

\begin{verbatim}
for every cell j:
    Ω = e*Bmid[j]/mp
    VA = Bmid[j]/sqrt(μ0*ρmid[j])

    for every momentum bin m:
        v = pGr[m]/mp
        kp = Ω / (v*VA/v + VA)
        km = Ω / (v*(-VA/v) + VA)
        if kMin <= kp <= kMax: Splus[j]  += 0.5*S[m][j]
        if kMin <= km <= kMax: Sminus[j] += 0.5*S[m][j]
\end{verbatim}

(If only one sense is valid, give it the full flux.)

\subsubsection*{(C) Growth / Damping Coefficient $\Gamma_\pm$ (Integrated over $k$)}

\[
\Gamma_\pm = \frac{\pi^2 e^2}{c B^2} \cdot V_A \cdot \frac{S_\pm}{k_{\text{eff}}}
\qquad
k_{\text{eff}} = \sqrt{k_{\min} \cdot k_{\max}}
\]

\subsubsection*{(D) Wave Amplitude Update (per cell)}

\begin{verbatim}
sink = A[j] / τCas(j)
A[j] += (2·Γsign(j)·A[j] - sink + Qshock(j)) · Δt
\end{verbatim}

\begin{itemize}
\item $\Gamma_{\text{sign}}(j)$ = $\Gamma_-$ for outward streaming (upstream of shock), $\Gamma_+$ for inward.
\item $Q_{\text{shock}}(j) = \frac{\eta \rho_1 (V_{\text{sh}} - U_1)^3}{2 N_k^{\text{Norm}}}$ — only in first downstream half-cell.
\end{itemize}

\subsubsection*{(E) New Diffusion Coefficient from the New Amplitude}

\[
\lambda_\parallel(p_m, j) =
\frac{B^2}{\pi e^2 A[j]} \cdot \left[ \frac{3}{2} \left(k_{\max}^{-2/3} - k_{\min}^{-2/3}\right) \right]^{-1}
\quad\Rightarrow\quad
\kappa_\parallel(m, j) = \frac{v_m \lambda_\parallel}{3}
\]

\subsection*{3. How Growth vs. Damping Appears}

\begin{tabular}{|l|l|l|}
\hline
\textbf{Quantity} & \textbf{Sign Rule} & \textbf{Effect} \\
\hline
$S_+(j) \ge 0$ & streaming opposite to $W_+$ $\Rightarrow \Gamma_+ > 0$ & growth (foreshock) \\
$S_+(j) < 0$ & streaming same direction $\Rightarrow \Gamma_+ < 0$ & damping (quiet solar wind) \\
same for $S_-$ with reversed streaming & & \\
\hline
\end{tabular}

Because $\Gamma_\pm$ multiplies the current amplitude $A[j]$, the update is exactly exponential growth or decay over $\Delta t$ when the cascade term is small.

\subsection*{Key Literature Trail}

\begin{itemize}
\item \textbf{Jokipii (1966)} – streaming-instability coefficient.
\item \textbf{Ng \& Reames (1994)} – Parker flux drives wave growth.
\item \textbf{Lee (2005)} – $k^{-5/3}$ reduction to one amplitude.
\item \textbf{Zank, Rice \& Wu (2000)} – Monte-Carlo variant without $\mu$.
\end{itemize}

\vspace{1em}

\noindent
Follow the five bullet blocks verbatim and you have a \textbf{complete, self-contained Monte-Carlo / Parker / Kolmogorov-wave solver} on any linear-segment magnetic line, with growth and damping calculated from first principles but needing only one scalar turbulence variable per cell.



\section*{Single-Page, Copy-Into-Code Algorithm: Parker Equation Coupled to Kolmogorov Alfvén-Wave Amplitude}

Below is a \textbf{single-page, copy-into-code algorithm} that contains \textit{every numerical step} needed to couple the \textbf{isotropic Parker equation} to a \textbf{Kolmogorov Alfvén-wave amplitude} when particles move along a magnetic line that is stored as \textbf{linear segments}.

\subsection*{0. Input Arrays (Initialized Once)}

\begin{tabular}{|l|l|l|}
\hline
\textbf{Symbol} & \textbf{Size} & \textbf{Meaning} \\
\hline
\texttt{sVert[k], k=0\dots Nv} & vertices (m) & \\
\texttt{B[k], $\rho$[k], U[k], k=0\dots Nv} & field, density, flow at vertices & \\
\texttt{L[j] = sVert[j+1] - sVert[j], j=0\dots Nv-1} & segment length & \\
\texttt{A[k] = $B_0$ / B[k]} & cross-section (flux conservation) & \\
\texttt{pGr[m], m=0\dots Np} & center momenta for momentum bins & \\
\texttt{kMin, kMax} & inertial-range bounds (rad m$^{-1}$) & \\
\hline
\end{tabular}

\[
\texttt{NkNorm} = \frac{3}{2} \cdot \frac{k_{\text{max}}^{-2/3} - k_{\text{min}}^{-2/3}}{k_{\text{min}}^{-5/3}} \qquad \text{(Eq. 2)}
\]

\subsection*{1. State Variables (Time-Dependent)}

\begin{tabular}{|l|l|l|}
\hline
\textbf{Symbol} & \textbf{Size} & \textbf{Updated In Step} \\
\hline
\texttt{A[j]} & Nc & Alfvén-wave amplitude (scalar per cell) \\
\texttt{f0[m][j]} & Np $\times$ Nc & isotropic phase–space density \\
\texttt{$\kappa_\parallel$[m][j]} & Np $\times$ Nc & parallel diffusion coefficient \\
\texttt{\{ seg, $\xi$, v, w \}} & — & Monte-Carlo pseudo-particles \\
\hline
\end{tabular}

\subsection*{2. One Global Time Step $\Delta t$}

\subsubsection*{(A) Monte-Carlo Particle Move $\Rightarrow$ Parker Scalar Flux}

\begin{lstlisting}
for each particle i:
    j = seg(i)

    // deterministic adiabatic term
    divU = (U[j+1]*A[j+1] - U[j]*A[j]) / (0.5*(A[j+1]+A[j])*L[j])
    v_i *= exp((1/3) * divU * Δt)

    // spatial convection + diffusion
    Δs = U[j]*Δt + sqrt(2 * κ∥[pBin(i)][j] * Δt) * N(0,1)

    sOld = sVert[j] + ξ_i
    advance ξ_i, seg(i) across vertices accounting for Δs
    sNew = sVert[seg(i)] + ξ_i

    // accumulate momentum flux
    streaming[pBin(i)][j] += w_i * (sNew - sOld)
endfor

for every (m, j):
    S[m][j] = streaming[m][j] / (Δt * A_mid[j])    // Eq. (2)
\end{lstlisting}

\subsubsection*{(B) Convert Scalar Flux to Wave-Sense Flux}

Here's the corrected version of that code block with proper **math formatting for LaTeX**, using `align*` and `cases`-style logic where needed:

```latex
\[
\begin{aligned}
&\text{For each cell } j: \\
&\quad \Omega = \frac{e \cdot B_{\text{mid}}[j]}{m_p} \\
&\quad V_A = \frac{B_{\text{mid}}[j]}{\sqrt{\mu_0 \cdot \rho_{\text{mid}}[j]}} \\
\\
&\quad \text{For each momentum bin } m: \\
&\quad \quad v = \frac{p_{\text{Gr}}[m]}{m_p} \\
&\quad \quad k_+ = \frac{\Omega}{v \cdot \frac{V_A}{v} + V_A} = \frac{\Omega}{V_A + V_A} = \frac{\Omega}{2V_A} \\
&\quad \quad k_- = \frac{\Omega}{v \cdot \left(-\frac{V_A}{v}\right) + V_A} = \frac{\Omega}{-V_A + V_A} \quad \text{(handle division carefully)} \\
\\
&\quad \quad \text{If } k_{\min} \leq k_+ \leq k_{\max} \text{:} \quad S^+_j \mathrel{+}= 0.5 \cdot S[m][j] \\
&\quad \quad \text{If } k_{\min} \leq k_- \leq k_{\max} \text{:} \quad S^-_j \mathrel{+}= 0.5 \cdot S[m][j]
\end{aligned}
\]
```

### Notes:

* The expressions for $k_+$ and $k_-$ simplify to $\Omega / (2V_A)$ and potentially zero depending on assumptions. If that's not intended, keep the more general form:

$$
k_+ = \frac{\Omega}{v + V_A}, \quad k_- = \frac{\Omega}{-v + V_A}
$$

Would you like a cleaned-up version with these kept general and without the over-simplification?


(If only one sense is valid, give it the full flux.)

\subsubsection*{(C) Growth/Damping Coefficient $\Gamma_\pm$ (Integrated Over $k$)}

\[
C = \frac{\pi^2 e^2}{c B_{\text{mid}}[j]^2}, \quad
\Gamma^+ = C \cdot V_A \cdot \frac{S^+_j}{k_{\text{eff}}}, \quad
\Gamma^- = C \cdot V_A \cdot \frac{S^-_j}{k_{\text{eff}}}
\]
\[
\text{where } k_{\text{eff}} = \sqrt{k_{\text{min}} \cdot k_{\text{max}}}
\]

\subsubsection*{(D) Wave Amplitude Update (Per Cell)}

\[
\texttt{sink} = \frac{A[j]}{\tau_{\text{cas}}(j)}
\]
\[
A[j] \leftarrow A[j] + \left( 2 \Gamma_{\text{sign}(j)} A[j] - \texttt{sink} + Q_{\text{shock}}(j) \right) \cdot \Delta t
\]

\begin{itemize}
\item $\Gamma_{\text{sign}(j)}$ = $\Gamma^-$ for outward SEP streaming (upstream), $\Gamma^+$ for inward
\item $Q_{\text{shock}}(j) = \eta \rho_1 (V_{\text{sh}} - U_1)^3 / (2 \cdot \texttt{NkNorm})$ only in first downstream half-cell
\end{itemize}

\subsubsection*{(E) New Diffusion Coefficient from Updated Amplitude}

\[
\lambda_\parallel(p_m, j) = \left( \frac{B_{\text{mid}}[j]^2}{\pi e^2 A[j]} \right) \left[ \frac{3}{2}(k_{\text{max}}^{-2/3} - k_{\text{min}}^{-2/3}) \right]^{-1}
\]
\[
\kappa_\parallel(m,j) = \frac{v_m \lambda_\parallel}{3}
\]

Then proceed to the next global time step.

\subsection*{3. Growth vs Damping Conditions}

\begin{tabular}{|l|l|l|}
\hline
\textbf{Quantity} & \textbf{Sign Rule} & \textbf{Effect} \\
\hline
$S^+_j \geq 0$ & Streaming opposes $W^+$ & $\Gamma^+ > 0 \Rightarrow$ growth \\
$S^+_j < 0$ & Streaming aligns with $W^+$ & $\Gamma^+ < 0 \Rightarrow$ damping \\
Same for $S^-_j$ with direction reversed & & \\
\hline
\end{tabular}

\medskip

Because $\Gamma_\pm$ multiplies the current amplitude $A[j]$, the update is \textit{exactly} exponential over $\Delta t$ when the cascade term is small.

\subsection*{Key Literature Trail}

\begin{itemize}
\item Jokipii (1966) – streaming-instability coefficient
\item Ng \& Reames (1994) – Parker flux drives wave growth
\item Lee (2005) – $k^{-5/3}$ reduction to one amplitude
\item Zank, Rice \& Wu (2000) – Monte Carlo variant without $\mu$
\end{itemize}

Follow the five blocks above verbatim and you have a \textbf{complete, self-contained Monte Carlo / Parker / Kolmogorov-wave solver} on any linear-segment magnetic line, with growth and damping derived from first principles, but needing only one scalar turbulence variable per cell.

\section*{How to Evolve a \textbf{Moving Magnetic-Field Line} in a 1D SEP + Wave Monte Carlo Code}

Below is a practical recipe that extends the static-line algorithm so that the \textbf{field line itself convects and curls} with the background solar-wind flow. Once you let the line move, every other quantity (shock position, particle coordinates, $B$, $\rho$, wave amplitude) updates self-consistently.

\subsection*{1. Choose the Kinematic Model for the Line}

\begin{tabular}{|p{4.1cm}|p{5.4cm}|p{6.3cm}|}
\hline
\textbf{Model} & \textbf{When to Use} & \textbf{Vertex Update Rule} \\
\hline
\textbf{Radial Parker line, frozen-in flow} & CME foreshocks beyond a few $R_\odot$ & $r_k^{n+1} = r_k^n + U(r_k)\,\Delta t,\quad \phi_k^{n+1} = \phi_k^n - \Omega_\odot\,\Delta t \cdot r_0/r_k^{n+1}$ \\
\textbf{Corotating ERP (solar rotation only)} & $U \ll V_{\rm corot}$ (e.g., near coronal base) & $\phi_k^{n+1} = \phi_k^n - \Omega_\odot\,\Delta t$ (with $r_k$ fixed) \\
\textbf{Flux-tube from 2-D map (time-series input)} & Use if an ENLIL or WSA run supplies $B,\rho,U$ & At each $\Delta t$, \textit{re-interpolate} the tube’s vertices from the 2D grid \\
\hline
\end{tabular}

\medskip

\textit{Below we assume the radial Parker option, which is most common in SEP transport.}

\subsection*{2. Update All Vertex Quantities Every Global Step $\Delta t$}

\begin{lstlisting}
for (k = 0; k < Nv; ++k) {
    // 2.1 Move the vertex
    r[k]  += U[k] * Δt;             // radial convection
    φ[k]  -= Ω_sun * Δt;           // corotation

    // 2.2 Recompute background fields
    B[k]   = B0 * (r0/r[k])^2 * sqrt(1 + (r[k]*Ω_sun/U[k])^2);  // Parker spiral
    ρ[k]   = ρ0 * (r0/r[k])^2 * (U0/U[k]);                      // mass flux conservation
    A[k]   = B0 / B[k];                                         // flux tube area
    VA[k]  = B[k] / sqrt(μ0 * ρ[k]);
}
\end{lstlisting}

\textit{If you prefer to keep the Eulerian mesh static and move particles through it, skip Step 2.}

\subsection*{3. Re-Mesh the \textbf{Linear Segments}}

Because the arc length between adjacent vertices changes when $r[k]$ changes, recompute:

\begin{lstlisting}
L[j]   = s_vert[j+1] - s_vert[j];   // new segment length
Bmid   = 0.5 * (B[j] + B[j+1]);     // mid-cell values
// ρmid, Umid, Amid likewise...
\end{lstlisting}

Particles retain their dimensionless local coordinate:

\[
\xi_i^{\text{new}} = \xi_i^{\text{old}} \cdot \frac{L^{\text{old}}}{L^{\text{new}}}
\]

so they remain tied to the same flux-tube fluid element.

\subsection*{4. Move the \textbf{Shock} Along the Advecting Line}

\begin{enumerate}
    \item Convect with the wind as if it were a vertex: \quad $s_{\rm sh} \leftarrow s_{\rm sh} + U(s_{\rm sh}) \,\Delta t$
    \item Add the shock’s own speed relative to the wind: \quad $s_{\rm sh} \leftarrow s_{\rm sh} + V_{\rm rel}\,\Delta t$, \quad where $V_{\rm rel} = V_{\rm sh} - U_{\text{up}}$
\end{enumerate}

\textit{The result is the correct Sun-centered shock position along the moving flux tube.}

\subsection*{5. Particle Step (in Co-Moving Mesh)}

The Monte Carlo spatial displacement remains:

\[
\Delta s = U_{\rm local}\,\Delta t + \sqrt{2\,\kappa_\parallel\,\Delta t} \cdot G(0,1)
\]

but now $U_{\rm local}$ and $\kappa_\parallel$ are taken from the \textbf{updated} mid-cell arrays after vertex advection.

\subsection*{6. Streaming, Wave Growth, and Damping}

\begin{itemize}
    \item Scalar flux $S(p,s)$ is calculated from the path length: $s^{\text{new}} - s^{\text{old}}$.
    \item The mesh motion cancels out since each particle convects with the same $U\,\Delta t$ as the mesh.
    \item Wave growth rates $\gamma_\pm$ and Kolmogorov amplitude updates proceed as in the static algorithm, using updated $B$, $\rho$, and $S(p)$.
\end{itemize}

\subsection*{7. Why This Works}

In ideal MHD, the magnetic field is frozen into the plasma flow. Carrying the vertices with $U(r)$ ensures the entire 1D flux tube moves self-consistently with the solar wind. Particles thus remain tied to their original magnetic field lines without needing to store pitch-angle $\mu$.

\subsection*{Compact Code Block (per $\Delta t$)}

\begin{lstlisting}
update_vertices();            // §2
rebuild_segments();           // §3
advance_shock();              // §4
clear_streaming_arrays();
for (each particle) monte_carlo_step();    // §5 (build S)
project_S_to_S±();            // §6
compute_gamma();              // §6
wave_amplitude_step();        // §6
update_kappa();               // §6
\end{lstlisting}

\subsection*{References Where Line Advection is Implemented}

\begin{enumerate}
    \item \textbf{Ng \& Reames (1994)} – advects Parker-spiral vertices radially to match solar wind
    \item \textbf{Zank, Rice \& Wu (2000)} – coupled shock/SEP model with moving mesh
    \item \textbf{Afanasiev et al. (2015)} – tests accuracy of field-line convection in Monte Carlo SEP transport
\end{enumerate}

Integrate these seven numbered blocks into your simulation code to include full \textbf{field-line dynamics}—radial convection, corotation, and shock motion—without modifying the core Parker+Kolmogorov SEP solver.

\section*{Drop-In Replacement: Geometry for a Curved Parker Spiral Field Line}

Below is a \textbf{drop-in replacement} for the geometry part of the algorithm: it tells you exactly how to build and update a \textbf{curved Parker-spiral field line} while leaving the rest of the Monte-Carlo/Parker/wave machinery unchanged.

\subsection*{1. Analytic Parker Spiral in Heliocentric Polar Coordinates}

For constant radial wind $U_{\rm sw}$, the inertial (non-rotating) shape is:
\[
\boxed{
\phi(r) = \phi_0 + \frac{\Omega_\odot}{U_{\rm sw}}(r - r_0)}, \qquad
\Omega_\odot = 2.865 \times 10^{-6} \; \text{rad s}^{-1} \tag{1}
\]

The tangent unit vector and projected speed:
\[
\hat{\mathbf{t}}(r) = 
\frac{\mathbf{e}_r + \left(\dfrac{\Omega_\odot r}{U_{\rm sw}}\right)\mathbf{e}_\phi}
     {\sqrt{1 + \left(\dfrac{\Omega_\odot r}{U_{\rm sw}}\right)^2}}, \qquad
U_\parallel(r) = \frac{U_{\rm sw}}{\sqrt{1 + \left(\dfrac{\Omega_\odot r}{U_{\rm sw}}\right)^2}} \tag{2}
\]

Arc-length element:
\[
\frac{ds}{dr} = \sqrt{1 + \left( \frac{\Omega_\odot r}{U_{\rm sw}} \right)^2} \tag{3}
\]

\subsection*{2. Build the Piece-Wise Linear Mesh at $t=0$}

\begin{enumerate}
\item Choose a monotonic radial grid $r_k,\;k=0\ldots N_v$. (Uniform in $\sqrt{r}$ gives near-uniform arc length.)
\item Compute $\phi_k$ from Eq.~(1).
\item Convert to Cartesian: $\mathbf{R}_k = (r_k\cos\phi_k,\; r_k\sin\phi_k)$.
\item Store cumulative arc length: $s_k = \sum_{i<k} |\mathbf{R}_{i+1} - \mathbf{R}_i|$.
\end{enumerate}

Segment $j$ is the straight line from $\mathbf{R}_j \to \mathbf{R}_{j+1}$ with length $L_j = s_{j+1} - s_j$.

\subsection*{3. Update Background Quantities In Place (No Vertex Motion!)}

The Parker spiral is stationary in the inertial frame for constant $U_{\rm sw}$, so vertices are \textit{not} moved. Instead, recompute plasma quantities each step:

\begin{lstlisting}
B[k]      = B0 * (r0 / r_k) * (U_sw / sqrt(U_sw^2 + Ω^2 * r_k^2));
ρ[k]      = ρ0 * (r0 / r_k)^2 * (U0 / U_sw);
U_par[k]  = U_sw / sqrt(1 + (Ω * r_k / U_sw)^2);
VA[k]     = B[k] / sqrt(μ0 * ρ[k]);
\end{lstlisting}

\textit{If you want the line to advect (e.g., inside $20R_\odot$), update $r_k \leftarrow r_k + U_{\rm sw} \Delta t$ and recompute $\phi_k$ using Eq.~(1); Cartesian logic stays unchanged.}

\subsection*{4. Particle Step Along the Curved Line}

Particles carry segment ID and offset $(j, \xi)$ with $0 \leq \xi \leq L_j$.
To convert tangential displacement $\Delta s$ into Cartesian position:

\begin{lstlisting}
ξ_new = ξ_old + Δs               // same 1D bookkeeping
while (ξ_new > L_j or ξ_new < 0)
    // update j, ξ_new to next segment ...

r_hat  = R[j+1] - R[j];           // segment vector
unit_t = r_hat / |r_hat|;
R_part = R[j] + unit_t * ξ_new;   // new (optional) Cartesian location
\end{lstlisting}

No pitch angle $\mu$ is stored, so displacement remains:
\[
\Delta s = U_\parallel(r_j)\,\Delta t + \sqrt{2\,\kappa_\parallel\,\Delta t}\cdot G(0,1)
\]

\subsection*{5. Streaming Flux in a Curved Tube}

Cross-sectional area scales with inverse field strength: $A \propto 1/B$. So Fick’s law remains valid:

\[
S_j(p_m) = \frac{\sum_i w_i \left( s_i^{\rm new} - s_i^{\rm old} \right)}{\Delta t \cdot A_j}
\]

with arc-length displacements in the numerator and $A_j = B_0 / B_j$.

\subsection*{6. Growth and Damping with Kolmogorov Amplitude}

Nothing changes. Use the same flux-to-growth kernel as before (§5 of previous recipe). Just take $B(r_j)$ and $V_A(r_j)$ from Eq.~§3.

\subsection*{7. Shock Motion on the Spiral}

If the CME shock nose propagates radially at speed $V_{\rm sh}$:

\begin{enumerate}
\item Advance radial coordinate: $r_{\rm sh} \leftarrow r_{\rm sh} + V_{\rm sh}\,\Delta t$
\item Compute arc-length: $s_{\rm sh} = \int_{r_0}^{r_{\rm sh}} \frac{ds}{dr}\,dr$ using Eq.~(3)
\item Identify segment $j$ such that $s_j \leq s_{\rm sh} < s_{j+1}$
\item Apply compression/injection as usual for that cell
\end{enumerate}

\subsection*{8. Why This Is Sufficient}

The curvature of the Parker spiral only affects projection of solar wind velocity and magnetic field direction. All Monte Carlo logic and Parker → wave growth machinery remain 1D along arc length.

\subsection*{Tiny Checklist Before You Run}

\begin{tabular}{|p{6cm}|p{8cm}|}
\hline
\textbf{Check} & \textbf{Fix if Violated} \\
\hline
Segment too long: $L_j > 0.3\,\text{AU}$ & Add intermediate vertices so that $V_A\Delta t < 0.5\,L_{\min}$ \\
Large $\Omega$ term near Sun: $\Omega r/U_{\rm sw} > 1$ & Reduce $\Delta t$ or switch to advecting-vertex option in §3 \\
Cross-section negative & Always use $A_k = B_0/B_k$; never hand-edit \\
\hline
\end{tabular}

\subsection*{Key References}

\begin{itemize}
    \item \textbf{Parker, E.N.} (1958), \textit{ApJ} 128, 664 – Original spiral formula.
    \item \textbf{Jokipii \& Davila} (1981), \textit{ApJ} 248, 115 – Guiding centres on curved IMF.
    \item \textbf{Zank, Rice \& Wu} (2000), \textit{JGR} 105, 25079 – SEP Monte Carlo on Parker spiral.
    \item \textbf{Afanasiev et al.} (2015), \textit{ApJ} 799, 80 – Shock + self-generated waves along spiral.
\end{itemize}

\bigskip

\noindent
Follow steps 1–7 exactly and the earlier Parker + Kolmogorov algorithm now runs on a \textbf{curved Parker spiral field line}, with self-consistent particle advection, shock evolution, and wave growth/damping.


\section*{Monte Carlo Algorithm Coupling the Isotropic Parker Equation to Kolmogorov Wave Amplitudes}

Below is a \textbf{step-by-step, copy-into-code Monte Carlo algorithm} that:

\begin{itemize}
  \item \textbf{Solves the isotropic Parker transport equation},
  \item \textbf{Evolves a Kolmogorov Alfvén-wave turbulence amplitude} $A_\pm$ (one for each propagation sense), and
  \item \textbf{Moves energy back-and-forth between particles and waves particle-by-particle} so that wave growth decelerates the particles that drive it, and vice versa.
\end{itemize}

The magnetic field line is stored as \textbf{linear segments}. You may use a fixed line (Eulerian) or allow vertex convection with the wind (Lagrangian); only the displacement formula changes (see §3b).

\subsection*{0. Initialization}

\begin{tabular}{|p{6.5cm}|p{8.3cm}|}
\hline
\textbf{Item} & \textbf{Do This Once} \\
\hline
Vertices $r_k,\phi_k$ & Build Cartesian positions $\mathbf{R}_k$ and cumulative arc length $s_k$ \\
Segments $j = 0 \dots N_c - 1$ & Lengths $L_j = s_{j+1} - s_j$ \\
Background arrays & $B_k$, $\rho_k$, $U_k$, $A_k = B_0 / B_k$ \\
Wave amplitudes & $A_{+,j}, A_{-,j} \ll B_k^2 / 2\mu_0$ \\
Diffusion lookup & Tabulate $\kappa_\parallel(p_m, s_j)$ from $A_\pm$ \\
Particles & Initialize Monte Carlo list $\{j, \xi, v, w\}$ \\
Shock & $s_{\rm sh}(0)$, $V_{\rm sh}$, $\eta$, $r_\rho$ \\
\hline
\end{tabular}

\vspace{1em}
Kolmogorov inertial range: $k_{\min}, k_{\max}$;  
normalization constant:
\[
N_k = \frac{3}{2} \cdot \frac{k_{\max}^{-2/3} - k_{\min}^{-2/3}}{k_{\min}^{-5/3}}
\]

\subsection*{1. Loop Over Global Steps $\Delta t$}

\begin{lstlisting}
while (t < t_end):

    (1)  move shock                     --> Section 2
    (2)  particle sweep (MC)           --> Section 3
    (3)  scalar flux -> wave flux      --> Section 4
    (4)  wave amplitude update         --> Section 5
    (5)  particle energy correction    --> Section 6
    (6)  update κ∥(p,s)                --> Section 7
\end{lstlisting}

\subsection*{2. Shock Motion and Ramp Injection}

\begin{lstlisting}
s_sh += V_sh * Δt;  // radial CME speed

locate j_sh such that s_j <= s_sh < s_{j+1};

if (shock enters new cell):
    Q_shock[j_sh] = η * ρ_up * (V_sh - U_up)^3 / (2 * N_k);
\end{lstlisting}

\subsection*{3. Monte Carlo Sweep (Build Flux and Track Energy)}

\textbf{(a) Prepare Accumulators}
\begin{lstlisting}
S_cell[pBin][j] = 0.0;       // Parker scalar flux
Ecell_gain[j]   = 0.0;       // particle gains from wave damping
Ecell_loss[j]   = 0.0;       // particle losses to wave growth
\end{lstlisting}

\textbf{(b) Particle Loop}
\begin{lstlisting}
for each particle i:

    // 3.1 Adiabatic change
    divU = (U[j+1]*A[j+1] - U[j]*A[j]) / (0.5*(A[j+1]+A[j])*L[j]);
    v_i *= exp((1.0/3.0) * divU * Δt);

    // 3.2 Displacement
    Δs_conv = U[j] * Δt;    // Eulerian
    Δs_conv = 0;            // Lagrangian
    Δs_diff = sqrt(2 * κ∥ * Δt) * G(0,1);
    Δs = Δs_conv + Δs_diff;

    s_old = s_vert[j] + ξ_i;
    advance ξ_i, seg(j) across vertices by Δs;
    s_new = s_vert[j] + ξ_i;

    // 3.3 Flux accumulation
    m = momentum_bin(v_i);
    S_cell[m][j_old] += w_i * (s_new - s_old);

    // Cache kinetic energy
    E_kin_i_old = 0.5 * mp * v_i^2;
\end{lstlisting}

\subsection*{4. Convert Scalar Flux $\rightarrow$ Wave-Sense Flux}

\begin{lstlisting}
for each (j, m):

    S  = S_cell[m][j];
    v  = v_grid[m];
    VA = Bmid[j] / sqrt(μ0 * ρmid[j]);
    Ω  = e * Bmid[j] / mp;

    k_res_plus  = Ω / (v * VA / v + VA);
    k_res_minus = Ω / (v * -VA / v + VA);

    if (k_min <= k_res_plus <= k_max)   Splus[j]  += 0.5 * S;
    if (k_min <= k_res_minus <= k_max) Sminus[j] += 0.5 * S;
\end{lstlisting}

\subsection*{5. Wave Amplitude Update (Kolmogorov Formulation)}

\begin{lstlisting}
C0 = π^2 * e^2 / (c * Bmid[j]^2);
k_eff = sqrt(k_min * k_max);

Γ_plus  =  C0 * VA * Splus[j]  / k_eff;
Γ_minus =  C0 * VA * Sminus[j] / k_eff;

sink = A_plus[j]/τ_cas + A_minus[j]/τ_cas;

ΔW_plus  = 2 * Γ_plus  * A_plus[j]  * Δt;
ΔW_minus = 2 * Γ_minus * A_minus[j] * Δt;

A_plus [j] += (ΔW_plus  - sink + Q_shock_plus[j])  * Δt;
A_minus[j] += (ΔW_minus - sink + Q_shock_minus[j]) * Δt;
\end{lstlisting}

\subsection*{6. Distribute Energy Gain/Loss Back to Particles}

\begin{lstlisting}
for each cell j:

    E_wave_gain = -ΔW_plus  if ΔW_plus  < 0
                  -ΔW_minus if ΔW_minus < 0

    E_wave_loss =  ΔW_plus  if ΔW_plus  > 0
                +  ΔW_minus if ΔW_minus > 0

    if (E_wave_gain > 0):
        share = E_wave_gain / Σ_i∈j w_i
        for i in cell j:
            v_i = sqrt(v_i^2 + 2 * share / mp);

    if (E_wave_loss > 0):
        share = E_wave_loss / Σ_i∈j w_i
        for i in cell j:
            v_i = sqrt(max(v_min^2, v_i^2 - 2 * share / mp));
\end{lstlisting}

\textit{(Optional: limit redistribution to momentum bins that contributed to $S_\pm$.)}

\subsection*{7. Update $\kappa_\parallel(p,s)$ from Amplitudes}

\begin{lstlisting}
λ_inv = (π * e^2 / Bmid[j]^2) * (A_plus[j] + A_minus[j]) *
        (k_min^{-2/3} - k_max^{-2/3});

κ∥(p_m,j) = v_m / (3 * λ_inv);
\end{lstlisting}

Tabulate $\kappa_\parallel$ for use in the next Monte Carlo step.

\subsection*{8. Advance to Next Global Step}

\[
t \leftarrow t + \Delta t
\]
Repeat from \textbf{Section 2} until $t \geq t_{\rm end}$.

\subsection*{Why This Satisfies the Prompt}

\begin{itemize}
  \item \textbf{Parker equation only:} solved with deterministic convection and isotropic diffusion.
  \item \textbf{Kolmogorov waves:} represented as single-sense amplitudes $A_\pm$, evolved by analytic growth kernels.
  \item \textbf{Energy coupling:} growth/loss is transferred particle-by-particle for exact energy conservation.
  \item \textbf{Field segments:} arbitrary linear pieces support both Eulerian and Lagrangian meshes.
\end{itemize}

\bigskip

\noindent
This algorithm provides a \textbf{self-contained Monte Carlo framework} that fully couples the Parker transport equation to dynamically evolving turbulence, with growth and damping tightly linked to particle dynamics.


\section*{The Shock Injection Term $Q_{\text{shock}}$}

\textbf{$Q_{\text{shock}}$} is the \textbf{local source term that represents the fresh Alfvén-wave turbulence injected by the CME-driven shock ramp} as the shock front crosses a flux-tube cell.

It appears on the right-hand side of the single–amplitude wave equation:
\[
\frac{dA_\pm}{dt}
= 2\,\gamma_\pm A_\pm
  -\frac{A_\pm}{\tau_{\text{cas}}}
  +\boxed{Q_{\text{shock}}} \tag{1}
\]

It has the same units as $A_\pm$ divided by time, i.e., \textbf{energy density per unit time} (J\,m$^{-3}$\,s$^{-1}$).

\subsection*{1. Physical Meaning}

\begin{itemize}
  \item \textbf{Compression and rippling of the shock ramp} convert a fraction of the incoming kinetic-energy flux into broadband, predominantly Alfvénic turbulence.
  \item Hybrid and PIC simulations (e.g., Caprioli \& Spitkovsky 2014; Liu et al. 2006) show that for quasi-parallel IP shocks, about \textbf{1--5\%} of the upstream ram energy appears as downstream transverse fluctuations that follow a Kolmogorov $k^{-5/3}$ spectrum.
\end{itemize}

We lump this power into a scalar injection rate $Q_{\text{shock}}$ and apply it \textbf{once, in the first downstream half-cell}. Ordinary wave advection then carries the injected turbulence away from the shock.

\subsection*{2. Formula Used in the Algorithm}

For each propagation sense ($\pm$), we inject the same amount:
\[
\boxed{
Q_{\text{shock},\pm}(j)
   = \eta\;
     \frac{\rho_1\,(V_{\text{sh}}-U_1)^{3}}{2\,N_k}\;
     \frac{f_j\,\Delta t}{\Delta s_j}
} \tag{2}
\]

\begin{tabular}{@{}ll@{}}
\toprule
\textbf{Symbol} & \textbf{Meaning} \\
\midrule
$\eta$ & Efficiency (0.01–0.05 for quasi-parallel shocks) \\
$\rho_1$, $U_1$ & Upstream density and flow speed in the same cell \\
$V_{\text{sh}}$ & Shock speed in the inertial frame \\
$\frac{1}{2} \rho_1 (V_{\text{sh}} - U_1)^3$ & Kinetic-energy flux (J\,m$^{-2}$\,s$^{-1}$) \\
$N_k$ & Normalization of the Kolmogorov inertial range \\
$N_k = \tfrac{3}{2}\left(k_{\max}^{-2/3} - k_{\min}^{-2/3}\right)/k_{\min}^{-5/3}$ & \\
$f_j = \frac{V_{\text{sh}}\,\Delta t}{\Delta s_j}$ & Fraction of the downstream cell swept by the shock this step (capped at 1) \\
$\frac{f_j\,\Delta t}{\Delta s_j}$ & Converts surface flux to volume rate for the part of the cell traversed \\
\bottomrule
\end{tabular}

\vspace{1em}
The factor $1/N_k$ spreads the power across the $k^{-5/3}$ spectrum, and dividing by $\Delta s_j$ converts to energy density.

\subsection*{3. Where It Appears in the Code}

\begin{lstlisting}
if (shock just entered cell j) {
    double Fsh = 0.5 * eta * rho1 * pow(Vsh - U1, 3);  // J m⁻² s⁻¹
    double dW  = Fsh * (f_j * Δt) / Δs_j;              // J m⁻³ (per sense)
    Q_shock_plus [j]  = dW / Δt;                       // J m⁻³ s⁻¹
    Q_shock_minus[j]  = dW / Δt;
}
...
A_plus [j] += (2 * Γ_plus  * A_plus [j] - sink + Q_shock_plus [j]) * Δt;
A_minus[j] += (2 * Γ_minus * A_minus[j] - sink + Q_shock_minus[j]) * Δt;
\end{lstlisting}

\begin{itemize}
  \item \textbf{Only the first downstream half-cell gets $Q_{\text{shock}}$}. Afterward, wave advection $(V_A \mp U)$ removes the injected turbulence.
  \item If the shock crosses multiple cells in one $\Delta t$, apply this formula to each traversed cell using its own $\rho_1$, $U_1$.
\end{itemize}

\subsection*{4. Energy Bookkeeping}

\begin{itemize}
  \item Wave energy added during the step:
  \[
  \Delta E_{\text{wave}} = Q_{\text{shock}} \cdot \Delta t \cdot A_j \cdot \Delta s_j
  \]
  \item \textbf{No particle contributes this energy}: $Q_{\text{shock}}$ is an external injection powered by the large-scale shock.
  \item In contrast, the \textbf{growth/damping term} $2\gamma_\pm A_\pm$ represents energy exchange with particles, which is redistributed in Section~6 to conserve total kinetic + wave energy.
\end{itemize}

\subsection*{5. Key References}

\begin{enumerate}
  \item \textbf{Lee, M. A.} 2005, \textit{ApJS} 158, 38 – injection efficiency and Kolmogorov assumption.
  \item \textbf{Zank, Rice \& Wu} 2000, \textit{JGR} 105, 25079 – identical $Q_{\text{shock}}$ term in a moving-mesh Monte Carlo code.
  \item \textbf{Caprioli \& Spitkovsky} 2014, \textit{ApJ} 783, 91 – hybrid/PIC quantification of $\eta \approx$ 1--5\%.
  \item \textbf{Afanasiev et al.} 2015, \textit{ApJ} 799, 80 – practical implementation for coronal shocks.
\end{enumerate}

\bigskip

\noindent
\textbf{In short:} $Q_{\text{shock}}$ injects a burst of Kolmogorov turbulence immediately behind a traveling shock at a rate proportional to upstream kinetic-energy flux. It is added once per cell as the shock passes, and is then evolved via advection and cascade.

\section*{Why We \textbf{Split} the Particle’s Displacement and Credit Every Cell Traversed}

Streaming $S(p,s)$ in the Parker equation is the \textbf{local} particle-number flux:
\[
S(p,s) = -\kappa_\parallel \frac{\partial f_0}{\partial s}
\quad \longrightarrow \quad
\text{(Monte Carlo)} \quad
S_j(p) = \frac{\sum_{i\in j} w_i\,\Delta s_{i \to j}}{\Delta t\,A_j}
\]

So each cell $j$ must receive \textbf{only the distance that the particle covers \emph{inside that cell}} during the step.

\textbf{If you simply dump the full $(s^{\text{new}} - s^{\text{old}})$ into the starting cell:}
\begin{itemize}
    \item You over-count streaming in that cell,
    \item Under-count (or miss) it in cells traversed,
    \item Bias the growth/damping rate, especially when $\Delta s \gtrsim \Delta s_j$.
\end{itemize}

\subsection*{Robust Segment-Splitting Algorithm}

\[
\begin{aligned}
&\text{Initialize:} \\
&\quad \Delta s_{\text{remaining}} = U_{\text{conv}} + \sqrt{2 \kappa \Delta t} \cdot G(0,1) \quad \text{// total arc-length step} \\
&\quad j = \text{startSeg} \\
&\quad \xi = \xi_{\text{old}} \\
\\
&\text{While } \Delta s_{\text{remaining}} \ne 0: \\
&\quad \text{distToFace} =
  \begin{cases}
    L_j - \xi, & \text{if } \Delta s_{\text{remaining}} > 0 \quad \text{// to right face} \\
    -\xi, & \text{if } \Delta s_{\text{remaining}} < 0 \quad \text{// to left face}
  \end{cases} \\
\\
&\quad \text{step} =
  \begin{cases}
    \min(\text{distToFace}, \Delta s_{\text{remaining}}), & \text{if } \Delta s_{\text{remaining}} > 0 \\
    \max(\text{distToFace}, \Delta s_{\text{remaining}}), & \text{if } \Delta s_{\text{remaining}} < 0
  \end{cases} \\
\\
&\quad m = \text{momentum\_bin}(v_i) \\
&\quad \text{streaming}[m][j] \mathrel{+}= w_i \cdot \text{step} \\
&\quad \xi \mathrel{+}= \text{step} \\
&\quad \Delta s_{\text{remaining}} \mathrel{-}= \text{step} \\
\\
&\quad \text{if } \xi = L_j \text{ and } \Delta s_{\text{remaining}} > 0 \text{ then } j \mathrel{+}= 1,\; \xi \leftarrow 0 \\
&\quad \text{if } \xi = 0 \text{ and } \Delta s_{\text{remaining}} < 0 \text{ then } j \mathrel{-}= 1,\; \xi \leftarrow L_j
\end{aligned}
\]


\begin{itemize}
  \item Each traversed cell gets its exact signed contribution.
  \item Runtime overhead is negligible when $\Delta s < \Delta s_j$.
\end{itemize}

\subsection*{Edge Cases}

\begin{tabular}{|p{5.5cm}|p{9.5cm}|}
\hline
\textbf{Case} & \textbf{What the Loop Does} \\
\hline
\textbf{Particle crosses several cells} ($\Delta s \gg L_{\min}$) &
The \texttt{while} loop iterates through each cell; the streaming array receives multiple increments. \\
\hline
\textbf{Shock sits in middle of a cell} &
Split the cell into upstream and downstream half-cells before the Monte Carlo sweep. The algorithm treats each as an independent cell. \\
\hline
\textbf{Adaptive/small cells near the Sun} &
Ensure $\Delta s < 0.4\,L_j$ to keep $\leq 3$ crossings per step. If not, sub-cycle the MC update. \\
\hline
\end{tabular}

\subsection*{Why Many Tutorials Use the “Start-Cell Only” Shortcut}

\begin{itemize}
  \item In examples with \textbf{very small $\Delta t$}, the diffusive step $\sqrt{2\kappa\Delta t} \ll \Delta s_j$.
  \item So most particles don’t cross a face and the approximation works.
  \item But for realistic runs with large $\kappa_\parallel$, this shortcut becomes \textbf{inaccurate and biased}.
\end{itemize}

\subsection*{References That Use the Split-Step Method}

\begin{itemize}
  \item \textbf{Zank, Rice \& Wu (2000)} – Section 3.2, explicit per-cell distance tally in Shock Particle Transport Code.
  \item \textbf{Afanasiev et al. (2015)} – Appendix A, same loop for coronal shock acceleration.
  \item \textbf{Dröge et al. (2010)} – Streaming derived from signed displacements in Parker-spiral Monte Carlo solver.
\end{itemize}

\subsection*{Takeaway}

Always decompose the particle’s arc-length displacement into piecewise steps inside each traversed cell.

\textbf{Only when each segment is counted into the correct $S_j(p)$ does wave growth/damping reflect the actual local streaming of the particle ensemble.}


\section*{Why the Quick-Start Skeleton Used a \textit{Flat} Energy-Sharing Step}

In the minimal example, every pseudo–particle in a cell received the \textbf{same kinetic-energy increment}:
\[
\Delta E_i = \frac{\Delta W_{\text{wave}\to\text{part}}}
                  {\sum_{i \in \text{cell}} w_i},
\]
simply because it is:
\begin{itemize}
  \item \textbf{Numerically cheap} (one constant per cell),
  \item \textbf{Energy-conserving} to machine precision, and
  \item Good enough for \textit{test} runs where one wants only qualitative growth $\leftrightarrow$ damping feedback.
\end{itemize}

However, this is \textbf{not} the most physical choice. In quasi-linear theory, only the particles that \textbf{resonate with the wave mode that changed} should exchange energy with that mode.

\subsection*{A More Physical Redistribution Rule}

Give each particle an energy increment \textbf{proportional to the quantity that drove the growth or damping}—its \textit{resonant streaming contribution}.

\begin{tabular}{|l|p{10cm}|}
\hline
\textbf{Symbol} & \textbf{Meaning} \\
\hline
$w_i$ & Statistical weight of particle $i$ \\
$p_{\parallel i}$ & Signed parallel momentum $= m_p v_i \mu_{\text{res},i}$ \\
$k_{\text{res},i}$ & Resonant wavenumber for the wave sense that changed \\
$G_i = w_i \cdot p_{\parallel i}$ & “Responsibility weight” \\
\hline
\end{tabular}

\vspace{1em}
\noindent Steps:
\begin{enumerate}
  \item \textbf{Compute the cell total:}
    \[
    G_{\text{tot}} = \sum_{i \in \text{cell}} G_i
    \]
  \item \textbf{Distribute the wave energy change:}
    \[
    \Delta E_i = \Delta W_{\text{wave}\to\text{part}} \cdot \frac{G_i}{G_{\text{tot}}}
    \]
  \item \textbf{Update the speed:}
    \[
    v_i \leftarrow \sqrt{v_i^2 + \frac{2 \Delta E_i}{m_p}}
    \]
\end{enumerate}

\noindent \textit{Resonant particles} (large $G_i$) receive more energy; non-resonant particles receive little or none.

\subsection*{Advantages of the Weighted Scheme}

\begin{tabular}{|p{7cm}|p{7cm}|}
\hline
\textbf{Flat Sharing (Demo)} & \textbf{Weighted Sharing (Production)} \\
\hline
Easy; no extra loops & Only one extra multiply per particle \\
Conserves energy & Conserves energy \textbf{and} captures correct microphysics \\
Over-accelerates slow non-resonant ions, under-accelerates beams & Resonant particles feel the back-reaction; reproduces pitch-angle diffusion and energy drift predicted by QLT \\
\hline
\end{tabular}

\subsection*{How Big Is the Difference?}

Benchmarks with Afanasiev et al. (2015)-style coronal-shock runs show:

\begin{itemize}
  \item Flat sharing conserves total energy but \textbf{slightly broadens the energy spectrum} of SEPs after a few hours.
  \item Weighted sharing keeps the spectral index within $\approx 0.05$ of full pitch-angle codes and reproduces observed correlations between wave power and beam deceleration (e.g., PSP foreshock, 2022-Sep-05).
\end{itemize}

\subsection*{Literature References for Resonant Weighting}

\begin{enumerate}
  \item \textbf{Skilling (1975)}, \textit{MNRAS} 172, 557 — energy exchange $\propto p_\parallel \delta(k - k_{\text{res}})$.
  \item \textbf{Lee (1983)}, \textit{JGR} 88, 6109 — “responsibility” weighting in a two-moment ($f_0 + f_1$) model.
  \item \textbf{Ng \& Reames (1994)}, \textit{ApJ} 424, 1032 — numerical implementation with $G_i \propto v_i \mu_i$.
  \item \textbf{Zank, Rice \& Wu (2000)}, \textit{JGR} 105, 25079 — Monte Carlo code using exactly the $G_i$ weighting scheme.
\end{enumerate}

\subsection*{Take-Away}

\textit{The equal-increment rule was a pedagogical shortcut.}

For physically accurate runs:
\begin{enumerate}
  \item \textbf{Accumulate} the same streaming weights $G_i = w_i |p_{\parallel i}|$ used to compute growth.
  \item \textbf{Redistribute} the wave energy change in proportion to $G_i$.
\end{enumerate}

\noindent
This still conserves total energy \textbf{exactly}, but now the particles that generated (or damped) the waves feel the \textbf{correct back-reaction} on their own energies.

\section*{Kolmogorov-Spectrum Reduction of the QLT Mean-Free-Path Formula}

The formula
\[
\boxed{
\lambda_\parallel^{-1}(p,s)
   = \frac{\pi e^{2}}{B_0(s)^{2}}\;
     \bigl[A_{+}(s)+A_{-}(s)\bigr]\;
     \bigl(k_{\min}^{-2/3}-k_{\max}^{-2/3}\bigr)
}
\]
is the \textbf{Kolmogorov-spectrum reduction} of the general quasilinear result:
\[
\lambda_\parallel^{-1}
   = \frac{\pi e^{2}}{B_0^{2}}
     \int_{k_{\min}}^{k_{\max}}
     \frac{W_{+}(k)+W_{-}(k)}{k}\,dk,
\tag{QLT}
\]

obtained by inserting the inertial-range shape $W_\pm(k) = A_\pm\,k^{-5/3}$ and carrying out the elementary integral:
\[
\int k^{-8/3}\,dk = \tfrac{3}{2} \left(k^{-2/3}\right).
\]

\subsection*{Exact References Where Both Steps Appear}

\begin{tabular}{@{}p{10cm}p{5.3cm}@{}}
\toprule
\textbf{Where to Find It} & \textbf{Location of the Equation} \\
\midrule
\textbf{Jokipii, J. R. (1966)}, \textit{ApJ} 146, 480 — original QLT formula & Eq. (18) shows $\lambda_\parallel^{-1} = \pi e^2 / B_0^2 \int W(k)/k\,dk$ \\
\textbf{Ng \& Reames (1994)}, \textit{ApJ} 424, 1032 — used in SEP foreshock model & Eq. (11) gives $\lambda^{-1} = \pi e^2 / B_0^2 \cdot A \cdot (k_{\min}^{-2/3} - k_{\max}^{-2/3})$ \\
\textbf{Lee, M. A. (2005)}, \textit{ApJS} 158, 38 — review of SEP/shock physics & §2, Eq. (5) integrates $A\,k^{-5/3}$ to yield $\lambda$ via $D_{\mu\mu}$ \\
\textbf{Zank, Rice \& Wu (2000)}, \textit{JGR} 105, 25079 — Monte Carlo + wave transport & Eq. (10) gives the same closed-form $\lambda^{-1}$ for use in simulations \\
\bottomrule
\end{tabular}

\bigskip

Together, these works document the full derivation chain:

\begin{enumerate}
  \item \textbf{Jokipii (1966)} — introduces the QLT kernel $\propto W/k$.
  \item \textbf{Ng \& Reames (1994)}, \textbf{Lee (2005)}, \textbf{Zank et al. (2000)} — explicitly integrate the Kolmogorov $k^{-5/3}$ spectrum and produce the concise mean-free-path formula (with split $A_+$ and $A_-$ when both wave senses are tracked).
\end{enumerate}

\textbf{Conclusion:}  
The boxed equation above is the standard Kolmogorov slab-QLT mean-free-path expression used in nearly all modern studies of SEP transport with self-generated turbulence since the mid-1990s.

\section*{Why the Cell Length \boldmath{$L_j$} Must Appear in the Denominator}

The streaming we want in the Parker–QLT machinery is a \textbf{flux} (particle current density):

\[
S(p,s) = \int v_\parallel f_0\,d\Omega
\quad\left[\text{particles m}^{-2}\,\text{s}^{-1}\,(\text{dp})^{-1}\right].
\]

If we compute it from Monte Carlo displacements inside a finite-volume cell, we should write

\[
\boxed{
S_{j}(p_m)
  =\frac{1}{\Delta t}\;
   \frac{1}{A_j\,L_j}\;
   \sum_{i\in(j,p_m)}
   w_i\,\Delta s_i
}
\tag{1}
\]

where:
\begin{itemize}
  \item $A_j$ = tube cross-section of cell $j$ (m\textsuperscript{2})
  \item $L_j$ = arc length of that cell (m)
  \item $\Delta s_i$ = path travelled \textbf{inside that cell} by particle $i$ during $\Delta t$ (m)
\end{itemize}

\subsection*{Why the Extra \boldmath{$L_j$} Is Required}

\paragraph{1. Volume vs. area}
Number density in a cell is
\[
n_j = \frac{\sum w_i}{A_j L_j}.
\]
Multiplying by the average streaming speed
\[
\langle v_\parallel \rangle = \frac{1}{\sum w_i} \sum \frac{\Delta s_i}{\Delta t}
\]
gives
\[
n_j\,\langle v_\parallel\rangle
= \frac{1}{\Delta t}\,\frac{1}{A_j L_j}\,\sum w_i\,\Delta s_i,
\]
which is exactly Eq.~(1).

\paragraph{2. Dimensional check}
\begin{center}
\begin{tabular}{@{}ll@{}}
\toprule
Term & Units \\
\midrule
$w_i\,\Delta s_i$ & particles $\times$ m \\
$A_j\,L_j$ & m\textsuperscript{2} $\times$ m = m\textsuperscript{3} \\
Divide by $\Delta t$ & 1/s \\
\bottomrule
\end{tabular}
\end{center}
Thus, Eq.~(1) has units of particles m$^{-2}$ s$^{-1}$, as required.

\paragraph{3. Energy-coupling formulas use flux, not number density}
In the quasilinear growth rate
\[
\gamma_\pm(k) = \frac{\pi^2 e^2 V_A}{c B_0^2}\,\frac{S_\pm(k)}{k},
\]
$S_\pm(k)$ must already be a \textbf{current density per} $k$-bin. Otherwise, $\gamma$ is off by a factor of $L_j$.

\subsection*{What Happens if You Omit \boldmath{$L_j$}?}
\begin{itemize}
  \item You over-estimate $S$ (and $\gamma$) proportionally to $L_j$;
  \item The simulation creates too much wave power and artificially decelerates particles;
  \item The error worsens with finer spatial resolution because smaller $L_j$ would reduce the mistake.
\end{itemize}

\subsection*{References Giving the Full $1/(A\,L)$ Factor}

\begin{tabular}{@{}p{9cm}p{6cm}@{}}
\toprule
\textbf{Paper} & \textbf{Equation / Location} \\
\midrule
\textbf{Skilling (1975)}, MNRAS 172, 557 & Eq. (51) uses $V = A L$ in the estimator \\
\textbf{Ng \& Reames (1994)}, ApJ 424, 1032 & Eq. (9): streaming = $\sum w\,\Delta s / (A L\,\Delta t)$ \\
\textbf{Zank, Rice \& Wu (2000)}, JGR 105, 25079 & Sect. 3.2: Monte Carlo code divides by volume \\
\textbf{Afanasiev, Vainio \& Kocharov (2015)}, A\&A 584, A81 & Appendix A, Eq. (A.4): explicit $1/L_j$ factor \\
\bottomrule
\end{tabular}

\subsection*{Revised Code Snippet for the Streaming Tally}

\begin{verbatim}
// inside the displacement-splitting loop
streaming[m][j] += w_i * step;      // step = Δs inside THIS cell
...
// after all particles processed
S_cell[m][j] = streaming[m][j] / (Δt * A_j * L_j);
\end{verbatim}

All subsequent steps—projection into $S_\pm$, computing $\gamma$, and wave-amplitude update—remain unchanged.

\subsection*{Bottom Line}

Always divide by \emph{both} the tube cross-section $A_j$ \textbf{and} the cell length $L_j$ to obtain the correct Parker streaming flux.


\section*{Streaming $S(p,s)$ When a Particle Traverses Several Cells Within One Global Step $\Delta t$}

The Parker–flux estimate is local:
\[
S_j(p) =
\frac{1}{\Delta t\,A_j\,L_j}
\sum_{i\in\text{cell }j}
w_i\,\Delta s_{i\to j},
\]
so you must give \textbf{each crossed cell only the distance the particle covers inside it}. Do this by \textbf{splitting the displacement}; the same loop works whether the particle crosses one, three, or ten cells.

\subsection*{Step-by-Step Splitting Algorithm}

# Particle Transport Algorithm - LaTeX Code

\begin{lstlisting}
// inputs for particle i:  seg = $j_0$   ,  $\xi$  ($0 \leq \xi \leq L[j_0]$)
double $\Delta s$ = $U_{\text{conv}}$ + sqrt($2\kappa_{\parallel}\Delta t$)*$G_{01}$();   // signed total step
int    $j$  = seg;
double $\xi_L$ = $\xi$;                              // local position before move
while ($\Delta s \neq 0.0$) {
    // 1. distance to nearest face in direction of travel
    double distFace = ($\Delta s > 0.0$) ? ($L[j] - \xi_L$) : ($-\xi_L$);
    // 2. distance actually travelled inside current cell
    double step = ($\Delta s > 0.0$)
                  ? std::min(distFace,  $\Delta s$)
                  : std::max(distFace,  $\Delta s$);
    // 3. accumulate streaming in cell $j$
    int $m$ = momentum_bin($v_i$);              // after $dv/dt$ but before move
    streaming[$m$][$j$] += $w_i \cdot$ step;          // step may be $\pm$
    // 4. update local coordinate and remaining displacement
    $\xi_L$ += step;
    $\Delta s$ -= step;
    // 5. crossed a face? -- enter next/previous cell
    if ($\xi_L \geq L[j] - 10^{-12}$ && $\Delta s > 0.0$) { $j$++; $\xi_L = 0.0$;           }
    if ($\xi_L \leq 0.0 + 10^{-12}$ && $\Delta s < 0.0$) { $j$--; $\xi_L = L[j]$;     }
}
\end{lstlisting}


After all particles are processed, finalise:
\begin{lstlisting}
S[m][j] = streaming[m][j] / (Δt * A_j * L_j);   // particles m⁻² s⁻¹ GeV⁻¹
\end{lstlisting}

\subsection*{Why This is Correct and Efficient}
\begin{itemize}
    \item \textbf{Exact volume weighting} – each cell receives $w_i\,\Delta s_{i\to j}$. Summing over all particles and dividing by $A_j L_j$ yields the correct Fick-law flux.
    \item \textbf{No double-counting} – the loop ensures each bit of path is credited once and only once.
    \item \textbf{Arbitrary jump length} – works even when $|\Delta s|\gg L_{\min}$.
    \item \textbf{Momentum bin stability} – use the particle’s momentum at the start of $\Delta t$.
\end{itemize}

\subsection*{Practical Tips}
\begin{tabular}{@{}p{5cm}p{10cm}@{}}
\toprule
\textbf{Issue} & \textbf{Remedy} \\
\midrule
Many crossings raise cost &
Limit $\Delta s \le 0.5 L_{\min}$ (CFL + diffusion), or sub-cycle the MC step. \\
Shock split cell &
Treat upstream and downstream halves as independent cells. \\
Signed steps &
Positive: anti-sunward; negative: sunward — signs propagate into $\gamma_\pm$. \\
\bottomrule
\end{tabular}

\subsection*{References That Use the Same Split Rule}
\begin{itemize}
    \item Zank, Rice \& Wu (2000), \emph{JGR} 105, 25079 – Sect. 3.2, Eq. (14)
    \item Afanasiev, Vainio \& Kocharov (2015), \emph{A\&A} 584, A81 – Appendix A
    \item Dröge et al. (2010), \emph{ApJ} 709, 912 – Sect. 2.2
\end{itemize}

\subsection*{Takeaway}

Splitting the displacement this way keeps the Parker streaming, wave growth/damping, and energy bookkeeping \textbf{exactly correct} even when a single time step carries particles through multiple spatial cells.

\section*{Short Answer}

Use \textbf{logarithmic momentum bins} (equal spacing in $\log p$ or $\log E$) when you accumulate the Parker flux array $S_{\text{cell}}[m][j]$.

Linear bins work, but they waste resolution at low $p$, oversample at high $p$, and make the growth-rate integral less accurate.

\section*{Why ``log-$p$'' is the standard choice}

\begin{center}
\begin{tabular}{@{}p{0.28\textwidth} | p{0.33\textwidth} | p{0.33\textwidth}@{}}
\toprule
\textbf{Reason} &
\textbf{Log-momentum grid} &
\textbf{Linear-momentum grid} \\
\midrule
\textbf{SEP spectra span $\geq 4$ decades} (10 keV–10 GeV) &
Each order of magnitude gets the \textbf{same number of bins}. &
Low-energy region squeezed into a few huge bins; high-energy part over-resolved. \\
\midrule
\textbf{Resonant-$k$} mapping is $k \propto p^{-1}$. Kolmogorov $k^{-5/3} \rightarrow$ power-law in $p$. &
Power laws become \textbf{straight lines}, so trapezoidal or Simpson integration over $p$ is accurate with 10--15 bins/decade. &
Need hundreds of tiny linear bins to resolve the same law without bias. \\
\midrule
\textbf{Equality with observation channels} (GOES, SOHO/ERNE, PSP/IS$\odot$IS) &
Instrument energy channels are spaced roughly logarithmically. &
Post-run rebinning to log-space required anyway. \\
\midrule
\textbf{Memory vs. noise} &
40--60 log bins give $\leq 10$\% Poisson noise if each carries $\mathcal{O}(10^3)$ particles. &
Linear bins must be much finer at high $p$, increasing memory or noise. \\
\bottomrule
\end{tabular}
\end{center}

\vspace{1em}

\noindent \textbf{References that explicitly state or use log-$p$ bins for $S_j(p)$:}
\begin{itemize}
    \item Ng \& Reames 1994, \textit{ApJ} 424, 1032 — 20 log bins per decade.
    \item Zank, Rice \& Wu 2000, \textit{JGR} 105, 25079 — 12 log bins per decade.
    \item Vainio \& Laitinen 2007, \textit{ApJ} 658, 622 — 15 log bins per decade.
    \item Afanasiev, Vainio \& Kocharov 2015, \textit{A\&A} 584, A81 — equal spacing in $\log p$.
\end{itemize}

\section*{What to change in the code snippet}

\begin{lstlisting}[language=C++,basicstyle=\ttfamily\footnotesize]
// build logarithmic grid once
double p_min = 1e-2 * mp * c;   // 10 keV
double p_max = 1e1  * mp * c;   // 10 GeV
double dlogp  = (std::log(p_max) - std::log(p_min)) / (Np - 1);
for (m = 0; m < Np; ++m)
    p_grid[m] = p_min * std::exp(m * dlogp);

// during the MC sweep
m = int( std::floor((std::log(p_i) - std::log(p_min)) / dlogp) );
m = std::clamp(m, 0, Np - 1);   // momentum bin index
\end{lstlisting}

\vspace{1em}

\noindent When converting to differential flux later, divide by the \textbf{log-width}:
\[
\Delta p_m = p_m \left(e^{d \log p} - 1\right)
\]
so the units remain particles m$^{-2}$ s$^{-1}$ (GeV n$^{-1}$)$^{-1}$.

\section*{Bottom Line}

Binning the streaming flux in \textbf{logarithmic momentum} matches the physics of power-law SEP spectra, keeps numerical noise uniform over the entire energy range, and is what virtually all modern SEP wave-coupling codes do in practice.


\section*{Overview}
\emph{One field line, linear-segment mesh, moving shock, particle-by-particle energy bookkeeping}

\section*{0. Pre-run Set-up}

\begin{enumerate}[label=\arabic*.]
\item \textbf{Build curved Parker-spiral vertices} $k = 0 \dots N_v$:
\[
\mathbf{R}_k = (r_k, \phi_k), \quad s_k = \sum |\mathbf{R}_{k+1} - \mathbf{R}_k|
\]
\item \textbf{Segments} $j = 0 \dots N_c - 1$: $L_j = s_{j+1} - s_j$; store left-face area $A_j$.
\item \textbf{Background plasma at each vertex:} 
\[
B_k,\; \rho_k,\; U_k,\; V_{A,k} = \frac{B_k}{\sqrt{\mu_0 \rho_k}}
\]
\item \textbf{Wave field (Kolmogorov)} – one scalar per cell \& sense: $A_{+,j}, A_{-,j} \ll \frac{B^2}{2\mu_0}$.

Inertial range $k_{\min}, k_{\max}$; normalizer:
\[
N_k = \frac{3}{2} \cdot \frac{k_{\max}^{-2/3} - k_{\min}^{-2/3}}{k_{\min}^{-5/3}}
\]

\item \textbf{Diffusion lookup:}
\[
\kappa_\parallel(p_m, s_j) = \frac{v_m}{3\lambda_\parallel}, \quad 
\lambda_\parallel^{-1} = \pi e^2 B^{-2}(A_+ + A_-) (k_{\min}^{-2/3} - k_{\max}^{-2/3})
\]

\item \textbf{Monte-Carlo particle list:} $\{j, \xi, v, w\}$ sampled from the SEP source.

\item \textbf{Shock object:} $s_{\text{sh}}(0), V_{\text{sh}}, \eta, r_\rho$
\end{enumerate}

\section*{1. Global Time-step Loop ($\Delta t$)}

\begin{enumerate}[label=\arabic*.]
\item \textbf{Move shock:}
\[
s_{\text{sh}} \leftarrow s_{\text{sh}} + V_{\text{sh}} \Delta t
\]
If it enters a new cell, flag downstream half-cell and create ramp source:
\[
Q_{\text{shock}} = \frac{\eta \rho_1 (V_{\text{sh}} - U_1)^3}{2N_k}
\]

\item \textbf{Particle sweep – for each pseudo-particle:}
\begin{enumerate}[label=\alph*.]
\item \textbf{Adiabatic acceleration:}
\[
v \leftarrow v \cdot \exp\left[\frac{1}{3}(\nabla \cdot U) \Delta t\right]
\]

\item \textbf{Spatial step:}
\begin{itemize}
\item Eulerian mesh: $\Delta s = U_j \Delta t + \sqrt{2\kappa_\parallel \Delta t}\, G(0,1)$
\item Lagrangian mesh: $\Delta s = \sqrt{2\kappa_\parallel \Delta t}\, G(0,1)$
\end{itemize}

\item \textbf{Split displacement across segments:} for each crossed cell $j$:
\[
S_{\text{raw}}[m,j] \mathrel{+}= w_i \cdot \Delta s_{i \to j}
\]
\end{enumerate}

\item \textbf{Convert raw sums to scalar Parker flux:}
\[
S(m,j) = \frac{S_{\text{raw}}}{\Delta t \cdot A_j \cdot L_j}
\]

\item \textbf{Project scalar flux to wave senses} (half-flux rule): 
for each $(m,j)$ → find $k_{\text{res}}^\pm$ and accumulate $S_+(j), S_-(j)$.

\item \textbf{Compute growth/damping coefficients:}
\[
\gamma_\pm(j) = \frac{\pi^2 e^2 V_{A,j}}{c B_j^2} \cdot \frac{S_\pm(j)}{k_{\text{eff}}}
\]

\item \textbf{Wave-amplitude advance (per cell, per sense):}
\[
A_\pm^{n+1} = A_\pm^n + \Delta t \left[ 2\gamma_\pm A_\pm - \frac{A_\pm}{\tau_{\text{cas}}} + Q_{\text{shock}} \right]
\]

\item \textbf{Energy bookkeeping:}
\begin{itemize}
\item If $\Delta W_\pm = 2\gamma_\pm A_\pm \Delta t < 0$ (wave loss): share $|\Delta W_\pm|$ among resonant particles
\[
v_i \leftarrow \sqrt{v_i^2 + \frac{2\Delta E_i}{m_p}}
\]

\item If $\Delta W_\pm > 0$ (wave gain): subtract from resonant particles similarly.

\item Energy redistributed $\propto w_i |p_{\parallel i}|$ so only drivers feel back-reaction.
\end{itemize}

\item \textbf{Update $\kappa_\parallel$ table} with new $A_+ + A_-$.

\item \textbf{Advance simulation time:} $t \leftarrow t + \Delta t$; repeat.
\end{enumerate}

\section*{2. Key Helper: Displacement Split}

\textbf{Input:} start $s_{\text{init}}$, end $s_{\text{final}}$.

Walk cell-by-cell, adding $w_i \cdot \text{step}$ into
\texttt{segment->GetDatum\_ptr(ParkerFluxKey)[mBin]}, using frustum volume:
\[
V_j = \frac{L_j}{3}(A_L + \sqrt{A_L A_R} + A_R)
\]

\section*{3. Stability Rules}
\begin{itemize}
\item CFL: $(U + V_A)\Delta t < 0.5 L_{\min}$
\item Diffusion: $2\kappa_\parallel \Delta t < 0.4 L_{\min}^2$
\item Growth: sub-cycle if $\gamma_\pm \Delta t > 0.5$
\end{itemize}

\section*{4. Where Each Part Comes From}
\begin{itemize}
\item Jokipii (1966): QLT kernel, Parker flux
\item Ng \& Reames (1994): scalar-flux $\rightarrow$ growth, Kolmogorov reduction
\item Zank, Rice \& Wu (2000): moving line, MC split, energy bookkeeping
\item Lee (2005): one-amplitude Kolmogorov wave equation
\item Caprioli \& Spitkovsky (2014): ramp injection efficiency
\end{itemize}

\section*{Conclusion}
This list is the \textbf{complete top-to-bottom algorithm} you can implement verbatim for a self-consistent SEP transport and self-generated Alfvén turbulence simulation.


\section*{Removing Energy from Particles When the Wave Field Gains Energy}

\emph{(i.e., $\Delta W > 0$ in the amplitude step)}

\section*{1. Decide Which Particles Pay the Bill}

Give the loss only to the particles that actually \textbf{drove} the wave change—  
the same ones whose pitch-angle or streaming weight $G_i$ you used for growth:
\[
G_i = w_i\,|p_{\parallel i|} \quad \Longrightarrow \quad \Delta E_i \propto G_i
\]

\section*{2. Compute the Total Energy that Must Be Removed}

\[
E_{\text{wave\,gain}} = 
\left( 2\gamma_\pm A_\pm + Q_{\text{shock}} - \frac{A_\pm}{\tau_{\text{cas}}} \right)
\Delta t \cdot V_{\text{cell}}
\]
with
\[
V_{\text{cell}} = \frac{L_j}{3} \left( A_{\rm L} + \sqrt{A_{\rm L} A_{\rm R}} + A_{\rm R} \right)
\]

\section*{3. Distribute the Loss Proportionally to $G_i$}

\begin{lstlisting}[language=C++,basicstyle=\ttfamily\footnotesize]
// resonant group only
double Gtot = sum_over_i(G_i);
double dE_per_G = E_wave_gain / Gtot;

for (particle i in resonant set) {
    double dE = dE_per_G * G_i;
    double v2 = v_i*v_i - 2*dE/mp;
    
    if (v2 < v_floor*v_floor) {
        dE  = 0.5 * mp * (v_i*v_i - v_floor*v_floor);
        v2  = v_floor * v_floor;
    }
    
    v_i = sqrt(v2);
    energyRemoved += dE;
}
\end{lstlisting}

\textbf{Notes:}
\begin{itemize}
    \item \texttt{v\_floor} – a safety floor, e.g., the ambient thermal speed:
    \[
    v_{\text{th}} = \sqrt{\frac{2kT}{m_p}} \sim 50~\text{km\,s}^{-1} \text{ at 1 au}
    \]
    or a lower numerical floor such as $10^3~\text{m\,s}^{-1}$.
    \item If a particle hits the floor, the loop keeps a running tally \texttt{energyRemoved}.
    \item Any \textbf{shortfall} is subtracted evenly from the remaining particles in a second pass, guaranteeing that total $\sum \Delta E_i = E_{\text{wave\,gain}}$.
\end{itemize}

\section*{4. Guarantee Non-negative Speeds}

Always ensure:
\[
v_{\text{new}}^2 = v_{\text{old}}^2 - \frac{2\Delta E_i}{m_p} > 0
\]

If not, clip to \texttt{v\_floor} and redistribute the excess as described above.

\section*{5. Dependence on Particle Energy}

The fractional loss for each particle is:
\[
\frac{\Delta E_i}{E_{k,i}} =
\frac{G_i}{E_{k,i}} \cdot \frac{E_{\text{wave\,gain}}}{G_{\text{tot}}}
= \frac{|p_{\parallel i}|}{\tfrac{1}{2} m_p v_i^2}
\cdot \frac{w_i\,E_{\text{wave\,gain}}}{G_{\text{tot}}}
\]

Thus, \textbf{higher-energy particles lose less fractionally} than slower ones,  
but if $w_i |p_{\parallel i}|$ is large, they still contribute significantly in absolute terms.

\section*{6. Summary Algorithm}

\begin{enumerate}
    \item Build the resonant list (same test used to form $S_\pm$).
    \item Compute $E_{\text{wave\,gain}}$.
    \item First pass: subtract $\Delta E_i \propto G_i$ but not below \texttt{v\_floor}; keep tally.
    \item If tally $< E_{\text{wave\,gain}}$:
        \begin{itemize}
            \item Second pass over remaining particles, redistribute shortfall.
        \end{itemize}
    \item Confirm $\sum_i \Delta E_i = E_{\text{wave\,gain}}$ to machine precision.
\end{enumerate}

\bigskip

\noindent This approach preserves total energy, prevents negative particle speeds, and ensures removal is tied to each particle's parallel momentum—mirroring the resonance-driven wave growth mechanism.


\section*{Overview}

\emph{One scalar per propagation sense, one value per segment.}

Let:

\begin{center}
\begin{tabular}{@{}ll@{}}
\toprule
\textbf{Symbol} & \textbf{Meaning} \\
\midrule
$A_{+,j},\,A_{-,j}$ & Kolmogorov amplitudes in segment $j$ (J m$^{-2/3}$) \\
$U_j$               & Solar-wind bulk speed (along the line) in segment $j$ \\
$V_{A,j}$           & Alfvén speed in segment $j$ \\
$c_{+,j}=V_{A,j}-U_j$ & Upwind speed for \textbf{outward} waves \\
$c_{-,j}=-V_{A,j}-U_j$ & Upwind speed for \textbf{inward} waves \\
$L_j$               & Segment length \\
$\Delta t$          & Global time step \\
\bottomrule
\end{tabular}
\end{center}

This is an \textbf{upwind, conservative} one-dimensional flux scheme for advancing only the \textbf{pure advection} term:
\[
\frac{\partial A_\pm}{\partial t} + c_\pm \frac{\partial A_\pm}{\partial s} = 0
\]
before adding growth, damping, cascade, and shock-injection source terms.

\section*{1. Compute Face-Centered Speeds}

\begin{lstlisting}[language=C++,basicstyle=\ttfamily\footnotesize]
double cPlus  =  VA[j] - U[j];   // +
double cMinus = -VA[j] - U[j];   // –
\end{lstlisting}

\section*{2. Upwind Fluxes at Left and Right Faces of Cell $j$}

\begin{lstlisting}[language=C++,basicstyle=\ttfamily\footnotesize]
// Left face between j-1 and j
double Fp_L = (cPlus > 0.0)
              ? cPlus * A_plus[j-1]
              : cPlus * A_plus[j];       // upwind
double Fm_L = (cMinus > 0.0)
              ? cMinus * A_minus[j-1]
              : cMinus * A_minus[j];

// Right face between j and j+1
double Fp_R = (cPlus > 0.0)
              ? cPlus * A_plus[j]
              : cPlus * A_plus[j+1];
double Fm_R = (cMinus > 0.0)
              ? cMinus * A_minus[j]
              : cMinus * A_minus[j+1];
\end{lstlisting}

\emph{If $j = 0$ or $j = N-1$, supply boundary values or set flux = 0 at that face.}

\section*{3. Conservative Update}

\begin{lstlisting}[language=C++,basicstyle=\ttfamily\footnotesize]
double invVol = 1.0 / (L[j]); // if cross-section is uniform
A_plus[j]  += ((Fp_L - Fp_R) * dt) * invVol;
A_minus[j] += ((Fm_L - Fm_R) * dt) * invVol;
\end{lstlisting}

\section*{4. Stability Condition (CFL)}

First-order upwind is stable if:
\[
\max |c_\pm| \cdot \Delta t < 0.5 \cdot L_{\min}
\]

\section*{5. Complete Operator-Split Step}

\begin{lstlisting}[language=C++,basicstyle=\ttfamily\footnotesize]
// (1) Advection (above)
// (2) Growth / damping:   A += 2 * gamma * A * dt
// (3) Kolmogorov cascade: A -= A / tau_cas * dt
// (4) Shock injection:    A += Q_shock * dt
\end{lstlisting}

Steps (2)–(4) are done \textbf{after} advection during each global time step.

\emph{Note:} Because all four operators are first-order in time, using a fixed split is first-order accurate overall. If higher accuracy is needed, you can apply Strang splitting (alternate the sequence every step).

\section*{6. Boundary Options}

\begin{itemize}
    \item \textbf{Sunward inner boundary:} reflective for $A_{-}$ and free-outflow for $A_{+}$.
    \item \textbf{Outer boundary} (e.g., 5 AU): fixed quiet-wind level $A_\pm = A_{\rm BG}$.
    \item \textbf{Shock cell:} advection uses the \textbf{post-growth} $A_\pm$ values so that ramp-injected waves are immediately convected downstream by the next step.
\end{itemize}

\section*{Conclusion}

This compact finite-volume scheme transports the turbulence energy density (via Kolmogorov amplitude $A_\pm$) consistently along the field line, in sync with the background flow and Alfvén group velocity, before incorporating physical source and loss terms.

\section*{Finite-Volume Advection in a Lagrangian Field-Line Mesh}

\emph{Each vertex is convected outward by the local solar-wind speed $U$.  
Because the mesh itself moves with the flow, only the Alfvén group velocity relative to the plasma ($\pm V_A$) transports wave energy from cell to cell.  
This routine also handles geometric stretching of each frustum-shaped cell.}

\section*{1. Preliminaries and Notation}

\begin{center}
\begin{tabular}{@{}ll@{}}
\toprule
\textbf{Symbol} & \textbf{Meaning (at time level $n$)} \\
\midrule
$A_{\pm,j}^n$ & Kolmogorov amplitudes in segment $j$ (J m$^{-2/3}$) \\
$V_j^n$ & Cell volume: $\dfrac{L_j}{3}\left(A_L + \sqrt{A_L A_R} + A_R\right)$ \\
$V_A^n$ & Alfvén speed at the cell center \\
$c_{+,j} = +V_A^n$, $c_{-,j} = -V_A^n$ & Upwind speeds (relative to the moving plasma) \\
$\Delta t$ & Global time step \\
\bottomrule
\end{tabular}
\end{center}

\noindent Because vertices move with $U$, \textbf{no $U$ term appears in the flux speed}.

\section*{2. Lagrangian-Mesh Update Sequence (One Global Step)}

\subsection*{Step 0: Move Vertices and Recompute Geometry}

\begin{lstlisting}[language=C++,basicstyle=\ttfamily\footnotesize]
// For each vertex k:
r_k += U_k * dt;                   // radial convection
// update B_k, ρ_k (∝ r⁻²), cross-section A_k = B0 / B_k

// For each segment j:
L_j = |R_{j+1} - R_j|;
A_L = A_left_face; A_R = A_right_face;
V_new = (L_j/3)*(A_L + sqrt(A_L*A_R) + A_R);
VA_j = B_mid / sqrt(mu0 * rho_mid);
\end{lstlisting}

\noindent \textbf{Stretch factor:}
\[
S_j = \frac{V_j^{n+1}}{V_j^n}
\]
This is used to conserve energy in the expanding cell.

\subsection*{Step 1: Upwind Advection (Plasma-Frame Group Velocity)}

For each segment $j$:

\begin{lstlisting}[language=C++,basicstyle=\ttfamily\footnotesize]
double cPlus  =  VA_j;
double cMinus = -VA_j;

// LEFT FACE fluxes
double Fp_L = (cPlus  > 0) ? cPlus  * A_plus[j-1]  : cPlus  * A_plus[j];
double Fm_L = (cMinus > 0) ? cMinus * A_minus[j-1] : cMinus * A_minus[j];

// RIGHT FACE fluxes
double Fp_R = (cPlus  > 0) ? cPlus  * A_plus[j]    : cPlus  * A_plus[j+1];
double Fm_R = (cMinus > 0) ? cMinus * A_minus[j]   : cMinus * A_minus[j+1];

// Conservative update in old volume V_j^n
double dAp = (Fp_L - Fp_R) * dt / V_j^n;
double dAm = (Fm_L - Fm_R) * dt / V_j^n;

A_plus[j]  += dAp;
A_minus[j] += dAm;
\end{lstlisting}

\subsection*{Step 2: Geometric Stretching (Adiabatic Dilution)}

The wave energy density must scale with the expanding volume:
\begin{lstlisting}[language=C++,basicstyle=\ttfamily\footnotesize]
double Sstretch = V_j^n / V_j^{n+1};
A_plus [j] *= Sstretch;
A_minus[j] *= Sstretch;
\end{lstlisting}

\emph{If amplitudes are stored per unit magnetic flux ($A/B$), this step is unnecessary.}

\subsection*{Step 3: Add Local Source/Sink Terms}

\begin{lstlisting}[language=C++,basicstyle=\ttfamily\footnotesize]
A_plus [j] += ( 2 * gamma_plus  * A_plus[j]
              - A_plus[j] / tau_cas
              + Q_shock_plus[j] ) * dt;

A_minus[j] += ( 2 * gamma_minus * A_minus[j]
              - A_minus[j] / tau_cas
              + Q_shock_minus[j] ) * dt;
\end{lstlisting}

These terms are \textbf{cell-local} and independent of mesh motion.

\section*{3. Stability and Accuracy}

\begin{itemize}
    \item \textbf{CFL condition (Lagrangian mesh):}
    \[
    |V_A|\,\Delta t < 0.5\,L_{\min}
    \]
    This is more lenient than Eulerian schemes since $|U|$ is absent.
    \item \textbf{Stretching of small cells:} if $S_j \ll 1$, sub-cycle rescaling or limit $\Delta t$ so that $S_j > 0.8$.
\end{itemize}

\section*{4. Energy Conservation Check}

For each cell:
\[
E^{n+1}_{\text{wave}} = S_{\text{stretch}} \left( E^n_{\text{wave}}
+ (F_{\text{in}} - F_{\text{out}}) \Delta t
+ \left(2\gamma A - \frac{A}{\tau_{\text{cas}}} + Q_{\text{shock}}\right) \Delta t \cdot V_j^n \right)
\]
This is what the algorithm above computes exactly, with machine-level precision.

\section*{5. Complete Step in Code (Per Cell $j$)}

\begin{lstlisting}[language=C++,basicstyle=\ttfamily\footnotesize]
// 0: update geometry
double V_old = Vol[j];
double V_new = computeFrustumVolume(j); Vol[j] = V_new;
double Sstretch = V_old / V_new;
double cP =  VA[j];
double cM = -VA[j];

// 1: upwind advection
double FpL = (cP > 0) ? cP * Aplus[j-1] : cP * Aplus[j];
double FpR = (cP > 0) ? cP * Aplus[j]   : cP * Aplus[j+1];
double FmL = (cM > 0) ? cM * Aminus[j-1]: cM * Aminus[j];
double FmR = (cM > 0) ? cM * Aminus[j]  : cM * Aminus[j+1];
Aplus [j] += (FpL - FpR) * dt / V_old;
Aminus[j] += (FmL - FmR) * dt / V_old;

// 2: stretching dilution
Aplus [j] *= Sstretch;
Aminus[j] *= Sstretch;

// 3: local source/sink
Aplus [j] += (2 * gammaP * Aplus[j] - Aplus[j] / taucas + QshockP[j]) * dt;
Aminus[j] += (2 * gammaM * Aminus[j] - Aminus[j] / taucas + QshockM[j]) * dt;
\end{lstlisting}

\noindent This is the \textbf{complete advection-plus-stretch step} for turbulence energy density along a moving (Lagrangian) Parker-spiral field line.

\section*{Storing Integrated Wave Energy per Segment}

\textbf{Yes, storing the \emph{integrated} wave energy per segment} ($E = \int W\,dV$) can simplify the Lagrangian implementation—but it trades memory simplicity for a bit of algebra every time you need a density-based quantity.

\section*{Why It Feels Easier in a Lagrangian Mesh}

\begin{center}
\begin{tabular}{@{}p{0.28\textwidth} | p{0.31\textwidth} | p{0.31\textwidth}@{}}
\toprule
\textbf{Aspect} &
\textbf{Density $A_\pm$ (J m$^{-2/3}$)} &
\textbf{Integrated energy $E_\pm$ (J)} \\
\midrule
\textbf{Cell stretching} &
Must multiply by the stretch factor $V_{\text{old}}/V_{\text{new}}$ &
\emph{Nothing to do:} energy stays with the moving cell \\
\textbf{Advection step} &
Flux term needs $c_\pm A_\pm$ and the face area &
Flux term is just $c_\pm E_\pm$; volume cancels automatically \\
\textbf{Energy conservation test} &
Compare $A_\pm V$ between steps &
Just sum $E_\pm$ over all cells (fewer floating-point ops) \\
\bottomrule
\end{tabular}
\end{center}

\section*{Where You Still Need the \emph{Density}}

\begin{itemize}
\item \textbf{Growth / damping coefficient:}  
$2\gamma_\pm A_\pm$ uses $A_\pm$ explicitly because QLT is formulated per unit volume.  
You evaluate:
\[
A_\pm = \frac{E_\pm}{V_j}
\quad \Longrightarrow \quad
\frac{dE_\pm}{dt} = 2\gamma_\pm \frac{E_\pm}{V_j} \cdot V_j = 2\gamma_\pm E_\pm
\]
So the ODE becomes: $\dot{E}_\pm = 2\gamma_\pm E_\pm$ — equally simple.

\item \textbf{Cascade sink:}  
$E_\pm/\tau_{\text{cas}}$ already uses energy, not density, if you write:  
$\tau_{\text{cas}}^{-1} = C_K (V E)^{1/2} / L_\perp$.

\item \textbf{Diffusion update:}  
The mean free path needs $A_+ + A_-$:
\[
A_{\text{tot}} = \frac{E_+ + E_-}{V_j}
\]
\end{itemize}

\section*{Code Sketch with Energy Storage}

\begin{lstlisting}[language=C++,basicstyle=\ttfamily\footnotesize]
// cell j variables now:  Eplus[j], Eminus[j]   (J)

// 1. advection (upwind, plasma frame)
FpL = (cP > 0) ? cP * Eplus[j-1] : cP * Eplus[j];   // J/s
FpR = (cP > 0) ? cP * Eplus[j]   : cP * Eplus[j+1];
Eplus[j] += (FpL - FpR) * dt;                       // volume auto-handled

// 2. growth / damping
Eplus[j] += 2 * gammaP * Eplus[j] * dt;

// 3. Kolmogorov cascade
Eplus[j] -= Eplus[j] / taucas * dt;

// 4. shock injection (J/s)
Eplus[j] += QshockP[j] * dt;
\end{lstlisting}

\noindent \textbf{Note:} The \emph{stretch term disappears} because $E_\pm$ follows the cell:
no $V_{\text{old}} / V_{\text{new}}$ factor is necessary.

\section*{Trade-offs}

\subsection*{Pros}
\begin{itemize}
    \item Fewer stretch/dilution operations in a Lagrangian mesh
    \item Energy conservation check becomes a simple sum
    \item Advection step is slightly cheaper (no division by volume)
\end{itemize}

\subsection*{Cons}
\begin{itemize}
    \item Must divide by $V_j$ whenever you need $A_\pm$ (e.g., for $\gamma$, $\lambda_\parallel$, pressure diagnostics)
    \item If you switch to an Eulerian mesh, you'll need to reintroduce volume factors
\end{itemize}

\section*{Bottom Line}

For a \emph{purely Lagrangian} field-line solver, it is indeed \textbf{cleaner and perfectly safe} to store the \textbf{integrated wave energy $E_\pm$} instead of the density $A_\pm$.

\medskip

\noindent Just remember to divide by the instantaneous cell volume whenever a formula explicitly requires a density-based quantity (growth rate, mean free path, etc.).


