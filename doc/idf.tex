\chapter{The model of internal degrees of freedom}


\chapter{Energy Distribution of a Colliding Pair in a Thermalized Gas}

\section{Introduction}

In a gas at thermal equilibrium at temperature $T$, the particles' velocities are distributed according to the Maxwell-Boltzmann distribution. Consider two particles selected at random from this gas. One may ask: what is the probability distribution of their \emph{relative kinetic energy}? Moreover, if we focus on pairs of particles that are actually colliding, the effective weighting of different relative speeds (and hence energies) changes due to the collision cross section and relative velocity dependence.

In this chapter, we derive the probability distributions for the relative energy of a pair of particles first in the absence of any collision-weighting factor (i.e., selecting pairs at random), and then considering a collision-weighted scenario where the collision frequency depends on the relative speed $g$ and the collision cross section $\sigma(g)$. We also consider the case where $\sigma(g)$ is a power-law function of $g$, i.e., $\sigma(g) \propto g^{v}$.

\section{Relative Velocity and Energy Distributions}

\subsection{Maxwell-Boltzmann Distribution for Relative Velocity}

Consider two particles with masses $m_1$ and $m_2$ in a gas at thermal equilibrium at temperature $T$. Define the reduced mass
\begin{equation}
  \mu = \frac{m_1 m_2}{m_1 + m_2}.
\end{equation}

The velocity distribution of each particle is Maxwell-Boltzmann. The relative velocity of the pair is given by
\begin{equation}
  \mathbf{g} = \mathbf{v}_1 - \mathbf{v}_2.
\end{equation}

In three dimensions, the probability density function (PDF) for the magnitude of the relative speed $g = |\mathbf{g}|$ (when the particles are independently drawn from a Maxwellian distribution) is:
\begin{equation}
  f(g) \propto g^{2} \exp\left(-\frac{\mu g^{2}}{2 k_B T}\right).
\end{equation}

This result stems from the convolution of two Maxwellian distributions and the definition of the reduced mass. The key point is that the relative velocity itself follows a Maxwellian form with mass parameter $\mu$.

\subsection{Energy Distribution for Randomly Chosen Pairs}

The relative kinetic energy of the pair is defined as
\begin{equation}
  E = \frac{1}{2}\mu g^{2}.
\end{equation}

To find the distribution in terms of $E$, we rewrite:
\begin{equation}
  g^{2} = \frac{2E}{\mu}.
\end{equation}
Thus,
\begin{equation}
  f(g) \propto \left(\frac{2E}{\mu}\right)^{\!1} \exp\left(-\frac{E}{k_B T}\right).
\end{equation}

We must also change variables from $g$ to $E$. Since $E = \frac{1}{2}\mu g^{2}$, the differential relation is
\begin{equation}
  dE = \mu g \, dg \implies dg = \frac{dE}{\mu g} = \frac{dE}{\mu \sqrt{\frac{2E}{\mu}}} = \frac{dE}{\sqrt{2\mu E}}.
\end{equation}

Substituting $g = \sqrt{2E/\mu}$ into $f(g)$ and including the Jacobian $dg/dE$, we have
\begin{align}
  f(E) &= f(g) \frac{dg}{dE} \nonumber \\
       &\propto \left(\frac{2E}{\mu}\right) \exp\left(-\frac{E}{k_B T}\right) \cdot \frac{1}{\sqrt{2\mu E}}.
\end{align}

Simplifying the powers of $E$:
\begin{equation}
  \left(\frac{2E}{\mu}\right) \frac{1}{\sqrt{2\mu E}} = E^{1 - \frac{1}{2}} = E^{1/2}.
\end{equation}

Thus, the unweighted distribution of the relative energy of two randomly chosen particles is
\begin{equation}
  f_{\text{pair}}(E) \propto E^{1/2} e^{-E/(k_B T)}.
\end{equation}

\section{Collision-Weighted Distribution}

\subsection{Inclusion of Collision Frequency}

When considering the energy distribution of pairs \emph{at the moment they collide}, we must weight the distribution by the collision frequency. The collision frequency for two particles at relative speed $g$ is proportional to
\begin{equation}
  \text{collision frequency} \propto g \sigma(g),
\end{equation}
where $\sigma(g)$ is the collision cross section. Physically, $g$ represents how fast one particle "encounters" the other, and $\sigma(g)$ represents the effective area for collision at that relative speed.

Therefore, the collision-weighted speed distribution becomes
\begin{equation}
  f_{\text{collision}}(g) \propto f(g) \, g \sigma(g).
\end{equation}

\subsection{Constant Cross Section Case}

First, consider the simplest case $\sigma(g) = \text{const}$. Let $\sigma(g) = \sigma_0$. In this case,
\begin{equation}
  f_{\text{collision}}(g) \propto f(g) \cdot g = g^{3} e^{-\frac{\mu g^{2}}{2 k_B T}}.
\end{equation}

Converting to energy:
\begin{equation}
  f_{\text{collision}}(E) \propto E e^{-E/(k_B T)},
\end{equation}
after going through the same change-of-variables steps as before. Notice that this differs from the unweighted distribution by an extra factor of $E^{1/2}$ turning into $E$, reflecting the bias towards higher relative speeds in actual collisions.

\subsection{General Case: $\sigma(g) \propto g^{v}$}

Now consider a more general scenario where the collision cross section depends on relative speed as a power-law:
\begin{equation}
  \sigma(g) \propto g^{v}.
\end{equation}

In that case, the collision-weighted distribution in terms of $g$ is
\begin{equation}
  f_{\text{collision}}(g) \propto f(g) \cdot g \cdot g^{v} = g^{3+v} \exp\left(-\frac{\mu g^{2}}{2 k_B T}\right).
\end{equation}

Again, we transform to energy $E = \tfrac{1}{2}\mu g^{2}$. Thus,
\begin{equation}
  g^{2} = \frac{2E}{\mu} \implies g = \sqrt{\frac{2E}{\mu}}.
\end{equation}
So
\begin{equation}
  g^{3+v} = \left(\frac{2E}{\mu}\right)^{\frac{3+v}{2}}.
\end{equation}

The collision-weighted distribution in terms of $E$ is
\begin{equation}
  f_{\text{collision}}(E) \propto \left(\frac{2E}{\mu}\right)^{\frac{3+v}{2}} e^{-E/(k_B T)} \cdot \frac{dE}{dg}.
\end{equation}

Recall:
\begin{equation}
  dg = \frac{dE}{\sqrt{2\mu E}},
\end{equation}
so
\begin{equation}
  \frac{dg}{dE} = \frac{1}{\sqrt{2\mu E}}.
\end{equation}

Substituting this, we get
\begin{align}
  f_{\text{collision}}(E) &\propto \left(\frac{2E}{\mu}\right)^{\frac{3+v}{2}} e^{-\frac{E}{k_B T}} \frac{1}{\sqrt{2\mu E}} \\
  &= \left(\frac{2E}{\mu}\right)^{\frac{3+v}{2}} \frac{1}{(2\mu E)^{1/2}} e^{-\frac{E}{k_B T}}.
\end{align}

Focus on the powers of $E$:
\begin{equation}
  E^{\frac{3+v}{2}} \cdot E^{-\frac{1}{2}} = E^{\frac{3+v}{2} - \frac{1}{2}} = E^{\frac{2+v}{2}} = E^{1 + \frac{v}{2}}.
\end{equation}

Thus, the final form is:
\begin{equation}
  f_{\text{collision}}(E) \propto E^{1 + \frac{v}{2}} e^{-\frac{E}{k_B T}}.
\end{equation}

\section{Discussion and Special Cases}

The exponent in the energy dependence is directly influenced by the exponent $v$ in the cross section's speed dependence:
\begin{itemize}
  \item For $v=0$ (constant cross section), we recover 
  \[
  f_{\text{collision}}(E) \propto E \, e^{-\frac{E}{k_B T}},
  \]
  which is the standard collision-energy distribution with no additional $g$-dependence in the cross section.

  \item For $v > 0$, higher speeds are more favorable, increasing the exponent of $E$ and thus placing more weight on higher energies.

  \item For $v < 0$, lower speeds are more favorable, decreasing the exponent of $E$ and thus placing more weight on lower energies.
\end{itemize}

\section{Conclusion}

In summary:
\begin{enumerate}
  \item A randomly chosen pair of particles from a Maxwellian gas has a relative energy distribution
  \[
  f_{\text{pair}}(E) \propto E^{1/2} e^{-\frac{E}{k_B T}}.
  \]

  \item Actual collisions are biased towards pairs with higher relative speeds. If the collision cross section is constant, the energy distribution of colliding pairs becomes
  \[
  f_{\text{collision}}(E) \propto E \, e^{-\frac{E}{k_B T}}.
  \]

  \item If the collision cross section scales as $\sigma(g) \propto g^{v}$, this modifies the distribution to
  \[
  f_{\text{collision}}(E) \propto E^{1 + \frac{v}{2}} e^{-\frac{E}{k_B T}}.
  \]
  \end{enumerate}

These results provide a comprehensive picture of how the underlying physics of collision cross sections and relative motion influences the observed distribution of collision energies in a thermalized gas.



\chapter{Energy Distribution of Colliding Pairs in a Two-Component Gas}

\section{Introduction}

In the previous chapters, we examined the energy distribution of colliding pairs in a single-component gas. Here, we generalize the discussion to a two-component gas mixture at thermal equilibrium, both components sharing the same temperature \( T \) but having different masses. Let the two species be labeled as 1 and 2, with masses \( m_1 \) and \( m_2 \), respectively.

We assume that each species follows a Maxwell-Boltzmann distribution at temperature \( T \). The collisions can occur between pairs of the same species (1–1 or 2–2) or between unlike species (1–2). We will see that the overall form of the energy distributions derived for the single-component case remains similar, but now each pair type involves a different reduced mass and potentially different relative abundances.

\section{Reduced Mass and Relative Speed Distributions}

\subsection{Reduced Mass for Different Pairings}

Consider two particles from the mixture. Depending on the pair:

\begin{itemize}
  \item Same-species pair of type 1–1: Reduced mass
    \[
    \mu_{11} = \frac{m_1}{2}.
    \]

  \item Same-species pair of type 2–2: Reduced mass
    \[
    \mu_{22} = \frac{m_2}{2}.
    \]

  \item Cross-species pair of type 1–2: Reduced mass
    \[
    \mu_{12} = \frac{m_1 m_2}{m_1 + m_2}.
    \]
\end{itemize}

Each of these reduced masses will determine the scaling of the relative velocity and energy distributions for that particular collision pair.

\subsection{Relative Speed Distributions}

In a gas at thermal equilibrium, the velocity distribution for each species is Maxwellian. For two particles drawn independently from these distributions, the relative velocity \(\mathbf{g} = \mathbf{v}_1 - \mathbf{v}_2\) also follows a Maxwell-Boltzmann type distribution but characterized by the appropriate reduced mass.

For a pair of species \( i \) and \( j \) (which could be the same or different), the probability density function (PDF) for the relative speed \( g = |\mathbf{g}| \) is:
\[
f_{ij}(g) \propto g^{2} \exp\left(-\frac{\mu_{ij} g^{2}}{2 k_B T}\right),
\]
where \(\mu_{ij}\) is the reduced mass of the pair. Explicitly:
\[
f_{11}(g) \propto g^{2} \exp\left(-\frac{\frac{m_1}{2} g^{2}}{2 k_B T}\right), \quad
f_{22}(g) \propto g^{2} \exp\left(-\frac{\frac{m_2}{2} g^{2}}{2 k_B T}\right),
\]
\[
f_{12}(g) \propto g^{2} \exp\left(-\frac{\mu_{12} g^{2}}{2 k_B T}\right).
\]

\section{Energy Distribution for Randomly Chosen Pairs}

The relative kinetic energy of the pair \( (i,j) \) is defined as
\[
E_{ij} = \frac{1}{2}\mu_{ij} g^{2}.
\]

When converting the speed distribution to an energy distribution for \emph{randomly chosen pairs} (i.e., without collision weighting), the derivation follows the same steps as in the single-component case. For any pair type \( (i,j) \), we find:
\[
f_{\text{pair},ij}(E) \propto E^{1/2} \exp\left(-\frac{E}{k_B T}\right).
\]

Notice that the \emph{functional form} of the distribution in terms of \( E \) does not change. The scaling and definitions shift only through the reduced mass \(\mu_{ij}\). However, since in the energy form we have factored out the \(\mu_{ij}\) dependence, the shape of \( f_{\text{pair},ij}(E) \) is universal. What differs in practice is the mapping between \( g \) and \( E \), and how likely different pairings are to occur, depending on the gas composition.

\section{Collision-Weighted Distributions}

\subsection{Collision Frequency and Cross Section}

When focusing on pairs that are actually colliding, we must weight the distribution by the collision frequency. For a pair with relative speed \( g \), the collision frequency is proportional to:
\[
g \sigma(g),
\]
where \(\sigma(g)\) is the collision cross section.

\subsection{Constant Cross Section}

For a constant cross section, \(\sigma(g)=\sigma_0\), the collision-weighted speed distribution for a pair \( (i,j) \) becomes:
\[
f_{\text{collision},ij}(g) \propto f_{ij}(g) \cdot g = g^{3} \exp\left(-\frac{\mu_{ij} g^{2}}{2 k_B T}\right).
\]

Changing variables to energy \( E_{ij} = \frac{1}{2}\mu_{ij} g^{2} \) and following the same procedure as before (including the Jacobian \( dg/dE \)), we find the collision-weighted energy distribution:
\[
f_{\text{collision},ij}(E) \propto E \exp\left(-\frac{E}{k_B T}\right).
\]

Thus, for each pair type, when \(\sigma(g)\) is constant, the energy distribution shifts from \( E^{1/2} e^{-E/(k_B T)} \) to \( E e^{-E/(k_B T)} \).

\subsection{Cross Section with Power-Law Dependence on \( g \)}

If the cross section depends on \( g \) as a power law:
\[
\sigma(g) \propto g^{v},
\]
then the collision-frequency factor is:
\[
g \sigma(g) \propto g^{1+v}.
\]

Thus, the collision-weighted speed distribution for pairs of type \( (i,j) \) is:
\[
f_{\text{collision},ij}(g) \propto g^{2} e^{-\frac{\mu_{ij} g^{2}}{2 k_B T}} \cdot g^{1+v} = g^{3+v} \exp\left(-\frac{\mu_{ij} g^{2}}{2 k_B T}\right).
\]

Converting to energy, we find the collision-weighted energy distribution for that pair type:
\[
f_{\text{collision},ij}(E) \propto E^{1 + \frac{v}{2}} \exp\left(-\frac{E}{k_B T}\right).
\]

\section{Overall Mixture Distribution}

In a two-component mixture, collisions can occur between any pair:
- 1–1 collisions,
- 2–2 collisions, and
- 1–2 collisions.

If the number densities of species 1 and 2 are \( n_1 \) and \( n_2 \), and if the relative frequencies of collisions depend on these densities and the cross sections, the total collision-energy distribution is a weighted sum:
\[
f_{\text{collision,total}}(E) = \alpha_{11} f_{\text{collision},11}(E) + \alpha_{22} f_{\text{collision},22}(E) + \alpha_{12} f_{\text{collision},12}(E),
\]
where \(\alpha_{ij}\) are weighting factors determined by the relative abundances of each type of collision event. These weighting factors depend on the partial pressures, number densities, and cross sections of the species involved.

\section{Discussion and Conclusion}

The key points are:
\begin{enumerate}
  \item The fundamental shapes of the distributions (either \( E^{1/2} e^{-E/(k_B T)} \) for randomly chosen pairs or \( E^{1+\frac{v}{2}} e^{-E/(k_B T)} \) for collision-weighted pairs with a power-law cross section) remain the same as in the single-species case.

  \item The difference lies in the reduced mass \(\mu_{ij}\), which changes the scaling of \( E \) with respect to \( g \). Although the final forms in terms of \( E \) look similar, the effective mass parameter differs.

  \item In a mixture, the total observed collision-energy distribution is a combination of the distributions from each pair type. The relative importance of each component depends on the composition of the mixture (i.e., the ratio \( n_1:n_2 \)) and the relative cross sections.
\end{enumerate}

In essence, transitioning from a single-component to a two-component gas does not alter the functional form of the energy distribution for collisions; it only introduces multiple channels (1–1, 2–2, and 1–2), each with its own reduced mass and associated weighting factor. The same logic used for a single species applies, but now you sum over the contributions from different pairs.



\section{Detailed Analysis of the Equilibrium Energy Distribution in case $\sigma(c_r) \propto \frac{d_1}{c_r^{v_1}} + \frac{d_2}{c_r^{v_2}}$} 

The collision cross-section quantifies the likelihood of collisions between particles as a function of their relative speed \(c_r\). It is a measure of the ``effective area'' that one particle presents to another for collision to occur:
\[
\sigma(c_r) \propto \frac{d_1}{c_r^{v_1}} + \frac{d_2}{c_r^{v_2}}
\]
This implies that the cross-section depends inversely on some power of the relative speed. Such dependencies can arise from different interaction potentials or mechanisms governing collisions.

\subsection{Impact on Collision Rate}
The \textbf{collision rate} \(R\) between particles depends on both the cross-section and the relative speed:
\[
R = n \langle \sigma(c_r) c_r \rangle
\]
where:
\begin{itemize}
    \item \(n\): number density of particles,
    \item \(\langle \cdot \rangle\): ensemble average over the distribution of relative speeds.
\end{itemize}
Thus, the form of \(\sigma(c_r)\) directly influences how often particles collide, especially at different energy (speed) levels.

\subsection{Transport Properties}
While the cross-section affects collision rates, it also influences macroscopic transport properties such as:
\begin{itemize}
    \item Viscosity,
    \item Thermal conductivity,
    \item Diffusion coefficients.
\end{itemize}
These properties depend on how momentum and energy are transported through the gas, which is governed by collision dynamics.



\subsection{Detailed Balance Principle}
\textbf{Detailed balance} is a condition that, at equilibrium, every microscopic process and its reverse occur at the same rate. This ensures that there is no net change in the population of any state, maintaining equilibrium. Mathematically, for any two states \(A\) and \(B\):
\[
W(A \to B) P(A) = W(B \to A) P(B)
\]
where:
\begin{itemize}
    \item \(W(X \to Y)\): transition rate from state \(X\) to \(Y\),
    \item \(P(X)\): probability of state \(X\).
\end{itemize}


\subsection{Derivation of the Fully Normalized Collision Energy Distribution}

We derive the \textbf{fully normalized collision energy distribution} \( P(E_{\text{col}}) \) for colliding particles in a gas, given the collision cross-section:
\[
\sigma(c_r) \propto \frac{d_1}{c_r^{v_1}} + \frac{d_2}{c_r^{v_2}}
\]
where:
\begin{itemize}
    \item \( d_1, d_2 > 0 \) are constants,
    \item \( c_r \) is the \textbf{relative speed} between colliding particles,
    \item \( v_1, v_2 \geq 0 \).
\end{itemize}

The \textbf{collision energy distribution}, \( P(E_{\text{col}}) \), where \( E_{\text{col}} = \frac{1}{2} m c_r^2 \), is proportional to:
\[
P(E_{\text{col}}) \propto \left( d_1 E_{\text{col}}^{1 - \frac{v_1}{2}} + d_2 E_{\text{col}}^{1 - \frac{v_2}{2}} \right) \exp\left( -\frac{E_{\text{col}}}{k_B T} \right)
\]
To obtain the \textbf{normalized distribution}, we calculate the normalization constant \( \mathcal{N} \) such that:
\[
\int_{0}^{\infty} P(E_{\text{col}}) \, dE_{\text{col}} = 1
\]
The final normalized form is:
\[
P(E_{\text{col}}) = \frac{1}{\mathcal{N}} \left( d_1 E_{\text{col}}^{1 - \frac{v_1}{2}} + d_2 E_{\text{col}}^{1 - \frac{v_2}{2}} \right) \exp\left( -\frac{E_{\text{col}}}{k_B T} \right)
\]

\section{Derivation of the Normalization Constant}
\subsection*{1. Express \( P(E_{\text{col}}) \) in Terms of Dimensionless Variables}
Introduce a dimensionless variable:
\[
x = \frac{E_{\text{col}}}{k_B T} \quad \Rightarrow \quad E_{\text{col}} = x \, k_B T \quad \text{and} \quad dE_{\text{col}} = k_B T \, dx
\]
Substituting into \( P(E_{\text{col}}) \):
\[
P(E_{\text{col}}) = \frac{1}{\mathcal{N}} \left( d_1 (x \, k_B T)^{1 - \frac{v_1}{2}} + d_2 (x \, k_B T)^{1 - \frac{v_2}{2}} \right) e^{-x}
\]

\subsection*{2. Factor Out Constants}
\[
P(E_{\text{col}}) = \frac{(k_B T)^{1 - \frac{v_1}{2}}}{\mathcal{N}} d_1 x^{1 - \frac{v_1}{2}} e^{-x} + \frac{(k_B T)^{1 - \frac{v_2}{2}}}{\mathcal{N}} d_2 x^{1 - \frac{v_2}{2}} e^{-x}
\]

\subsection*{3. Set Up the Normalization Condition}
\[
\int_{0}^{\infty} P(E_{\text{col}}) \, dE_{\text{col}} = 1
\]
Expressing this in terms of \( x \):
\[
\frac{(k_B T)^{1 - \frac{v_1}{2}} d_1}{\mathcal{N}} \int_{0}^{\infty} x^{1 - \frac{v_1}{2}} e^{-x} \, dx + \frac{(k_B T)^{1 - \frac{v_2}{2}} d_2}{\mathcal{N}} \int_{0}^{\infty} x^{1 - \frac{v_2}{2}} e^{-x} \, dx = 1
\]

\subsection*{4. Evaluate the Integrals}
The integral:
\[
\int_{0}^{\infty} x^{n-1} e^{-x} \, dx = \Gamma(n)
\]
applies to the Gamma function, defined for \( n > 0 \). Using this:
\[
\int_{0}^{\infty} x^{1 - \frac{v_i}{2}} e^{-x} \, dx = \Gamma\left(2 - \frac{v_i}{2}\right), \quad \text{for } i = 1, 2
\]
For convergence:
\[
2 - \frac{v_i}{2} > 0 \quad \Rightarrow \quad v_i < 4
\]

\subsection*{5. Solve for \( \mathcal{N} \)}
Substitute the evaluated integrals back into the normalization condition:
\[
\frac{1}{\mathcal{N}} \left[ d_1 (k_B T)^{1 - \frac{v_1}{2}} \Gamma\left(2 - \frac{v_1}{2}\right) + d_2 (k_B T)^{1 - \frac{v_2}{2}} \Gamma\left(2 - \frac{v_2}{2}\right) \right] = 1
\]
Solve for \( \mathcal{N} \):
\[
\mathcal{N} = d_1 (k_B T)^{1 - \frac{v_1}{2}} \Gamma\left(2 - \frac{v_1}{2}\right) + d_2 (k_B T)^{1 - \frac{v_2}{2}} \Gamma\left(2 - \frac{v_2}{2}\right)
\]

\section{Final Normalized Collision Energy Distribution}
Substituting \( \mathcal{N} \) into \( P(E_{\text{col}}) \):
\[
P(E_{\text{col}}) = \frac{d_1 E_{\text{col}}^{1 - \frac{v_1}{2}} + d_2 E_{\text{col}}^{1 - \frac{v_2}{2}}}{d_1 \Gamma\left(2 - \frac{v_1}{2}\right) (k_B T)^{1 - \frac{v_1}{2}} + d_2 \Gamma\left(2 - \frac{v_2}{2}\right) (k_B T)^{1 - \frac{v_2}{2}}} \exp\left( -\frac{E_{\text{col}}}{k_B T} \right)
\]

\section{Summary}
The normalized collision energy distribution is:
\[
P(E_{\text{col}}) = \frac{d_1 \, E_{\text{col}}^{1 - \frac{v_1}{2}} + d_2 \, E_{\text{col}}^{1 - \frac{v_2}{2}}}{d_1 \Gamma\left(2 - \frac{v_1}{2}\right) (k_B T)^{1 - \frac{v_1}{2}} + d_2 \Gamma\left(2 - \frac{v_2}{2}\right) (k_B T)^{1 - \frac{v_2}{2}}} \exp\left( -\frac{E_{\text{col}}}{k_B T} \right)
\]
where:
\begin{itemize}
    \item \( \Gamma(n) \): Gamma function,
    \item \( d_1, d_2 \): constants from the collision cross-section,
    \item \( v_1, v_2 \): exponents from the collision cross-section,
    \item \( k_B \): Boltzmann's constant,
    \item \( T \): absolute temperature,
    \item \( E_{\text{col}} \): collision energy.
\end{itemize}




In simulations of colliding particles within a gas, accurately sampling the collision energy \( E_{\text{coll}} \) is crucial for modeling collision dynamics and predicting transport properties. This chapter presents the mathematical foundation and a numerical algorithm for sampling \( E_{\text{coll}} \) from a specified distribution:

\[
P(E_{\text{coll}}) \propto \left( d_1 \cdot E_{\text{coll}}^{1 - \frac{v_1}{2}} + d_2 \cdot E_{\text{coll}}^{1 - \frac{v_2}{2}} \right) \cdot (E - E_{\text{coll}})^a
\]

where:
\begin{itemize}
    \item \( E > E_{\text{coll}} \) is the total energy available for the collision,
    \item \( d_1, d_2 > 0 \) are constants determining the weight of each term,
    \item \( v_1, v_2 \geq 0 \) are exponents dictating the energy dependence of the collision cross-section,
    \item \( a \geq -1 \) modifies the influence of the remaining energy \( (E - E_{\text{coll}}) \).
\end{itemize}

The sampling procedure leverages the Gamma distribution to construct a mixture of Beta distributions, enabling efficient and accurate generation of \( E_{\text{coll}} \) values.

\section{Mathematical Formulation}

\subsection*{Target Distribution}

The goal is to sample \( E_{\text{coll}} \) from the distribution:

\[
P(E_{\text{coll}}) = \frac{ \left( d_1 \cdot E_{\text{coll}}^{1 - \frac{v_1}{2}} + d_2 \cdot E_{\text{coll}}^{1 - \frac{v_2}{2}} \right) \cdot (E - E_{\text{coll}})^a }{ \mathcal{N} }, \quad \text{for } 0 < E_{\text{coll}} < E
\]

where \( \mathcal{N} \) is the normalization constant ensuring that \( P(E_{\text{coll}}) \) integrates to 1 over the interval \( [0, E] \).

\subsection*{Dimensionless Transformation}

Introduce a dimensionless variable \( x \):

\[
x = \frac{E_{\text{coll}}}{E} \quad \Rightarrow \quad E_{\text{coll}} = E \cdot x, \quad \text{where } 0 < x < 1
\]

Substituting into \( P(E_{\text{coll}}) \):

\[
P(x) = \frac{ \left( d_1 \cdot (E x)^{1 - \frac{v_1}{2}} + d_2 \cdot (E x)^{1 - \frac{v_2}{2}} \right) \cdot (E - E x)^a }{ \mathcal{N} } \cdot E = \frac{ d_1 \cdot x^{1 - \frac{v_1}{2}} \cdot (1 - x)^a + d_2 \cdot x^{1 - \frac{v_2}{2}} \cdot (1 - x)^a }{ \mathcal{N}' }
\]

where \( \mathcal{N}' \) encapsulates constants involving \( E \).

\subsection*{Mixture of Beta Distributions}

The transformed distribution \( P(x) \) can be expressed as a mixture of two Beta distributions:

\[
P(x) = p \cdot \text{Beta}(x; \alpha_1, \beta) + (1 - p) \cdot \text{Beta}(x; \alpha_2, \beta)
\]

where:
\begin{itemize}
    \item \( \alpha_1 = 2 - \frac{v_1}{2} \),
    \item \( \alpha_2 = 2 - \frac{v_2}{2} \),
    \item \( \beta = a + 1 \),
    \item \( p = \frac{d_1 \cdot \Gamma(\alpha_1)}{d_1 \cdot \Gamma(\alpha_1) + d_2 \cdot \Gamma(\alpha_2)} \).
\end{itemize}

\textbf{Gamma Function Relationship:} A Beta(\( \alpha \), \( \beta \)) distribution can be sampled using two independent Gamma(\( \alpha \), 1) and Gamma(\( \beta \), 1) random variables \( X \) and \( Y \):
\[
\text{Beta}(\alpha, \beta) = \frac{X}{X + Y}
\]

\section{Sampling Procedure}

The procedure to sample \( E_{\text{coll}} \) is as follows:

\begin{enumerate}
    \item \textbf{Input Parameters:}
    \begin{itemize}
        \item Constants \( d_1, d_2 > 0 \),
        \item Exponents \( v_1, v_2 < 4 \),
        \item Exponent \( a \geq -1 \),
        \item Total energy \( E > 0 \).
    \end{itemize}
    \item \textbf{Compute Beta Distribution Parameters:}
    \[
    \alpha_1 = 2 - \frac{v_1}{2}, \quad \alpha_2 = 2 - \frac{v_2}{2}, \quad \beta = a + 1
    \]
    \item \textbf{Calculate Mixture Weight \( p \):}
    \[
    p = \frac{d_1 \cdot \Gamma(\alpha_1)}{d_1 \cdot \Gamma(\alpha_1) + d_2 \cdot \Gamma(\alpha_2)}
    \]
    \item \textbf{Sample from Mixture of Beta Distributions:}
    \begin{itemize}
        \item Generate a uniform random number \( u \in [0, 1] \),
        \item If \( u < p \), sample \( x \) from Beta(\( \alpha_1, \beta \)),
        \item Else, sample \( x \) from Beta(\( \alpha_2, \beta \)).
    \end{itemize}
    \item \textbf{Transform to Collision Energy:}
    \[
    E_{\text{coll}} = E \cdot x
    \]
    \item \textbf{Output \( E_{\text{coll}} \):} The sampled collision energy lies within \( (0, E) \).
\end{enumerate}

\section{C++ Implementation}

The following C++ function \texttt{sample\_E\_coll} implements the sampling procedure using the C++11 \texttt{<random>} library.

\begin{lstlisting}[language=C++,caption={C++ Implementation of Collision Energy Sampling}]
#include <iostream>
#include <random>
#include <stdexcept>
#include <cmath>    // For std::tgamma

double sample_E_coll(double d1, double d2, double v1, double v2, double a, double E, std::mt19937 &gen) {
    if (d1 <= 0.0 || d2 <= 0.0 || v1 >= 4.0 || v2 >= 4.0 || a < -1.0 || E <= 0.0) {
        throw std::invalid_argument("Invalid parameters.");
    }

    double alpha1 = 2.0 - (v1 / 2.0);
    double alpha2 = 2.0 - (v2 / 2.0);
    double beta = a + 1.0;

    double gamma_alpha1 = std::tgamma(alpha1);
    double gamma_alpha2 = std::tgamma(alpha2);

    double p = (d1 * gamma_alpha1) / (d1 * gamma_alpha1 + d2 * gamma_alpha2);

    std::uniform_real_distribution<double> uniform_dist(0.0, 1.0);
    double u = uniform_dist(gen);

    std::gamma_distribution<double> gamma_dist_alpha1(alpha1, 1.0);
    std::gamma_distribution<double> gamma_dist_alpha2(alpha2, 1.0);
    std::gamma_distribution<double> gamma_dist_beta(beta, 1.0);

    double x;
    if (u < p) {
        double X = gamma_dist_alpha1(gen);
        double Y = gamma_dist_beta(gen);
        x = X / (X + Y);
    } else {
        double X = gamma_dist_alpha2(gen);
        double Y = gamma_dist_beta(gen);
        x = X / (X + Y);
    }

    return E * x;
}
\end{lstlisting}

\section{Conclusion}

This method leverages the relationship between Beta and Gamma distributions to efficiently sample \( E_{\text{coll}} \) values. Accurate sampling ensures realistic simulations of collision dynamics and transport properties in gases.
