\chapter{The model of internal degrees of freedom}


\chapter{Energy Distribution of a Colliding Pair in a Thermalized Gas}

\section{Introduction}

In a gas at thermal equilibrium at temperature $T$, the particles' velocities are distributed according to the Maxwell-Boltzmann distribution. Consider two particles selected at random from this gas. One may ask: what is the probability distribution of their \emph{relative kinetic energy}? Moreover, if we focus on pairs of particles that are actually colliding, the effective weighting of different relative speeds (and hence energies) changes due to the collision cross section and relative velocity dependence.

In this chapter, we derive the probability distributions for the relative energy of a pair of particles first in the absence of any collision-weighting factor (i.e., selecting pairs at random), and then considering a collision-weighted scenario where the collision frequency depends on the relative speed $g$ and the collision cross section $\sigma(g)$. We also consider the case where $\sigma(g)$ is a power-law function of $g$, i.e., $\sigma(g) \propto g^{v}$.

\section{Relative Velocity and Energy Distributions}

\subsection{Maxwell-Boltzmann Distribution for Relative Velocity}

Consider two particles with masses $m_1$ and $m_2$ in a gas at thermal equilibrium at temperature $T$. Define the reduced mass
\begin{equation}
  \mu = \frac{m_1 m_2}{m_1 + m_2}.
\end{equation}

The velocity distribution of each particle is Maxwell-Boltzmann. The relative velocity of the pair is given by
\begin{equation}
  \mathbf{g} = \mathbf{v}_1 - \mathbf{v}_2.
\end{equation}

In three dimensions, the probability density function (PDF) for the magnitude of the relative speed $g = |\mathbf{g}|$ (when the particles are independently drawn from a Maxwellian distribution) is:
\begin{equation}
  f(g) \propto g^{2} \exp\left(-\frac{\mu g^{2}}{2 k_B T}\right).
\end{equation}

This result stems from the convolution of two Maxwellian distributions and the definition of the reduced mass. The key point is that the relative velocity itself follows a Maxwellian form with mass parameter $\mu$.

\subsection{Energy Distribution for Randomly Chosen Pairs}

The relative kinetic energy of the pair is defined as
\begin{equation}
  E = \frac{1}{2}\mu g^{2}.
\end{equation}

To find the distribution in terms of $E$, we rewrite:
\begin{equation}
  g^{2} = \frac{2E}{\mu}.
\end{equation}
Thus,
\begin{equation}
  f(g) \propto \left(\frac{2E}{\mu}\right)^{\!1} \exp\left(-\frac{E}{k_B T}\right).
\end{equation}

We must also change variables from $g$ to $E$. Since $E = \frac{1}{2}\mu g^{2}$, the differential relation is
\begin{equation}
  dE = \mu g \, dg \implies dg = \frac{dE}{\mu g} = \frac{dE}{\mu \sqrt{\frac{2E}{\mu}}} = \frac{dE}{\sqrt{2\mu E}}.
\end{equation}

Substituting $g = \sqrt{2E/\mu}$ into $f(g)$ and including the Jacobian $dg/dE$, we have
\begin{align}
  f(E) &= f(g) \frac{dg}{dE} \nonumber \\
       &\propto \left(\frac{2E}{\mu}\right) \exp\left(-\frac{E}{k_B T}\right) \cdot \frac{1}{\sqrt{2\mu E}}.
\end{align}

Simplifying the powers of $E$:
\begin{equation}
  \left(\frac{2E}{\mu}\right) \frac{1}{\sqrt{2\mu E}} = E^{1 - \frac{1}{2}} = E^{1/2}.
\end{equation}

Thus, the unweighted distribution of the relative energy of two randomly chosen particles is
\begin{equation}
  f_{\text{pair}}(E) \propto E^{1/2} e^{-E/(k_B T)}.
\end{equation}

\section{Collision-Weighted Distribution}

\subsection{Inclusion of Collision Frequency}

When considering the energy distribution of pairs \emph{at the moment they collide}, we must weight the distribution by the collision frequency. The collision frequency for two particles at relative speed $g$ is proportional to
\begin{equation}
  \text{collision frequency} \propto g \sigma(g),
\end{equation}
where $\sigma(g)$ is the collision cross section. Physically, $g$ represents how fast one particle "encounters" the other, and $\sigma(g)$ represents the effective area for collision at that relative speed.

Therefore, the collision-weighted speed distribution becomes
\begin{equation}
  f_{\text{collision}}(g) \propto f(g) \, g \sigma(g).
\end{equation}

\subsection{Constant Cross Section Case}

First, consider the simplest case $\sigma(g) = \text{const}$. Let $\sigma(g) = \sigma_0$. In this case,
\begin{equation}
  f_{\text{collision}}(g) \propto f(g) \cdot g = g^{3} e^{-\frac{\mu g^{2}}{2 k_B T}}.
\end{equation}

Converting to energy:
\begin{equation}
  f_{\text{collision}}(E) \propto E e^{-E/(k_B T)},
\end{equation}
after going through the same change-of-variables steps as before. Notice that this differs from the unweighted distribution by an extra factor of $E^{1/2}$ turning into $E$, reflecting the bias towards higher relative speeds in actual collisions.

\subsection{General Case: $\sigma(g) \propto g^{v}$}

Now consider a more general scenario where the collision cross section depends on relative speed as a power-law:
\begin{equation}
  \sigma(g) \propto g^{v}.
\end{equation}

In that case, the collision-weighted distribution in terms of $g$ is
\begin{equation}
  f_{\text{collision}}(g) \propto f(g) \cdot g \cdot g^{v} = g^{3+v} \exp\left(-\frac{\mu g^{2}}{2 k_B T}\right).
\end{equation}

Again, we transform to energy $E = \tfrac{1}{2}\mu g^{2}$. Thus,
\begin{equation}
  g^{2} = \frac{2E}{\mu} \implies g = \sqrt{\frac{2E}{\mu}}.
\end{equation}
So
\begin{equation}
  g^{3+v} = \left(\frac{2E}{\mu}\right)^{\frac{3+v}{2}}.
\end{equation}

The collision-weighted distribution in terms of $E$ is
\begin{equation}
  f_{\text{collision}}(E) \propto \left(\frac{2E}{\mu}\right)^{\frac{3+v}{2}} e^{-E/(k_B T)} \cdot \frac{dE}{dg}.
\end{equation}

Recall:
\begin{equation}
  dg = \frac{dE}{\sqrt{2\mu E}},
\end{equation}
so
\begin{equation}
  \frac{dg}{dE} = \frac{1}{\sqrt{2\mu E}}.
\end{equation}

Substituting this, we get
\begin{align}
  f_{\text{collision}}(E) &\propto \left(\frac{2E}{\mu}\right)^{\frac{3+v}{2}} e^{-\frac{E}{k_B T}} \frac{1}{\sqrt{2\mu E}} \\
  &= \left(\frac{2E}{\mu}\right)^{\frac{3+v}{2}} \frac{1}{(2\mu E)^{1/2}} e^{-\frac{E}{k_B T}}.
\end{align}

Focus on the powers of $E$:
\begin{equation}
  E^{\frac{3+v}{2}} \cdot E^{-\frac{1}{2}} = E^{\frac{3+v}{2} - \frac{1}{2}} = E^{\frac{2+v}{2}} = E^{1 + \frac{v}{2}}.
\end{equation}

Thus, the final form is:
\begin{equation}
  f_{\text{collision}}(E) \propto E^{1 + \frac{v}{2}} e^{-\frac{E}{k_B T}}.
\end{equation}

\section{Discussion and Special Cases}

The exponent in the energy dependence is directly influenced by the exponent $v$ in the cross section's speed dependence:
\begin{itemize}
  \item For $v=0$ (constant cross section), we recover 
  \[
  f_{\text{collision}}(E) \propto E \, e^{-\frac{E}{k_B T}},
  \]
  which is the standard collision-energy distribution with no additional $g$-dependence in the cross section.

  \item For $v > 0$, higher speeds are more favorable, increasing the exponent of $E$ and thus placing more weight on higher energies.

  \item For $v < 0$, lower speeds are more favorable, decreasing the exponent of $E$ and thus placing more weight on lower energies.
\end{itemize}

\section{Conclusion}

In summary:
\begin{enumerate}
  \item A randomly chosen pair of particles from a Maxwellian gas has a relative energy distribution
  \[
  f_{\text{pair}}(E) \propto E^{1/2} e^{-\frac{E}{k_B T}}.
  \]

  \item Actual collisions are biased towards pairs with higher relative speeds. If the collision cross section is constant, the energy distribution of colliding pairs becomes
  \[
  f_{\text{collision}}(E) \propto E \, e^{-\frac{E}{k_B T}}.
  \]

  \item If the collision cross section scales as $\sigma(g) \propto g^{v}$, this modifies the distribution to
  \[
  f_{\text{collision}}(E) \propto E^{1 + \frac{v}{2}} e^{-\frac{E}{k_B T}}.
  \]
  \end{enumerate}

These results provide a comprehensive picture of how the underlying physics of collision cross sections and relative motion influences the observed distribution of collision energies in a thermalized gas.
