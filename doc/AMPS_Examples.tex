\chapter{Examples}


\section{Comet 67P/Churyumov-Gerasimenko}

\paragraph{}
This application of AMPS is used to model the rarefied atmosphere, the so-called coma, of the Rosetta target comet 67P/Churyumov-Gerasimenko in 3D. Both the gas and dust coma can be modeled using a realistic nucleus shape derived from inversion of a light curve obtained with Hubble Space Telescope measurements (Lamy et al. 2006). The gas flux distribution and the surface temperature of the nucleus are adapted from the thermophysical model from Davidsson and Gutierrez (2004, 2005, 2006), previously used in Tenishev et al. (2008, 2011), from a spherical nucleus to a more realistic shape using the angle between the local normal and the direction of the Sun.

\paragraph{}
Input commands related to this application need to be marked by {\tt \#block ../cg.pl} at the beginning and {\tt \#endblock} at the end.

\begin{itemize}

\item {\bf Gravity3D}=[off,on] 

Defines the gravity mode that will be used, two models are available:
\\{\tt Gravity3D=off} - the computation of the gravity acceleration will be done using a spherical approximation of the nucleus,
\\{\tt Gravity3D=on} - the computation of the gravity acceleration will be done summing over the contribution of about 1250 tetrahedrons meshing the nucleus shape from Lamy et al. (2006).

\item {\bf HeliocentricDistance}={\it the comet�s heliocentric distance}

Defines the heliocentric distance at which the comet is located. The user can used the \_AU\_ notation give the distance in astronomical unit.
\\Examples: {\tt HeliocentricDistance=3.3*\_AU\_} for a comet located at 3.3 AU from the Sun.

\item {\bf SubsolarPointAzimuth}={\it the angle between the Sun position and the x-axis}

Defines the angle between the x-axis and the Sun�s position in the Z-plane, enabling to reproduce different geometries. The value of the angle must be given in radians. A value of 0 will align the Sun with the positive values of the x-axis. 
\\Examples: {\tt SubsolarPointAzimuth=Pi/2} places the Sun 90 degrees with respect to the x-axis. 

\item {\bf RadiativeCoolingMode}=[off,Crovisier]

Selects the water radiative cooling mode that will be used in the simulation, two modes are available:
\\{\tt RadiativeCoolingMode=off} - no radiative cooling,
\\{\tt RadiativeCoolingMode=Crovisier} - uses the radiative cooling computed by Crovisier (1984).

\item {\bf ndist}=[0,1,2,3,4]

Determines the column that will be used from the table of inner boundary conditions derived from the thermophysical model from Davidsson and Gutierrez (2004, 2005, 2006). Five different distributions are available:
\\{\tt ndist=0} - corresponds to the 1.3 AU distribution,
\\{\tt ndist=1} - corresponds to the 2.0 AU distribution,
\\{\tt ndist=2} - corresponds to the 2.7 AU distribution,
\\{\tt ndist=3} - corresponds to the 3.0 AU distribution,
\\{\tt ndist=4} - corresponds to the 3.3 AU distribution,

\item {\bf BjornProductionRate}

Gives the gas production rate that will be distributed using the thermophysical model from Davidsson and Gutierrez (2004, 2005, 2006).
\\Examples: {\tt  BjornProductionRate(H2O)=1.0e24}

\item {\bf DustTotalMassProductionRate}

Gives the total dust mass flux that will be injected in the simulation.
\\Examples: {\tt DustTotalMassProductionRate=1.0}

\item {\bf PowerLawIndex}

Defines the index used in the power law of the dust grain size distribution that will be injected: $f=\left(a/a_{0}\right)^{-N}$ with N the index given in argument. Typical values of N vary between 2.7 and 4.5. It is necessary to point out that this size distribution corresponds to the injected particles, and that since the expansion velocity of the grains depends on their size, the size distribution in the coma will be different than the one injected. The 
\\Examples: {\tt PowerLawIndex=4}

\item {\bf dustRmin}

Defines the minimal radius size of the dust grains.
\\Examples: {\tt dustRmin=1.0e-7}

\item {\bf dustRmax}

Defines the maximal radius size of the dust grains.
\\Examples: {\tt dustRmax=1.0e-2}

\item {\bf nDustRadiusGroups}

Defines the number of dust sampling bins that will be used in the simulation.
\\Examples: {\tt nDustRadiusGroups=10}